% Paper Style Packages
% \documentclass[twoside, 10pt, a4paper]{book}

% % \def\dbar{{\mathchar'26\mkern-12mu d}} 
% % \usepackage{tocbibind}
% % \setlength{\tolerance}{1}                   % Prevent hyphenation / overflowing
% % \setlength{\emergencystretch}{\maxdimen}    % -//-
% % \setlength{\hyphenpenalty}{10000}           % -//-
% % \setlength{\hbadness}{10000}                % -//-
% % \usepackage{fancyhdr}           % Page numerating style
% % \setlength{\headheight}{15pt}   % -//-
% % \pagestyle{fancy}               % -//-
% % \fancyhf{}                      % -//-
% % \fancyhead[L]{\leftmark}        % -//-
% % \fancyhead[R]{\thepage}         % -//-
% \usepackage[pscoord]{eso-pic}
% \usepackage{minitoc}
% \usepackage{lscape}
% \usepackage{afterpage}
% % Set the depth and numbering
% % for table of contents (toc)
% % \setcounter{tocdepth}{5}
% % \setcounter{secnumdepth}{2}
% \usepackage{titlesec}

% %%% FORMAT THE STYLE OF CHAPTERS, SECTIONS ETC %%%
% % Add a horizontal line in the header
% \newpagestyle{main}{%
%   \sethead[\thepage][][\chaptertitle]{\thesection\ \sectiontitle}{}{\thepage}
%   \headrule
% }
% \pagestyle{main}

% % Display format of the chapter
% \titleformat{\chapter}[display]
% {\large}
% {\filleft\MakeUppercase{\chaptertitlename} \Huge\thechapter}
% {2ex}
% {\LARGE\bfseries\filleft}
% [\vspace{2ex}%
% \titlerule]

% % Display format of the section
% \titleformat{\section}[block]
% {\filcenter\large
% \addtolength{\titlewidth}{2pc}%
% \titleline*[c]{\titlerule*[.6pc]{\tiny\textbullet}}%
% \addvspace{6pt}%
% \normalfont\bfseries\sffamily}
% {\thesection}{1em}{}
% \titlespacing{\section}
% {5pc}{*2}{*2}[5pc]

% % Display format of the subsection
% \titleformat{\subsection}[block]
% {\normalfont\sffamily}
% {\thesubsection}{.5em}{\titlerule\\[.8ex]\bfseries}

% % Display format of the subsubsection
% \titleformat{\subsubsection}[wrap]
% {\normalfont\fontseries{b}\selectfont\filright}
% {\thesubsubsection.}{.5em}{}
% \titlespacing{\subsubsection}
% {12pc}{1.5ex plus .1ex minus .2ex}{1pc}

% %%% END OF FORMATING %%%

% \usepackage[              % Left and right margins
%     DIV=14,               % You can play with these numbers for a better
%     BCOR=1cm,             % reference look into KOMA-Script CTAN manual
%     headinclude=true,
%     footinclude=false,
%     paper=A4
%     ]{typearea}


% % Language Packages
% \usepackage[unicode]{hyperref}
% \hypersetup{colorlinks,linkcolor={black},citecolor={black},urlcolor={blue}}
% \usepackage[LGR, T1]{fontenc}
% \usepackage[utf8]{inputenc}
% \usepackage[english, greek]{babel}
% \usepackage{alphabeta}
% \DeclareTextCompositeCommand{\acctonos}{PU}{\textalpha}{\9003\254}
% \DeclareTextCompositeCommand{\acctonos}{PU}{\textepsilon}{\9003\255}
% \DeclareTextCompositeCommand{\acctonos}{PU}{\texteta}{\9003\256}
% \DeclareTextCompositeCommand{\acctonos}{PU}{\textiota}{\9003\257}
% \DeclareTextCompositeCommand{\acctonos}{PU}{\textomicron}{\9003\314}
% \DeclareTextCompositeCommand{\acctonos}{PU}{\textomega}{\9003\316}
% \DeclareTextCommand{\textfinalsigma}{PU}{\9003\302}
% \usepackage{csquotes}
% \usepackage{dirtytalk}
% \usepackage{amsmath, esint}
% \usepackage{amssymb}
% \usepackage{mathtools}
% \usepackage{xcolor}
% \usepackage{cancel}
% \usepackage{enumitem}
% \usepackage{subcaption}
% \usepackage{mathtools}
% \newcommand{\dbar}{{d\mkern-7mu\mathchar'26\mkern-2mu}} % Define inexact differential

% % Bibliography Packages
% \usepackage[backend=biber, style=numeric, natbib=true, sorting=none]{biblatex}
% \usepackage{biblatex}
% \addbibresource{bibliography.bib}


% % Graphic Packages
% \usepackage{graphicx}
% \usepackage{caption}



% % Paragraph Packages
% \usepackage{indentfirst}            % Indent at the start of every paragraph
% \setlength{\parindent}{0em}         % Size of indent
% \setlength{\parskip}{1em}           % Space between paragraphs


% % Document Details
% \title{Εισαγωγή στην Αστροφυσική}
% \author{Χανλαρίδης Σάββας}


% \begin{document}
%     \begin{titlepage}
    \newcommand{\HRule}{\rule{\linewidth}{0.5mm}}
    \center
    % \textsl{\Huge Όνομα Σχολής}\\[0.5cm] 
    % \textsl{\Large Όνομα Τμήματος}\\[2cm] 
    \makeatletter
    \HRule \\[0.6cm]
    { \huge \bfseries \@title}\\[0.3cm] 
    \HRule \\[2cm]
    \large
    \vspace{5cm}
    \center 
    \includegraphics[width=\linewidth]{title/skinakas_star_trails.jpg}
    
   
    % \begin{minipage}{0.45\textwidth}
    % 	\begin{flushleft}
    %         \emph{Συγγραφέας:}\\
    %         \@author \\
    %     \end{flushleft}
    % \end{minipage}
    % ~
    % \begin{minipage}{0.45\textwidth}
    % 	\begin{flushright}
    %         \emph{Υπεύθυνος Καθηγητής:} \\
    %         \textup{Όνομα Καθηγητή}
    %     \end{flushright}
    % \end{minipage}\\[3cm]
    
    % \makeatother
    
    % {\large Η εργασία κατατέθηκε για το μάθημα:}\\[0.2cm]
    % {\Large \emph{Όνομα Μαθήματος}}\\[1cm]
    % {\large \today}\\[2cm]
    % \vfill 
    
\end{titlepage}
    
%     \selectlanguage{english}
    
%     % Define names of report parts
%     \renewcommand{\contentsname}{Περιεχόμενα}
%     \renewcommand{\listfigurename}{Λίστα Σχημάτων}
%     \renewcommand{\listtablename}{Λίστα Πινάκων}
%     \renewcommand{\chaptername}{Κεφάλαιο}
%     \renewcommand{\appendixname}{Παράρτημα}
%     \renewcommand{\bibname}{Βιβλιογραφία}
    
%     \pagenumbering{roman}
%     %\setlength{\parskip}{0.5em}     % Space between paragraphs
    

%     \tableofcontents
%     % \listoffigures
%     % \listoftables
    
%     \pagenumbering{arabic}
%     %\setlength{\parskip}{1em}       % Space between paragraphs

 
%     \chapter{Φωτομετρία \& Φασματοσκοπία Αστέρων}
\label{ch:Chapter3}
{\hypersetup{linkcolor=black, pdfborder=0 0 1}
	\minitoc
	%\newpage
}
\section{Βασικά στοιχεία φωτομετρίας} 
% Συστήματα φίλτρων, ορισμός βολομετρικών μεγεθών, βολομετρική διόρθωση, υπεροχή χρώματος κτλ. %
Ο όρος ``φωτομετρία'' αναφέρεται στην τεχνική μέτρησης της ηλεκτρομαγνητικής ροής ενός ουράνιου αντικειμένου. Συνήθως αυτή πραγματοποιείται χωρίζοντας το φάσμα του αστέρα σε διάφορες ζώνες διέλευσης (φίλτρα) και στη συνέχεια καταγράφεται η ροή σε κάθε φίλτρο χρησιμοποιώντας κάποιο φωτοευαίσθητο όργανο (π.χ. κάμερα CCD). Το σύνολο των ζωνών διέλευσης, που αντιστοιχεί σε ένα πολύ μεγάλο εύρος μηκών κύματος της ακτινοβολίας, ονομάζεται ``φωτομετρικό σύστημα''. 
Συνδυάζοντας φωτομετρικές μετρήσεις σε διάφορα φίλτρα μπορούμε να προσδιορίσουμε την φωτεινότητα του αστέρα (αν η απόσταση του είναι γνωστή) καθώς και άλλες φυσικές ιδιότητές του όπως η επιφανειακή του θερμοκρασία.

% \underline{Ορισμός βολομετρικών μεγεθών}
\subsection{Ορισμός βολομετρικών μεγεθών}
Ακολουθώντας την ίδια λογική που χρησιμοποιήσαμε για να εξάγουμε το φαινόμενο (απόλυτο) μέγεθος ενός αστέρα, μπορούμε να ορίσουμε και το βολομετρικό φαινόμενο (απόλυτο) μέγεθος ως:

\begin{itemize}
    \item Ολική (ή βολομετρική) φωτεινότητα αστέρων 
    $$F_{\text{bol}} = \int_0^{\infty}F(\lambda) d\lambda = \frac{L_{\text{bol}}}{4 \pi d^2} = \frac{4\pi R^2 \sigma T_{\text{eff}}^4}{4\pi d^2}$$
    Το $F(\lambda)$ είναι το παρατηρούμενο φάσμα του αστέρα. Στο βαθμό που ο αστέρας συμπίπτει με μέλαν σώμα είναι ουσιαστικά το $B_{\lambda}$ της σχέσης του Planck.
    \item Ολικό (ή βολομετρικό) \textit{φαινόμενο} μέγεθος αστέρα 
    $$m_{\text{bol}} = -2.5 \log \left( \frac{F_{\text{bol}}}{F_{\odot}} \right) = 2.5 \log F_{\odot} - 2.5 \log F_{\text{bol}}$$
    
    Στην περίπτωση που μιλάμε για βολομετρικά μεγέθη, η σταθερά βαθμονόμησης ορίζεται βάσει της φωτεινότητας του Ήλιου και όχι του Vega!
    \item Ολικό (ή βολομετρικό) \textit{απόλυτο} μέγεθος αστέρα
    $$M_{\text{bol}} = m_{\text{bol}} \ \text{όταν} \ d = 10 \ \text{pc} \longrightarrow M_{\text{bol}} = -2.5 \log \left( \frac{F_{\text{bol}, 10 \text{pc}}}{F_{\odot}} \right)$$
\end{itemize}
\hrule 
\underline{\textbf{Παράδειγμα}}:
\textbf{Γνωρίζοντας ότι το απόλυτο βολομετρικό μέγεθος του Ήλιου είναι $M_{\odot}^{\text{bol}} = 4.74$, βρείτε μία σχέση που να συνδέει το απόλυτο μέγεθος ενός αστέρα με την λαμπρότητά του καθώς και αυτή του Ήλιου.}

Εξ' ορισμού το απόλυτο βολομετρικό μέγεθος του Ήλιου θα είναι
$$M_{\odot}^{\text{bol}} = -2.5 \log \left( \frac{F_{\odot, \text{10pc}}^{\text{bol}}}{F_{\odot}} \right) \Rightarrow 2.5 \log F_{\odot} - 2.5 \log F_{\odot, \text{10pc}}^{\text{bol}} = 4.74 $$

Για έναν τυχαίο αστέρα, το απόλυτο βολομετρικό μέγεθός του θα είναι κατά αντιστοιχία
$$M_{\text{bol}} = 2.5 \log F_{\odot} - 2.5 \log F_{\text{bol, 10pc}}$$

Πρέπει να εμφανίσουμε τον όρο $F_{\odot, \text{10pc}}^{\text{bol}}$, το οποίο το καταφέρνουμε με το να τον προσθαφαιρέσουμε από τη σχέση του απόλυτου βολομετρικού μεγέθους του αστέρα μας. Έτσι έχουμε:

\begin{eqnarray*}
    M_{\text{bol}} &=& {\color{blue} 2.5 \log F_{\odot}} - 2.5 \log F_{\text{bol, 10pc}} + 2.5 \log F_{\odot, \text{10pc}}^{\text{bol}} - {\color{blue} 2.5 \log F_{\odot, \text{10pc}}^{\text{bol}}} = \\\\
    &=& {\color{blue}4.74} - 2.5 \log F_{\text{bol, 10pc}} + 2.5 \log F_{\odot, \text{10pc}}^{\text{bol}} \Rightarrow \\\\
    &\Rightarrow & 4.74 - M_{\text{bol}} = 2.5 \log \left[ \frac{L_{\text{bol}}}{4\pi (\text{10pc})^2} \right] - 2.5 \log \left[ \frac{L_{\odot}}{4\pi (\text{10pc})^2} \right] = \\\\
    &=& 2.5 \log \left( \frac{L_{\text{bol}}}{L_{\odot}} \right) \Rightarrow 0.4(4.74 - M_{\text{bol}}) = \log \left( \frac{L_{\text{bol}}}{L_{\odot}} \right) \Rightarrow \\\\
    &\Rightarrow & \boxed{\frac{L_{\text{bol}}}{L_{\odot}} = 10^{0.4(4.74 - M_{\text{bol}})}}
\end{eqnarray*}

Βάσει αυτής της σχέσης, μπορούμε να υπολογίσουμε αμέσως πόσες φορές πιο φωτεινός είναι ένας αστέρας από τον Ήλιο, αν γνωρίζουμε το απόλυτο μέγεθός του. \\
%\hrule 

\subsection{Φωτομετρικά συστήματα}
Ο υπολογισμός συνολικών φωτεινοτήτων είναι πολύπλοκη διαδικασία και εξαιρετικά χρονοβόρα όταν μελετάμε συστήματα με χιλιάδες αστέρια, π.χ. σφαιρωτά σμήνη. Η συλλογή φωτός σε όλα τα μήκη κύματος, η προσαρμογή τους σε μέλανα σώματα κτλ για κάθε ένα από τα μέλη ενός σμήνους είναι απαγορευτική.
Γι' αυτό το λόγο χρησιμοποιούμε φίλτρα (ηθμούς) τα οποία μας επιτρέπουν την παρατήρηση σε ένα συγκεκριμένο εύρος συχνοτήτων/μηκών κύματος.

Υπάρχουν πολλά συστήματα φίλτρων όπως π.χ. το ``Johnson-Cousins''. Σε αυτό το σύστημα υπάρχουν 5 φίλτρα, τα U, B, V, R, I. Το διάγραμμα (σχήμα \ref{fig:filter_curves}) δείχνει τις \textit{καμπύλες διαπερατότητας} $S(\lambda)$ αυτών των φίλτρων.

\begin{figure}
    \centering
    \includegraphics[scale=0.4]{Figures/filter_curves.jpg}
    \caption{Καμπύλες διαπερατότητας για το σύστημα φίλτρων Johnson-Cousins. Ο άξονας y δείχνει την \textit{διαφάνεια}. Το φίλτρο V (visual) αφήνει να καταγραφούν τα ίδια φωτόνια (πάνω-κάτω) με αυτά που αντιλαμβάνονται τα μάτια μας. Δηλαδή με άλλα λόγια, η καμπύλη φασματικής ευαισθησίας του φίλτρου V είναι παρόμοια μ' εκείνη του ματιού μας.}
    \label{fig:filter_curves}
\end{figure}

Όταν παρατηρούμε ένα αστέρι μέσω ενός φίλτρου (π.χ. το V) τότε εμείς μετράμε:
\begin{equation}
    F_V = \int_0^{\infty} F(\lambda) S_V(\lambda) d\lambda
\end{equation}
όπου $F(\lambda)$ είναι το φάσμα που επέμπει ο αστέρας, και $S_V(\lambda)$ είναι η καμπύλη φασματικής ευαισθησίας, του φίλτρου και του ανιχνευτή.

Έτσι, μπορούμε να ορίσουμε τα μεγέθη ενός αστέρα στα διάφορα φίλτρα ως εξής:
\begin{eqnarray*}
    V \equiv m_V = -2.5 \log \left( \frac{F_V}{F_{V,0}} \right) = 2.5 \log F_{V,0} - 2.5 \log F_V = C_V - 2.5 \log F_V \\\\
    B \equiv m_B = -2.5 \log \left( \frac{F_B}{F_{B,0}} \right) = 2.5 \log F_{B,0} - 2.5 \log F_B = C_B - 2.5 \log F_B \\\\
    U \equiv m_U = -2.5 \log \left( \frac{F_U}{F_{U,0}} \right) = 2.5 \log F_{U,0} - 2.5 \log F_U = C_U - 2.5 \log F_U 
\end{eqnarray*}
όπου οι σταθερές $C_V, C_B, C_U$ κτλ ορίζονται με βάση τη φωτεινότητα του Vega στα αντίστοιχα μήκη κύματος, δηλαδή:
\begin{eqnarray*}
    C_V  &=& 2.5 \log(F_{V, Vega}) \\\\
    C_B  &=& 2.5 \log(F_{B, Vega}) \\\\
    C_U  &=& 2.5 \log(F_{U, Vega}) \hspace{1cm} \text{κτλ}
\end{eqnarray*}
Αυτό σημαίνει ότι εξ' ορισμού ισχύει $\boxed{m_{V,Vega} = m_{B,Vega} = m_{U,Vega} = \dots = 0}$
\\\\
\hrule 
\underline{\textbf{Παράδειγμα}}:
\textbf{Ποιό είναι το βολομετρικό μέγεθος ενός αστέρα αν υποθέσουμε ότι εκπέμπει μόνο στα U,B,V,R;}

Αν κάποιος υποθέσει ότι το συνολικό μέγεθος $m_{\text{bol}}$ είναι απλώς το άθροισμα των επιμέρους μεγεθών στα διάφορα φίλτρα, τότε είναι λάθος! Με άλλα λόγια: $$m_{\text{bol}} \neq m_U + m_B + m_V + m_R$$
γιατί το μέγεθος είναι λογαριθμική ποσότητα. Αυτό που ισχύει είναι ότι η \textit{συνολική ροή θα ισούται με το άθροισμα των ροών στα αντίστοιχα φίλτρα}. Έτσι, αν γνωρίζουμε τα $m_U, m_B$ κτλ και θέλουμε να βρούμε το $m_{\text{bol}}$ dουλεύουμε ως εξής:
$$m_B = -2.5 \log \left( \frac{F_B}{F_{B,0}} \right) \Rightarrow -0.4m_B = \log \left( \frac{F_B}{F_{B,0}} \right) \Rightarrow F_B = F_{B,0} 10^{-0.4m_B}$$
Παρόμοια βρίσκουμε και τα $F_U, F_V, F_R$ και άρα $F_{\text{bol}} = F_B+F_V+F_U+F_R$. Τελικά $$m_{\text{bol}} = -2.5 \log \left( \frac{F_{\text{bol}}}{F_0} \right)$$
%\hrule 

\subsubsection{Η βολομετρική διόρθωση}
Ο υπολογισμός της συνολικής φωτεινότητας ενός αστέρα είναι ο λόγος που κάνουμε παρατηρήσεις με περισσότερα από ένα φίλτρα. Παρόλα αυτά, υπάρχει και ένας άλλος τρόπος υπολογισμού, μέσω του $m_V$ και μόνο. Για αυτό τον λόγο θα πρέπει να ορίσουμε μία νέα ποσότητα η οποία είναι γνωστή ως \textit{βολομετρική διόρθωση} (bolometric correction) ή αλλιώς συντελεστής συνολικής διόρθωσης ως εξής:
\begin{equation}
    \boxed{BC = m_{\text{bol}} - m_V = M_{\text{bol}} - M_V}
\end{equation}

Η ιδέα για το BC είναι ότι μπορεί να υπολογιστεί \textit{θεωρητικά}. Γνωρίζοντας ότι για τα περισσότερα άστρα το φάσμα τους προσαρμόζεται αρκετά ικανοποιητικά από το φάσμα ενός μέλανος σώματος, αν γνωρίζουμε τη θερμοκρασία ενός αστέρα, άρα γνωρίζουμε και το φάσμα του μέλανος σώματος που αντιστοιχεί σε αυτή τη θερμοκρασία, μπορούμε να ολοκληρώσουμε το φάσμα ως προς όλα τα μήκη κύματος για να βρούμε την ολική λαμπρότητα του αστέρα, να ολοκληρώσουμε έπειτα το φάσμα μόνο στην περιοχή του φίλτρου V, και να υπολογίσουμε θεωρητικά τη ποσότητα BC για διάφορες θερμοκρασίες.
Άρα, παρατηρώντας το $m_V$ ενός αστέρα, μπορούμε να προσθέσουμε το BC που αντιστοιχεί στη θερμοκρασία του εν λόγω αστέρα (το οποίο έχει υπογισθεί θεωρητικά) και έτσι έχουμε μία εκτίμηση για το ολικό μέγεθος του αστέρα.

Βολομετρική διόρθωση μπορεί να οριστεί και σε άλλα μήκη κύματος πέρα του ορατού. Για παράδειγμα, σε μερικά ψυχρά άστρα όπου το μέγιστο της ενέργειάς τους εκπέμπεται στα υπέρυθρα μήκη κύματος, μπορούμε να εφαρμόσουμε ένα διαφορετικό σύνολο βολομετρικών διορθώσεων στο απόλυτο μέγεθος στα υπέρυθρα, αντί για το απόλυτο μέγεθος στα ορατά μήκη κύματος. Έτσι, $BC_K = M_{\text{bol}} - M_K$, όπου $BC_K$ και $M_K$ είναι η βολομετρική διόρθωση και το απόλυτο μέγεθος στην K-band αντίστοιχα.  

\newpage
{\color{red} \hrule}
\begin{center}
	\large Παρατηρήσεις
\end{center}

Για τον Ήλιο ισχύει ότι $M_{\text{bol}} = 4.74$ και $M_V = 4.83$. Άρα, $BC = -0.09$.
Παρατηρούμε ότι ο BC πρέπει πάντα να είναι αρνητικός καθώς τα μεγέθη ορίζονται με ένα μείον (μικρότερο μέγεθος συνεπάγεται πιο φωτεινός ο αστέρας).

Για αστέρια με $T_{\text{eff}} \simeq 6700 \ \text{K}$ έχουμε $BC \simeq 0$. Αυτό σημαίνει ότι το $\lambda_{\text{max}}$ που αντιστοιχεί σε αυτή τη θερμοκρασία, δηλαδή το μεγαλύτερο μέρος της εκπεμπόμενης ακτινοβολίας διέρχεται μέσω του φίλτρου V.

Για αστέρια με $T_{\text{eff}} > 6700 \ \text{K}$, το $\lambda_{\text{max}}$ μετατοπίζεται σε μικρότερα μήκη κύματος και άρα ``χάνουμε'' φωτόνια από εκεί.

Για αστέρια με $T_{\text{eff}} < 6700 \ \text{K}$, αντίστοιχα, ``χάνουμε'' φωτόνια από τα μεγαλύτερα μήκη κύματος.\\
{\color{red} \hrule}



\subsubsection{Δείκτες χρώματος}
Αφού λοιπόν θα μας αρκούσαν παρατηρήσεις μόνο σε ένα φίλτρο για να υπολογίσουμε τη συνολική φωτεινότητα ενός αστέρα, προκύπτει η εύλογη απορία τι μας χρειάζεται να παρατηρούμε στα άλλα φίλτρα. Η απάντηση είναι επειδή θέλουμε να υπολογίσουμε την επιφανειακή θερμοκρασία, $T_{\text{eff}}$ ενός αστέρα (χωρίς να χρειαστεί να έχουμε το συνολικό φάσμα του αστέρα), γνώση που είναι απαραίτητη για να βρούμε το BC.

\begin{figure}[h]
    \centering
    \includegraphics[scale=0.3]{Figures/Planck_law_log_log_scale.png}
    \caption{Φάσματα μέλανων σωμάτων για διάφορες θερμοκρασίες. Το διάγραμμα είναι σε λογαριθμικούς άξονες γι' αυτό η μορφή των καμπυλών είναι διαφορετική.}
    \label{fig:planck_law_log_scale}
\end{figure}

Στο διάγραμμα \ref{fig:planck_law_log_scale} απεικονίζεται --ποιοτικά-- το εύρος των συχνοτήτων που καλύπτει το φίλτρο B (με μπλε χρώμα), ενώ με πορτοκαλί φαίνεται το εύρος συχνοτήτων που καλύπτει το φίλτρο V. Άρα, η ροή του αστέρα στο φίλτρο B θα είναι το εμβαδόν της επιφάνειας που ορίζεται κάτω από την καμπύλη καθώς από τον ορισμό ξέρουμε ότι είναι το ολοκλήρωμα ως προς τα μήκη κύματος επι τη φασματική ευαισθησία του φίλτρου. Λόγω της μορφής των καμπυλών τα εμβαδά αυτά δεν είναι σταθερά.

Αυτό που μας ενδιαφέρει είναι το εμβαδόν (η ροή) και στα δύο φίλτρα. Αν πάρουμε ως παράδειγμα το μέλαν σώμα θερμοκρασίας 6000 Κ, παρατηρούμε ότι το εμβαδόν στο B φίλτρο είναι ελαφρώς μεγαλύτερο από το εμβαδόν στο V φίλτρο. Άρα η διαφορά στα μεγέθη $B - V \equiv m_B - m_V < 0$ καθώς $F_B > F_V$.
Για το μέλαν σώμα θερμοκρασίας 3000 Κ, το εμβαδόν στο V φίλτρο είναι μεγαλύτερο από το εμβαδόν στο B φίλτρο. Άρα $B - V > 0$ καθώς $F_B < F_V$.
Η διαφορά στα μεγέθη των δύο φίλτρων δεν θα έχει το ίδιο πρόσημο! Έτσι, μετρώντας τη ροή του αστέρα σε δύο φίλτρα και παίρνοντας τη διαφορά στα μεγέθη τους, μπορούμε να υπολογίσουμε τη θερμοκρασία του αστέρα.


Έχοντας κατανοήσει το πως επηρεάζεται η παρατηρούμενη ροή, μέσα από ένα φίλτρο, από την θερμορκρασία γίνεται αντιληπτό γιατί χρειαζόμαστε παρατηρήσεις σε 2 τουλάχιστον φίλτρα. Παίρνοντας την διαφορά $B - V$, $U - V$ κτλ, βρίσκουμε τη θερμοκρασία του αστέρα. Γνωρίζοντας τη θερμοκρασία, έχουμε μία τιμή για τη βολομετρική διόρθωση που χρειαζόμαστε και κατά συνέπεια μπορούμε να υπολογίσουμε τη συνολική λαμπρότητα του αστέρα (αν γνωρίζουμε την απόσταση).

Αυτές οι διαφορές στα μεγέθη (π.χ. $B - V$) ονομάζονται ``\textit{δείκτες χρώματος}'' και συνήθως ορίζονται ως η διαφορά μεγέθους σε μικρότερο μήκος κύματος μείον το μέγεθος σε μεγαλύτερο μήκος κύματος. Συνηθισμένοι δείκτες χρώματος είναι οι $$B - V \equiv m_B - m_V, U - B \equiv m_U - m_B, V - R \equiv m_V - m_R$$

Θεωρώντας ότι ένα αστέρι εκπέμπει ως μέλαν σώμα, μπορούμε για κάθε θερμοκρασία να υπολογίσουμε θεωρητικά τι δείκτη χρώματος περιμένουμε. Στην πράξη, χρησιμοποιούμε διάφορες εμπειρικές σχέσεις όπως την παρακάτω για τον δείκτη $B-V$: 
\begin{equation}
    T_{\text{eff}} = \frac{9000 \ \text{K}}{(B - V) + 0.93}
\end{equation}
που ισχύει για αστέρια με δείκτη χρώματος $-0.1 \leq B-V \leq 1.4$ ή ισοδύναμα $4000 \ \text{K} \leq T_{\text{eff}} \leq 11000 \ \text{K}$.

\begin{itemize}
    \item Για τον Vega, $T_{\text{eff}} \approx 10000 \ \text{K}$ και εξ' ορισμού $B-V = U-B = \dots = 0$.
    \item Για αστέρια με $T_{\text{eff}} > 10000 \ \text{K} \longrightarrow B-V < 0$, ενώ για αστέρια με $T_{\text{eff}} < 10000 \ \text{K} \longrightarrow B-V > 0$. Αυτό το συμπέρασμα προκύπτει εύκολα αν σκεφτούμε ότι μεγαλύτερη θερμοκρασία συνεπάγεται $\lambda_{\text{max}}$ σε μικρότερη μήκη κύματος και άρα $F_B > F_V \Rightarrow m_B < m_V \Rightarrow B-V < 0$. Αντίστοιχη λογική ακολουθείται και για όταν $T_{\text{eff}} < 10000 \ \text{K}$.
\end{itemize}

\hrule
\underline{\textbf{Παράδειγμα}}:
\textbf{Ας υποθέσουμε ότι έχουμε ένα αστέρι ψυχρότερο από τον Vega και ότι ισχύει $\displaystyle \frac{F_B}{F_V} < \left( \frac{F_B}{F_V} \right)_{Vega}$. Να δείξετε ότι σε αυτή την περίπτωση $B-V > 0$.}

Γνωρίζουμε ότι 
\begin{eqnarray*}
\frac{F_B}{F_V} &<& \frac{F_{B,Vega}}{F_{V,Vega}} \Rightarrow \log \left( \frac{F_B}{F_V} \right) < \log \left( \frac{F_{B,Vega}}{F_{V,Vega}} \right) \Rightarrow \\\\\
&\Rightarrow & - \log \left( \frac{F_B}{F_V} \right) > - \log \left( \frac{F_{B,Vega}}{F_{V,Vega}} \right) \Rightarrow \\\\
&\Rightarrow & - 2.5 \log \left( \frac{F_B}{F_V} \right) > - 2.5 \log \left( \frac{F_{B,Vega}}{F_{V,Vega}} \right) \Rightarrow \\\\
&\Rightarrow & - 2.5 \log \left( \frac{F_B}{F_V} \right) + 2.5 \log \left( \frac{F_{B,Vega}}{F_{V,Vega}} \right) > 0 \Rightarrow \\\\
&\Rightarrow & {\color{blue} -2.5 \log F_B} + {\color{purple} 2.5 \log F_V} + {\color{blue} 2.5 \log F_{B,Vega}} {\color{purple} - 2.5 \log F_{V,Vega}} > 0 \Rightarrow \\\\
&\Rightarrow & -2.5 \left( \log F_B - \log F_{B,Vega} \right) + 2.5 \left( \log F_V - \log F_{V,Vega} \right) > 0 \Rightarrow \\\\
&\Rightarrow & \underbrace{-2.5 \log \left( \frac{F_B}{F_{B,Vega}} \right)}_{m_B} + \underbrace{2.5 \log \left( \frac{F_V}{F_{V,Vega}} \right)}_{- m_V} > 0 \Rightarrow \\\\
&\Rightarrow & m_B - m_V > 0 \Rightarrow \boxed{B - V > 0}
\end{eqnarray*}
%\hrule 
\subsubsection{Μεσοαστρική ερυθρή χρώση}
Υπάρχει όμως λόγος να μελετάμε τα άστρα σε παραπάνω από 2 φίλτρα; Η απάντηση είναι πως ναι, καθώς μέχρι τώρα υποθέσαμε ότι δεν παρεμβάλεται τίποτα μεταξύ του παρατηρητή και του αστέρα. Στην πραγματικότητα, και ιδιαίτερα για τα άστρα που βρίσκονται στο γαλαξιακό επίπεδο, υπάρχουν αέρια και σκόνη που απορροφούν μέρος του φωτός.

Η απορρόφηση του οπτικού φωτός από τη μεσοαστρική σκόνη γίνεται με διαφορετικό τρόπο στα διάφορα μήκη κύματος. Η απορρόφηση είναι μεγαλύτερη στα μικρότερα μήκη κύματος (στο ``μπλε'' φως) οπότε τα αστέρια εμφανίζονται περισσότερο κόκκινα απ 'οτι είναι στην πραγματικότητα. Αυτό το φαινόμενο ονομάζεται ``\textit{μεσοαστρική ερυθρή χρώση}'' (interstellar reddening). Αν όμως έχουμε μετρήσεις του μεγέθους των αστέρων σε διάφορα φίλτρα, τότε μπορούμε να υπολογίσουμε την απορρόφηση στα διάφορα μήκη κύματος, να διορθώοσυμε τους παρατηρούμενες δείκτες χρώματος, και άρα να υπολογίσουμε τη σωστή $T_{\text{eff}}$ και τον συντελεστή βολομετρικής διόρθωσης BC.


Λόγω του φαινομένου της μεσοαστρικής ερυθρής χρώσης, ένας αστέρας παρατηρείται πιο κόκκινος απ' ότι είναι στην πραγματικότητα, δηλαδή με αλλοιωμένο δείκτη χρώματος. Αν $(B-V)$ είναι ο παρατηρούμενος και $(B-V)_0$ ο πραγματικός δείκτης χρώματος, τότε η διαφορά τους
\begin{equation}
    E_{(B-V)} = (B-V) - (B-V)_0
\end{equation}
ονομάζεται ``\textit{υπεροχή χρώματος}'' (color excess). Από την υπεροχή χρώματος υπολογίζεται τελικά η απορρόφηση σε αστρικά μεγέθη ($A_{\text{V}}$ για το οπτικό και $A_{\text{B}}$ για το κυανό) από τις εμπειρικές σχέσεις
\begin{align}
    A_{\text{V}} & = 3 E_{(B-V)} \\\nonumber\\
    A_{\text{B}} & = 4 E_{(B-V)}
\end{align}


\section{Φασματική ανάλυση}
Στα μέσα του 19ου αιώνα οι Kirchhoff και Bunsen παρατήρησαν ότι τα φάσματα των φυσικών σωμάτων ακολουθούν τους εξής δύο γενικούς κανόνες:
\begin{enumerate}
    \item Τα στερεά και τα υγρά σώματα εκπέμπουν συνεχές φάσμα, ενώ τα (αραιά) αέρια εκπέμπουν γραμμικό φάσμα.
    \item Όταν ένα αέριο παρεμβάλλεται μεταξύ μιας (θερμότερης από αυτό) πηγής συνεχούς φάσματος και του παρατηρητή, δημιουργούνται γραμμές απορρόφησης στο συνεχές φάσμα. Οι γραμμές αυτές έχουν το ίδιο μήκος κύματος με τις γραμμές εκπομπής του φάσματος του αερίου του κανόνα (1).
\end{enumerate}

Σε αυτό το κεφάλαιο θα αναλύσουμε κάποια χαρακτηριστικά των αστρικών φασμάτων δίνοντας έμφαση στο σχηματισμό της γραμμικής συνιστώσας, ενώ το πως μια αέρια μάζα --όπως ένας αστέρας-- μπορεί να έχει και συνεχές φάσμα, που έρχεται σε αντίθεση με τον κανόνα (1), θα γίνει αντιληπτό στο επόμενο κεφάλαιο.

\subsection{Αστρικά φάσματα}

Μπορούμε να πάρουμε το φάσμα μιας πηγής χρησιμοποιώντας ένα πρίσμα ή ένα φράγμα περίθλασης. Παρόλο που η λειτουργία των δύο αυτών οργάνων βασίζεται σε εντελώς διαφορετικά φυσικά φαινόμενα, τα αποτελέσματα που μας δίνουν είναι τα ίδια: αναλύουν μία παράλληλη πολυχρωματική δέσμη φωτός, σε μία αποκλίνουσα δέσμη, σε κάθε διεύθυνση της οποίας αντιστοιχεί φως μια συγκεκριμένης συχνότητας. Στη συνέχεια το φάσμα καταγράφεται φωτογραφικά ή φωτοηλεκτρικά.

Το φάσμα μιας πηγής χωρίζεται σε δύο συνιστώσες: τη \textbf{συνεχής} συνιστώσα και τη \textbf{γραμμική} συνιστώσα. Η τελευταία αποτελείται από \textbf{φασματικές γραμμές}, δηλαδή στενές φασματικές περιοχές πλάτους $\Delta \lambda$, όπου $\Delta \lambda \ll \lambda$, στις οποίες η φωτεινή ένταση έχει τιμή πολύ μεγαλύτερη (φωτεινές γραμμές) ή πολύ μικρότερη (σκοτεινές γραμμές) από τη μέση τιμή των γειτονικών της περιοχών.
Η συνεχής συνιστώσα αποτελείται από την εξομαλυμένη καμπύλη που προκύπτει αν ``αφαιρέσουμε'' από το φάσμα τις φασματικές γραμμές, ή αν ``προσαρμόσουμε'' στην πειραματική καμπύλη του φάσματος μια συνεχής ``θεωρητική'' καμπύλη σύμφωνα με κάποιο λογικό κριτήριο. Τέτοια θεωρητική καμπύλη μπορεί να είναι για παράδειγμα μία καμπύλη Planck (μέλανος σώματος) αν έχουμε λόγους να πιστεύουμε ότι το φωτοβόλο σώμα εκπέμπει θερμικά\footnote{Θερμική ακτινοβολία είναι η ακτινοβολία που εκπέμπεται από ένα σώμα λόγω της θερμικής του κατάστασης και η έντασή της περιγράφεται από τον νόμο του Planck.} και βρίσκεται σε θερμοδυναμική ισορροπία, ένα πολυώνυμο αν πιστεύουμε ότι το σώμα εκπέμπει ακτινοβολία σύγχροτρον\footnote{Ακτινοβολία σύγχροτρον είναι η ακτινοβολία που εκπέμπεται όταν ηλεκτρόνια με σχετικιστικές ταχύτητες κινούνται μέσα σε μαγνητικό πεδίο. Η ακτινοβολία αυτή δημιουργεί συνεχές φάσμα και εκπέμπεται μέσα στα όρια ενός στενού κώνου γωνίας $\alpha = 2m_0 c^2/E$, με άξονα κάθετο στη διεύθυνση του μαγνητικού πεδίου, όπου $m_0$ η μάζα ηρεμίας του ηλεκτρονίου και $E$ η ενέργειά του.} ή ακόμη και αυτή που προκύπτει από μία απλή γραφική εξομάλυνση των δεδομένων.

Η συνεχής συνιστώσα των φασμάτων των αστέρων μοιάζει πολύ με το φάσμα μέλανος σώματος, εφόσον ως θερμοκρασία του αστέρα πάρουμε την ενεργό θερμοκρασία του (σχήμα \ref{fig:continuous_spectra}). Οι αποκλίσεις της συνεχούς συνιστώσας από το μέλαν σώμα οφείλονται στο γεγονός ότι η ακτινοβολία προέρχεται από διάφορα βάθη, με διάφορες θερμοκρασίες. Αυτό σημαίνει ότι το συνεχές φάσμα μοιάζει να είναι μια σύνθεση φασμάτων πολλών μελανών σωμάτων διαφορετικών θερμοκρασιών. Επίσης, στη φωτόσφαιρα του άστρου παρουσιάζονται αποκλίσεις από τη θερμοδυναμική ισορροπία που προϋποθέτει ο νόμος του Planck και παρουσιάζει σημαντική διαφάνεια (δεν επικρατεί θερμοδυναμική ισορροπία σε όλο το εξωτερικό στρώμα του αστέρα από το οποίο προέρχεται η ακτινοβολία).

\begin{figure}[h]
    \centering
    \includegraphics[width=\linewidth]{Figures/contin_spectra.png}
    \caption{\textbf{Αριστερά:} Προσαρμογή συνεχούς φάσματος θερμοκρασίας $ T \sim 5777 \ K$ στο φάσμα του Ήλιου. \textbf{Δεξιά:} Προσαρμογή συνεχούς φάσματος θερμοκρασίας $ T \sim 6660 \ K$ στο φάσμα του αστέρα HD134083. Και στις δύο περιπτώσεις είναι χαρακτηριστική η παρουσία φασματικών γραμμών απορρόφησης.}
    \label{fig:continuous_spectra}
\end{figure}

Επειδή το πλάτος $\Delta \lambda$ των φασματικών γραμμών είναι συνήθως πολύ μικρό συγκρικά με το πλάτος της οπτικής περιοχής, η φωτεινή ενέργεια που αντιστοιχεί σ' αυτές είναι πολύ μικρή, έστω κι αν η έντασή τους είναι πολύ μεγαλύτερη (ή μικρότερη) από την τιμή της συνεχούς συνιστώσας στην οπτική περιοχή. Έτσι, η ύπαρξη των φασματικών γραμμών δεν επηρεάζει σημαντικά τη φωτεινότητα ενός αστέρα και κατ' επέκταση το μέγεθός του ($m_U, m_B, m_V$ κτλ). Παρόλα αυτά, η ένταση και το πλάτος των φασματικών γραμμών των αστρικών φασμάτων είναι φορείς που μας μεταφέρουν τις σημαντικότερες πληροφορίες που έχουμε σήμερα στη διάθεσή μας για του αστέρες: φυσικές συνθήκες στην επιφάνεια και την ατμόσφαιρά τους (πίεση και θερμοκρασία), χημική σύσταση, ταχύτητα περιστροφής, ακτινική ταχύτητα ως προς τον παρατηρητή. Για παράδειγμα, η μέτρηση της μετατόπισης του μήκους κύματος των φασματικών γραμμών από τα διάφορα τμήματα της επιφάνειας του αστέρα, προς μικρότερα ή μεγαλύτερα μήκη κύματος (λόγω φαινομένου Doppler) μας επιτρέπει τη μέτρηση της ακτινικής συνιστώστας της ταχύτητας του αστέρα. Αντίστοιχα, η μέτρηση του εύρους των φασματικών γραμμών μας επιτρέπει τον προσδιορισμό της γωνιακής ταχύτητας περιστροφής του αστέρα.Γενικά πάντως, η πεπλάτυνση των φασματικών γρμαμών οφείλεται σε πολλούς παράγοντες, όπως η θερμική κίνηση των συστατικών του αερίου. Σε αυτή την περίπτωση βέβαια, η πεπλάτυνση δεν είναι ίδια για όλες τις φασματικές γραμμές που αντιστοιχούν στα διάφορα χημικά στοιχεία παρόντα στην ατμόσφαιρα του αστέρα. Η ταχύτητα περιστροφής όμως επηρεάζει το εύρος όλων των φασματικών γραμμών το ίδιο.

Η δημιουργία της συνεχούς συνιστώσας του φάσματος αν και έχει θεωρητικό ενδιαφέρον, οι ιδιότητές της λίγο εξαρτώνται από το μηχανισμό παραγωγής της και κατ' επέκταση από τη χημική σύσταση του αστέρα.

\subsection{Σχηματισμός φασματικών γραμμών}
Οι φασματικές γραμμές δημιουργούνται όταν μεταβάλλεται η ενέργεια ενός ατόμου ή ενός μορίου (ή ελεύθερης ρίζας) μεταξύ κβαντισμένων ενεργειακών σταθμών. Στην περίπτωση των μορίων, αυτό συμβαίνει είτε όταν μεταβάλλεται το πλάτος της ταλάντωσης ή η σχετική θέση των ατόμων τους, είτε όταν μεταβάλλεται η στροφορμή τους. Η ενέργεια ενός ατόμου, από την άλλη μεριά, μεταβάλλεται όταν μεταβληθεί ένας από τους τέσσερις κβαντικούς αριθμούς που χαρακτηρίζουν την ενεργειακή κατάσταση ενός ηλεκτρονίου. Η μεταβολή αυτή της ενέργειας κατά $\Delta E$ μπορεί να γίνει είτε όταν το άτομο απορροφά ($\Delta E > 0$) είτε όταν εκπέμπει $\Delta E < 0$ ένα φωτόνιο συχνότητας $\nu$, η οποία δίνεται από τη σχέση $|\Delta E| = h\nu$, όπου $h$ η σταθερά του Planck.

Όταν από μια δέσμη Η/Μ ακτινονολίας, που αποτελείται από συνεχή κατανομή συχνοτήτων, απορροφώνται φωτόνια συχνότητας $\nu_0$, τότε στο παρατηρούμενο φάσμα της δημιουργείται μία \textit{γραμμή απορρόφησης}. Όταν προστίθενται φωτόνια συχνότητας $\nu_0$, τότε στο φάσμα της επιπροστίθεται μία \textit{γραμμή εκπομπής}. Το αν η φασματική γραμμή που δημιουργείται σε κάθε περίπτωση θα είναι γραμμή εκπομπής ή απορρόφησης εξαρτάται από τη θερμοδυναμική κατάσταση της ύλης, δηλαδή των ίδιων των ατόμων.

\begin{figure}[h]
    \centering
    \includegraphics[scale=0.1]{Figures/energy_transitions.jpeg}
    \caption{Ενεργειακές μεταπτώσεις στις οποίες οφείλονται η δημιουργία φασματικών γραμμών.}
    \label{fig:energy_transitions}
\end{figure}

\subsubsection{Φασματικές σειρές του Υδρογόνου}
Για λόγους απλότητας, θα θεωρήσουμε ότι η μοναδική αιτία αλλαγής της ενεργειακής κατάστασης ενός ατόμου είναι η μετάπτωση ενός ηλεκτρονίου από μία στοιβάδα σε μία άλλη, δηλαδή η αλλαγή του κύριου κβαντικού αριθμού, $(n = 1,2,3, \dots)$. Με άλλα λόγια, δεν θα μιλήσουμε καθόλου για ενεργειακές μεταπτώσεις μορίων ή ενεργειακές μεταπτώσεις υπέρλεπτης υφής (μεταπτώσεις που προκαλούνται λόγω της αλληλεπίδρασης του πυρήνα με το ηλεκτρονιακό νέφος, π.χ. γραμμή εκπομπής 21-cm του Υδρογόνου).

Το Υδρογόνο είναι με διαφορά το στοιχείο με τη μεγαλύτερη αφθονία τοσο στο σύμπαν όσο και στους αστέρες. Σε συνδυασμό με το γεγονός ότι είναι και το πιο εύκολα μελετήσιμο λόγω της απλής δομής του, τα αποτελέσματα αυτά είναι εξαιρετικά χρήσιμα.

Η ενέργεια του μοναδικού ηλεκτρονίου του ατόμου του Υδρογόνου δίνεται από τη σχέση
\begin{equation}
    E = - \frac{1}{n^2} \frac{2\pi^2 m_e e^4}{h^2}
\end{equation}
όπου $m_e$, $e$ και $n$ είναι η μάζα ηρεμίας, το φορτίο και ο κύριος κβαντικός αριθμός του ηλεκτρονίου αντίστοιχα.
Αν το ηλεκτρόνιο μεταβεί από μία στάθμη με κύριο κβαντικό αριθμό $n_i$ σε μία άλλη με κύριο κβαντικό αριθμό $n_f$, τότε η ενέργειά του μεταβάλλεται κατά 
\begin{equation}
    \Delta E = E_2 - E_1 = - \frac{2\pi^2 m_e e^4}{h^2} \left( \frac{1}{n_f^2} - \frac{1}{n_i^2} \right)
\end{equation}
Όταν $n_i < n_f$ τότε $\Delta E > 0 $, οπότε έχουμε απορρόφηση φωτονίου από το άτομο το οποίο μεταβαίνει από τη στάθμη χαμηλότερης ενέργειας ($n_i$) στη στάθμη υψηλότερης ενέργειας ($n_f$). Στην αντίθετη περίπτωση όπου $n_i > n_f$, έχουμε εκπομπή φωτονίου. Και στις δύο περιπτώσεις, η συχνότητα του φωτονίου δίνεται από τη σχέση

\begin{equation}
    \nu = \frac{|\Delta E|}{h} = \frac{2 \pi^2 m_e e^4}{h^3} \left| \frac{1}{n_f^2} - \frac{1}{n_i^2} \right|
\end{equation}
και το μήκος κύματος από τη σχέση του Rydberg

\begin{equation}
    \label{eq:rydberg_formula}
    \frac{1}{\lambda} = Z^2 R \left| \frac{1}{n_f^2} - \frac{1}{n_i^2} \right|
\end{equation}
όπου $Z$ είναι ο ατομικός αριθμός του ατόμου και $R$ η σταθερά του Rydberg η οποία δίνεται από τη σχέση $$R = \frac{2\pi^2 m_e e^4}{h^3 c}$$

Εφαρμόζοντας τη σχέση \eqref{eq:rydberg_formula} για το άτομο του Υδρογόνου ($Z=1$) τότε προκύπτουν οι εξής σειρές (δες και σχήμα \ref{fig:hydrogen_lines}):

\begin{itemize}
    \item Για $n_f = 1$ και $n_i=2,3, \dots$ τότε έχουμε τη σειρά \textit{εκπομπής Lyman}. Στην περίπτωση που $n_i = 1$ και $n_f = 2,3, \dots$ τότε έχουμε τη σειρά \textit{απορρόφησης Lyman}.
    \item Για $n_f = 2$ και $n_i=3,4, \dots$ τότε έχουμε τη σειρά εκπομπής Balmer. Στην περίπτωση που $n_i = 2$ και $n_f = 3,4, \dots$ τότε έχουμε τη σειρά απορρόφησης Balmer. Αυτή η σειρά είναι και η πιο σημαντική στην Οπτική Αστρονομία καθώς τα μήκη κύματος αυτής της σειράς βρίσκονται στην οπτική περιοχή. 
    
    Η γραμμή που προκύπτει από την μετάπτωση $n_i = 3 \rightarrow n_f = 2$ ονομάζεται γραμμή εκπομπής $H_{\alpha}$. Αντίστοιχα, η αντίστροφη μετάβαση $n_i = 2 \rightarrow n_f = 3$ ονομάζεται γραμμή απορρόφησης $H_{\alpha}$. Με την ίδια λογική μπορούμε να ορίσουμε τη γραμμή Balmer $H_{\beta}$ που αντιστοιχεί στην μετάβαση $n_i = 4 \rightarrow n_f = 2$ (ή το αντίστροφο), την γραμμή $H_{\gamma}$ για την μετάβαση $n_i = 5 \rightarrow n_f = 2$ κτλ.
    
    Το μήκος κύματος της σειρά Balmer ξεκινάει από $\lambda(H_{\alpha}) = 6563$Å και μειώνεται όσο αυξάνει η τάξη της γραμμής τείνοντας σε μια οριακή τιμή $\lambda(H_{\infty}) = 3546$Å. Στην περιοχή του οριακού αυτού μήκους κύματος οι γραμμές απορρόφησης είναι τόσο κοντά η μία με την άλλη, ώστε αλληλοεπικαλύπτονται με αποτέλεσμα το υπόβαθρο στην περιοχή αυτή να εμφανίζει μία ασυνέχεια, η οποία ονομάζεται \textbf{ασυνέχεια Balmer} και το ύψος της εξαρτάται από τη θερμοκρασία του αστέρα προσφέροντας έτσι ένα ακόμα παρατηρησιακό εργαλείο για τη μέτρησή της. 
    \item Για $n_f = 3$ και $n_i=4,5, \dots$ τότε έχουμε τη σειρά \textit{εκπομπής Paschen} (και αντίστροφα για την γραμμή απορρόφησης).
    \item Για $n_f = 4$ και $n_i=5,6, \dots$ τότε έχουμε τη σειρά \textit{εκπομπής Brackett} (και αντίστροφα για την γραμμή απορρόφησης).
    \item Για $n_f = 5$ και $n_i=6,7, \dots$ τότε έχουμε τη σειρά \textit{εκπομπής Pfund} (και αντίστροφα για την γραμμή απορρόφησης).
    \item Για $n_f = 6$ και $n_i=7,8, \dots$ τότε οι σειρές που προκύπτουν δεν έχουν κάποια συγκεκριμένη ονομασία.
\end{itemize}


\begin{figure}[h]
    \centering
    \includegraphics[scale=0.5]{Figures/hydrogen_spectral_lines.png}
    \caption{Σειρές Lyman, Balmer και Paschen για το άτομο του Υδρογόνου. Οι σειρές Brackett και Pfund, καθώς και άλλες ανώτερες σειρές, δεν παρουσιάζονται στο συγκεκριμένο διάγραμμα.}
    \label{fig:hydrogen_lines}
\end{figure}



\subsection{Φαινόμενο Zeeman}
Όταν τα άτομα στα οποία οφείλεται η δημιιουργία των φασματικών γραμμών βρίσκονται σε μαγνητικό πεδίο, τότε οι ενεργειακές τους στάθμες μεταβάλλονται, έτσι ώστε κάθε φασματική γραμμή διασπάται σε δύο, ή περισσότερες, πολωμένες συνιστώσες. Το φαινόμενο αυτό ονομάζεται ``φαινόμενο Zeeman'' (το ηλεκτρικό ανάλογο αυτού του φαινομένου είναι το φαινόμενο Stark όπου οι φασματικές γραμμές διασπόνται υπό την παρουσία ηλεκτρικού πεδίου) και επειδή η απόσταση των διάφορων συνιστωσών εξαρτάται από την ένταση του μαγνητικού πεδίου, το εκμεταλλευόμαστε για την μέτρηση αστρικών, μεσοαστρικών και γαλαξιακών μαγνητικών πεδίων. 

Στην απλούστερη περίπτωση (το ομαλό φαινόμενο Zeeman) μία φασματική γραμμή συχνότητας $\nu_{\circ}$ διασπάται σε τρεις συνιστώσες με συχνότητες $\nu_{\circ} - \Delta \nu$, $\nu_{\circ}$, και $\nu_{\circ} + \Delta \nu$. Αν η ένταση του μαγνητικού πεδίου, $B$, μετριέται σε Gauss, τότε η διαφορά συχνότητας $\Delta \lambda$, δίνεται απο τη σχέση 
\begin{equation}
    \Delta \nu =  \frac{eB}{4\pi m c} = 1.4 \times 10^6 \ B \ \ Hz
\end{equation}

Οι συσκευές μέτρησης των μαγνητικών πεδίων που βασίζονται σε παρατηρήσεις Η/Μ ακτινοβολίας ονομάζονται \textit{μαγνητογράφοι}. 



\section{Φασματική Ταξινόμηση}
Όπως είδαμε, Οι παράγοντες που καθορίζουν το φάσμα ενός αστέρα είναι η επιφανειακή θερμοκρασία ($ T_{eff}$), η ταχύτητα του αστέρα ($\boldsymbol{u}$), η γωνιακή ταχύτητα περιστροφής ($\boldsymbol{\omega}$), η ένταση του μαγνητικού πεδίου ($\boldsymbol{B}$), και η πίεση στην επιφάνεια του αστέρα. Τα αστρικά αυτά φάσματα κυριαρχούνται από γραμμές απορρόφησης (ορισμένα άστρα εμφανίζουν και αδύναμες γραμμές εκπομπής) με τις συνηθέστερες γραμμές απορρόφησης να είναι αυτές της σειράς Balmer του Υδρογόνου. Επειδή η ένταση των γραμμών απορρόφησης εξαρτάται άμεσα από την ενεργό θερμοκρασία του αστέρα, μπορούμε να ταξινομήσουμε τα αστέρια με βάση την ένταση των γραμμών απορρόφησης του Υδρογόνου και να καθορίσουμε με αυτό τον τρόπο την ενεργό θερμοκρασία τους.


\subsection{Ταξινόμηση κατα Harvard}
Αρχικά οι αστρονόμοι ταξινόμησαν τα αστρικά φάσματα αλφαβητικά σε φασματικούς τύπους (spectral types, Sp) με πρώτο κριτήριο την ένταση της γραμμής $H_{\alpha}$ της σειράς Balmer. Έτσι τα αστέρια τύπου Α είχαν την εντονότερη γραμμή $H_{\alpha}$, οι τύπου Β αστέρες είχαν λιγότερο έντονη γραμμή $H_{\alpha}$ κ.ο.κ.
Για τους αστέρες που είχαν τόσο ασθενή γραμμή $H_{\alpha}$, ώστε να μην μπορεί να χρησιμοποιηθεί ως κριτήριο ταξινόμησης, οι αστρονόμοι εισήγαγαν ως συμπληρωματικά κριτήρια την παρουσία ή την απουσία και άλλων φασματικών γραμμών (π.χ. ουδέτερα άτομα, ιόντα, μοριακές ταινίες).

Με την ανάλυση πολλών φασμάτων, διαπιστώθηκε ότι οι γραμμές απορρόφησης -εκτός του Υδρογόνου- δεν έδειχναν να μεταβάλλονται ομαλά από τον έναν φασματικό τύπο στον επόμενο. Αντίθετα, έδειχναν να εμφανίζονται και να εξαφανίζονται απότομα. Έτσι, οδηγήθηκαν στην αναδιάταξη της -αρχικά αλφαβητικής- ακολουθίας των φασματικών τύπων σε μία νέα ακολουθία όπου το χαρακτηριστικό της ήταν η συνεχής και μονότονη μεταβολή της ενεργού θερμοκρασίας. Η φασματική ταξινόμηση που προέκυψε ονομάστηκε ταξινόμηση κατα Harvard (σχήμα \ref{fig:spectral_classes_harvard}) επειδή προτάθηκε από ερευνητές του Πανεπιστημίου του Harvard.

Κάθε φασματικός τύπος της συγκεκριμένης ταξινόμησης διαιρείται σε δέκα υποκατηγορίες που χαρακτηριζόνται από τους αριθμούς $0,1,2,\dots , 9$. Ένα αστέρι φασματικού τύπου $ F1$ είναι πιο θερμό από ένα αστέρι $ F3$ αλλά πιο ψυχρό από ένα αστέρι $ A9$. Έτσι, οι θερμότεροι αστέρες που έχουν παρατηρηθεί μέχρι σήμερα είναι αστέρες φασματικού τύπου $ O5$ $ T_{eff} = 50000 \ K$, ενώ οι ψυχρότεροι αστέρες (ερυθροί νάνοι) είναι φασματικού τύπου $ M9$ με $ T_{eff} = 2900 \ K$. Ένας γενικός μνημονικός κανόνας για την συγκεκριμένη ταξινόμηση είναι 
\begin{center}
    \textbf{O}h, \textbf{B}e \textbf{A} \textbf{F}ine \textbf{G}irl/\textbf{G}uy, \textbf{K}iss \textbf{M}e
\end{center}


\begin{figure}[h]
    \centering
    \includegraphics[scale=0.6]{Figures/spectral_classes.png}
    \caption{Φασματική ταξινόμηση κατά Harvard βάσει την ενεργό θερμοκρασία των άστρων. Κάθε τύπος αποτελείται από 10 υποκατηγορίες από το 0 εώς το 9. Οι τύποι P και Q δεν είναι απαραίτητο να ακολουθούν την μονότονη αύξηση της επιφανειακής θερμοκρασίας βάσει της οποίας ταξινομούνται τα αστέρια στους υπόλοιπους τύπους.}
    \label{fig:spectral_classes_harvard}
\end{figure}

Όπως φαίνεται και στο σχήμα \ref{fig:spectra_comparisson_table}, η ένταση των γραμμών Balmer ξεκινάει σχεδόν από το μηδεν ($ Sp = O5$), φτάνει σε κάποιο μέγιστο ($ Sp = A$) και καταλήγει πάλι κοντά στο μηδέν ($ Sp = M$). Αυτό συμβαίνει γιατί σε αστέρια με πολύ υψηλές θερμοκρασίες ($ > 10000 \ K$) το άτομο του Υδρογόνου έχει χάσει το μοναδικό του ηλεκτρόνιο και άρα δεν μπορεί να δώσει μετάπτωση. Τα φάσματα τέτοιων θερμών αστέρων θα περιέχουν ιόντα από σχετικά απλά άτομα που χρειάζονται πολύ ενέργεια για να διώξουν τα εξωτερικά τους ηλεκτρόνια σε σύγκριση με πιο βαριά άτομα που τα εξωτερικά τους ηλεκτρόνια είναι πιο ``χαλαρά'' δεμένα με το δυναμικό του πυρήνα.
Στην αντίθετη περίπτωση που το άστρο είναι πολύ ψυχρό, το ηλεκτρόνιο βρίσκεται στη βασική του στοιβάδα οπότε πάλι δεν μπορεί να δώσει την μετάπτωση που απαιτείται για τις γραμμές Balmer. Η χαμηλή θερμοκρασία εξηγεί και την ύπαρξη μοριακών ταινιών στα φάσματα αυτών των αστέρων.\\


{\color{red} \hrule}
\textbf{Ερμηνεία της κατάταξης κατά Harvard}\\
Η φασματική ακολουθία $ O, B, \dots, M$ είναι στην πραγματικότητα μία μονότονη και φθίνουσα συνάρτηση της θερμοκρασίας. Καθώς η θερμοκρασία ελλατώνεται βαθμιαία από τους αστέρες τύπου Ο προς τους αστέρες τύπου Μ, η αριθμητική πυκνότητα των ιονισμένων ατόμων ελλατώνεται, ενώ αυξάνεται η αριθμητική πυκνότητα των ουδέτερων ατόμων. Έτσι, οι γραμμές των ιονισμένων ατόμων που κυριαρχούν στους αστέρες τύπου Ο, παραχωρούν τη θέση τους στις γραμμές των ουδέτερων ατόμων, και τελικά στις μοριακές ταινίες. Η ερμηνεία αυτή βασίζεται στον νόμο του Saha.\\
{\color{red} \hrule}

\begin{figure}[h]
   \centering
\begin{subfigure}[h]{0.45\textwidth}
	\centering
   	 \includegraphics[angle=270,width=\textwidth]{Figures/spectral_class_table.png} 
\end{subfigure}
\begin{subfigure}[h]{0.5\textwidth}
	\centering
	\includegraphics[scale=0.5]{Figures/spectra_comparison_harvard.png} 
    \end{subfigure}
    \caption{\textbf{Αριστερά}: Πίνακας με στοιχεία για κάθε φασματικό τύπο. \textbf{Δεξιά}: Σύγκριση φασμάτων αστέρων της κύριας ακολουθίας.}
    \label{fig:spectra_comparisson_table}
\end{figure}

Οι έξι συμπληρωματικοί φασματικοί τύποι (κόκκινα γράμματα στο σχήμα \ref{fig:spectral_classes_harvard}) περιγρέφουν μη-συνηθισμένους αστέρες. Ο τύπος W χαρακτηρίζει αστέρες τύπου \textbf{Wolf-Rayet} οι οποίοι είναι όμοιοι με αυτούς του τύπου Ο, αλλά με ευρείες γραμμές εκπομπής, οι οποίες οφείλονται σε ανώμαλα εκτεταμένη ατμόσφαιρα.
Ο τύπος P χαρακτηρίζει \textbf{πλανητικά νεφελώματα}, ένα από τα τελικά στάδια της εξέλιξης ενός αστέρα μικρής μάζας, τα οποία αποτελούνται από εξαιρετικά αραιό αέριο και σκόνη.
Ο τύπος Q ορίστηκε για την περιγραφή των φασμάτων \textbf{καινοφανών αστέρων} (novae), αστέρες που εμφανίζουν ξαφνική αύξηση της φωτεινότητάς τους κατά πολλά μεγέθη λόγω εκρηκτικής ανάφλεξης πυρηνικού καυσίμου. Αυτό μπορεί να συμβεί είτε σε διπλό σύστημα λόγω μεταφοράς μάζας από τον ένα αστέρα στον άλλον (δες Κεφάλαιο \ref{ch:Chapter7}), είτε και σε απλούς αστέρες όταν διάφοροι μηχανισμοί πρόσμιξης μεταφέρουν φρέσκο υλικό από ανώτερα στρώματα, στα στρώματα που γίνεται καύση βαρύτερων υλικών. Πάντως ο φασματικός τύπος Q χρησιμοποιείται σπάνια σήμερα.
Τέλος, οι αστέρες φασματικού τύπου R, N και S έχουν ανώμαλη χημική σύσταση, με τα φάσματα των αστέρων R και N να περιέχουν ασυνήθιστα υψηλή περιεκτικότητα σε άνθρακα (μοριακές ταινίες της ελεύθερης ρίζας CH και CN αντίστοιχα) και να ονομάζονται \textbf{αστέρες άνθρακα}. Από την άλλη, τα αστέρια φασματικού τύπου S, έχουν έντονες μοριακές ταινίες των οξειδίων του τιτανίου, ζιρκονίου και άλλων σπάνιων γαιών. Αυτη η ανώμαλη χημική σύσταση αυτών των αστέρων μάλλον οφείλεται σε κάποιον μιχανσιμό ανάμιξης της επιφανειακής ύλης με ύλη που προέρχεται από το εσωτερικό του αστέρα μέσω ρευμάτων μεταφοράς. 
Ας σημειωθεί εδώ ότι οι έξι αυτοί φασματικοί τύποι που αναλύσαμε δεν αποτελούν γνήσια επέκταση της αρχικής φασματικής ταξινόμησης, καθώς δεν παρουσιάζουν αμφιμονοσήμαντη αντιστοιχία μεταξύ των φασματικών τύπων και των ενεργών θερμοκρασιών. Τα αστέρια που ανήκουν σε αυτές ή αλλες φασματικές κατηγορίες που δεν εμφανίζονται εδώ (π.χ. L και T κατηγορίας καφέ νάνων) αποτελούν πολύ μικρό ποσοτό της τάξης $< 1 \%$.

Παρόλα αυτά, υπάρχουν αστέρια που φυσικά η ενεργός τους θερμοκρασία εμπίπτει στα όρια $ 3000 - 50000 \ K$ αλλά το φάσμα τους δεν ταιριάζει με κανέναν από τους παραπάνω φασματικούς τύπους. Γι' αυτό γενικά η χρήση της θερμοκρασίας ή του δείκτη χρώματος (που είναι συνάρτηση της θερμοκρασίας) προτιμάται έναντι του φασματικού τύπου.



\subsection{Διάγραμμα Hertzsprung-Russell}
Αν τοποθετήσουμε τους αστέρες με γνωστά φάσματα σε ένα δισδιάστατο διάγραμμα που έχει τετμημένη τον φασματικό τύπο (Sp) και τεταγμένη το απόλυτο μέγεθος του αστέρα, τότε παρατηρούμε ότι οι αστέρες δεν είναι τυχαία διασκορπισμένοι στο επίπεδο, αλλά συγκεντρώνονται σε συγκεκριμένες δομές. Το διάγραμμα αυτό ονομάζεται \textbf{διάγραμμα Hertzprung-Russel} ή για συντομία διάγραμμα H-R.
Η μορφή του διαγράμματος H-R μας δείχνει ότι υπάρχουν φυσικοί νόμοι που συνδέεουν τη λαμπρότητα του αστέρα (απόλυτο μέγεθος) με την ενεργό θερμοκρασία του (φασματικό τύπο) και ότι όλα τα βασικά παρατηρησιακά μεγέθη ενός αστέρα δεν έχουν τυχαίες τιμές, αλλά συνδέονται με σχέσεις που μπορούν να εξηγηθούν και θεωρητικά. Φυσικά μία σχέση της μορφής $$ L = f(T_{eff})$$ που αποκαλύπτει το διάγραμμα H-R είναι προσεγγιστική καθώς οι αστέρες χαρακτηρίζονται και από άλλες ποσότητες, όπως η χημική σύσταση, που σε πρώτη φάση αγνοούνται. Γι' αυτό άλλωστε και οι αστέρες δεν βρίσκονται στο διάγραμμα H-R κατά μήκος μονοδιάστατων καμπυλών, αλλά παρουσιάζουν μία κάποια διασπορά.

\begin{figure}[h]
    \centering
    \includegraphics[scale=0.4]{Figures/HRDiagram.png}
    \caption{Διάγραμμα Hertzprung-Russel.}
    \label{fig:HRD}
\end{figure}

Από το σχήμα \ref{fig:HRD} παρατηρούμε ότι τα περισσότερα αστέρια είναι συγκεντρωμένα σε μία ζώνη, που διατρέχει το διάγραμμα διαγώνια και η οποία ονομάζεται \textbf{κύρια ακολουθία}. Πάνω από την κύρια ακολουθία υπάρχει μία άλλη ζώνη αστέρων, που ονομάζεται \textbf{κλάδος γιγάντων}. Τέλος, κάτω από τον κλάδο της κύριας ακολουθίας υπάρχει μία συγκέντρωση αστέρων, που βρίσκεται αριστερά του διαγράμματος και σε αυτή ανήκουν οι \textbf{λευκοί νάνοι}. Αν προεκτείνουμε την γραμμή του κλάδου των γιγάντων προς τα αριστερά, παρατηρούμε ότι τέμνει την κύρια ακολουθία σ' ένα σημείο που αντιστοιχεί περίπου στον φασματικό τύπο A0 και απόλυτο μέγεθος $ M = +1$. Οι αστέρες που βρίσκονται αριστερά από το φασματικό τύπο Α0 ονομάζονται αστέρες \textbf{προγενέστερου φασματικού τύπου} (π.χ. κυανοί γίγαντες), ενώ οι αστέρες που βρίσκονται δεξιά από τον φασματικό τύπο Α0 ονομάζονται αστέρες \textbf{μεταγενέστερου φασματικού τύπου} (π.χ. ερυθροί νάνοι, νάνοι της κύριας ακολουθίας κτλ). Οι αστέρες μεταγενέστερου φασματικού τύπου που βρίσκονται πάνω από την κύρια ακολουθία ονομάζονται \textbf{ερυθροί γίγαντες}. Η διάκριση των αστέρων σε αστέρες προγενέστερου και μεταγενέστερου φασματικού τύπου δεν έχει σήμερα καμία άλλη σημασία παρά μόνο την περιγραφή της θέσης τους στο διάγραμμα H-R. Αντίθετα, η διάκριση των αστέρων σε νάνους και γίγαντες έχει φυσική σημασία.

\textbf{Αστρικά σμήνη}\\
Μπορούμε να διακρίνουμε δύο ειδών αστρικών σμηνών. Τα \textbf{σφαιρωτά σμήνη} είναι πυκνές συγκεντρώσεις μεγάλου αριθμού αστέρων ($10^4 - 10^6$) με σφαιρικό σχήμα, που κινούνται σε ελειπτικές τροχιές στην άλω γύρω από το κέντρου του Γαλαξία. Το χαρακτηριστικό των σφαιρωτών σμηνών είναι ότι επειδή σχηματίστηκαν από την βαρυτική κατάρρευση του ίδιου μοριακού νέφους, αποτελούνται σε πρώτη προσέγγιση από άστρα της ίδιας χημικής σύστασης (ίδια ``μεταλλικότητα''), της ίδιας ηλικίας, και βρίσκονται στην ίδια απόσταση από τη Γη.
Η άλλη κατηγορία αστρικών σμηνών είναι τα \textbf{ανοικτά σμήνη} τα οποία είναι συγκεντρώσεις το πολύ μερικών χιλιάδων άστρων που σχηματίστηκαν και αυτά από το ίδιο μοριακό νέφος, αλλά τα αστέρια είναι πληθυσμού I, και άρα πολύ νεώτερα συγκριτικά με τα αστέρια στα σφαιρωτά σμήνη που είναι πληθυσμού ΙΙ. Τα ανοικτά σμήνη βρίσκονται κυρίως στο Γαλαξιακό επίπεδο και όχι στην άλω.


\begin{figure}[h]
    \centering
    \includegraphics[scale=0.4]{Figures/cmd.png}
    \caption{Διάγραμμα Χρώματος-Μεγέθους για το σφαιρωτό σμηνος Μ3. Στο διάγραμμα φαίνεται μια ισόχρονη καμπύλη των 10Gyr.}
    \label{fig:CMD}
\end{figure}

Είναι προφανές ότι δεν μπορούμε να πάρουμε το φάσμα για κάθε ένα από τα αστέρια-μέλη ενός σμήνους. Γι' αυτό το λόγο, όταν θέλουμε να κάνουμε το διάγραμμα H-R στην περίπτωση σμήνους αστέρων, εκφράζουμε την τετμημένη σε όρους δείκτη χρώματος (που είναι συνάρτηση της θερμοκρασίας και μπορεί να υπολογιστεί για χιλιάδες αστέρια που βρίσκονται στο ίδιο οπτικό πεδίο) αντί για φασματικό τύπο, και την τεταγμένη σε μέγεθος αντί για λαμπρότητα. Έτσι προκύπτει το λεγόμενο \textbf{διάγραμμα χρώματος-μεγέθους} (color-magnitude diagram) που φαίνεται στο σχήμα \ref{fig:CMD}.

Αν η συνάρτηση αρχικής μάζας (initial mass funtion) ενός σμήνους είναι γνωστή, τότε μπορούμε να υπολογίσουμε θεωρητικά μια ισόχρονη καμπύλη για οποιαδήποτε ηλικία θέλουμε, με το να προσομειώσουμε την εξέλιξη του κάθε αστέρα του αστρικού πληθυσμού και κάνοντας το διάγραμμα H-R για αυτά τα αστέρια. Συγκρίνοντας τις θεωρητικές αυτές καμπύλες που αντιστοιχούν σε συγκεκριμένες ηλικίες με το παρατηρησιακό διάγραμμα χρώματος-μεγέθους, μπορούμε να έχουμε μία εκτίμηση για την ηλικία του σμήνους.





\subsection{Ταξινόμηση κατα Yerkes}
Αστέρια ίδιου φασματικού τύπου (ίδιας ενεργού θερμοκρασίας) εμφανίζουν κατανομή στις λαμπρότητές τους, όπως φαίνεται και από το διάγραμμα H-R (σχήμα \ref{fig:HRD}). Για παράδειγμα, ο εγγύτατος του Κενταύρου έχει την ίδια επιφανειακή θερμόκαρσία με αυτή του Betelgeuse, άρα ανήκουν στον ίδιο φασματικό τύπο, αλλά τελείως διαφορετικές λαμπρότητες με τον εγγύτατο να έχει 100 φορές μικρότερη λαμπρότητα από τον Ήλιο και τον Betelgeuse 100,000 φορές μεγαλύτερη λαμπρότητα.

Το 1930, οι Morgan \& Keenan, πρότειναν ένα νέο σύστημα ταξινόμησης των αστέρων, το οποίο λειτουργεί συμπληρωματικά του συστήματος ταξινόμησης κατα Harvard. Το νέο σύστημα βασίζεται στην έννοια της \textit{Κατηγορίας λαμπρότητας}. Σύμφωνα με αυτή την ταξινόμηση, αστέρια της \textbf{ίδιας} ενεργούς θερμοκρασίας (ίδια φασματική τάξη), χωρίζονται σε έξι κατηγορίες λαμπρότητας, με βάση το \textbf{πλάτος} των γραμμών απορρόφησης (σχήμα \ref{fig:spectral_classes_yerkes}).

\begin{figure}[h]
   \centering
\begin{subfigure}[h]{0.42\textwidth}
	\centering
   	 \includegraphics[angle=270,origin=c,scale=0.3]{Figures/luminosity_variations_yerkes.png} 
\end{subfigure}
\begin{subfigure}[h]{0.55\textwidth}
	\centering
	\includegraphics[scale=0.6]{Figures/spectral_classes_yerkes.jpg} 
    \end{subfigure}
    \caption{\textbf{Αριστερά}: Διακύμανση του πλάτους των γραμμών απορρόφησης για τέσσερια αστέρια φασματικού τύπου Α0 (HR 1040, $\eta$ Leo, $\alpha$ Dra, $\alpha$ Lyr). \textbf{Δεξιά}: Φασματική ταξινόμηση κατά Yerkes, σε έξι κατηγορίες λαμπρότητας. Για μεγαλύτερη κατηγορία λαμπρότητας ($ I, II, \dots VI$) το πλάτος της γραμμής απορρόφησης αυξάνεται, ενώ η ακτίνα και η λαμπρότητα μειώνονται. Ο Ήλιος ανήκει στην κατηγόρια G2V.}
    \label{fig:spectral_classes_yerkes}
\end{figure}

\vspace{0.5cm}
{\color{red} \hrule}
\textbf{Ερμηνεία της κατάταξης κατά Yerkes}\\
Εφόσον μιλάμε για αστέρια ίδιου φασματικού τύπου, το εύρος της γραμμής απορρόφησης δεν οφείλεται σε διαφορές στην ενεργό θερμοκρασία (ούτε στην περιστροφή των αστέρων) αλλά στην πίεση στην επιφάνεια των αστέρων. Κάνοντας μερικές απλές παραδοχές, μπορεί να δειχτεί ότι η ατμοσφαιρική πίεση σε έναν γίγαντα αστέρα είναι μικρότερη από την πίεση σε έναν νάνο αστέρα. Συνδυάζοντας αυτό το αποτέλεσμα με τον νόμο του Saha που μας δίνει τον βαθμό ιονισμού ενός αερίου, μπορούμε να εξηγήσουμε τις φασματικές διαφορές ανάμεσα σε αστέρια του ίδιου φασματικού τύπου.\\
{\color{red} \hrule}


%     \include{Chapters/Chapter3}
%     \chapter{Αστρικές ατμόσφαιρες}
\label{ch:Chapter4}
{\hypersetup{linkcolor=black, pdfborder=0 0 1}
	\minitoc
	%\newpage
}

\section{Πεδία ακτινοβολίας και ιδιότητες}
Στα προηγούμενα κεφάλαια μιλήσαμε για ένταση και ροή ακτινοβολίας χωρίς να έχουμε ορίσει αυστηρά τι σημαίνουν αυτά τα μεγέθη. Σε αυτό το κεφάλαιο θα μιλήσουμε για τις ατμόσφαιρες των άστρων ξεκινώντας με το να δώσουμε επακριβείς ορισμούς συγκεκριμένων μεγεθών που χαρακτηρίζουν και περιγράφουν ένα πεδίο ακτινοβολίας. Όπως θα δούμε, το πιο βασικό μέγεθος είναι αυτό της έντασης της ακτινοβολίας και όλα τα υπόλοιπα προκύπτουν με φυσικό τρόπο από τον ορισμό της.

\subsection{Η στερεά γωνία}
{\color{red} \hrule}
\underline{Με λίγα λόγια}:
Στερεά γωνία είναι το κομμάτι της επιφάνειας που καλύπτει ένα αντικείμενο πάνω σε μία σφαίρα, όπως το βλέπει ένας παρατηρητής στο κέντρο της σφαίρας. Με άλλα λόγια, η στερεά γωνία δείχνει το οπτικό πεδίο που καταλαμβάνει το εν λόγω αντικείμενο από ένα συγκεκριμένο σημείο, είναι δηλαδή ένα μέτρο του πόσο μεγάλο φαίνεται το αντικείμενο στον παρατηρητή.\\
 {\color{red} \hrule}
Πριν ορίσουμε το τι είναι η στερεά γωνία, ας θυμηθούμε πως ορίζουμε τις επίπεδες γωνίες (plane angles) στις δύο διαστάσεις. Η γωνία ορίζεται ως ο λόγος του μήκους τόξου (arc length) που φαίνεται από αυτή τη γωνία προς την ακτίνα του κύκλου (σχήμα \ref{fig:plane_and_solid_angle}). 
\begin{equation}
\label{eq:plane_angle}
    \theta = \frac{l}{r}
\end{equation}

\begin{figure}[h]
   \centering
\begin{subfigure}[h]{0.45\textwidth}
	\centering
   	 \includegraphics[scale=0.3]{Figures/plane_angle_def.png} 
\end{subfigure}
\begin{subfigure}[h]{0.5\textwidth}
	\centering
	\includegraphics[scale=0.3]{Figures/solid_angle_def.png} 
    \end{subfigure}
    \caption{Ορισμός α) επίπεδης γωνίας και β) στερεάς γωνίας.}
    \label{fig:plane_and_solid_angle}
\end{figure}


Για τον μοναδιαίο κύκλο ($r=1$), ένας πλήρης κύκλος έχει $2\pi$ ακτίνια (rad). Κατα τα γνωστά, η επιφάνεια του κύκλου (ή πιο σωστά δίσκου) θα είναι $A = \pi r^2$ και η περιφέρειά του $C = 2 \pi r$.

Κατα αντιστοιχία, η στερεά γωνία είναι η μεταφορά της επίπεδης γωνίας στον τρισδιάστατο χώρο, και όπως η επίπεδη γωνία είναι και αυτή αδιάστατο μέγεθος. Ορίζεται ως ο λόγος της επιφάνειας που προβάλλεται πάνω σε μία σφαίρα προς το τετράγωνο της ακτίνας της σφαίρας (σχήμα \ref{fig:plane_and_solid_angle}).

\begin{equation}
    \label{eq:solid_angle}
    \Omega = \rm \frac{projected \ area}{distance ^2}=\frac{A}{r^2}
\end{equation}

Η αδιάστατη μονάδα με την οποία μετράμε τις στερεές γωνίες, ονομάζεται \textit{στερακτίνιο ή steradian (sr)}. Υποθέτοντας για απλότητα ότι η επιφάνεια $A$ είναι σφαίρα και γνωρίζοντας ότι η επιφάνεια σφαίρας δίνεται από τη σχέση $A = 4\pi r^2$, γίνεται αντιληπτό πως αν η επιφάνεια $A$ καλύπτει όλη τη σφαίρα, τότε $\Omega = 4 \pi$. Άρα μία σφαίρα έχει $4 \pi$ sr.\\


\begin{figure}[h]
    \centering
    \includegraphics[scale=0.6]{Figures/solid_angle_projection.png}
    \caption{Υπολογισμός στερεάς γωνίας $\Delta \Omega$ μέσω της διανυσματικής επιφάνειας $S_2$.}
    \label{fig:solid_angle_projection}
\end{figure}

Στο σχήμα \ref{fig:solid_angle_projection} φαίνονται δύο διανυσματικές επιφάνειες $\boldsymbol{S}_1 = S_1 \Delta \boldsymbol{A}_1$ και $\boldsymbol{S}_2 = S_2 \Delta \boldsymbol{A}_2$, όπου $\Delta \boldsymbol{A}_1$ και $\Delta \boldsymbol{A}_2$ είναι τα διανύσματα κάθετα (normal) στις επιφάνειες $S_1$ και $S_2$, ενώ απέχουν αποστάσεις $r_1$ και $r_2$ αντίστοιχα από κέντρο σφαίρας Q.
Το διάνυσμα $\Delta \boldsymbol{A}_1$ είναι παράλληλο με το ακτινικό μοναδιαίο διάνυσμα $\boldsymbol{\hat{r}}$, όπου $\boldsymbol{\hat{r}} = \frac{\boldsymbol{r}}{r}$ και το οποίο έχει διεύθυνση τον άξονα του κώνου που σχηματίζει στερεά γωνία $\Delta \Omega$. Με άλλα λόγια, η επιφάνεια $S_1$ είναι κάθετη στο ακτινικό διάνυσμα $\boldsymbol{\hat{r}}$.

Έτσι, η στερεά γωνία υπολογισμένη για την επιφάνεια $S_1$ θα είναι:
$$\Delta \Omega = \frac{\boldsymbol{S}_1 \cdot \boldsymbol{\hat{r}}}{r_1^2} = \frac{S_1 \cancelto{1}{|\boldsymbol{\hat{r}}|} \cancelto{1}{\cos(\Delta \boldsymbol{A}_1, \boldsymbol{\hat{r}})}}{r_1^2} = \frac{S_1}{r_1^2}$$

Το διάνυσμα $\Delta \boldsymbol{A}_2$ βρίσκεται υπο γωνία $\theta$ σε σχέση με το ακτινικό μοναδιαίο διάνυσμα $\boldsymbol{\hat{r}}$. Γι' αυτό το λόγο, πρέπει να προβάλουμε την επιφάνεια $\Delta \boldsymbol{A}_2$ σε μια επιφάνεια $\Delta \boldsymbol{A}_{2n}$ η οποία θα έχει κατεύθυνση ίδια με αυτή του $\boldsymbol{\hat{r}}$. Η προβολή θα είναι: $$\Delta \boldsymbol{A}_{2n} = \rm proj_{\boldsymbol{\hat{r}}} \boldsymbol{S}_2 = \frac{\boldsymbol{S}_2 \cdot \boldsymbol{\hat{r}}}{|\boldsymbol{\hat{r}}|^2}\boldsymbol{\hat{r}} = \frac{S_2 \cancelto{1}{|\boldsymbol{\hat{r}}|} \cos (\Delta \boldsymbol{A}_2, \boldsymbol{\hat{r}})}{\cancelto{1}{|\boldsymbol{\hat{r}}|^2}}\boldsymbol{\hat{r}} = S_2 \cos \theta \ \boldsymbol{\hat{r}}$$

Άρα, η στερεά γωνία υπολογισμένη για την επιφάνεια $S_2$ θα είναι:
$$\Delta \Omega = \frac{\Delta \boldsymbol{A}_{2n} \cdot \boldsymbol{\hat{r}}}{r_2^2} = \frac{S_2 \cos \theta \cancelto{1}{|\boldsymbol{\hat{r}}|} \cancelto{1}{\cos (\boldsymbol{\hat{r}}, \boldsymbol{\hat{r}})}}{r_2^2} = \frac{S_2 \cos \theta}{r_2^2}$$



Είναι χρήσιμο όταν θέλουμε να υπολογίσουμε ολοκληρώματα στερεών γωνιών να τα μετατρέψουμε σε σφαιρικές συντεταγμένες. Από το σχήμα \ref{fig:diff_solid_angle} προκύπτει ότι η γωνία $d \theta$ βάσει του ορισμού της επίπεδης γωνίας (σχέση \eqref{eq:plane_angle}) θα είναι $d \theta = \frac{s_{\theta}}{r}$ όπου $s_{\theta}$ είναι το μήκος του τόξου (μπλε γραμμή στο σχήμα) που βλεπει τη γωνία $d \theta$. Αντίστοιχα, για την αζιμούθια γωνία\footnote{Δες και Παράρτημα \ref{apx:coordinates} αναφορικά με τις σφαιρικές συντεταγμένες.} $d \phi = \frac{s_{\phi}}{r \sin \theta}$. Συνεπώς η διαφορική επιφάνεια $dA$ είναι:
$$dA = s_{\theta} \times s_{\phi} = r d\theta \times r \sin \theta d \phi = r^2 \sin \theta d \phi d \theta$$
και άρα 
\begin{equation}
    \label{eq:diff_solid_angle}
    d \omega = \frac{dA}{r^2} = \sin \theta d \theta d \phi
\end{equation}
Η στερεά γωνία τότε είναι
\begin{equation}
    \label{eq:solid_angle_integral}
    \Omega = \iint_S d \omega = \int_{\phi_1}^{\phi_2} \int_{\theta_1}^{\theta_2} \sin \theta d \theta d \phi = (\phi_2 - \phi_1)(\cos \theta_1 - \cos \theta_2)
\end{equation}


\begin{figure}[h]
    \centering
    \includegraphics[scale=0.3]{Figures/diferential_solid_angle.png}
    \caption{Διαφορική στερεά γωνία σε σφαιρικές συντεταγμένες.}
    \label{fig:diff_solid_angle}
\end{figure}
\hrule
\underline{\textbf{Παράδειγμα}}:
\textbf{Έχοντας ως δεδομένα πως η μέση απόσταση Γης-Σελήνης και Γης-Ήλιου είναι $D_{EM} = 3.84 \times 10^5 \ \rm km$ και $D_{ES} = 1.496 \times 10^8 \ \rm km$ αντίστοιχα, και επιπρόσθετα $R_M = 1.74 \times 10^3 \ \rm km$, $R_S = 6.96 \times 10^5 \ \rm km$ είναι η ακτίνα της Σελήνης και του Ήλιου αντίστοιχα, να βρείτε: α) Ποιά είναι η γωνιακή διάμετρος του Ήλιου και της Σελήνης, β) Ποιά είναι η στερεά γωνία υπό την οποία φαίνονται ο Ήλιος και η Σελήνη από τη Γη, και γ) Ποιό από τα δύο αντικείμενα εμφανίζεται μεγαλύτερο;}

\begin{figure}[h]
    \centering
    \includegraphics[scale=0.5]{Figures/solid_angle_problem.png}
    \caption{Γεωμετρικό πλαίσιο για τον υπολογισμό της στερεάς γωνίας υπό την οποία φαίνεται σφαίρα ακτίνας R της οποιάς το κέντρο βρίσκεται σε απόσταση D από τον παρατηρητή.}
    \label{fig:solid_angle_problem}
\end{figure}

Για το πρώτο ερώτημα, χρησιμοποιώντας της σχέση \eqref{eq:angular_diameter} και από το σχήμα \ref{fig:solid_angle_problem} μπορούμε να γράψουμε:
$$\theta_{\rm max} = \arcsin \left( \frac{R}{D} \right)$$ από την οποία προκύπτει $2\theta_{\rm max}^{M} = 0.52^{\circ}$ για τη Σελήνη και $2\theta_{\rm max}^{S} = 0.53^{\circ}$.

Για το δεύτερο ερώτημα, η στερεά γωνία με άνοιγμα $2\theta$ θα είναι βάσει της σχέσης \eqref{eq:diff_solid_angle}:
$$\Omega = \int_{0}^{2\pi} \int_{0}^{\theta} \sin \theta d \theta d \phi = 2\pi (1 - \cos \theta)$$ από την οποία προκύπτει $\Omega_M = 6.5 \times 10^{-5} \ \rm sr$ για τη Σελήνη και $\Omega_S = 6.8 \times 10^{-5} \ \rm sr$ για τον Ήλιο.

Για το τρίτο ερώτημα, προκύπτει ότι ο Ήλιος καταλαμβάνει 5\% μεγαλύτερη στερεά γωνία από τη Σελήνη. Το γεγονός ότι χρησιμοποίησαμε τις μέσες αποστάσεις Γης-Σελήνης και Ήλιου-Σελήνης για τους παραπανω υπολογισμούς και ότι γενικά αυτές οι τιμές δεν είναι σταθερές, εξηγεί την ύπαρξη ολικών εκλείψεων Ηλίου.\\
{\hrule}


\newpage
\subsection{Η έννοια της έντασης ακτινοβολίας}
Όπως έχουμε πει, ο ρυθμός με τον οποίο τα αστέρια ακτινοβολούν ενέργεια ονομάζεται \textit{λαμπρότητα} (luminosity) και μετριέται σε J/s (Watt) ή erg/s. Η λαμπρότητα είναι μία ενδογενής ιδιότητα του αστέρα. 

Τι συμβαίνει όμως στο φως που ακτινοβολείται από τα αστέρια όσο αυτό κινείται προς το μέρος μας; Οι ακτίνες φωτός ``ανοίγουν" με την απόσταση που σημαίνει ότι λιγότερο φως περνάει από έναν ανιχνευτή μιας συγκεκριμένης διατομής όσο μεγαλώνει η απόσταση. Με άλλα λόγια, ο αστέρας εμφανίζεται αμυδρότερος. Έτσι, ορίζουμε την ενέργεια που περνάει \textit{\color{red} ανεξαρτήτως διεύθυνσης} από μια επιφάνεια $\rm 1 m^2$ ανα δευτερόλεπτο ως \textit{ροή} (flux, irradiance ή radiosity). Η ροή μετριέται σε W/$\rm m^2$ ή $\rm ergs \ s^{-1} \ cm^{-2}$. 



\subsubsection{Ορισμός ειδικής έντασης ακτινοβολίας}
Η ροή ως μέγεθος δεν μας δίνει πολλές πληροφορίες καθώς δεν μας λέει από ποιά διεύθυνση προέρχεται το φως. Γι' αυτό ορίζουμε ένα νέο μέγεθος, την \textit{ειδική ένταση} ακτινοβολίας (specific intensity, spectral radiance ή spectral brightness) η οποία ορίζεται ως η ροή των φωτονίων συχνότητας $d \nu$ που διέρχονται \textit{\color{red} κάθετα} από σημείο, Ο, μιας επιφάνειας $dA_{\perp}$ και που εμπεριέχονται σε κάποια στερεά γωνία $d \omega$ της οποίας η διεύθυνση είναι $\boldsymbol{\hat{\Omega}}$. Από τον ορισμό της, η ειδική ένταση μετριέται σε $\rm W \ m^{-2} \ sr^{-1} \ Hz^{-1}$ ή $\rm ergs \ s^{-1} \ cm^{-2} \ sr^{-1} \ Hz^{-1}$ και δίνεται από τη σχέση:
\begin{equation}
    I_{\nu}(O, \boldsymbol{\hat{\Omega}}) = \frac{dE}{dt d\nu d\omega dA_{\perp}} = \frac{dP}{d\nu d\omega dA_{\perp}}
\end{equation}

Η ειδική ένταση της ακτινοβολίας χαρακτηρίζει σχεδόν πλήρως το πεδίο της ακτινοβολίας αφού η μόνο πληροφορία που δεν εμπεριέχεται είναι αυτή της πόλωσης του φωτός. Αν εξαιρέσουμε την πόλωση, η ειδική ένταση μας δείχνει πόσα φωτόνια και με τι ενέργειες είναι παρόντα καθώς και σε ποιά συγκεκριμένη διεύθυνση κινούνται. Όλες οι άλλες ποσότητες (π.χ. η ροή μέσω κάποιας συγκεκριμένης επιφάνειας προσανατολισμένης προς κάποια συγκεκριμένη διεύθυνση, η πίεση της ακτινοβολίας, η πυκνότητα ενέργειας κτλ) μπορούν να βρεθούν μέσω της έντασης (για λόγους συντομίας θα χρησιμοποιείται ο όρος ``ένταση" για να εκφράσουμε την ειδική ένταση. Όταν αναφερόμαστε στην ένταση όλων των συχνοτήτων θα γίνεται χρήση του όρου ``ολική ένταση").

Αν η επιφάνεια από την οποία διέρχονται τα φωτόνια βρίσκεται υπό γωνία, τότε η ροή υπολογίζεται βάσει της προβολής της επιφάνειας στο επίπεδο που βρίσκεται κάθετα στη διεύθυνση των ακτίνων ενώ ισχύει $$dA_{\perp} = dA \cos \theta \leq dA$$ όπου $\theta$ είναι γωνία που σχηματίζεται μεταξύ του κάθετος διανύσματος, $\boldsymbol{\hat{\eta}}$, στην επιφάνεια $dA$ και την κάθετη (στην πορεία των φωτονίων) επιφάνεια $dA_{\perp}$ με κατεύθυνση αυτή του διανύσματος $\boldsymbol{\hat{\Omega}}$ (νόμος συνημιτόνων Lambert, σχήμα \ref{fig:intensity_definition}). Άρα, η ειδική ένταση της ακτινοβολίας γράφεται γενικά:
\begin{equation}
    \label{eq:specific_intensity_def}
    I_{\nu}(O, \theta, \phi) = \frac{dE}{dt d\nu d\omega dA_{\perp}} = \frac{dP}{d\nu d\omega dA \cos \theta} \longrightarrow I_{\nu}(\theta) = I_{\nu}(0)\cos \theta
\end{equation}
όπου πλέον η επιφάνεια $dA$ είναι μία τυχαία επιφάνεια που σχηματίζει γωνία $\theta$ με την κάθετη επιφάνεια κατά τη διεύθυνση της ακτινοβολίας, $dA_{\perp}$. Η διεύθυνση δίνεται από τις γωνίες $\theta$ (πολική) και $\phi$ (αζιμούθια).

\begin{figure}[h]
   \centering
\begin{subfigure}[h]{0.45\textwidth}
	\centering
   	 \includegraphics[scale=0.4]{Figures/specific_intensity_def_3D.png} 
\end{subfigure}
\begin{subfigure}[h]{0.5\textwidth}
	\centering
	\includegraphics[scale=0.5]{Figures/specific_intensity_example.png} 
    \end{subfigure}
    \caption{Ορισμός της ειδικής έντασης ακτινοβολίας. Η συνεισφορά της ροής φωτονίων υπο γωνία $\theta$ πρέπει να ``ζυγιστεί'' με το συνημίτονο της γωνίας καθώς ισχύει $\cos \theta = \boldsymbol{\hat{\eta}} \cdot \boldsymbol{\hat{\Omega}}$. Η προβαλλόμενη επίπεδη επιφάνεια $dA_{\perp}$ είναι κάθετη στην διεύθυνση $(\theta_r, \phi_r)$, ενώ η επιφάνεια $dS$ που ``βλέπει'' το σημείο Ο ορίζει τη στερεά γωνία $d \omega = dS/R^2$, όπου $R$ η ακτίνα της σφαίρας με κέντρο το σημείο Ο.}
    \label{fig:intensity_definition}
\end{figure}

Το ότι μια τυχαία επιφάνεια $dA$ καταλαμβάνει μεγαλύτερη έκταση από την κάθετη επιφάνεια στη διεύθυνση διάδοσης της ακτινοβολίας φαίνεται καλύτερα στο σχήμα \ref{fig:oblique_rays}. Επειδή η ένταση της ακτινοβολίας δεν εξαρτάται από την απόσταση, το ίδιο ποσό ενέργειας μοιράζεται σε μεγαλύτερη επιφάνεια στην περίπτωση που αυτή δεν είναι κάθετη στην διεύθυνση διάδοσης της δέσμης.

\begin{figure}[h]
    \centering
    \includegraphics[scale=0.15]{Figures/oblique_rays.png}
    \caption{Γραφική απεικόνιση της εξάρτησης της έντασης ακτινοβολίας από τη γωνία που σχηματίζει η επιφάνεια με τη διεύθυνση διάδοσης της ακτινοβολίας.}
    \label{fig:oblique_rays}
\end{figure}



\subsubsection{Ιδιότητες της ειδικής έντασης ακτινοβολίας}
Η ένταση της ακτινοβολίας παραμένει αναλλοίωτη (σταθερή) κατά τη διεύθυνση της δέσμης με την προϋπόθεση ότι δεν παρεμβάλλονται άλλες πηγές ή καταβόθρες μεταξύ της πηγής και του δέκτη. Αυτό αποδεικνύεται γεωμετρικά ως εξής: θεωρούμε δύο επιφάνειες $dA$ και $dS$ όπου η μία εκπέμπει φως ενω ή άλλη λειτουργεί ως δέκτης (σχήμα \ref{fig:intensity_invariance}).

\begin{figure}[h]
    \centering
    \includegraphics[width=\linewidth]{Figures/intensity_invariance.png}
    \caption{Γεωμετρική αναπαράσταση της διατήρησης της έντασης ακτινοβολίας κατα μήκος μιας ακτίνας φωτός.}
    \label{fig:intensity_invariance}
\end{figure}

Έστω ότι $d\omega$ είναι η στερεά γωνία που σχηματίζει η επιφάνεια του δέκτη $dS$, όπως φαίνεται από το κέντρο της επιφάνειας $dA$ (πηγή).

Αντίστοιχα, έστω ότι $d \omega^{\prime}$ είναι η στερεά γωνία που σχηματίζει η επιφάνεια της πηγής $dA$, όπως φαίνεται από το κέντρο της επιφάνειας $dS$ (δέκτης). Τότε, σύμφωνα με τον ορισμό της στερεάς γωνίας (σχέση \eqref{eq:solid_angle}) θα έχουμε:

\begin{eqnarray*}
    d\omega = \frac{dS \cos \phi}{r^2} \\ \\
    d\omega^{\prime} = \frac{dA \cos \theta}{r^2}
\end{eqnarray*}

Η ενέργεια που διέρχεται μέσα από την επιφάνεια $dA$ με στερεά γωνία $d\omega$ σύμφωνα με τη σχέησ \eqref{eq:specific_intensity_def} είναι:
\begin{equation*}
    dE = I_{\nu} dt d\nu d\omega dA \cos \theta = I_{\nu} dt d\nu \left( \frac{dS \cos \phi}{r^2} \right) dA \cos \theta
\end{equation*}

Αντίστοιχα, η ενέργεια που διέρχεται μέσα από την επιφάνεια του δέκτη $dS$ με στερεά γωνία $d\omega^{\prime}$ είναι:
\begin{equation*}
    dE^{\prime} = I_{\nu}^{\prime} dt d\nu d\omega^{\prime} dS \cos \phi = I_{\nu}^{\prime} dt d\nu \left( \frac{dA \cos \theta}{r^2} \right) dS \cos \phi
\end{equation*}

Εφόσον η ενέργεια που εκπέμπφθηκε από την πηγή δεν απορροφήθηκε από πουθενά θα ισχύει $$dE = dE^{\prime} \longrightarrow I_{\nu} = I_{\nu}^{\prime}$$

Το παραπάνω αποτέλεσμα έχει δύο πολύ σημαντικές ιδιότητες
\begin{itemize}
    \item Η ένταση της ακτινοβολίας είναι ανεξάρτητη της απόστασης.
    \item Η ένταση της ακτινοβολίας είναι ίδια και στην πηγή και στον δέκτη. Άρα μπορούμε να σκεφτούμε την ένταση με όρους ενέργειας που πηγάζει από κάπου ή ως ενέργεια που διέρχεται μέσα από έναν ανιχνευτή.
\end{itemize}

Η ολική ένταση 
\begin{equation}
    I \equiv \int_{0}^{\infty} I_{\nu} d\nu = \int_{0}^{\infty} I_{\lambda} d\lambda
\end{equation}
διατηρείται επίσης, ενώ ισχύει κατά τα γνωστά ότι $|I_{\nu} d\nu| = |I_{\lambda} d\lambda|$.




\subsection{Πυκνότητα ροής, μέση ένταση, πυκνότητα και πίεση ακτινοβολίας}

\subsubsection{Πυκνότητα ροής}
Εαν η πηγή είναι διακριτή, δηλαδή καταλαμβάνει μία καλώς ορισμένη στερεά γωνία, τότε η ``πυκνότητα ροής'', $S_{\nu}$,  είναι η ισχύς που λαμβάνει ένας ανιχνευτής ανα μονάδα προβαλλόμενης επιφάνειας και ανα συχνότητα. 
Απο τη σχέση \eqref{eq:specific_intensity_def} προκύπτει ότι
\begin{equation*}
    \frac{dP}{d\nu dA} = I_{\nu} \cos \theta d\omega
\end{equation*}
και ολοκληρώνοντας ώστε να πάρουμε την ενέργεια ανα μονάδα χρόνου και ανα συχνότητα που διαπερνάει μία οποιαδήποτε επιφάνεια ανεξάρτητα της κατεύθυνσης καταλήγουμε:

\begin{equation}
    \label{eq:flux_density_def}
    S_{\nu} = \int I_{\nu}(\theta, \phi) \cos \theta d\omega
\end{equation}

Αν το γωνιακό μέγεθος της πηγής είναι $\ll 1 \ \rm rad$ τότε $\cos \theta \approx 1$ και η σχέση \eqref{eq:flux_density_def} απλοποιείται στην:
\begin{equation}
    \label{eq:flux_density_simple}
    S_{\nu} = \int I_{\nu}(\theta, \phi) d\omega
\end{equation}
κάτι το οποίο ισχύει σχεδόν πάντα καθώς δεν είθισται να χρησιμοποιούμε την πυκνότητα ροής για μεγάλους στόχους στους οποίους θα χρειαζόταν να κρατήσουμε τον όρο του συνημιτόνου (π.χ. εκπομπή από τον Γαλαξία).\\

{\color{red} \hrule}
\textbf{Πότε όμως χρησιμοποιούμε την ειδική ένταση και πότε την πυκνότητα ροής για να περιγράψουμε μία πηγή};

Εάν η πηγή έχει πολύ μικρό γωνιακό μέγεθος ώστε να μπορεί να θεωρηθεί σημειακή, η πυκνότητα ροής μπορεί να μετρηθεί αλλά όχι η ειδική ένταση.
Εάν η πηγή έχει πολύ μεγάλο γωνιακό μέγεθος, τότε η ειδική ένταση μπορεί να μετρηθεί απευθείας σε κάθε σημείο της πηγής, αλλά η πυκνότητα ροής πρέπει να υπολογισθεί ολοκληρώνοντας την παρατηρήσιμη ένταση σε όλη τη στερεά γωνία που καταλαμβάνει η πηγή. Έτσι, η πυκνότητα ροής χρησιμοποιείται μόνο για να περιγράψει σχετικά μικρές πηγές.\\
{\color{red} \hrule}

Η πυκνότητα ροής μετριέται σε $\rm W \ m^{-2} \ Hz^{-1}$ ή $\rm ergs \ s^{-1} \ m^{-2} \ Hz^{-1}$. Επειδή όμως αυτές οι μονάδες είναι πολύ μεγάλες για τις τιμές που τυπικά μετράμε από αστρονομικά αντικείμενα, έχει υιοθετηθεί μία άλλη μονάδα, το Jansky (Jy) 
$$\rm 1 \ Jy \equiv 10^{-26} W \ m^{-2} \ Hz^{-1} \equiv 10^{-23} ergs \ s^{-1} \ m^{-2} \ Hz^{-1}$$

Σε αντίθεση με την ένταση της ακτινονολίας, η πυκνότητα ροής δεν είναι ανεξάρτητη της απόστασης. Επειδή $$\int d\omega \propto d^{-2} \longrightarrow S_{\nu} \propto d^{-2}$$ το οποίο είναι γνωστό ως νόμος του αντίστροφου τετραγώνου. Άρα κάποια αστέρια φαίνονται αμυδρά επειδή καταλαμβάνουν μικρότερη στερεά γωνία στον ουράνιο θόλο (και άρα έχουν μικρότερη ροή) και όχι επειδή η ένταση της ακτινοβολίας μειώνεται με την απόσταση.

Ολοκληρώνοντας σε όλες τις συχνότητες, παίρνουμε την γνωστή (ολική) ροή (ισχύς ανά μονάδα επιφάνειας) ακτινοβολίας, αν και πολλές φορές ο όρος ``ροή'' χρησιμοποιείται έναντι του όρου ``πυκνότητα ροής'' για λόγους συντομίας:
\begin{equation}
    F_{\rm rad} \equiv S = \int_{0}^{\infty} S_{\nu} d\nu
\end{equation}

Τέλος, η λαμπρότητα σε ένα εύρος συχνοτήτων ορίζεται ως ισχύς ανά συχνότητα
(για πηγή που εκπέμπει ισοτροπικά):
\begin{equation}
    L_{\nu} = 4\pi d^2 S_{\nu} \longrightarrow L \equiv \int_{0}^{\infty} L_{\nu} d\nu
\end{equation}
και είναι ενδογενής ιδιότητα της πηγής καθώς δεν εξαρτάται από την απόσταση.



\subsubsection{Μέση ένταση ακτινοβολίας}
Η μέση ένταση της ακτινοβολίας είναι η \textit{ροπή μηδενικής τάξης της έντασης ως προς την ποσότητα $\cos \theta$}. Με άλλα λόγια, είναι η μέση τιμή της έντασης σε όλες τις στερεές γωνίες. Έτσι, έχουμε:
\begin{eqnarray*}
    J_{\nu} &=& \frac{\oint I_{\nu} d\omega}{\oint d\omega}, \ \rm \ με \ \ \oint d\omega = \int_{0}^{2\pi} \int_{0}^{\pi} \sin \theta d\theta d\phi = 4\pi \\ \\
    J_{\nu} &=& \frac{1}{4\pi} \int_{0}^{2\pi} \int_{0}^{\pi} I_{\nu}(\theta, \phi) \sin \theta d\theta d\phi
\end{eqnarray*}
Αν υποθέσουμε και αζιμουθαϊκή συμμετρία, τότε η παραπάνω σχέση γράφεται:
\begin{equation*}
    J_{\nu} = \frac{1}{4\pi} \oint I_{\nu}(\theta, \phi) d\omega = \frac{2\pi}{4\pi} \int_{0}^{\pi} I_{\nu}(\theta) \sin \theta d\theta
\end{equation*}
Κάνοντας την αντικατάσταση $\mu = \cos \theta \longrightarrow d\mu = - \sin \theta d\theta$, τα όρια της ολοκλήρωσης γίνονται $\mu_1 = \cos(0) = 1, \ \mu_2 = \cos(\pi) = -1$, και έχουμε τελικά:

\begin{equation}
    \label{eq:mean_intensity}
    J_{\nu} = \frac{1}{4\pi} \oint I_{\nu}(\theta, \phi) d\omega = \frac{1}{2} \int_{-1}^{1} I_{\nu} d\mu
\end{equation}

\textit{Παρατήρηση}: Αντίστοιχα, μπορούμε να ορίσουμε και την ``\textbf{ροή Eddington}'', $H_{\nu}$, η οποία είναι η ροπή πρώτης τάξης της έντασης ως προς την ποσότητα $\cos \theta$, ως εξής:
\begin{equation}
    \label{eq:eddington_flux}
    H_{\nu} = \frac{\oint I_{\nu}(\theta, \phi) \cos \theta d\omega}{\oint d\omega} = \frac{S_{\nu}}{4\pi} = \frac{1}{2} \int_{-1}^{1} I_{\nu} \mu d\mu
\end{equation}



\subsubsection{Πυκνότητα ακτινοβολίας}
Ένα από τα βασικά μεγέθη που χαρακτηρίζουν το Η/Μ πεδίο είναι η ``πυκνότητα ακτινοβολίας'', $u_{\nu}$. Το μέγεθος αυτό παριστάνει τη χρονική μέση τιμή της Η/Μ ενέργειας ανά μονάδα συχνότητας που περιέχεται στη μονάδα του όγκου. Αν σκεφτούμε την Η/Μ ακτινοβολία σύμφωνα με το σωματιδιακό μοντέλο, η πυκνότητα ακτινοβολίας παριστάνει το γινόμενο του πλήθους των φωτονίων με συχνότητες μεταξύ $\nu$ και $\nu + d\nu$, τα οποία περιέχονται κατά μέσο όρο στη μονάδα του όγκου, επί την ενέργεια του κάθε φωτονίου $(h\nu)$ δια το στοιχειώδες διάστημα συχνότητας $d\nu$.

Μαθηματικά αυτό αποδεικνύεται ως εξής: σε χρονικό διάστημα $dt$ ένα φωτόνιο διανύει απόσταση $dl = c dt$, οπότε από τον ορισμό της ειδικής έντασης της ακτινοβολίας (σχέση \eqref{eq:specific_intensity_def}) έχουμε:
\begin{equation*}
    \frac{I_{\nu} (\theta, \phi)}{c} = \frac{dE_{\nu}}{dl d\nu d\omega dA \cos \theta}
\end{equation*}
όπου το γινόμενο $dl dA \cos \theta$ ισούται με το στοιχειώδη όγκο $dV$ που καλύπτει το ``μέτωπο'' ενός Η/Μ κύματος σε χρόνο $dt$. Έτσι, από την προηγούμενη σχέση βρίσκουμε ότι η πυκνότητα ακτινοβολίας (ενέργεια ανά μονάδα συχνότητας και μονάδα όγκου), η οποία οφείλεται στη διάδοση της ακτινοβολίας κατά μία συγκεκριμένη διεύθυνση, δίνει από τη σχέση:
\begin{equation}
    \label{eq:differential_radiation_density}
    du_{\nu} = \frac{dE_{\nu}}{d\nu dV} = \frac{I_{\nu}(\theta, \phi)}{c} d\omega  
\end{equation}

Το ολοκλήρωμα της σχέσης \eqref{eq:differential_radiation_density} ως προς τη στερεά γωνία θα μας δώσει τη συνολική πυκνότητα ακτινοβολίας, η οποία οφείλεται στην ακτινοβολία που διαδίδεται προς όλες τις διευθύνσεις.

\begin{equation}
    \label{eq:radiation_density}
    u_{\nu} = \frac{1}{c} \oint I_{\nu}(\theta, \phi) d\omega = \frac{4\pi}{c} J_{\nu}
\end{equation}

Παρατηρούμε ότι η πυκνότητα της ακτινοβολίας είναι ανάλογη της μηδενικής ροπής της έντασης ως προς τον παράγοντα $\cos \theta$, με συντελεστή αναλογία $4\pi/c$.



\subsubsection{Πίεση ακτινοβολίας}
Το Η/Μ πεδίο μπορεί να θεωρηθεί ως ένα (σχετικιστικό) αέριο φωτονίων. Από την θεωρία των τέλειων αερίων γνωρίζουμε ότι η πίεση ενός αερίου είναι ανάλογη της πυκνότητας ενέργειας. Έτσι, κατά αντιστοιχία, μπορούμε να ορίσουμε την \textit{πίεση ακτινοβολίας}. Η πίεση γενικά ορίζεται ως η μεταβολή της ορμής, $p$, που διέρχεται κάθετα προς τη μονάδα της επιφάνειας ανά μονάδα χρόνου, ενώ ισχύει ότι $p = E/c$. Η Η/Μ ενέργεια που διέρχεται ανά μονάδα επιφάνειας προς τη διεύθυνση $\theta$ είναι: $$dS_{\nu} = I_{\nu}(\theta, \phi) \cos \theta d\omega$$. Η ορμή που μεταφέρει αυτή η ενέργεια είναι $$dp_{\nu} = \frac{dS_{\nu}}{c} = \frac{I_{\nu}(\theta, \phi)}{c} \cos \theta d\omega$$ ενώ η ορμή που διέρχεται κάθετα προς τη μοναδιαία επιφάνεια είναι $$(dp_{\nu})_{\perp} = \frac{dS_{\nu}}{c} \cos \theta = \frac{I_{\nu}(\theta, \phi)}{c} \cos^2 \theta d\omega$$

Η συνολική μεταβολή της ορμής η οποία ισούται με την πίεση, βρίσκεται με την ολοκλήρωση ως προς όλες τις δυνατές στερεές γωνίες:
\begin{equation}
    \label{eq:radiation_pressure}
    P_{\nu} = \frac{1}{c} \oint I_{\nu}(\theta, \phi) \cos^2 \theta d\omega
\end{equation}

Παρατηρούμε ότι η πίεση της ακτινοβολίας είναι ανάλογης της ροπής δεύτερης τάξης της ειδικής έντασης ως προς την ποσότητα $\cos \theta$:

\begin{eqnarray*}
    K_{\nu} &=& \frac{\oint I_{\nu}(\theta, \phi) \cos^2 \theta d\omega}{\oint d\omega} = \frac{1}{4\pi} \oint I_{\nu}(\theta, \phi) \cos^2 \theta d\omega = \frac{1}{2} \int_{-1}^{1} I_{\nu} \mu^2 d\mu  \Rightarrow \\ \\
    &\Rightarrow & K_{\nu} = \frac{c}{4\pi} P_{\nu}
\end{eqnarray*}

\textit{Παρατήρηση}: Όσον αφορά την εξάρτηση της πίεσης από τη διεύθυνση της στοιχειώδους επιφάνειας στην οποία υπολογίζεται, πρέπει να τονίσουμε ότι όταν η ένταση δεν είναι ισοτροπική, τότε η πίεση είναι τανυστικό μέγεθος και χρειάζεται γενικά ο υπολογισμός εννέα συνιστωσών για τον πλήρη προσδιορισμό της. 

Επίσης, παρόλο που η ένταση είναι ανεξάρτητη της απόστασης για κάθε πεδίο ακτινοβολίας, οι τρεις παραπάνω ροπές είναι γενικά συναρτήσεις της απόστασης, όταν η ένταση δεν είναι ισοτροπική.


\subsection{Ισοτροπική ένταση και μέλαν σώμα}
Για να υπολογιστούν τα ολοκληρώματα των σχέσεων \eqref{eq:mean_intensity}, \eqref{eq:eddington_flux}, \eqref{eq:radiation_density} και \eqref{eq:radiation_pressure}, με τα οποία ορίζονται η μέση ένταση, η πυκνότητα ροής (ή ροή Eddington), η πυκνότητα ακτινοβολίας και η πίεση της ακτινοβολίας, θα πρέπει γενικά να γνωρίζουμε τη μορφή της συνάρτησης $I_{\nu}(\theta, \phi)$. Στην ειδική περίπτωση όμως που η ειδική ένταση είναι ανεξάρτητη της διεύθυνσης, όταν δηλαδή το πεδίο της ακτινοβολίας είναι ισοτροπικό, μας αρκεί το γεγονός ότι η ένταση δεν είναι συνάρτηση των $\theta$ και $\phi$. Έτσι προκύπτουν τα εξής συμπεράσματα:

\begin{itemize}
    \item Η μέση ειδική ένταση θα είναι σύμφωνα με τη σχέση \eqref{eq:mean_intensity}
    \begin{equation}
        J_{\nu} = \frac{1}{4\pi} I_{\nu} \oint d\omega = I_{\nu} 
    \end{equation}
    \item Η πυκνότητα ροής γίνεται σύμφωνα με τη σχέση \eqref{eq:eddington_flux}
    \begin{equation}
        H_{\nu} = \frac{S_{\nu}}{4\pi} = \frac{1}{2} I_{\nu} \int_{-1}^{1} \mu d\mu = 0
    \end{equation}
    Άρα η ροή σε ένα ισοτροπικό πεδίο ακτινοβολίας είναι μηδέν. Αυτό συμβαίνει γιατί έχουμε ορίσει ως ροή την ``καθαρή'' ροη (net flux) που διέρχεται από μία επιφάνεια, ώστε $$S_{\nu} = S_{\nu}^{+} - S_{\nu}^{-}$$
    
    Σε ένα ισοτροπικό πεδίο, η ροή $S_{\nu}^{+}$ που διέρχεται από μία στοιχειώδη επιφάνεια προς τη θετική κατεύθυνση που έχουμε ορίσει εμείς είναι ακριβώς ίση με τη ροή $S_{\nu}^{-}$ που διέρχεται προς την αρνητική κατεύθυνση.
    
    Αν θέλουμε για παράδειγμα να υπολογίσουμε τη πυκνότητα ροής στην επιφάνεια ενός αστέρα, υποθέτωντας ότι η ένταση κατευθύνεται μόνο προς τα ``έξω'', τότε ολοκληρώνοντας μόνο στη μισή σφαίρα θα είχαμε:
    $$S_{\nu}^{+} = I_{\nu} \int_{0}^{2\pi} d\phi \int_{0}^{\pi/2} \cos \theta \sin \theta d\theta = \pi I_{\nu}$$
    Κάποιοι ορίζουν την \textit{αστροφυσική (πυκνότητα) ροή} (astrophysical flux) ως τον λόγο $$\mathcal{F} = \frac{S_{\nu}}{\pi}$$ ώστε να ισούται αριθμητικά με την ένταση.
    \item Για την πυκνότητα ακτινοβολίας προκύπτει αβίαστα απο τη σχέση \eqref{eq:radiation_density} ότι ισχύει
    \begin{equation}
        u_{\nu} = \frac{4\pi}{c} J_{\nu} = \frac{4\pi}{c} I_{\nu}
    \end{equation}
    \item Τέλος, για την πίεση της ακτινοβολίας από τη σχέση \eqref{eq:radiation_pressure} προκύπτει
    \begin{equation}
        P_{\nu} = \frac{2\pi I_{\nu}}{c} \int_{0}^{\pi} \cos^2 \theta \sin \theta d\theta = \frac{4\pi I_{\nu}}{3c} = \frac{1}{3} u_{\nu}
    \end{equation}
    
    Άρα, στην περίπτωση ισοτροπικού πεδίου ακτινοβολίας, η πίεση της ακτινοβολίας είναι ανάλογη της πυκνότητας ακτινοβολίας με συντελεστή αναλογίας τον καθαρό αριθμό 1/3 (σε αντιδιαστολή με την περίπτωση ενός τέλειου αερίου, όπου ο συντελεστής αναλογίας είναι 2/3)\footnote{Δες και Παράρτημα \ref{apx:kinetic_theory}.}.
\end{itemize}

Εάν μιλάμε για μέλαν σώμα, το οποίο εξ' ορισμού εκπέμπει ισοτροπικά, τότε έχουμε και την αναλυτική έκφραση της ειδικής έντασης της ακτινοβολίας καθώς αυτή θα δίνεται από τον νόμο του Planck. Σε αυτή την περίπτωση ισχύει ότι $$J_{\nu} = I_{\nu} = B_{\nu}$$
και επομένως η ολική ένταση θα ισούται με
\begin{equation}
    I = \int_{0}^{\infty} I_{\nu} d\nu = \int_{0}^{\infty} B_{\nu} d\nu = \frac{\sigma T^4}{\pi}
\end{equation}
 όπου $\sigma$ είναι η γνωστή σταθερά των Stefan-Boltzmann.
 
 Η ολική πυκνότητα ακτινοβολίας είναι 
 \begin{equation}
    \label{eq:photon_energy_density}
     u = \frac{4\pi}{c} \int_{0}^{\infty} I_{\nu} d\nu = \int_{0}^{\infty} B_{\nu} d\nu = \alpha T^4
 \end{equation}
όπου $\alpha = \frac{4\sigma}{c}$ είναι η σταθερά της ακτινοβολίας.

Αντίστοιχα, η ολική (θετική) ροή ισούται με
\begin{equation}
    F^{+} = \pi \int_{0}^{\infty} I_{\nu} d\nu = \pi \int_{0}^{\infty} B_{\nu} d\nu = \sigma T^4
\end{equation}

Τέλος, για την ολική πίεση της ακτινοβολίας, χρησιμοποιώντας τη σχέση $P_{\nu} = u_{\nu}/3$ βρίσκουμε ότι
\begin{equation}
    \label{eq:radiation_pressure_black_body}
    P = \int_{0}^{\infty} P_{\nu} d\nu = \frac{1}{3}\int_{0}^{\infty} u_{\nu} d\nu = \frac{1}{3} \alpha T^4
\end{equation}




\section{Απόσβεση και εκπομπή ακτινοβολίας}
Αν το Σύμπαν ήταν άδειο, τότε το πρόβλημα της μεταφοράς της ακτινοβολίας θα ήταν απλό καθώς η ένταση της ακτινοβολίας στο σημείο του παρατηρητή θα ήταν ίδια με την ένταση της ακτινοβολίας στην πηγή. Με άλλα λόγια η ένταση θα ήταν σταθερή. Σε ένα ρεαλιστικό σενάριο όμως, το Σύμπαν δεν είναι άδειο, αλλά αποτελείται από ύλη που εν γένει μπορεί να αποσβέσει (απορρόφηση φωτονίων από άτομα/ιόντα, σκέδαση φωτονίων σε διευθύνσεις άλλες από αυτή της γραμμής παρατήρησης) ή να εκπέμψει φωτόνια (αυθόρμητη εκπομπή ή εκπομπή laser). Η περιγραφή του πως αλλάζει η ένταση της ακτινοβολίας καθώς αυτή κινείται στον χώρο δίνεται από την \textit{εξίσωση μεταφοράς} της ακτινοβολίας (radiative transfer equation). Αυτή είναι και η θεμελιώδης εξίσωση που διέπει τις αστρικές ατμόσφαιρες (και μία από τις πιο βασικές σε όλη την αστροφυσική).



\subsection{Εξίσωση μεταφοράς}
Με τον όρο απόσβεση της ακτινοβολίας (extinction or attenuation) εννοούμε το συνολικό αποτέλεσμα των μικροσκοπικών (ατομικών) μηχανισμών που ευθύνονται για την μείωση της ακτινοβολούμενης ενέργειας κατά τη διεύθυνση διάδοσής της.
Αντίθετα, με τον όρο εκπομπή ακτινοβολίας θα θεωρούμε το συνολικό αποτέλεσμα μικροσκοπικών μηχανισμών που ευθύνονται για την αύξηση της ακτινοβολούμενης ενέργειας κατά τη διεύθυνση διάδοσης της.

Η σκέδαση του φωτός μπορεί απλά να να οφείλεται σε αλληλεπίδραση των φωτονίων με σωματίδια σκόνης που αλλάζουν την κατεύθυνση διάδοσης του φωτός. Παρόλα αυτά, όταν ένα άτομο απορροφά ένα φωτόνιο συγκεκριμένης ενέργειας και διεγείρεται, η αποδιέγερση του ατόμου υφίσταται σχεδόν ακαριαία (της τάξης των νανοδευτερελέπτων) μέσω της εκπομπής ενός φωτονίου με την ίδια συχνότητα του αρχικού φωτονίου που απορροφήθηκε. Σε αυτά τα πλαίσια, δηλαδή της απορρόφησης και άμεσης επανεκπομπής ενός φωτονίου σε μία άλλη κατεύθυνση, η διαδικασία αυτή μπορεί να θεωρηθεί ως σκέδαση. Αν όμως το διεγερμένο άτομο συγκρουστεί με κάποιο γειτονικό του πριν αποδιεγερθεί (μέσω ελαστικής κρούσης) τότε υπάρχει περίπτωση να αποδιεγερθεί μετατρέποντας την ενέργεια που θα εξέπεμπε με τη μορφή φωτονίου, σε κινητική ενέργεια των συγκρουόντων ατόμων. Αυτή η διαδικασία είναι πραγματική απορρόφηση καθώς αφαιρεί φωτόνια.

Η σκέδαση όμως μπορεί να λειτουργήσει και σαν μηχανισμός εκπομπής ακτινοβολίας στην περίπτωση που το σκεδαζόμενο φωτόνιο εισέρχεται στη διεύθυνση παρατήρησης. Η σκέδαση μαζί με την αυθόρμητη εκπομπή (spontaneous emission) είναι οι κύριοι μηχανισμοί που θεωρούμε ότι συνεισφέρουν (συλλογικά) στην αύξηση της ενέργειας της ακτινοβολίας, ενώ μηχανισμοί όπως η εξαναγκασμένη εκπομπή (stimulated emission, or laser emission) που και αυτοί συνεισφέρουν θετικά στην ενέργεια της ακτινοβολίας, θα αγνοούνται.

Σε μικροσκοπικό επίπεδο, η ύλη δεν μπορεί να θεωρηθεί ότι είναι ομοιογενής. Παρόλα αυτά, μπορούμε να θεωρήσουμε ομοιογενή στοιχεία όγκου τα οποία εκπέμπουν και απορροφούν\footnote{Πολλές φορές θα χρησιμοποιούμε τον όρο απορρόφηση για να αναφερθούμε σε οποιοδήποτε μηχανισμο απόσβεσης της ακτινοβολίας.} ακτινοβολούμενη ενέργεια, έτσι ώστε όλη η πληροφορία που εμπεριέχεται στο ατομικό επίπεδο να μπορεί να αποδοθεί με τη χρήση κάποιων μακροσκοπικών συντελεστών μεταφοράς.


\begin{figure}[h]
    \centering
    \includegraphics{Figures/radiative_transfer_scheme.png}
    \caption{Διάδοση ακτινοβολίας μέσα από ένα οπτικό μέσο. Ο παράγοντας $4\pi$ εμφανίζεται αν λάβουμε υπόψιν ότι ο συντελεστής επομπής (αφετική ικανότητα) $j_{\nu}$ ορίζεται ανά στερεά γωνία και ότι σε τοπική θερμοδυναμική ισορροπία, η εκπομπή ακτινοβολίας είναι ισοτροπική.}
    \label{fig:radiative_transfer_scheme}
\end{figure}


\subsubsection{Απορρόφηση \& Σκέδαση}
Θα προσπαθήσουμε να περιγράψουμε την μείωση στην ειδική ένταση της ακτινοβολίας κατά μήκος μια διαδρομής ds, στην κατεύθυνση $\boldsymbol{\hat{\eta}}$ λόγω της απορρόφησης της Η/Μ ακτινοβολίας από ένα οπτικό μέσο (σχήμα \ref{fig:radiative_transfer_scheme}). Στην περίπτωση που έχουμε ένα αδύναμο Η/Μ πεδίο και ένα αραιό οπτικό μέσο, τότε η μείωση της έντασης θα είναι ανάλογη της έντασης της εισερχόμενης ακτινοβολίας
\begin{equation}
    \label{eq:linear_absorption_coefficient}
    \frac{dI_{\nu}(\boldsymbol{\hat{\eta}})}{ds} \propto I_{\nu}(\boldsymbol{\hat{\eta}}) \longrightarrow \alpha_{\nu}(\boldsymbol{\hat{\eta}}) \left. \right|_{\rm absorption} \equiv - \frac{1}{I_{\nu}(\boldsymbol{\hat{\eta}})} \frac{dI_{\nu}(\boldsymbol{\hat{\eta}})}{ds}
\end{equation}
όπου ο συντελεστής αναλογίας $\alpha_{\nu}(\boldsymbol{\hat{\eta}}) \left. \right|_{\rm absorption}$, ονομάζεται \textit{γραμμικός συντελεστής απορρόφησης} και εκφράζει το κλάσμα της μονοχρωματικής ακτινοβολίας που αφαιρείται από την προσπίπτουσα δέσμη κατά μήκος μιας διεύθυνσης λόγω απορρόφησης.

Αντίστοιχα, μπορούμε να ορίσουμε και τον \textit{γραμμικό συντελεστή σκέδασης}, $\alpha_{\nu}(\boldsymbol{\hat{\eta}}) \left. \right|_{\rm scattering}$, ο οποίος εκφράζει το κλάσμα της μονοχρωματικής ακτινοβολίας ανά μονάδα μήκους που παρεκλίνει από την διεύθυνση, $\boldsymbol{\hat{\eta}}$, της προσπίπτουσας δέσμης λόγω σκέδασης. Έτσι, ορίζουμε τον \textit{γραμμικό συντελεστή απόσβεσης}
\begin{equation}
    \label{eq:linear_extinction_coefficient}
    \chi_{\nu} (\boldsymbol{\hat{\eta}}) \equiv \alpha_{\nu}(\boldsymbol{\hat{\eta}}) \left. \right|_{\rm absorption} + \alpha_{\nu}(\boldsymbol{\hat{\eta}}) \left. \right|_{\rm scattering}
\end{equation}
ο οποίος αναφέρεται στο συνολικό φαινόμενο της αφαίρεσης ειδικής έντασης από την προσπίπτουσα δέσμη ανά μονάδα μήκους. Από τη διαστασιακή ανάλυσης της σχέσης \eqref{eq:linear_absorption_coefficient}, προκύπτει ότι $$\left[\chi_{\nu} (\boldsymbol{\hat{\eta}})\right] = \left[\alpha_{\nu}(\boldsymbol{\hat{\eta}}) \left. \right|_{\rm absorption}\right] = \left[\alpha_{\nu}(\boldsymbol{\hat{\eta}}) \left. \right|_{\rm scattering}\right] = \left[L^{-1}\right]$$.

Το νόημα των μακροσκοπικών αυτών συντελεστών γίνεται εμφανές αν σκεφτούμε τι σημαίνει μείωση της έντασης της ακτινοβολίας σε μικροσκοπικό επίπεδο. Στην ατομική κλίμακα, η μείωση της έντασης οφείλεται στην αλληλεπίδραση των φωτονίων με ηλεκτρόνια, άτομα, μόρια ή ακόμα και μεγαλύτερες δομές που ονομάζουμε σκόνη. Ο συντελεστής απόσβεσης λοιπόν εξαρτάται από δύο παράγοντες: την πιθανότητα ένα φωτόνιο να απορροφηθεί ή να σκεδαστεί από κάποιο σωματίδιο, και τον αριθμό των ατόμων που περιέχονται σε έναν στοιχειώδη όγκο τα οποία είναι ικανά να απορροφήσουν ή να σκεδάσουν το εν λόγω φωτόνιο. 

Ο πρώτος παράγοντας, η πιθανότητα δηλαδή το φωτόνιο να αλληλεπιδράσει με κάποιο σωματίδιο, σχετίζεται με την \textbf{ενεργό διατομή} που είναι ο λόγος του αριθμού των φωτονίων που απορροφούνται/σκεδάζονται προς τη ροή των εισερχόμενων φωτονίων. Η ενεργός διατομή ενός σωματιδίου μπορεί να γίνει αντιληπτή αν θεωρήσουμε πως κάθε σωματίδιο ορίζει στον χώρο μία σφαίρα επιρροής, τέτοια ώστε η διέλευση ενός φωτονίου μέσα από αυτή να συνεπάγεται αλληλεπίδραση με το σωματίδιο. Το εμβδαδόν ενός μέγιστου κύκλου μιας τέτοιας σφαίρας, είναι η ενεργός διατομή του σωματιδίου, $\sigma_{\nu}$, η οποία γενικά εξαρτάται από τη συχνότητα και είναι μία ιδιότητα των σωματιδίων με διαστάσεις $[L^2]$. Ο δεύτερος παράγοντας, εκφράζεται από την αριθμητική πυκνότητα των σωματιδίων, $\eta$ και εξαρτάται από την θερμοδυναμική κατάσταση του οπτικού μέσου.

Είναι σαφές ότι εν γένει υπάρχουν πολλά σωματίδια το κάθε ένα από αυτά με τη δική του ενεργό διατομή για μια συγκεκριμένη διεργασία όπως η σκέδαση ή η απορρόφηση. Για λόγους απλότητας, ας υποθέσουμε ότι υπάρχει ένα μόνο είδος σωματιδίων ανά μονάδα όγκου και ότι κάθε ένα από αυτά έχει μία ενεργό διατομή για τη συλλογική απόσβεση της ακτινονολίας ίση με $\sigma_{\nu}$. Τότε, ο γραμμικός συντελεστής απόσβεσης θα ισούται 
\begin{equation}
    \label{eq:atomic_absorption_coefficient}
    \chi_{\nu} = \sigma_{\nu} \eta
\end{equation}
Από την παραπάνω έκφραση, βλέπουμε ότι ο γραμμικός συντελεστής απόσβεσης μπορεί να θεωρηθεί και ως ενεργός διατομή ανά μονάδα όγκου. Ισοδύναμα, μπορούμε να ορίσουμε τον γραμμικό συντελεστή απόσβεσης ως τη μείωση της έντασης της ακτινοβολίας ανά μονάδα μάζας (ή ενεργό διατομή ανά μονάδα μάζας), ώστε
\begin{equation}
    \label{eq:mass_extinction_coefficient}
    \chi_{\nu} = \kappa_{\nu} \rho
\end{equation}
όπου $\rho$ είναι η πυκνότητα μάζας του οπτικού μέσου με διαστάσεις $[M L^{-3}]$ και $\kappa_{\nu}$ ονομάζεται ο συντελεστής απόσβεσης μάζας (mass extinction coefficient) με διαστάσεις $[L^2 M^{-1}]$. Στην αστροφυσική, ο ορισμός ανά μονάδα μάζας είναι αυτός που χρησιμοποείται συνήθως στην μελέτη των αστρικών ατμοσφαιρών, με τον συντελεστή $\kappa_{\nu}$ να ονομάζεται \textbf{αδιαφάνεια} (opacity).

Βάσει των παραπάνω, η σχέση \eqref{eq:linear_absorption_coefficient} μπορεί να γραφτεί ισοδύναμα
\begin{equation}
    \label{eq:radiative_equation_extinction}
    \boxed{dI_{\nu} = - \chi_{\nu} I_{\nu} ds = - \sigma_{\nu} \eta I_{\nu} ds = - \kappa_{\nu} \rho I_{\nu} ds}
\end{equation}
και όπως έχουμε πει εκφράζει το πόσο μειώνεται η ειδική ένταση της ακτινοβολίας όταν διέρχεται από ένα οπτικό μέσο, σε σχέση με την αρχική ένταση.

Το αντίστροφο του μακροσκοπικού συντελεστή απόσβεσης συνδέεται με τη \textbf{μέση ελεύθερη διαδρομή}, $\ell$ , η οποία είναι η μέση απόσταση που διανύει ένα σωματίδιο σε ένα μέσο μεταξύ δύο διαδοχικών αλληλεπιδράσεων.
\begin{equation}
    \label{eq:mean_free_path}
    \ell \equiv \frac{1}{\chi_{\nu}} = \frac{1}{\sigma_{\nu} \eta} = \frac{1}{\kappa_{\nu} \rho}
\end{equation}

Εάν ο γραμμικός συντελεστής απόσβεσης μεταβάλλεται με την απόσταση, τότε το ολοκλήρωμα του γραμμικού συντελεστή απόσβεσης ως προς την απόσταση ορίζει ένα νέο (αδιάστατο) μέγεθος, το οποίο ονομάζουμε \textbf{οπτικό βάθος}, $\tau_{\nu}$
\begin{equation}
    \label{eq:optical_depth}
    \tau_{\nu} \equiv \int_{0}^{s} \chi_{\nu}(s^{\prime}) ds^{\prime} = \int_{0}^{s} \kappa_{\nu}(s^{\prime}) \rho(s^{\prime}) ds^{\prime} 
\end{equation}
και αποτελεί ένα μέτρο της "απορροφητικότητας" ενός οπτικού μέσου μέχρι ένα συγκεκριμένο βάθος.





\subsubsection{Εκπομή}
Έστω ένα μικρό στοιχείο όγκου $dV = dA ds$, όπου το $d$s έχει την ίδια διεύθυνση με τη διεύθυνση της δέσμης της ακτινοβολίας $\boldsymbol{\hat{\eta}}$, και η επιφάνεια $dA$ είναι κάθετη σε αυτή. Τότε η ενέργεια ανά μονάδα όγκου που προστίθεται στην αρχική δέσμη της ακτινοβολίας σε χρονικό διάστημα $dt$, με συχνοτικό εύρος $d\nu$ και μέσα σε μια στερεά γωνία $d\omega$ προς τη διεύθυνση $\boldsymbol{\hat{\eta}}$ της δέσμης θα είναι
\begin{equation}
    \label{eq:emission_coefficient}
    dE_{\nu}(\boldsymbol{\hat{\eta}}) = \eta_{\nu}(\boldsymbol{\hat{\eta}}) dt d\nu dA d\omega ds \longrightarrow dI_{\nu}(\boldsymbol{\hat{\eta}}) = \eta_{\nu}(\boldsymbol{\hat{\eta}}) ds
\end{equation}
όπου ο συντελεστής $\eta_{\nu}(\boldsymbol{\hat{\eta}})$ ονομάζεται (μονοχρωματικός) \textit{συντελεστής εκπομπής} του μέσου και στο σύστημα cgs έχει διαστάσεις $\rm \left[ erg \ s^{-1} \ Hz^{-1} \ cm^{-3} \ sr^{-1} \right]$. Επίσης, μπορούμε να ορίσουμε την ενέργεια που προστίθεται στην αρχική δέσμη ανά μονάδα μάζας με τη βοήθεια του συντελεστή \textit{αφετικής ικανότητας}, $j_{\nu}$ ώστε
\begin{equation}
    \label{eq:emissivity_coefficient}
    \eta_{\nu}(\boldsymbol{\hat{\eta}}) = j_{\nu}(\boldsymbol{\hat{\eta}}) \rho
\end{equation}
όπου $\rho$ είναι η πυκνότητα μάζας του μέσου. Η αφετική ικανότητα (emissivity) έχει διαστάσεις στο cgs $\rm \left[ erg \ s^{-1} \ Hz^{-1} \ sr^{-1} \ gr^{-1} \right]$. Με τη βοήθεια της σχέσης \eqref{eq:emissivity_coefficient}, μπορούμε να γράψουμε την εξίσωση \eqref{eq:emission_coefficient} ως
\begin{equation}
    \label{eq:radiative_equation_emission}
    \boxed{dI_{\nu}(\boldsymbol{\hat{\eta}}) = \eta_{\nu}(\boldsymbol{\hat{\eta}}) ds = j_{\nu}(\boldsymbol{\hat{\eta}}) \rho ds}
\end{equation}
όπου έχουμε θεωρήσει ότι ο συντελεστής εκπομπής (ή αφετική ικανότητα) συμπεριλαμβάνουν τη συνεισφορά στην αύξηση της ειδικής έντασης τόσο της αυθόρμητης εκπομπής όσο και της σκέδασης των φωτονίων από μία αρχική διεύθυνση $\boldsymbol{\hat{\eta}^{\prime}}$ στην διεύθυνση $\boldsymbol{\hat{\eta}}$ της δέσμης.

Συγκρίνοντας τις σχέσεις \eqref{eq:radiative_equation_extinction} και \eqref{eq:radiative_equation_emission} για την απόσβεση και εκπομπή ακτινοβολίας αντίστοιχα, παρατηρούμε ότι στην πρώτη περίπτωση η ενέργεια που αφαιρείται από τη δέσμη εξαρτάται από το πόσο ισχυρή ήταν η δέσμη αρχικά (αν δεν έχουμε καθόλου ενέργεια αρχικά τότε προφανώς τίποτα δεν μπορεί να αφαιρεθεί). Αντίθετα, η ενέργεια που προστίθεται στη δέσμη είναι ανεξάρτητη της έντασης. Η μόνη εξαίρεση σε αυτό αποτελεί η εξαναγκασμένη εκπομπή. Συνήθως αυτό το φαινόμενο αντιμετωπίζεται ως "αρνητική απορρόφηση" καθώς είναι εκπομπή που εξαρτάται από την ένταση της προσπίπτουσας ακτινοβολίας. Παρόλα αυτά, δεν θα ασχοληθούμε με αυτού του είδους τη συνεισφορά. 

Σε αυτό το σημείο μπορούμε να ορίσουμε τον λόγο 
\begin{equation}
    \label{eq:source_function}
    s_{\nu} \equiv \frac{\eta_{\nu}}{\chi_{\nu}} = \frac{j_{\nu}}{\kappa_{\nu}}
\end{equation}
ο οποίος ονομάζεται \textbf{συνάρτηση πηγής}\footnote{Οι περισσότεροι συγγραφείς αναφέρονται στη συνάρτηση πηγής με το σύμβολο $S_{\nu}$. Επειδή χρησιμοποιήσαμε το συγκεκριμένο σύμβολο για την πυκνότητα ροής, επιλέξαμε να χρησιμοποιήσουμε αντ' αυτού το πεζό σύμβολο $s_{\nu}$ προς αποφυγή σύγχησης.} (source function) και ουσιαστικά αποτελεί ένα μέτρο για το πόσα φωτόνια σε μία δέσμη αφαιρούνται και προστίθονται σε αυτή καθώς διέρχεται από ένα οπτικό μέσο. Οι διαστάσεις της συνάρτησης πηγής στο σύστημα cgs είναι $\rm \left[ erg \ s^{-1} \ Hz^{-1} \ cm^{-2} \ sr^{-1} \right]$, ίδιες δηλαδή με αυτές τις ειδικής έντασης.

Συνδυάζοντας τις σχέσεις \eqref{eq:radiative_equation_extinction}, \eqref{eq:radiative_equation_emission} και \eqref{eq:source_function}, προκύπτει ότι η διαφορική εξίσωση που περιγράφει τη διάδοση της ακτινοβολίας μέσα σε ένα οπτικό μέσο είναι
\begin{equation}
    \label{eq:radiative_transfer_equation}
    \frac{dI_{\nu}}{ds} = -\chi_{\nu} I_{\nu} + \eta_{\nu} =  - \kappa_{\nu} \rho I_{\nu} + j_{\nu} \rho
\end{equation}
ή χρησιμοποιώντας ισοδύναμα την έκφραση για το οπτικό βάθος (σχέση \eqref{eq:optical_depth})
\begin{eqnarray}
    \label{eq:radiative_transfer_equation_optical_depth_expressed}
    \frac{dI_{\nu}}{\kappa_{\nu} \rho ds} &=& - I_{\nu} + \frac{j_{\nu}}{\kappa_{\nu}} \Rightarrow  \boxed{\frac{dI_{\nu}}{d \tau_{\nu}} = s_{\nu} - I_{\nu}}
\end{eqnarray}



\subsubsection{Γενική λύση της εξίσωσης μεταφοράς}



Η εξίσωση \eqref{eq:radiative_transfer_equation_optical_depth_expressed} είναι μία μη-ομογενής, γραμμική διαφορική εξίσωση πρώτου βαθμού της μορφής $$P_0 (x) y^{\prime} + P_1 (x) y = G(x), \hspace{1cm} G(x) \neq 0$$
Γνωρίζουμε ότι η γενική λύση μίας τέτοιας διαφορικής εξίσωσης αποτελείται από δύο μέρη. Το πρώτο σκέλος είναι η λύση της αντίστοιχης ομογενούς διαφορικής εξίσωσης ενώ το δεύτερο σκέλος είναι μία μερική λύση της μη-ομογενούς σύμφωνα με το παρακάτω θεώρημα:\\

{\color{red} \hrule} 
\textbf{Θεώρημα διαφορικών εξισώσεων}\\
Έστω μία μη-ομογενής γραμμική διαφορική εξίσωση της μορφής $$y^{\prime} + p(x) y = f(x)$$ 
Αν $y_h$ είναι μία μη-τετριμμένη (non-trivial) λύση της αντίστοιχης ομογενούς διαφορικής εξίσωσης $(f(x) = 0)$ και $y_p$ μία μερική (particular) λύση της μη-ομογενούς, τότε η γενική λύση της αρχικής διαφορικής εξίσωσης είναι $$y = y_h + y_p$$
όπου η μερική λύση $y_p$, σε αντίθεση με τη λύση $y_h$ της ομογενούς, \textit{δεν} περιέχει σταθερές.\\
{\color{red} \hrule} 


\underline{\textit{Λύση της αντίστοιχης ομογενούς}}\\
Αν η συνάρτηση πηγής $s_{\nu} = 0$, τότε η εξίσωση μεταφοράς γράφεται
\begin{equation*}
    \frac{dI_{\nu}}{d\tau_{\nu}} + I_{\nu} = 0 \Rightarrow \frac{dI_{\nu}}{I_{\nu}} = - d\tau_{\nu} \Rightarrow \ln{I_{\nu}} + c = - \int_{0}^{\tau_{\nu}} d\tau^{\prime} 
\end{equation*}
ή
\begin{equation}
    \label{eq:homogeneous_DE_solution}
    I_{\nu} = \rm c e^{- \tau_{\nu}} = I_{\nu} (0) e^{- \tau_{\nu}}
\end{equation}
όπου για τον προσδιορισμό της σταθεράς $c$, χρησιμοποιήσαμε την αρχική συνθήκη $\tau_{nu} = 0 \longrightarrow I_{\nu} = I_{\nu} (0)$. Παρατηρούμε ότι η σχέση \eqref{eq:homogeneous_DE_solution} δεν είναι άλλη από τη γνωστή σχέση των Beer-Lambert.

Η λύση της ομογενούς εξίσωσης μεταφοράς είναι για την περίπτωση όπου η απόσβεση της ακτινοβολίας καθώς αυτή διέρχεται μέσα από ένα μέσο είναι σημαντική αλλά η εκπομπή ακτινοβολίας μπορεί να αγνοηθεί. 'Ενα τέτοιο παράδειγμα είναι η διάδοση φωτός μέσα από το διαστρικό μέσο (interstellar medium). Στα οπτικά μήκη κύματος, το διαστρικό μέσο δεν συνεισφέρει τίποτα στην ακτινοβολία καθώς η ύλη είναι πολύ ψυχρή. Η σκόνη όμως που παρεμβάλλεται μεταξύ της πηγής και του παρατηρητή μπορεί να προκαλέσει σημαντική απόσβεση της έντασης της ακτινοβολίας, αν το μήκος του μέσου είναι αρκετά μεγάλο.

\underline{\textit{Μερική λύση της μη-ομογενούς}}\\
Θα αναζητήζουμε λύσεις της μορφής $y_p(\tau_{\nu}) = u(\tau_{\nu}) y_h (\tau_{\nu})$, όπου $y_h$ είναι η μη-τετριμμένη λύση της ομογενούς (σχέση \eqref{eq:homogeneous_DE_solution}) και η $u(\tau_{\nu})$ πρέπει να προσδιοριστεί.
Η μέθοδος αυτή που στηρίζεται σε μια λύση της ομογενούς είναι γνωστή και ως "μεταβολή των σταθερών" (variation of parameters).

Εφόσον η $y_p (\tau_{\nu})$ είναι λύση της εξίσωσης μεταφοράς (σχέση \eqref{eq:radiative_transfer_equation_optical_depth_expressed}), θα πρέπει να την ικανοποιεί
\begin{eqnarray*}
    \frac{d}{d\tau_{\nu}} \left[ u(\tau_{\nu}) y_h (\tau_{\nu}) \right] + u(\tau_{\nu}) y_h (\tau_{\nu}) &=& s_{\nu} \Rightarrow \\\\
    \frac{du(\tau_{\nu})}{d \tau_{\nu}} y_h (\tau_{\nu}) + u(\tau_{\nu}) \frac{d y_h (\tau_{\nu})}{d \tau_{\nu}} + u(\tau_{\nu}) y_h(\tau_{\nu}) &=& s_{\nu} \Rightarrow \\\\
    \frac{d u(\tau_{\nu})}{d \tau_{\nu}} I_{\nu}(0) \rm e^{- \tau_{\nu}} - \cancel{u(\tau_{\nu}) I_{\nu}(0) e^{-\tau_{\nu}}} + \cancel{u(\tau_{\nu}) I_{\nu}(0) \rm e^{- \tau_{\nu}}} &=& s_{\nu} \Rightarrow \\\\
    \frac{d u(\tau_{\nu})}{d \tau_{\nu}} = \frac{s_{\nu}}{I_{\nu}(0)} \rm e^{\tau_{\nu}} \Rightarrow u(\tau_{\nu}) &=& \int_{0}^{\tau_{\nu}} \frac{s_{\nu}}{I_{\nu}(0)} \rm e^{\tau^{\prime}} d \tau^{\prime} 
\end{eqnarray*}

Άρα. η μερική λύση της μη-ομογενούς εξίσωσης μεταφοράς είναι
\begin{eqnarray}
    \label{eq:inhomogeneous_DE_solution}
    y_p (\tau_{\nu}) &=& u(\tau_{\nu}) y_h (\tau_{\nu}) = \cancel{I_{\nu}(0)} \rm e^{- \tau_{\nu}} \int_{0}^{\tau_{\nu}} \frac{s_{\nu}}{\cancel{I_{\nu}(0)}} \rm e^{\tau^{\prime}} d \tau^{\prime} \Rightarrow \nonumber \\ \nonumber \\
    &\Rightarrow & y_p (\tau_{\nu}) = \int_{0}^{\tau_{\nu}} s_{\nu} \rm e^{\tau^{\prime} - \tau_{\nu}} d \tau^{\prime}
\end{eqnarray}

Τελικά, συνδυάζοντας τις λύσεις \eqref{eq:homogeneous_DE_solution} και \eqref{eq:inhomogeneous_DE_solution}, καταλήγουμε ότι η γενική λύση της εξίσωσης μεταφοράς είναι
\begin{equation}
    \label{eq:radiative_transfer_general_solution}
    \boxed{I_{\nu} (\tau_{\nu}) = I_{\nu} (0) \rm e^{- \tau_{\nu}} + \int_{0}^{\tau_{\nu}} s_{\nu} \rm e^{\tau^{\prime} - \tau_{\nu}} d \tau^{\prime}}
\end{equation}
όπου ο πρώτος όρος μας λέει ότι η ένταση της αρχικής δέσμης μειώθηκε κατά έναν παράγοντα $\rm e^{\tau_{\nu}}$ λόγω απόσβεσης. Όμως ένα οπτικό μέσο μπορεί να εκπέμπει και αυτό ακτινοβολία η ένταση της οποίας υπόκειται και αυτή σε απορρόφηση ή σκέδαση στη θέση $\tau^{\prime}$. Αυτή η διαδικασία περιγράφεται από τον δεύτερο όρο της λύσης της εξίσωσης μεταφοράς.

Επιπροσθέτως, εαν η συνάρτηση πηγής είναι ανεξάρτητη της θέσης μέσα στο οπτικό μέσο, μπορεί να βγει έξω από το ολοκλήρωμα της σχέσης \eqref{eq:radiative_transfer_general_solution}. Τότε η λύση της εξίσωσης μεταφοράς μπορεί να γραφτεί με την απλοποιημένη μορφή
\begin{equation}
    \label{eq:radiative_transfer_simple_solution}
    \boxed{I_{\nu}(\tau_{\nu}) = I_{\nu} (0) \rm e^{- \tau_{\nu}} + s_{\nu} \left( 1 - \rm e^{- \tau_{\nu}} \right)}
\end{equation}

Για να γίνει αντιληπτή η φυσική σημασία της λύσης \eqref{eq:radiative_transfer_simple_solution}, θα εξετάσουμε τις ακόλουθες οριακές περιπτώσεις

\begin{itemize}
    \item Όταν $\tau_{\nu} = 0$, σημαίνει ότι είτε δεν υπάρχει καθόλου ύλη, οπότε $\rho = 0$, είτε η ύλη είναι εντελώς διαφανής σε ακτινοβολίας συχνότητας $\nu$, οπότε $\kappa_{\nu} = 0$. Τότε, η λύση της εξίσωσης μεταφοράς μας δίνει ότι $$I_{\nu} = I_{\nu}(0)$$ πράγμα που σημαίνει ότι ο παρατηρητής βλέπει μόνο την ακτινοβολία της πηγής. Τότε ο ορισμός της συνάρτησης πηγής του οπτικού μέσου (αν υπάρχει) είναι δυνατό να ικανοποιείται μόνο αν $j_{\nu} = 0$. Ένα εντελώς διάφανο μέσο λοιπόν δεν εκπέμπει καθόλου ακτινοβολία.
    \item Στην περίπτωση που $\tau_{\nu} \gg 1$, τότε τα εκθετικά στη σχέση \eqref{eq:radiative_transfer_simple_solution} είναι προσεγγιστικά ίσα με μηδέν, και η λύση της εξίσωσης μεταφοράς γράφεται $$I_{\nu} = s_{\nu}$$ πράγμα που σημαίνει ότι ο παρατηρητής βλέπει μόνο την επιφάνεια του οπτικού μέσου και δεν βλέπει καθόλου την φωτεινή πηγή πίσω από αυτό (δες και σχήμα \ref{fig:optical_depth_examples}). Το οπτικό μέσο δηλαδή είναι τελείως αδιαφανές και χαρακτηρίζεται ως "οπτικά παχύ" (optically thick).
    Στην αντίθετη περίπτωση, δηλαδή όταν $\tau_{\nu} \ll 1$, τότε λέμε ότι το μέσο είναι \textit{οπτικά λεπτό} (optically thin) και υπάρχει μικρή απόσβεση της ακτινοβολίας.
    \item Τέλος, αν $\tau_{\nu} \approx 1$, τότε η φωτεινή ένταση που φτάνει στον παρατηρητή οφείλεται και στην πηγή και στο οπτικό μέσο και η λύση της εξίσωσης μεταφοράς είναι $$I_{\nu} = 0.368 I_{\nu}(0) + 0.632 s_{\nu}$$
\end{itemize}

\begin{figure}
    \centering
    \includegraphics[width=\textwidth]{Figures/optical_depth_examples.png}
    \caption{Φασματικές γραμμές από ένα ομοιογενές οπτικό μέσο με συνάρτηση πηγής $s_{\nu}$. Καμία γραμμή δεν εμφανίζεται όταν το μέσο είναι οπτικά παχύ (πάνω αριστερά). Όταν είναι οπτικά λεπτό, εμφανίζονται γραμμές εκπομπής όταν δεν υπάρχει πηγή ακτινοβολίας από πίσω ($I_{\nu}(0) = 0$, πάνω αριστερά), ή όταν προσπίπτει ακτινοβολία με $I_{\nu}(0) < s_{\nu}$. Γραμμές απορρόφησης εμφανίζονται μόνο όταν το μέσο είναι οπτικά λεπτό και ακτινοβολείται από δέσμη με ένταση $I_{\nu}(0) > s_{\nu}$. Οι εμφανιζόμενες γραμμές εμφανίζουν κορεσμό στο $I_{\nu} \approx s_{\nu}$ όταν το μέσο είναι οπτικά παχύ στη συχνότητα $\nu_0$.}
    \label{fig:optical_depth_examples}
\end{figure}


\subsection{Νόμος του Kirchhoff}
{\color{red} \hrule}
\underline{Με λίγα λόγια}: Ο νόμος της φασματοσκοπίας του Kirchhoff συνδέει το ρυθμό εκπομπής ακτινοβολίας ενός σώματος σε θερμοδυναμική ισορροπία με το ρυθμό απορρόφησης της ακτινοβολίας από το ίδιο σώμα.\\
{\color{red} \hrule}

Η φυσική σημασία της συνάρτησης πηγής $s_{\nu}$ γίνεται φανερή, αν υποθέσουμε ότι το οπτικό μέσο βρίσκεται σε θερμοδυναμική ισορροπία, δηλαδή ύλη και ακτινοβολία χαρακτηρίζονται από την ίδια θερμοκρασία $T$. Τότε η ένταση της ακτινοβολίας $I_{\nu}$ εξ' ορισμού δεν μεταβάλλεται στο εσωτερικο του μέσου, δηλαδή όσο ακτινοβολία απορροφάται σε κάθε σημείο του εσωτερικού του, τόση και εκπέμπεται. Στην περίπτωση αυτή $dI_{\nu}/ds = 0$, οπότε από τη σχέση \eqref{eq:radiative_transfer_equation} προκύπτει ότι 
\begin{equation}
    I_{\nu} = s_{\nu} = B_{\nu} (T) = \frac{2h \nu^3}{c^2} \frac{1}{\exp \left( \frac{h \nu}{kT} \right) - 1}
\end{equation}

Η τελευταία ισότητα της παραπάνω σχέσης ισχύει καθώς αφού το πεδίο ακτινοβολίας βρίσκεται σε θερμοδυναμική ισορροπία, εξαρτάται μόνο από τη θερμοκρασία του οπτικού μέσου και όχι από οποιαδήποτε άλλη φυσική ιδιότητά του, και η ένταση του θα δίνεται από τον νόμο του Planck. Με άλλα λόγια, στην περίπτωση που το οπτικό μέσο βρίσκεται σε θερμοδυναμική ισορροπία, η συνάρτηση πηγής είναι ανεξάρτητη από το υλικό του οπτικού μέσου και ισούται με την ένταση της ακτινοβολίας μέλανος σώματος. Αυτή η πρόταση αποτελεί το \textit{νόμο της φασματοσκοπίας του Kirchhoff}. 


\section{Αδιαφάνεια και αστρικά φάσματα}
Όπως αναφέραμε και προηγουμένως, οι μικροφυσικοί μηχανισμοί των οποίων το αποτέλεσμα εκφράζεται φαινομενολογικά ως συντελεστής απόσβεσης και εκπομπής, είναι κατά βάση κβαντικά φαινόμενα, αφού σχετίζονται με απορρόφηση, σκέδαση, και εκπομπή φωτονίων. Τα κύρια κβαντικά φαινόμενα που αποτελούν πηγές αδιαφάνειας στην Αστρονομία είναι οι μεταπτώσεις ηλεκτρονίων και μορίων από μία ενεργειακή κατάσταση σε μία άλλη. Επειδή οι μοριακές μεταπτώσεις είναι σημαντικές μόνο σε χαμηλές θερμοκρασίες, όπως π.χ. στις ατμόσφαιρες των ψυχρότερων αστέρων και στα νεφελώματα, η αδιαφάνεια που οφείλεται σε τέτοιου είδους μεταπτώσεις δεν θα μας απασχολήσει σε αυτό το επίπεδο.

\subsection{Μηχανισμοί αδιαφάνειας}
 Οι ηλεκτρονικές μεταπτώσεις που είναι υπεύθυνες για την αστρική αδιαφάνεια οφείλονται σε τέσσερις βασικούς μηχανισμούς:
\begin{enumerate}
    \item \textbf{Μεταπτώσεις από δέσμια σε ελεύθερη κατάσταση (bound-free transition ή φωτοϊονισμός) ή από ελεύθερη σε δέσμια κατάσταση (επανασύνδεση ή recombination)} 
    
    Η περίπτωση της επανασύνδεσης αναφέρεται στην σύγκρουση και απορρόφηση ενός ηλεκτρονίου με ένα ιόν, εκπέμποντας ένα φωτόνιο κατά τη διαδικασία αυτή. Η ενέργεια αυτού του φωτονίου θα ισούται με την κινητική ενέργεια του ηλεκτρονίου συν την ενέργεια σύνδεσης του ηλεκτρονίου που πλέον είναι μέρος του ιόντος. Επειδή η κινητική ενέργεια που είχε το ηλεκτρόνιο δεν οφείλει να είναι κβαντισμένη, η μετάπτωση επανασύνδεσης αποτελεί πηγή συνεχούς αδιαφάνειας (continuum spectrum). Παρόμοια λογική ισχύει και για την περίπτωση της δέσμιας-ελεύθερης μετάπτωσης.
    
    \item \textbf{Μεταπτώσεις από ελεύθερη σε ελεύθερη κατάσταση με την παρουσία ατομικών πυρήνων (free-free transition ή Bremsstrahlung)}
    
    Η ακτινοβολία πέδησης όπως αποκαλείται οφείλεται σε επιβράδυνση των ηλεκτρονίων (ή οποιουδήποτε φορτίου) τα οποία σκεδάζονται από άλλα άτομα ή ιόντα, χωρίς να συλλαμβάνονται από αυτά. Αυτή η διαδικασία αποτελεί ακόμα μία πηγή συνεχούς αδιαφάνειας.
    
    \item \textbf{Μεταπτώσεις από ελεύθερη σε ελεύθερη κατάσταση λόγω σκεδασμού φωτονίων από ηλεκτρόνια (σκέδαση Thomson και φαινόμενο Compton)}
    
    Στην σκέδαση Thomson, ένα φωτόνιο σκεδάζεται ελαστικά από ένα φορτισμένο σωματίδιο (συνήθως ηλεκτρόνιο) χωρίς να αλλάζει η κινητική ενέργεια του σωματιδίου ούτε η συχνότητα του φωτονίου. Η σκέδαση Thomson αποτελεί το όριο χαμηλών ενεργείων του πιο γενικού φαινομένου Compton.
    
    Στη σκέδαση Compton, ένα φωτόνιο σκεδάζεται από ένα φορτισμένο σωματίδιο (συνήθως ηλεκτρόνιο) με αποτέλεσμα να μειωθεί η ενέργειά του (να αυξηθεί το μήκος κύματος). Μέρος της ενέργειας του φωτονίου μεταφέρεται στο ανακρουόμενο ηλεκτρόνιο. Το αντίστροφο φαινόμενο Compton λαμβάνει χώρα όταν ένα φορτισμένο σωματίδιο μεταφέρει μέρος της ενέργειάς του σε ένα φωτόνιο.
    
    Καθώς και τα δύο αυτά φαινόμενα σκέδασης περιλαμβάνουν ελεύθερα σωματίδια, αποτελούν πηγές συνεχούς αδιαφάνειας.
    
    \item \textbf{Μεταπτώσεις από δέσμια σε δέσμια κατάσταση (διέγερση ατόμων ή ιόντων)}
    Η μετάπτωση από κάποια ενεργειακή στοιβάδα σε μία άλλη ενεργειακή στοιβάδα θα παράξει φωτόνιο καθορισμένης ενέργειας και άρα αυτού του είδους η μετάπτωση δεν αποτελεί πηγή συνεχούς αδιαφάνειας.
\end{enumerate}

Είναι φανερό ότι οι μηχανισμοί 1-3 προϋποθέτουν την ύπαρξη ελεύθερων ηλεκτρονίων και ιονισμένων ατόμων σε σημαντικές αριθμητικές πυκνότητες, γεγονός που προϋποθέτει υψηλές θερμοκρασίες. Δεδομένου ότι η ενέργεια ιονισμού του Υδρογόνου είναι $\rm 13.6 \ eV$, η θερμοκρασία που πρέπει να επικρατεί είναι της τάξης των $\rm 11600 \ K$. Τέτοιες θερμοκρασίες συναντώνται κατ' εξοχήν στο εσωτερικό των αστέρων  και στις επιφάνειες αστέρων προγενέστερου φασματικού τύπου. 

Ο τέταρτος μηχανισμός προϋποθέτει την ύπαρξη ουδέτερων ατόμων, που ακολουθώντας την ίδια συλλογιστική, μας οδηγεί στο συμπέρασμα ότι είναι κατ' εξοχήν σημαντικός στην επιφάνεια των αστέρων (φωτόσφαιρα) και στην ατμόσφαιρα των αστέρων, εκεί δηλαδή όπου παράγεται το συνεχές φάσμα και οι φασματικές γραμμές απορρόφησης (και ενδεχομένως εκπομπής) των αστέρων.

\subsection{Η συνεχής συνιστώσα}
Τόσο το συνεχές όσο και το γραμμικό φάσμα, δημιουργούνται ταυτόχρονα σε όλο το βάθος των εξωτερικών στρωμάτων ενός αστέρα, από τη φωτόσφαιρα μέχρι την κορυφή της ατμόσφαιρας.
Από τον νόμο του Kirchhoff, γνωρίζουμε ότι ένα σώμα που εκπέμπει συνεχές φάσμα πρέπει να έχει κάποια πηγή συνεχούς αδιαφάνειας. Οι τρεις πρώτες περιπτώσεις που αναφέραμε παραπάνω αποτελούν παραδείγματα τέτοιων μηχανισμών, και επομένως καθεμία από αυτές θα μπορούσε να είναι υπεύθυνη για τη συνεχή εκπομπή. Επειδή τα αστέρια αποτελούνται κατά βάση από Υδρογόνο και Ήλιο, η ασυνεχής αδιαφάνεια θα πρέπει να είναι συνδεδεμένη με τον ιονισμό του Υδρογόνου ή/και του Ηλίου στην επιφάνεια των αστέρων. Όμως ο ιονισμός του Υδρογόνου απαιτεί θερμοκρασίες άνω των $\rm 10000 \ K$, όπως έχουμε αναφέρει, ενώ ο ιονισμός του Ηλίου ακόμα μεγαλύτερες θερμοκρασίες. Τέτοιες θερμοκρασίες συναντώνται πραγματικά στις ατμόσφαιρες των αστέρων προγενέστερου φασματικού τύπου αλλά όχι και στις ψυχρότερες ατμόσφαιρες των αστέρων μεταγενέστερου φασματικού τύπου, όπως ο Ήλιος.

Σήμερα γνωρίζουμε ότι κύρια πηγή αδιαφάνειας στην ατμόσφαιρα του Ήλιου είναι ο φωτοϊονισμός, αλλά όχι στη συνηθισμένη εκδοχή του. Είναι η μετάπτωση από δέσμια σε ελεύθερη κατάσταση του επιπλέον ηλεκτρονίου του \textit{αρνητικού ιόντος} του Υδρογόνου $\rm H^{-}$
$$\rm H^{-} + h\nu \leftrightharpoons H + e^{-}$$
Αυτή η διαδικασία αναφέρεται ως "φωτοδιάσπαση" γιατί δίνει ένα ουδέτερο άτομο αντί για ένα ιόν. Επομένως, όταν το ανιόν του Υδρογόνου διασπάται, απορροφά φωτόνια συνεχούς ενέργειας, και άρα είναι πηγή συνεχούς αδιαφάνειας, ενώ όταν συντίθεται εκπέμπει φωτόνια συνεχούς ενέργειας και άρα είναι πηγή συνεχούς εκπομπής.

Το $\rm H^{-}$ έχει ενέργεια σύνδεσης $\rm 0.75 \ eV$ που αντιστοιχεί σε θερμοκρασία περίπου $\rm 8700 \ K$. Άρα, σε αστέρες με ενεργό θερμοκρασία μικρότερη από $\rm 8700 \ K$ η δημιουργία του $\rm H^{-}$ ευνοείται θερμοδυναμικά καθώς η δέσμια κατάσταση είναι κατά $\rm 0.75 \ eV$ ενεργειακά μικρότερη από την ενέργεια ηρεμίας ενός ατόμου Υδρογόνου και ενός ελεύθερου ηλεκτρονίου. Αντίθετα, σε αστέρες με ενεργό θερμοκρασία μεγαλύτερη από $\rm 8700 \ K$ η δημιουργία του $\rm H^{-}$ δεν ευνοείται καθώς πάρα πολλά φωτόνια έχουν ενέργεια μεγαλύτερη από $\rm 0.75 \ eV$ και προκαλούν ανισορροπία στις αντιδράσεις σχηματισμού-καταστροφής του $\rm H^{-}$, ευνοώντας τις τελευταίες. Είναι προφανές, πως αν η αριθμητική πυκνότητα των ιόντων $\rm H^{-}$ είναι μικρή, τότε η συμμετοχή τους στην παραγωγή της συνεχούς συνιστώσας του φάσματος εκπομπής είναι αμελητέα.

Αυτός ο μηχανισμός εξηγεί την δημιουργία του συνεχούς φάσματος του Ήλιου μόνο για τα μήκη κύματος για τα οποία ισχύει η σχέση $\rm h\nu > 0.75 \ eV$, τα οποία περιλαμβάνουν το ορατό φάσμα και το εγγύς υπέρυθρο. Η ένταση της συνεχούς ακτινοβολίας του Ήλιου τα ραδιοφωνικά μήκη κύματος, μέρος του υπεριώδους και στις ακτίνες Χ ειναι πολύ μεγαλύτερη από αυτή που προβλέπει ο νόμος του Planck και δεν μπορεί να εξηγηθεί με τον παραπάνω μηχανισμό. Αυτό συμβαίνει γιατί η ακτινοβολία στις συγκεκριμένες συχνοτικές περιοχές, δεν είναι θερμικής φύσης δηλαδή προέρχονται από υλικό που δεν είναι σε θερμοδυναμική ισορροπία, και επομένως δεν ισχύει στην περίπτωση αυτή ούτε ο νόμος του Planck, ούτε ο νόμος του Kirchhoff.



\section{Νόμοι Boltzmann \& Saha}
\subsection{Νόμος του Boltzmann}
Η κατανομή των ατόμων ενός στοιχείου (ουδέτερων ή ιονισμένων) στις διάφορες ενεργειακές στάθμες ακολουθεί τη στατιστική Maxwell-Boltzmann, είναι συνάρτηση της απόλυτης θερμοκρασίας $T$ της ατμόσφαιρας και δίνεται από τον νόμο του Boltzmann\footnote{Δες Παράρτημα \ref{apx:kinetic_theory} για απόδειξη της σχέσης.}
\begin{eqnarray}
    \label{eq:boltmann_law}
    \frac{n_{i,j}}{n_{0,j}} = \frac{g_{i,j}}{g_{0,j}} e^{-E_i/(kT)}
\end{eqnarray}
όπου η τιμή 0 υποδηλώνει τη θεμελιώδη στάθμη του ατόμου. Η μεταβλητή $n_{i,j}$ παριστάνει την αριθμητική πυκνότητα (αριθμός ατόμων ανά $\rm cm^3$) των ατόμων που είναι διεγερμένα στη στάθμη i $(i=0,1,\dots)$ και βρίσκονται στην κατάσταση ιονισμού j ($j = 0$ για ουδέτερα άτομα, $j=1$ για απλά ιονισμένα κτλ). Η μεταβλητή $E_i$ είναι η ενέργεια διέγερσης από τη στάθμη 0 στη στάθμη i, και k είναι η σταθερά του Boltzmann. Τέλος, $g_{i,j}$ είναι η πολλαπλότητα (ή στατιστικό βάρος) της στάθμης i, και παριστάνει το πλήθος των ενεργειακών υποσταθμών της στάθμης i υπό την παρουσία μαγνητικού πεδίου (δες φαινόμενο Zeeman). Ο συντελεστής $g_{i,j}$ συνδέεται με την ολική στροφορμή J του ατόμου με τη σχέση 
\begin{eqnarray}
    \label{eq:total_angular_momentum}
    g_{i,j} = 2J + 1
\end{eqnarray}

Ειδικά για το άτομο του Υδρογόνου ισχύει $$g_{i,0} = g_i = 2(i+1)^2 \ i = 0,1,\dots$$ αφού ο δείκτης j δεν μπορεί να πάρει τιμή διάφορη του 0 (το Υδρογόνο έχει ένα μόνο ηλεκτρόνιο και, αν ιονισθεί, δεν υπάρχει άλλο έτσι ώστε οι αλλαγές των κβαντικών αριθμών του οποίου να αλλάζουν την κβαντική κατάσταση του ιόντος του Υδρογόνου).

Είναι φανερό ότι το άθροισμα

\begin{equation}
    \label{eq:number_density_of_atoms_in_j_state}
    n_j = \sum_{i=0}^{\infty} n_{i,j} = \frac{n_{0,j}}{g_{0,j}} \sum_{i=0}^{\infty} g_{i,j} e^{-E_i/(kT)} = \frac{n_{0,j}}{g_{0,j}} Z_j(T)
\end{equation}
όπου $Z_j(T)$ η συνάρτηση επιμερισμού. Η σχέση \eqref{eq:number_density_of_atoms_in_j_state} παριστάνει την αριθμητική πυκνότητα των ατόμων που βρίσκονται στην κατάσταση ιονισμού j, ανεξάρτητα από τη στάθμη διέγερσης.

Διαιρώντας κατά μέλη τις σχέσεις \eqref{eq:boltmann_law} και \eqref{eq:number_density_of_atoms_in_j_state}, προκύπτει ο νόμος του Boltzmann στη συνηθισμένη μαθηματική μορφή του
\begin{equation}
    \label{eq:boltzmann_law_general}
    \frac{n_{i,j}}{n_j} = \frac{g_{i,j}}{Z_{j}(T)} e^{-E_i/(kT)}
\end{equation}
η οποία δίνει το λόγο της αριθμητικής πυκνότητας των ατόμων ενός στοιχείου, σε μια συγκεκριμένη κατάσταση διέγερσης και ιονισμού, προς την αριθμητική πυκνότητα των ατόμων του ίδιου στοιχείου, στην ίδια κατάσταση ιονισμού, ανεξάρτητα από την κατάσταση διέγερσης. Ο παράγοντας $e^{-E_i/(kT)}$ ονομάζεται και παράγοντας Boltzmann.



\subsection{Νόμος του Saha}
Η συναρτησιακή μορφή της σχέσης που μας δίνει τα ποσοστά των ιονισμένων ατόμων μπορεί να υπολογισθεί από τις βασικές αρχές των νόμων της χημικής ισορροπίας. Αν η ύλη στην ατμόσφαιρα του αστέρα βρίσκεται σε θερμοδυναμική ισορροπία, τότε ο ρυθμός του (απλού) ιονισμού των ατόμων ενός στοιχείου είναι ίσος με το ρυθμό επανασύνδεσής τους. Αυτή η κατάσταση παριστάνεται συμβολικά με τη χημική εξίσωση $$\text{A} + h\nu \leftrightharpoons \text{A}^+ + e^-$$ όπου $\text{A}$, $\text{A}^+$ και $e^-$ παριστάνουν αντίστοιχα το ουδέτερο άτομο, το ιονισμένο άτομο και το ηλεκτρόνιο. Ας υποθέσουμε ότι $N_0, N_i, N_e$ είναι ο αριθμός των ουδέτερων ατόμων, των ιονισμένων ατόμων και των ηλεκτρονίων αντίστοιχα, τα οποία βρίσκονται σε ένα κουτί όγκου $V$. Τότε, η εξίσωση του Saha θα είναι:
\begin{equation}
    \label{eq:saha_function}
    \frac{N_e N_i}{N_0} = S (T,P)
\end{equation}

Η εξίσωση του Saha είναι συνάρτηση της πίεσης και της θερμοκρασίας, με την υψηλή θερμοκρασία να ευνοεί τον ιονισμό ενώ η υψηλή πίεση την επανασύνδεση (recombination). Η εξίσωση μας λέει ότι οι σχετικοί αριθμοί τριών τύπων σωματιδίων (με άλλα λόγια ο βαθμός ιονισμού) σε μια κατάσταση ισορροπίας, όταν ο ρυθμός ιονισμού είναι ίσος με τον ρυθμό επανασύνδεσης.

Γνωρίζουμε ότι ο αριθμός των σωματιδίων σε ένα συγκεκριμένο ενεργειακό επίπεδο είναι ανάλογος του παράγοντα Boltzmann για το συγκεκριμένο επίπεδο, και ο ολικός αριθμός των σωματιδίων είναι ανάλογος του αθροίσματος των παραγόντων Boltzmann για όλα τα ενεργειακά επίπεδα --- δηλαδή, ανάλογος της συνάρτησης επιμερισμού. Έτσι, η εξίσωση του Saha γράφεται

\begin{equation}
    \label{eq:saha_partition_functions}
    \frac{N_e N_i}{N_0} = \frac{Q_e Q_i}{Q_0}
\end{equation}

Όπως είπαμε, η συνάρτηση επιμερισμού είναι το άθροισμα όλων των παραγόντων Boltzmann για όλες τις ενεργειακές καταστάσεις, μεταφορικές (χωρίς περιστροφή) και εσωτερικές (ηλεκτρονιακή δομή). Η ολική ενέργεια ενός σωματιδίου είναι το άθροισμα της μεταφορικής και εσωτερικής ενέργειας, έτσι ώστε η ολική συνάρτηση επιμερισμού να είναι το γινόμενο των μεταφορικών και εσωτερικών συναρτήσεων επιμερισμού\footnote{Δες σχετικό κεφάλαιο στο Παράρτημα \ref{apx:kinetic_theory}}, για τις οποίες θα χρησιμοποιήσουμε το σύμβολο $u$. Έτσι έχουμε:
\begin{equation}
    \frac{N_e N_i}{N_0} = \left( \frac{2\pi m k T}{h^2} \right)^{\frac{3}{2}} V \frac{u_e u_i}{u_0}
\end{equation}
όπου $m = \frac{m_e m_i}{m_0}$, και διαφέρει πολύ λίγο από το $m_e$.

Η εσωτερική συνάρτηση επιμερισμού του ηλεκτρονίου ισούται με το στατιστικό του βάρος $2s + 1 = 2$. 'Ετσι, μπορούμε να γράψουμε την παραπάνω εξίσωση με όρους αριθμητικής πυκνότητας $(n = N/V)$ και να καταλήξουμε στη συνηθισμένη μορφή της εξίσωσης του Saha:
\begin{equation}
    \label{eq:saha_equation_full}
    \frac{n_e n_i}{n_0} = \left( \frac{2\pi m k T}{h^2} \right)^{\frac{3}{2}} \frac{2u_i}{u_0} \exp \left( - \frac{\chi_i}{kT} \right)
\end{equation}
όπου $\chi_i$ είναι η ενέργεια ιονισμού.

Η εξίσωση Saha έπεξε καθοριστικό ρόλο στην κατανόηση των αστρικών φασμάτων. Όπως έχουμε αναφέρει, η φασματική ταξινόμηση $\text{O,B,A,F,G, \dots}$ είναι αποτέλεσμα του βαθμού ιονισμού και διέγερσης των χημικών στοιχείων ως συνάρτηση της θερμοκρασίας, ενώ η διαφορά στο βαθμό ιονισμού μεταξύ αστέρων της κύριας ακολουθίας και αστέρες-γίγαντες μιας συγκεκριμένης θερμοκρασίας είναι το αποτέλεσμα του υψηλότερου βαθμού ιονισμού στις σχετικά χαμηλής πίεσης ατμόσφαιρες των αστέρων-γιγάντων.



\section{Ο Ήλιος ως τυπικός αστέρας}
\subsection{Μακροσκοπικά χαρακτηριστικά}
Από τις παρατηρήσεις που έχουμε για τον Ήλιο, γνωρίζουμε πως η φαινόμενη διάμετρός του είναι
$$d = 32^\prime = 9.3 \times 10^{-3} \hspace{0.5cm} \text{rad}$$ που αντιστοιχεί σε ακτίνα
$$R_{\odot} = 1 \ \text{AU} \times \sin{\frac{d}{2}} = 6.96 \times 10^{10} \hspace{0.25cm} \text{cm}$$ και η οποία είναι σε γενικές γραμμές σταθερή (πέρα από μερικές μικρές ταλαντώσεις). Η βολομετρική φαινόμενη λαμπρότητα του Ήλιου, δηλαδή η ολική φωτεινή ροή της ακτινοβολίας του Ήλιου σε απόσταση 1 AU ονομάζεται \textbf{ηλιακή σταθερά}, $f$, και ισούται με 
$$f=1.36 \times 10^6 \hspace{0.25cm} \text{erg sec$^{-1}$ cm$^{-2}$}$$

Υποθέτωντας πως ο Ήλιος ακτινοβολεί ισοτροπικά, η ηλιακή σταθερά ισούται με τη λαμπρότητα του Ήλιου (δηλαδή την ισχύ που ακτινοβολεί σε όλο το εύρος του ηλεκτρομαγνητικού φάσματος) διαμοιρασμένη στην επιφάνεια μιας σφαίρας με ακτίνα την αστρονομική μονάδα
$$L_{\odot} = 4\pi f (1 \ \text{AU})^2 = 3.9 \times 10^{33} \hspace{0.25cm} \text{erg sec$^{-1}$}$$
Η ενεργός θερμοκρασία του Ήλιου, όπως υπολογίζεται με τη βοήθεια της σχέσης
$$L_{\odot} = 4\pi R^2 \sigma T_{\text{eff}}^4$$ 
είναι $T_{\text{eff}} = 5800 \hspace{0.25cm} \text{K}$, ο δε φασματικός τύπος του \textbf{G2V}.

Η μάζα του Ήλιου μπορεί να υπολογιστεί είτε από τον τρίτο νόμο του Kepler 
$$\frac{A^3}{P^3} = G \frac{M_{\odot} + M_{\oplus}}{4 \pi ^2}$$ αν θέσουμε $A = 1 \ AU$, $P = 1 \ \text{yr}$ και αγνοήσουμε τη μάζα της Γης σε σύγκριση με τη μάζα του Ήλιου, θέτοντας $M_{\oplus} = 0$. Έτσι βρίσκουμε $M_{\odot} \simeq 2 \times 10^{33} \hspace{0.25cm} \text{gr}$. Ένας άλλος τρόπος είναι από την περιφορά της Γης γύρω από τον Ήλιο αν εξισώσουμε την κεντρομόλο δύναμη με την δύναμη της βαρύτητας:
$$F = ma \Rightarrow G \frac{M_\odot M_\oplus}{d^2} = M_\oplus \frac{v^2}{d} \Rightarrow M_{\odot} = \frac{v^2 d}{G}$$ όπου $d$ είναι η μέση απόσταση Γης-Ήλιου, και $v$ η ταχύτητα περιφοράς της Γης γύρω από αυτόν. Η ταχύτητα της περιφοράς της Γης γύρω από τον Ήλιο βρίσκεται πολύ απλά
$$v = \frac{2\pi d}{365 \ \text{days}} = 30 \ \text{km s$^{-1}$}$$
Έτσι, προκύπτει πάλι $M_{\odot} \simeq 2 \times 10^{33} \hspace{0.25cm} \text{gr}$.

Η ταχύτητα περιστροφής του Ήλιου γύρω από τον άξονά του μπορεί να μετρηθεί με περισσότερες από μία μεθόδους, λόγω της μικρής απόστασής του από τη Γη. Ο χρόνος που απαιτείται ώστε τα διάφορα μακρόβια φαινόμενα της επιφάνειάς του (π.χ. κηλίδες) να επανέλθουν στο ίδιο σημείο είναι μία από αυτές. Η μέτρηση της μετατόπισης Doppler των φασματικών γραμμών του φωτός που προέρχεται από το χείλος του Ήλιου είναι μία άλλη. Όλες οι μέθοδοι δίνουν την ίδια γενική εικόνα, αλλά οι τιμές για την ταχύτητα περιστροφής μπορεί να διαφέρουν αισθητά. Οι διαφορές αυτές οφείλονται στο ότι κάθε μέθοδος μετρά την ταχύτητα περιστροφής του ηλιακού υλικού σε διαφορετικές αποστάσεις από το κέντρο του Ήλιου (ή, ισοδύναμα, σε διαφορετικά βάθη από την επιφάνειά του), και αυτές διαφέρουν επειδή ο Ήλιος δεν περιστρέφεται σαν στερεό σώμα, αλλά εκτελεί διαφορική περιστροφή (differential rotation). Ο Ήλιος περιστρέφεται διαφορικά όχι μόνο καθ' ύψος (ως συνάρτηση δηλαδή της απόστασης από το κέντρο), αλλά και κατά ηλιογραφικό πλάτος. Η περίδος περιστροφής ενός στοιχείου της επιφάνειας του Ήλιου στην περιοχή του Ισημερινού του είναι περίπου 25 μέρες, που αντιστοιχεί σε γραμμική ταχύτητα $2 \ \text{km s$^{-1}$}$, ενώ η περίοδος κοντά στους πόλους είναι μεγαλύτερη από 27 μέρες. Η διαφορά αυτή, αν και δεν έχει εξηγηθεί πλήρως, παίζει κύριο ρόλο στη δημιουργία του μαγνητικού πεδίου του Ήλιου ($\alpha - \omega $ dynamo).

Οι αστέρες μεταγενέστερου φασματικού τύπου, όπως ο Ήλιος, φαίνεται να έχουν παραπλήσιες γραμμικές ταχύτητες $(u \sim 2 \ \text{km/s})$. Οι αστέρες όμως προγενέστερων φασματικών τύπων φαίνεται ότι περιστρέφονται πολύ ταχύτερα, με γραμμικές ταχύτητες της τάξης των $200 \ \text{km/s}$, που αντιστοιχούν σε περιόδους περιστροφής της τάξης των λίγων ημερών. Η μεγάλη αυτή διαφορά στις ταχύτητες περιστροφής μεταξύ αστέρων προγενέστερων και μεταγενέστερων φασματικών τύπων δεν έχει εξηγηθεί πλήρως, αλλά πιστεύεται ότι έχει σχέση με τον τρόπο δημιουργίας των αστέρων και με την ύπαρξη ή όχι πλανητικών συστημάτων.



\subsection{Η επιφάνεια του Ήλιου}
Σε αυτό το σημείο θα προσπαθήσουμε να απαντήσουμε στα εξής ερωτήματα: Ποιό είναι το ποσοστό των φωτονίων που παράγονται και τα οποία φτάνουν σε εμάς; Γιατί ο Ήλιος φαίνεται να έχει πολύ καλά καθορισμένο περίγραμμα και γιατί στην άκρη του φαίνεται πιο αμυδρός απ' ότι στο κέντρο του;

Έστω επιφάνεια στο εσωτερικό του Ήλιου από την οποία εκπέμπονται φωτόνια στο οπτικό μήκος κύματος $(\lambda \sim 5000 \ \text{\AA})$ και διέρχονται από τα ανώτερα στρώματα της ατμόσφαιρας του Ήλιου. Έστω ότι το οπτικό βάθος για αυτή την επιφάνεια είναι $\tau_{\nu = 5000\text{\AA}} \equiv \tau = 6.9$. 
Από τον όρο που μας δίνει την απόσβεση της ακτινοβολίας, βρίσκουμε ότι:
$$I_{\nu = 5000\text{\AA}} \equiv I = I(0) e^{- \tau} \Rightarrow \frac{I}{I(0)} = e^{-6.9} = 0.001$$
Άρα, μόνο το $0.1 \%$ των παραγόμενων φωτονίων σε αυτή την επιφάνεια φτάνουν σε εμάς. Με άλλα λόγια, όταν κοιτάμε τον Ήλιο, δεν βλέπουμε αυτή την επιφάνεια καθώς όλα τα φωτόνια που εκπέμθηκαν από αυτή, έχουν απορροφηθεί. Με αντίστοιχη λογική βρίσκουμε ότι
\begin{itemize}
    \item Για επιφάνεια σε οπτικό βάθος $\tau = 4.5$, το $1 \%$ των φωτονίων φτάνουν σε εμάς.
    \item Για επιφάνεια σε οπτικό βάθος $\tau = 1$, το $37 \%$ των φωτονίων φτάνουν σε εμάς.
    \item Για επιφάνεια σε οπτικό βάθος $\tau = 0.7$, το $50 \%$ των φωτονίων φτάνουν σε εμάς.
\end{itemize}
Τελικά, τα φωτόνια που βλέπουμε εμείς, από ποιά επιφάνεια προήλθαν; Η απάντηση είναι απο όλες, απλά από την επιφάνεια που αντιστοιχεί σε οπιτκό βάθος $\tau = 4.5$, ο αριθμός παραγωγής φωτονίων είναι αρκετός ώστε να αρχίσουμε να τα ανιχνεύουμε.

\textbf{Προσοχή}: Αυτά είναι ποσοστά! Το 10 \% μιας επιφάνειας μπορεί να αντιστοιχεί σε περισσότερα φωτόνια από το 80 \% μιας άλλης επιφάνειας. Πρέπει να θυμόμαστε ότι υπάρχει εξάρτηση και από τον αριθμό των φωτονίων που παράγει η κάθε επιφάνεια. Ο αριθμός αυτός εξαρτάται από την θερμοκρασία όπως γνωρίζουμε από τον νόμο του Planck. Όσο προχωράμε προς το κέντρο του Ήλιου, η θερμοκρασία αυξάνεται και άρα παράγονται περισσότερα φωτόνια. Το τελευταίο συμπέρασμα προκύπτει αβίαστα, καθώς είναι ο μόνος τρόπος για να ερμηνεύσουμε το παρατηρησιακό φαινόμενο της \textbf{αμαύρωση του χείλους} του Ήλιου, κατά το οποίο τα χείλη του Ηλιακού δίσκου φαίνονται στα οπτικά μήκη κύματος πιο αμυδρά (σκοτεινότερα) από το κέντρο του δίσκου. Αν δεχτούμε ότι ως παρατηρητές από τη Γη βλέπουμε φωτόνια που προέρχονται από ένα μέσο βάθος $\tau \approx 1$, τότε όταν κοιτάμε το κέντρο του Ηλιακού δίσκου, η επιφάνεια που αντιστοιχεί σε αυτό το οπτικό βάθος (επιφάνεια Α) έχει θερμοκρασία $\rm T_{HI}$. Όταν κοιτάμε όμως το χείλος του Ηλιακού δίσκου, η επιφάνεια που αντιστοιχεί σε οπτικό βάθος $\tau \approx 1$ (επιφάνεια Β) βρίσκεται πιο πάνω από την επιφάνεια Α, και έχει θερμοκρασία $\rm T_{LO}$. Για να είναι λιγότερα τα φωτόνια που εκπέμπει η επιφάνεια Β σε σχέση με την επιφάνεια Α, θα πρέπει αναγκαστικά $\rm T_{LO} < T_{HI}$ (σχήμα \ref{fig:limb_darkening}).


\begin{figure}[h]
    \centering
    \includegraphics[scale=0.08]{Figures/limb_darkening.png}
    \caption{Εμηνεία του φαινομένου της αμαύρωσης του χείλους του 'Ηλιου σε οπτικά μήκη κύματος. Το εξωτερικό όριο αντιστοιχεί σε ακτίνα στην οποία τα φωτόνια που εκπέμπει ο αστέρας δεν υπόκεινται σε απορρόφηση. Ο παρατηρητής βλέπει στρώματα του Ήλιου που αντιστοιχούν σε οπτικό βάθος $\tau \approx 1$ (απόσταση L). Όταν παρατηρεί προς το κέντρο του Ηλιακού δίσκου, βλέπει βαθύτερα και θερμότερα, και άρα λαμπρότερα στρώματα, απ' ότι όταν παρατηρεί προς το χείλος του Ηλιακού δίσκου.}
    \label{fig:limb_darkening}
\end{figure}

Κάνοντας αναλύτικά τους υπολογισμούς, βρίσκουμε ότι ο μεγαλύτερος αριθμός φωτονίων με $\lambda = 5000 \ \text{\AA}$ προέρχονται από τον φλοιό που ορίζεται από την επιφάνεια για την οποία το οπτικό βάθος είναι $\tau = 4.5$ και την επιφάνεια για την οποία το οπτικό βάθος είναι $\tau = 0.7$. Επιφάνειες που βρίσκονται σε ακόμα μικρότερο βάθος από την επιφάνεια με $\tau = 0.7$, ουσιαστικά δεν παράγουν αρκετά οπτικά φωτόνια για να τις συμπεριλάβουμε και απλά θεωρούμε ότι τα φωτόνια που παράγονται στα κατώτερα στρώματα απλά διέρχονται από αυτές. Τον φλοιό αυτό τον ονομάζουμε \textbf{φωτόσφαιρα}.

Το ερώτημα τώρα είναι πόσο είναι το πάχος αυτού του σφαιρικού φλοιού που ορίζει τη φωτόσφαιρα του Ήλιου.
Θεωρώντας τα $\kappa_{\nu=5000 \ \text{\AA}} \equiv \kappa, \rho$ σταθερά σε αυτό το πάχος του φλοιού, τότε
$$\Delta S = S_{4.5} - S_{0.7} = \left. \frac{\tau}{\kappa \rho} \right|_{4.5} - \left. \frac{\tau}{\kappa \rho} \right|_{0.7} = \frac{4.5 - 0.7}{\kappa \rho} $$
Αντικαθιστώντας τις τιμές $\kappa = 0.03 \ \text{m$^2 $kg$^{-1}$}$ και $\rho = 2.1 \times 10^{-4} \ \text{kg m$^{-3}$}$ για την αδιαφάνεια και την μέση πυκνότητα του Ήλιου, τις οποίες γνωρίζουμε από παρατηρήσεις, προκύπτει ότι το πάχος του φλοιού από τον οποίο προέρχονται η πλειοψηφία των οπτικών φωτονίων είναι $$\Delta S \simeq 600 \ \text{km}$$
Λόγω του πολύ μικρού πάχους του φλοιού αυτού, το οποίο αντιστοιχεί στο $0.1 \%$ της ηλιακής ακτίνας, η φωτόσφαιρα πολλές φορές θεωρείται ως η επιφάνεια του Ήλιου. Όι επιφάνειες (και όλες οι ενδιάμεσες) που ορίζουν την φωτόσφαιρα του Ήλιου, χαρακτηρίζονται όπως είπαμε από διαφορετικές θερμοκρασίες καθώς αντιστοιχούν σε διαφορετικά βάθη. Γίνεται αντιληπτό πως η ενεργός θερμοκρασία του Ήλιου δεν χαρακτηρίζει κάποιο συγκεκριμένο στρώμα της φωτόσφαιρας, αλλά αποτελεί έναν εύχρηστο μέσο όρο.

Το εξαιρετικά μικρό πάχος της φωτόσφαιρας απαντάει στο ερώτημα του γιατί ο Ήλιος έχει σαφώς καθορισμένο περίγραμμα καθώς τα τηλεσκόπια δεν έχουν την απαραίτητα διακριτική ικανότητα να ξεχωρίσουν τον μικρό αυτό φλοιό σε σχέση με την ακτίνα του Ήλιου. Αν μπορούσαμε να διακρίνουμε αποστάσεις 600 χιλιομέτρων στον Ήλιο, τότε σαφώς το περίγραμμά του δεν θα ήταν τόσο καθαρό.




\subsection{Σύνοψη}
Η φασματική κατανομή μέλανος σώματος θερμοκρασίας $T_{\text{eff}}$ δίνεται από τον νόμο του Planck και σε καλή προσέγγιση συμφωνεί με την παρατηρούμενη φασματική κατανομή της ηλιακής ακτινοβολίας για  $T_{\text{eff}} = 5800 \ \text{K}$. Παρόλα αυτά, δεν συμπίπτουν τέλεια με τις διαφορές να οφείλονται στους εξής λόγους:
\begin{enumerate}
    \item Η ακτινοβολία του Ήλιου που φτάνει σε εμάς προέρχεται από διαφορετικά βάθη των εξωτερικών στρωμάτων του Ήλιου, τα οποία έχουν διαφορετικές θερμοκρασίες. Έτσι, η φωτεινή ένταση σε μια φασματική περιοχή είναι το άθροισμα των εντάσεων κατανομών που αντιστοιχούν σε διαφορετικές θερμοκρασίες.
    \item Στα μεγάλα (ραδιοφωνικά) και μικρά μήκη κύματος (υπεριώδες, ακτίνες-Χ κτλ) η ακτινοβολία του Ήλιου δεν είναι θερμικής φύσης, οπότε η έντασή της δεν έχει κανένα λόγο να ακολουθεί τον νόμο του Planck.
    \item Το ηλιακό φάσμα παρουσιάζει γραμμές απορρόφησης (γνωστές και ως γραμμές Fraunhofer) οι οποίες ελαττώνουν τη μέση ένταση της ακτινοβολίας σητν περιοχή που εμφανίζονται. Το φαινόμενο αυτό είναι ιδιαίτερα εμφανές στην περιοχή του ορίου συσσώρευσης των γραμμών της σειράς Balmer, που ονομάζεται ασυνέχεια Balmer.
    \item Τέλος, η ύλη που ακτινοβολεί θερμικά δεν βρίσκεται σε τοπική θερμοδυναμική ισορροπία, και άρα δεν ισχύει ακριβώς ο νόμος του Planck. Αξίζει να σημειωθεί όμως ότι η απόκλιση της κατανομής της έντασης της ηλιακής ακτινοβολίας από αυτή του μελανού σώματος, λόγω της μη ακριβούς θερμοδυναμικής ισορροπίας, είναι ασήμαντη σε σχέση με τις τρεις προηγούμενες αιτίες.
\end{enumerate}












%     \chapter{Αστρικά κατάλοιπα}
\label{ch:Chapter6}
{\hypersetup{linkcolor=black, pdfborder=0 0 1}
	\minitoc
	%\newpage
}

Όταν σταματήσουν οι θερμοπυρηνικές αντιδράσεις στο εσωτερικό των άστρων, τότε ο πυρήνας αρχίζει να ψύχεται, επειδή δεν αναπληρώνονται τα ποσά της ενέργειας που ρέουν προς τα εξωτερικά στρώματα του αστέρα. Η ψύξη του πυρήνα, όμως, συνεπάγεται πτώση της θερμικής πίεσης στο εσωτερικό του, οπότε η πίεση των υπερκείμενων στρωμάτων αρχίζει να υπερισχύει της θερμικής πίεσης του αερίου, με αποτέλεσμα ο πυρήνας να αρχίσει να συστέλλεται. Αν η μάζα του είναι μικρή ($M < 1\,M_\odot$), η συστολή του δεν συνοδεύεται, συνήθως, από καταστροφικά φαινόμενα. Αντίθετα, η ύλη αστέρων μεγάλης μάζας υφίσταται καταστροφική "κατάρρευση", η οποία συνήθως ακολουθείται από έκρηξη, και η ισορροπία των δυνάμεων που διέπουν την ύπαρξη της τελικής κατάστασης, στην οποία θα περιπέσουν αυτοί οι αστέρες, είναι πολύ λεπτή.

Με τις σημερινές γνώσεις της Φυσικής πιστεύουμε ότι είναι δυνατόν να υπάρξουν τριών ειδών τελικές καταστάσεις, όταν σταματήσει οριστικά η παραγωγή ενέργειας από θερμοπυρηνικές αντιδράσεις, στις οποίες γενικά αναφερόμαστε ως \textbf{συμπαγείς αστέρας} (compact stars) επειδή έχουν μικρές τυπικές διαστάσεις και μεγάλες πυκνότητες. Μία τέταρτη περίπτωση κατά την οποία ο αστέρας διαλύεται, με την ύλη να διασκορπίζεται στον μεσοαστρικό χώρο χωρίς να αφήνει πίσω κάποιο κατάλοιπο, θα συζητηθεί στο Κεφάλαιο \ref{ch:Chapter7}.


\section{Λευκοί νάνοι}
 Οι λευκοί νάνοι (white dwarfs) είναι συμπαγείς αστέρες οι οποίοι δεν εξελίσσονται πλέον, δεδομένου ότι στον πυρήνα τους δεν συμβαίνουν πια θερμοπυρηνικές αντιδράσεις. Στην τελική αυτή κατάσταση θα καταλήξουν όλοι οι αστέρες των οποίων η αρχική μάζα (κατά τη στιγμή της εγκατάστασής τους στην κύρια ακολουθία) δεν υπερβαίνει τις $\sim 5\,M_\odot$. Τέτοιες μάζες έχει το μεγαλύτερο ποσοστό ($\sim 90\,\%$) των αστέρων, μεταξύ των οποίων συμπεριλαμβάνεται και ο Ήλιος.
 
 Οι λευκοί νάνοι είναι το προϊόν της αναχαίτησης της βαρυτικής κατάρρευσης ενός κοινού αστέρα από την πίεση των \textit{εκφυλισμένων ηλετρονίων} του πυρήνα του. Περισσότερες πληροφορίες για τις ιδιότητες της εκφυλισμένης ύλης δίνονται στο Παράρτημα\,\ref{apx:kinetic_theory}. Η κατάρρευση αυτή αρχίζει όταν εξαντληθεί όλο το διαθέσιμο ``καύσιμο' ' του πυρήνα του αστέρα. Επομένως το εσωτερικό των λευκών νάνων αποτελείται κατά βάση είτε από ήλιο, είτε από μείγμα άνθρακα και οξυγόνου.
 
 Το μαγνητικό πεδίο στην επιφάνεια των λευκών νάνων είναι ιδιαίτερα ισχυρό ($\sim 10^6\,\text{G}$). Αυτό οφείλεται στην διατήρηση (κατά την τελική συστολή ενός ερυθρού γίναντα προς δημιουργία λευκού νάνου) της επιφανειακής μαγνητικής ροής, $\sim BR^2$, όπου $B$ είναι το μαγνητικό πεδίο στην επιφάνεια του αστέρα και $R$ η ακτίνα του.
 
 Ο πρώτος λευκός νάνος που παρατηρήθηκε ήταν ο Σείριος Β ο οποίος είναι ο συνοδός αστέρας του λαμπρού Σείριου Α (Sirius, a CMi), που είναι ένας από τους κοντινότερους αστέρας. Το σύστημα αυτό αποτελεί έναν αστρομετρικό διπλό σύστημα που μας επέτρεψε να μετρήσουμε τις μάζες των δύο αστέρων (δες Κεφάλαιο\,\ref{ch:Chapter7}). Για τον Σείριο Β προέκυψε ότι πρέπει να έχει μάζα $M_{\text{Sb}} \simeq 0.97\,M_\odot$, επιφανειακή θερμοκρασία $T_{\text{eff, Sb}} \simeq 27000\,\text{K}$, λαμπρότητα $L_{\text{Sb}} \simeq 0.03\,L_\odot$, και ακτίνα $R_{\text{Sb}} \simeq 0.008\,R_\odot \simeq 0.9\,R_\oplus$. Χρησιμοποιώντας αυτές τις τιμές μπορούμε να έχουμε μία εκτίμηση για την μέση πυκνότητα που επικρατεί στον Σείριο Β
 $$\bar{\rho}_{\text{Sb}} = \frac{3M_{\text{Sb}}}{4\pi R_{\text{Sb}}^3} = 2 \times 10^6\,\rho_\odot = 2 \times 10^9\,\text{kg m$^{-3}$}$$η οποία είναι εξαιρετικά μεγάλη. Το αμέσως επόμενο εύλογο ερώτημα είναι ποιά είναι η απαιτούμενη πίεση ώστε να υποστηρίξει τον αστέρα. Ξέρουμε ότι ο Σείριος Β είναι ευσταθής άρα θα βρίσκεται σε κατάσταση υδροστατικής ισορροπίας. Αυτό σημαίνει ότι μπορούμε να πάρουμε πολύ προσεγγιστικά μία τιμή για την πίεση που επικρατεί στο κέντρο του χρησιμοποιώντας την εξίσωση υδροστατικής ισορροπίας
 $$\frac{dP}{dr} = - G \frac{m(r) \rho(r)}{r^2} \longrightarrow \frac{P_s - P_c}{R_s - R_c} \approx -G \frac{M \bar{\rho}}{R^2} \rightarrow P_c \approx 5 \times 10^{17}\,\text{atm}$$Αυτή λοιπόν είναι προσεγγιστικά η πίεση που πρέπει να επικρατεί στο κέντρο του Σείρου Β ώστε να βρίσκεται σε υδροστατική ισορροπία. Ποιά είναι όμως η πηγή αυτής της πίεσης;
 
 Αν η πίεση αυτή οφείλεται στη θερμική κίνηση των σωματιδίων τότε μπορούμε εύκολα να υπολογίσουμε την θερμοκρασία που θα έπρεπε να επικρατεί στο κέντρο του Σείριου Β χρησιμοποιώντας την καταστατική εξίσωση των τέλειων αερίων (υποθέτοντας για λόγους απλότητας ότι η ύλη αποτελείται εξ' ολοκλήρου από άνθρακα)
 $$P_{\text{gas}} = n k_B T = \frac{\rho}{\mu m_H} k_B T \longrightarrow T_c \approx 6 \times 10^9\,\text{K}$$Η τιμή αυτή της θερμοκρασίας είναι πολύ μεγάλη και δεν είναι δυνατόν να αντιστοιχεί στην πραγματική τιμή της θερμοκρασίας που επικρατεί στο κέντρο του Σείριου Β. Ο λόγος είναι ότι σε τέτοιες υψηλές θερμοκρασίες θα είχε ήδη ξεκινήσει η πυρηνική σύντηξη του άνθρακα και άρα ο Σείρος Β θα έπρεπε να εμφανίζεται να έχει λαμπρότητα πολλές τάξεις μεγέθους μεγαλύτερη από αυτή που παρατηρούμε. Άρα ο μηχανισμός που παρέχει την πίεση σίγουρα δεν είναι θερμικής φύσης!
 
 Έχοντας ως δεδομένο ότι η θερμοκρασία στο κέντρο του Σείριου Β δεν μπορεί, σε καμία περίπτωση, να υπερβαίνει σημαντικά τους $10^7\,\text{K}$, επειδή στη θερμοκρασία των $10^8\,\text{K}$ αρχίζουν οι θερμοπυρηνικές αντιδράσεις καύσης των στοιχείων του πυρήνα του (ηλίου ή άνθρακα και οξυγόνου), προκύπτει ότι ούτε η πίεση της ακτινοβολίας 
 $$P_{\text{rad}} = \frac{1}{3} \alpha T^4$$
 είναι αρκετή για την δημιουργία της κατάλληλης πίεσης ωστε ο Σείριος Β να είναι σε κατάσταση υδροστατικής ισορροπίας. Αυτό σημαίνει ότι πρέπει να υπάρχει στο εσωτερικό του μία πρόσθετη πηγή πίεσης.

\subsection{Η πίεση εκφυλισμένου αερίου}
Για να αντιληφθούμε τη φύση αυτής της πρόσθετης πηγής πίεσης, πρέπει να λάβουμε υπόψη ότι στην πραγματικότητα οι αστέρες δεν αποτελούνται ακριβώς από τέλειο ρευστό. Καθώς ο αστέρας συστέλλεται, η πίεση και η πυκνότητα αυξάνονται σε τέτοιο βαθμό, ώστε τα άτομα διαμελίζονται σε ηλεκτρόνια και γυμνούς πυρήνες. Ο διαμελισμός αυτός των ατόμων εξακολουθεί να ισχύει, ακόμα και όταν ο αστέρας ψυχθεί, λόγω της απώλειας ενέργειας με ακτινοβολία, σε θερμοκρασία μικρότερη από τη θερμοκρασία ιονισμού των ατόμων και, για το λόγο αυτό, ονομάζεται \textit{ιονισμός πίεσης}. Όταν ο ιονισμός πίεσης είναι πλήρης, τα ηλεκτρόνια κινούνται μέσα στο πλέγμα των βαρύτερων και, συνεπώς πρακτικά ακίνητων, πυρήνων έτσι, ώστε το υλικό του αστέρα έχεις τις ιδιότητες ενός μετάλλου. Κάτω από τις συνθήκες αυτές, σε πλήρη αναλογία με τα μέταλλα, η ενέργεια η παραγόμενη στο εσωτερικό του αστέρα διαδίδεται με \textit{αγωγιμότητα} παρά με ακτινοβολία ή ρεύματα μεταφοράς. Το σύνολο των ηλεκτρονίων προσομοιάζεται με ένα αέριο, το \textit{ηλεκτρονικό αέριο}, για το οποίο, η καταστατική εξίσωση είναι ανεξάρτητη από την απόλυτη θερμοκρασία (σε αντίθεση με την εξίσωση κατάστασης των τέλειων αερίων). Αυτή η \textit{οριακή} μορφή της καταστατικής εξίσωσης του ηλεκτρονικού αερίου ονομάζεται κατάσταση \textbf{πλήρους εκφυλισμού}. Το βασικό χαρακτηριστικό της είναι ότι κάτω από συνθήκες υψηλής πίεσης και χαμηλής θερμοκρασίας, όπως οι παραπάνω, η ενέργεια της θερμικής κίνησης ενός ηλεκτρονίου είναι πολύ μικρότερη από την ενέργεια ηρεμίας του και, επομένως, τα ηλεκτρόνια μεταπίπτουν στις χαμηλότερες δυνατές ενεργειακές στάθμες τους. Όλες οι ενεργειακές στάθμες των ηλεκτρονίων, μέχρι μιας ανώτερης δυνατής, είναι κατειλημμένες, ενώ όλες οι ανώτερες απ' αυτήν είναι κενές. Η ενέργεια της ανώτερης κατειλημένης στάθμης ονομάζεται \textbf{ενέργεια Fermi} του συστήματος, $\epsilon_F$.

Η λεπτομερής ποσοτική περιγραφή των ιδιοτήτων του εκφυλισμένου αερίου ηλεκτρονίων κάτω από συνθήκες πολύ υψηλής πίεσης και πολύ χαμηλής θερμοκρασίας απαιτεί την χρήση των αρχών της στατιστικής κβαντομηχανικής. Μία τέτοια περιγραφή επιχειρείται στο Παράρτημα\,\ref{apx:kinetic_theory}. Σύμφωνα με την ανάλυση που γίνεται εκεί, καθώς η βαρυτική κατάρρευση προχωρεί, τα ηλεκτρόνια συμπιέζονται συνεχώς, ώστε σταδιακά καταλαμβάνονται όλες οι ενεργειακές καταστάσεις με ενέργεια μικρότερη της $\epsilon_F$. Όταν αυτό συμβεί, όλα τα ηλεκτρόνια αντιδρούν σε οποιαδήποτε συστολή του αστέρα και δημιουργούν μία πρόσθετη προς τα εξώ πίεση, την \textit{πίεση των εκφυλισμένων ηλεκτρονίων}, η οποία είναι δυνατό να αναχαιτίσει την κατάρρευση. Κάτω από συνηθισμένες συνθήκες το κβαντομηχανικό αυτό φαινόμενο είναι αμελητέο, σε υψηλές όμως πυκνότητες γίνεται σημαντικό. Μπορεί να δειχτεί ότι για ένα (μη-σχετικιστικό) εκφυλισμένο αέριο, η πίεση εκφυλισμού θα δίνεται από την καταστατική εξίσωση
\begin{equation}
	P_{\text{deg}} = K \rho^{5/3}
    \label{eq:nr_deg_eos}
\end{equation}όπου $K$ μία σταθερά. Στην περίπτωση του σχετικιστικού εκφυλισμένου αερίου, ο εκθέτης παίρνει την τιμή 4/3.

Θα πρέπει να τονιστεί ιδιαίτερα ότι, σύμφωνα με τη σχέση \eqref{eq:nr_deg_eos}, η πίεση των εκφυλισμένων ηλεκτρονίων δεν είναι θερμική, δηλαδή για τη δημιουργία της δεν απαιτείται μεγάλη θερμική ενέργεια και, επομένως, η πίεση διατηρείται και όταν ο αστέρας ψυχθεί εντελώς. Πρακτικά, λοιπόν, η πίεση των εκφυλισμένων ηλεκτρονίων δεν εξαρτάται από τη θερμοκρασία, και η ύλη θεωρείται ``ψυχρή'' με την έννοια ότι η ανώτερη επιτρεπόμενη θερμοκρασία δεν αρκεί, ώστε να μεταβάλλει αισθητά τις ιδιότητές της (να άρει τον εκφυλισμό).

\subsection{Σχέση μάζας-ακτίνας}
Όπως διαπιστώσαμε και στην προηγούμενη παράγραφο, η πιεση στο εσωτερικό των λευκών νάνων είναι, πρακτικά, ανεξάρτητη από τη θερμοκρασία. Εξαρτάται μόνο από το μέσο μοριακό βάρος ανά ηλεκτρόνιο, $\mu_e = A/Z$ (μέσω της σταθεράς $K$), και την πυκνότητα, $\rho$. Κάτω από αυτές τις συνθήκες μπορεί εύκολα να αποδείχτεί ότι ο όγκος που καταλαμβάνει ένας λευκός νάνος είναι αντιστρόφως ανάλογος προς τη μάζα του.

Πράγματι, συνδυάζοντας την καταστατική εξίσωση του εκφυλισμένου αερίου με την εξίσωση υδροστατικής ισορροπίας πρόκύπτει
\begin{equation*}
	\begin{rcases}
    	P \propto \frac{M}{R^2} \rho \\\\
        P \propto \rho^{5/3}
    \end{rcases}
    \rho^{5/3} \propto \frac{M}{R^2} \rho \Rightarrow \rho^{2/3} \propto \frac{M}{R^2} \Rightarrow \left( \frac{M}{R^3} \right)^{2/3} \propto \frac{M}{R^2} \Rightarrow R \propto M^{-1/3}
\end{equation*}Χρησιμοποιώντας παρατηρησιακά δεδομένα, καταλήγουμε στην εμπειρική σχέση
\begin{equation}
	R_{\text{WD}} = 0.01\,R_\odot \left( \frac{M_{\text{WD}}}{0.7\,M_\odot} \right)^{-1/3} 
    \label{eq:wd_mass_radius}
\end{equation}
Η σχέση \eqref{eq:wd_mass_radius} μας ξαφνιάζει, γιατί αποδεικνύει ακριβώς το αντίθετο από τις συνηθισμένες μας εμπειρίες, όπου γνωρίζουμε ότι αύξηση της μάζας ενός σώματος συνεπάγεται (για σταθερή πυκνότητα) αύξηση και των γραμμικών του διαστάσεων.
$$M^{1/3} R = \text{σταθερό} \Rightarrow MR^3 = \text{σταθερό} \Rightarrow MV = \text{σταθερό}$$
Είναι φανερό λοιπόν ότι η (μέση) πυκνότητα των λευκών νάνων γίνεται τόσο μεγαλύτερη, όσο μεγαλύτερη είναι η μάζα τους. Με άλλα λόγια, όσο αυξάνει η μάζα ενός λευκού νάνου τόσο θα πρέπει να μειώνεται ο όγκος του ώστε να διατηρηθεί η υδροστατική ισορροπία.


\subsection{Όριο μάζας Chandrasekhar}
Είδαμε ότι αν η μάζα ενός λευκού νάνου για κάποιον λόγο αυξάνεται (π.χ. προσάυξηση μάζας από συνοδό αστέρα σε κάποιο διπλό σύστημα), τότε θα πρέπει ο λευκός νάνος να γίνεται πιο συμπαγής, να μειώνει δηλαδή την ακτίνα του ώστε να φέρει τα ηλεκτρόνια πιο κοντά μεταξύ τους και να αυξήσει, με αυτόν τον τρόπο, την πίεση των εκφυλισμένων ηλεκτρονίων που παρέχει την κατάλληλη πίεση για να μην καταρρεύσει.  Όσο συμπιέζεται και τα ηλεκτρόνια έρχονται πιο κοντά μεταξύ τους, τότε σύμφωνα με την αρχή της απροσδιοριστίας του Heisenberg αυξάνεται και η ορμή (ταχύτητα) των ηλεκτρονίων. Η ταχύτητα όμως των σωματιδίων δεν μπορεί να υπερβεί την ταχύτητα του φωτός στο κενό. Όσο η ταχύτητα των ηλεκτρονίων γίνεται σχετικιστική, η πίεση των εκφυλισμένων ηλεκτρονίων θα πρέπει να αυξάνει με συνεχώς αργότερο ρυθμό, καθώς αυξάνει η μάζα του λευκού νάνου. Αυτός είναι, άλλωστε, και ο λόγος που εκθέτης στην καταστατική εξίσωση του σχετικιστικού εκφυλισμένου αερίου γίνεται 4/3 από 5/3 που είναι στην περίπτωση του μη-σχετικιστικού εκφυλισμένου αερίου. 
Άρα πρέπει να υπάρχει ένα ανώτατο όριο μάζας που μπορεί να υποστηρίξει ο λευκός νάνος και θα αντιστοιχεί σε ακτίνα τέτοια ώστε τα ηλεκτρόνια να κινούνται με ταχύτητα σχεδόν αυτή της ταχύτητας του φωτός.

Για έναν τυπικό (και μη-περιστρεφόμενο) λευκό νάνο που αποτελείται από άνθρακα, το ανώτατο αυτό όριο, που ονομάζεται \textbf{όριο Chandrasekhar} είναι ίσο με $M_{\text{ch}} = 1.4\,M_\odot$.  Για μάζα μεγαλύτερη από το όριο Chandrasekhar η πίεση των εκφυλισμένων ηλεκτρονίων δεν είναι αρκετή για να υποστηρίξει το βάρος του αντικειμένου και, ο λευκός νάνος είτε θα διαλυθεί (πιθανότατα εκρηγνυόμενος) είτε θα μεταπέσει σε αστέρα νετρονίων ή μελανή οπή (βλέπε παρακάτω).

\subsection{Ψύξη λευκών νάνων}
Γενικά οι λευκοί νάνοι έχουν μεγάλη πυκνότητα, μικρές διαστάσεις, υψηλή επιφανειακή θερμοκρασία και (λόγω των μικρών διαστάσεών τους) μικρή απόλυτη λαμπρότητα. Επομένως, καταλαμβάνουν την κάτω αριστερά γωνία του διαγράμματος Hertzsprung-Russel. Η επιφανειακή αυτή θερμοκρασία διατηρείται σε ένα επιφανειακό στρώμα, το πάχος του οποίου είναι μόλις το 1\,\% της ακτίνας, και όπου η ύλη δεν είναι εκφυλισμένη. Κάτω από το στρώμα αυτό και μέχρι το κέντρο του λευκού νάνου η θερμοκρασία διατηρείται ομοιόμορφη. Αυτό οφείλεται στη μεγάλη μέση ελεύθερη διαδρομή των εκφυλισμένων ηλεκτρονίων\footnote{Η μεγάλη μέση ελεύθερη διαδρομή των ηλεκτρονίων οφείλεται στο ότι δεν μπορουν να αλληλεπιδράσουν. Αλληλεπίδραση θα σήμαινε αλλαγή στην ενέργεια του ηλεκτρονίου αλλά λόγω του ότι όλες οι διαθέσιμες ενεργειακές στάθμες είναι κατειλημένες αυτό είναι αδύνατο.}, η οποία συνεπάγεται μεγάλη θερμική αγωγιμότητα και, συνεπώς, ομοιόμορφη θερμοκρασία στο εσωτερικό του λευκού νάνου.

Η περαιτέρω εξέλιξη των λευκών νάνων δεν είναι δυνατή και αυτοί συνεχώς ψύχονται. Το ερώτημα που προκύπτει τώρα είναι ποιά είναι η πηγή της ακτινοβολούμενης ενέργειας. Αυτή δεν μπορεί να είναι οι πυρηνικές αντιδράσεις στο εσωτερικό του καθώς ο λευκός νάνος είναι αδρανής. Επίσης δεν μπορεί να είναι ούτε η βαρυτική συστολή καθώς βρίσκεται σε υδροστατική ισορροπία\footnote{Περαιτέρω βαρυτική συστολή του λευκού νάνου είναι πηγή ακτινοβολούμενης ενέργειας, κυρίως για λευκούς νάνους μικρής ηλικίας, το εσωτερικό των οποίων δεν έχει φτάσει ακόμα στην κατάσταση του πλήρους εκφυλισμού.}, ούτε η εκπομπή ακτινοβολίας με τη μορφή νετρίνο καθώς αυτή είναι σημαντική μόνο στις αρχικές φάσεις σχηματισμού του λευκού νάνου όπου επικρατούν υψηλές θερμοκρασίες. Τέλος, η ακτινοβολούμενη ενέργεια δεν μπορεί να οφείλεται στην απελευθέρωση της θερμικής ενέργειας των ηλεκτρονίων καθώς αυτά δεν μπορούν να χάσουν ενέργεια λόγω του ότι όλες οι ενεργειακές στάθμες είναι κατειλημένες. Η εκπεμπόμενη ενέργεια λοιπόν γίνεται εις βάρος ποιάς πηγής; Πρέπει να θυμηθούμε ότι ο λευκός νάνος δεν αποτελείται μόνο από ηλεκτρόνια, αλλά και από ιόντα τα οποία δεν είναι καθόλου εκφυλισμένα. Ο εκφυλισμός των ηλεκτρονίων συμβαίνει πολύ νωρίτερα από τον οποιοδήποτε εκφυλισμό των ιόντων λόγω της πολύ μικρότερης μάζας τους. Αυτό σημαίνει ότι το αέριο των ιόντων το οποίο χαρακτηρίζεται από μία συγκεκριμένη θερμοκρασία, περιγράφεται ικανοποιητικά από την καταστατική εξίσωση των τέλειων αερίων και άρα η κύρια πηγή ακτινοβολούμενης ενέργειας είναι η θερμική ενέργεια των ιόντων.
Αποδεικνύεται ότι η εκπεμπόμενη λαμπρότητα του λευκού νάνου εξαρτάται από τη θερμοκρασία στο κέντρο του μέσω της σχέσης
\begin{equation}
	L_{\text{WD}}^{\text{emit}} = C T_c^{7/2}
    \label{eq:wd_cooling_lum_temp}
\end{equation}Εξ' ορισμού, αυτή η λαμπρότητα είναι ο ρυθμός με τον οποίο ελλατώνεται η ολική θερμική ενέργεια των ιόντων ($E_{\text{int}}$) οπότε
\begin{equation}
	L_{\text{WD}} = - \frac{dE_{\text{int}}}{dt}, \hspace{0.25cm} \text{όπου} \hspace{0.15cm} E_{\text{int}} = N_{\text{ions}} \times \frac{3}{2}k_BT_c \simeq \frac{M_{\text{WD}}}{Am_H}\frac{3}{2}k_BT_c
    \label{eq:wd_internal_energy_ions}
\end{equation}όπου $A$, ο ατομικός αριθμός του αερίου των ιόντων. Άρα συνδυάζοντας τις σχέσεις \eqref{eq:wd_cooling_lum_temp}, \eqref{eq:wd_internal_energy_ions} προκύπτει ότι 
\begin{align}
	\nonumber CT_c^{7/2} &= - \frac{dE_{\text{int}}}{dt} \Rightarrow CT_c^{7/2} = - \frac{d}{dt} \left( \frac{M_{\text{WD}}}{Am_H}\frac{3}{2}k_BT_c \right) \Rightarrow \\ \nonumber \\
    &\Rightarrow \boxed{T_c(t) = T_0\left( 1 + \frac{t}{\tau_0} \right)^{-2/5}}
    \label{eq:wd_cooling_time}
\end{align}όπου χρησιμοποιήσαμε τις αρχικές συνθήκες $t = 0 \rightarrow T_c = T_0$, και την αντικατάσταση $\displaystyle \frac{1}{\tau_0} = \frac{5}{3} \frac{A m_H C T_0^{5/2}}{M_{\text{WD}} k_B}$.
Η σχέση \eqref{eq:wd_cooling_time} μας δείχνει πως η θερμοκρασία στο εσωτερικό ενός λευκού νάνου μειώνεται (ψύχεται) με τον χρόνο. Με αντικατάσταση της σχέσης \eqref{eq:wd_cooling_time} στην σχέση \eqref{eq:wd_cooling_lum_temp}, μπορούμε να βρούμε το πως η λαμπρότητα ενός λευκού νάνου ελλατώνεται με τον χρόνο.

Η ενέργεια της θερμικής κίνησης των ιόντων αποτελεί την κύρια πηγή της ακτινοβολούμενης ενέργειας, μόνο εφόσον η θερμοκρασία του εσωτερικού του λευκού νάνου βρίσκεται εντός κάποιων ορίων. Για χαμηλότερες θερμοκρασίες, φαινόμενα στερεάς κατάστασης γίνονται σημαντικά, όπως είναι η \textbf{κρυσταλλοποίηση} του (μεταλλικού) πλέγματος των θετικών ιόντων. Στην περίπτωση αυτή οι \textit{δονήσεις πλέγματος} των ιόντως και όχι οι θερμικές κινήσεις τους αποτελούν την κύρια πηγή της ακτινοβολούμενης ενέργειας και, επομένως, ψύξης του λευκού νάνου. Η χαρακτηριστική θερμοκρασία, στην οποία συμβαίνει η κρυσταλλοποίηση, είναι $\sim 10^7\,\text{K}$ και ονομάζεται \textbf{θερμοκρασία Debye}. Λόγω της κρυσταλλοποίησης του πλέγματος, η ψύξη του λευκού νάνου είναι πολύ ταχύτερη, σε συμφωνία με τα παρατηρησιακά δεδομένα ψύξης αμυδρών λευκών νάνων ($L \leq 10^{-3}\,L_\odot$). 

\subsection{Περίοδος περιστροφής λευκών νάνων}
Ένα από τα εξαιρετικά δύσκολα προβλήματα σχετικά με την παρατήρηση των λευκών νάνων είναι η μέτρηση της περιόδου αξονικής περιστροφής τους. Αποδεικνύεται ότι η ελάχιστη δυνατή περίοδος περιστροφής ενός τυπικού λευκού νάνου είναι $\sim 10\,\text{s}$. Πάρα τη δυσκολία μέτρησης της περιόδου περιστροφής, είναι γνωστό ότι η ένταση της ακτινοβολίας πολλών λευκών νάνων παρουσιάζει ορισμένες μεταβολές με τυπικές περιόδους $\sim 10^2 - 10^3\,\text{s}$. Οι περίδοι των μεταβολών αυτών είναι σταθερές με διακύμανση μικρότερη από $6 \times 10^{-14}\,\text{s/s}$. Λευκοί νάνοι με την ιδιότητα αυτή ονομάστηκαν \textbf{παλλόμενοι} λευκοί νάνοι.

Οι δονήσεις που φαίνεται ότι παρουσιάζουν οι παλλόμενοι λευκοί νάνοι δεν μπορεί να είναι σφαιρικά συμμετρικές ακτινικές δονήσεις (όπως π.χ. στους κλασσικούς Κηφείδες και τους μεταβλητούς τύπου RR Lyrae). Είναι δυνατό να αποδειχτεί ότι η περίοδος των σφαιρικά συμμετρικών ακτινικών δονήσεων των λευκών νάνων είναι της τάξης μεγέθους του χρόνου που απαιτείται, ώστε ένα ηχητικό κύμα να διαπεράσει ολόκληρο τον αστέρα, αλλά ο χρόνος αυτός είναι σημαντικά μικρότερος από $10^2 - 10^3\,\text{s}$. 

Έχει αποδειχτεί ότι οι δονήσεις των λευκών νάνων είναι δυνατό να προκαλούνται από μικρά κυματίδια που διαδίδονται στην επιφάνεια του λευκού νάνου κατά τη φορά περιστροφής του και δεν προκαλούν μεταβολή της ακτίνας και της δυναμικής ενέργειάς του. Η αιτία δημιουργίας των κυματιδίων είναι αστάθειες της \textit{στρωματομένης ατμόσφαιρας} (stratified atmosphere) του λευκού νάνου,  που οφείλονται σε μερικό ιονισμό του He της και συμβαίνουν, εφόσον η επιφανειακή θερμοκρασία του αστέρα βρίσκεται επιλεκτικά στις περιοχές $(1.6 - 1.9) \times 10^4\,\text{K}$ και $(2.6 - 2.9) \times 10^4\,\text{K}$. Σύμφωνα με αυτά τα θεωρητικά αποτελέσματα, ένας αρκετά θερμός λευκός νάνος είναι δυνατόν, κατά τη διάρκεια ψύξης του, να περάσει από το στάδιο του παλλόμενου λευκού νάνου περισσότερες από μία φορές.

% {\color{red} \hrule}
% Λευκός νάνος είναι ο εκφυλισμένος, γυμνός πυρήνας ενός αστέρα με σχετικά μικρή αρχική μάζα ($M \leq 5\,M_\odot$). Αρχικά, ο λευκός νάνος έχει πολύ υψηλή θερμοκρασία. Με την πάροδο του χρόνου, όμως, η θερμοκρασία του συνεχώς μειώνεται, μέχρις ότου πάψει να ακτινοβολεί θερμικά. Στην περίπτωση αυτή η υδροστατική ισορροπία του αστέρα εξασφαλίζεται από την (κβαντομηχανικής και όχι θερμικής προέλευσης) πίεση των ηλεκτρονίων. \\
% {\color{red} \hrule}




\section{Αστέρες νετρονίων}
Οι αστέρες νετρονίων αποτελούνται από την πλέον συμπαγή μορφή ύλης η οποία είναι δυνατό να περιγραφεί με γνωστούς νόμους της Φυσικής. Οι αστέρες αυτοί, όπως και οι λευκοί νάνοι, δεν εξελίσσονται πλέον, δεδομένου ότι ούτε και σ' αυτούς συμβαίνουν θερμοπυρηνικές αντιδράσεις. Στην τελική αυτή κατάσταση είναι δυνατόν να καταλήξουν αστέρες των οποίων η αρχική μάζα (κατά την εγκατάσταση στην κύρια ακολουθία) υπερβαίνει τις $5 M_\odot$ αλλά όχι τις $20 M_\odot$. Η μέση τιμή της μάζας των αστέρων αυτών είναι $1 - 2 M_\odot$, η μέση πυκνότητά τους είναι $\sim 10^{14}\,\text{gr cm}^{-3}$ και η μέση θερμοκρασία του εσωτερικού τους είναι $\sim 10^7\,\text{K}$. Η πυκνότητα αυτή των αστέρων νετρονίων είναι συγκρίσιμη με την πυρηνική πυκνότητα που υπάρχει σε έναν ατομικό πυρήνα. Γι' αυτό οι αστέρες νετρονίων, θεωρούνται, πολλές φορές, ως ενιαίοι ατομικοί πυρήνες, στους οποίους οι ισχυρές πυρηνικές δυνάμεις (των μεσονίων) έχουν αντικατασταθεί από τη δύναμη της βαρύτητας.

\subsection{Σχηματισμός αστέρων νετρονίων}
Κατά τα τελευταία στάδια της εξέλιξης ενός αστέρα μεγάλης μάζας δημιουργείται στο κέντρο του ένας εκφυλισμένος πυρήνας που αποτελείται κυρίως από νικέλιο και σίδηρο. Ο αστέρας τότε λέμε ότι βρίσκεται στο όριο ικανότητας ισορροπίας, διότι οι πυρήνες των δύο αυτών στοιχείων είναι οι πιο ευσταθείς από όλους, δηλαδή έχουν τη μεγαλύτερη ενέργεια σύνδεσης ανα νουκλεόνιο, και, επομένως, στον πυρήνα δεν είναι δυνατές περαιτέρω αντιδράσεις παραγωγής ενέργειας. Ο κεντρικός πυρήνας θα βρίσκεται σε ισορροπία, εφόσον η μάζα του είναι αρκετά μικρή. Λόγω όμως της συνεχούς αύξησης της μάζας του (λόγω της καύσης που πραγματοποιούνται στους φλοιούς που περικλύουν τον εν λόγω πυρήνα), ο πυρήνας γίνεται βαρυτικά ασταθής και αρχίζει να καταρρέει όταν η μάζα του ξεπεράσει το όριο Chandrasekhar. Κατά τη διάρκεια της κατάρρευσης του πυρήνα καταστρέφεται και η μηχανική ισορροπία του υπόλοιπου αστέρα, διότι το βάρος των υπερκείμενων στρωμάτων δεν αντισταθμίζεται από την πίεση του αερίου. Κατά συνέπεια, τα εξωτερικά στρώματα πέφτουν προς το κέντρο και σύντομα η κινητική τους ενέργεια μετατρέπεται σε θερμική με αποτέλεσμα την αύξηση της θερμοκρασία σε μικρό χρονικό διάστημα και την εκτόξευση του μανδύα σε μία έκρηξη υπερκαινοφανούς τύπου ΙΙ (δες και Κεφάλαιο \ref{ch:Chapter7}).

Η κατάρρευση ενός πυρήνα σιδήρου δεν είναι ο μοναδικός τρόπος σχηματισμού ενός αστέρα νετρονίων. Ένας άλλος πολύ δημοφιλής μηχανισμός βασίζεται στην συσσώρευση μάζας στην επιφάνεια ενός λευκού νάνου -μέρος ενός διπλού συστήματος- λόγω προσαύξησης (accretion) από τον συνοδό αστέρα (δες και Κεφάλαιο \ref{ch:Chapter7}). Λόγω της σχέσης μάζας-ακτίνας που βγάλαμε για τους λευκούς νάνους, η αύξηση της μάζας συνοδεύεται από συστολή του λευκού νάνου στην προσπάθειά του να επανέλθει σε υδροστατική ισορροπία. Όταν η μάζα του λευκού νάνου ξεπεράσει το όριο Chandrasekhar, το αέριο των ηλεκτρονίων καταρρέει κάτω από την τεράστια δύναμη πίεσης που αναπτύσσεται κατά τη συστολή του αστέρα. Τα ηλεκτρόνια ``εισχωρούν'' στους ατομικούς πυρήνες, εξουδετερώνοντας τα πρωτόνια και δημιουργούν ένα αέριο \textit{εκφυλισμένων νετρονίων} (αφού και τα νετρόνια είναι φερμιόνια και, άρα, υπακούουν στην αρχή του Pauli). Η πίεση των εκφυλισμένων νετρονίων μπορεί να υπολογιστεί ακολουθώντας παρόμοια συλλογιστική με αυτή που χρησιμοποιήθηκε για να υπολογίσουμε την πίεση των εκφυλισμένων ηλεκτρονίων. Η πολυτροπική καταστατική εξίσωση που περιγράφει το εκφυλισμένο (μη-σχετικιστικό) αέριο νετρονίων έχει την ίδια μορφή με την σχέση \eqref{eq:nr_deg_eos}.

\subsection{Σχέση μάζας-ακτίνας}
Για τους αστέρες νετρονίων ισχύει μία σχέση μεταξύ μάζας και ακτίνας που, όπως και στην περίπτωση των λευκών νάνων, υιοθετεί την μορφή 
\begin{equation*}
	R \propto M^{-1/3}
\end{equation*}
Η αντίστοιχη εμπειρική σχέση μάζας-ακτίνας για τους αστέρες νετρονίων είναι
\begin{equation}
	R_{\text{NS}} = 11\,\text{km} \left( \frac{M_{\text{NS}}}{1.4\,M_\odot} \right)^{-1/3}
\end{equation}
Αξίζει να παρατηρήσουμε ότι στην παραπάνω σχέση η ακτίνα μετριέται σε km ενώ στην αντίστοιχη εμπειρική σχέση για τους λευκούς νάνουν, η ακτίνα μετριέται σε όρους $R_\odot$. Αυτό υποδεικνύει και το πόσο πιο συμπαγή αντικείμενα είναι οι αστέρες νετρονίων συγκριτικά με τους λευκούς νάνους.

\subsection{Όριο μάζας Tolman-Oppenheimer-Volkoff}
Όσο η μάζα ενός αστέρα νετρονίων αυξάνει, τόσο τα νετρόνια που τον αποτελούν γίνονται όλο και πιο σχετικιστικά, με αποτέλεσμα να μειώνεται (όπως και στην περίπτωση των λευκών νάνων) η τιμή της αδιαβατικής σταθερής γ (εκθέτης στην καταστατική εξίσωση), από 5/3 προς το όριο 4/3. Επομένως θα υπάρχει και για τη μάζα των αστέρων νετρονίων έναν ανώτατο όριο, $M_{\text{TOV}}$, ανάλογο με το όριο Chandrasekhar, $M_{\text{Ch}}$, των λευκών νάνων. Η ακριβής τιμή του ορίου αυτού δεν είναι σήμερα γνωστή, λόγω της αβεβαιότητας στη μορφή των καταστατικής εξίσωσης των σχετικιστικών νετρονίων και του μέσου μοριακού βάρους της ύλης των αστέρων νετρονίων. Οι τιμές του ορίου $M_{\text{TOV}}$ που έχουν προταθεί βρίσκονται στο διάστημα $0.7\,M_\odot < M_{\text{TOV}} < 3.2\,M_\odot$.

\subsection{Ψύξη αστέρων νετρονίων}
Όπως και στην περίπτωση των λευκών νάνων, η ψύξη ενός αστέρα νετρονίων δεν μπορεί να οφείλεται στην απελευθέρωση της θερμικής ενέργειας των νετρονίων καθώς αυτά βρίσκονται σε συνθήκες πλήρους εκφυλισμού και άρα είναι αδύνατον να χάσουν ενέργεια και να μεταπέσουν σε χαμηλότερη ενεργειακή στάθμη. Επίσης, ο αστέρας νετρονίων αποτελείται σχεδόν εξ' ολοκλήρου από νετρόνια και άρα δεν υπάρχει κάποιο αέριο ιόντων --όπως στην περίπτωση των λευκών νάνων-- που να απελευθέρωνει θερμική ενέργεια.

Αποδεικνύεται ότι το εσωτερικό των αστέρων νετρονίων ψύχεται λόγω ακτινοβολίας νετρίνο τα οποία παράγονται από τις ασθενείς αντιδράσεις
\begin{eqnarray*}
	n \longrightarrow p + e^{-} + \bar{\nu}_e \\ \\
    p + e^{-} \longrightarrow n + \nu_e
\end{eqnarray*}
και τα οποία διαφεύγουν στο διάστημα.

Παρόλα αυτά, οι αστέρες νετρονίων δεν αποτελούνται εξολοκλήρου από νετρόνια: στην επιφάνειά τους υπάρχει ένα λεπτό στρώμα πλάσματος, πάχους λίγων εκατοστών, που αποτελεί την ``ατμόσφαιρα'' του αστέρα νετρονίων (crust). Αυτή η κρούστα πλάσματος στην επιφάνεια του αστέρα νετρονίων έχει θερμοκρασία $T_{\text{eff}}\sim 10^6 - 10^7\,\text{K}$ και άρα η επιφάνεια του αστέρα νετρονίων ψύχεται μέσω εκπομπής φωτονίων.

Το πάχος της ατμόσφαιρας αυτής είναι τόσο μικρό, επειδή η επιτάχυνση της βαρύτητας είναι πολύ μεγάλη. Για το λόγο αυτό, άλλωστε, και η ταχύτητα διαφυγής, $v_{\text{esc}}$, από έναν αστέρα νετρονίων είναι εντυπωσιακά μεγάλη. Από το ολοκλήρωμα της ενέργειας
\begin{equation*}
	\frac{1}{2} m v_{\text{esc}}^2 + \left( - G \frac{M m}{R} \right) = 0
\end{equation*}
βρίσκουμε ότι η ταχύτητα διαφυγής δίνεται από τη σχέση
\begin{equation}
	v_{\text{esc}} = \sqrt{\frac{2GM}{R}}
    \label{eq:escape_velocity}
\end{equation}
Για τιμές μάζας και ακτίνας ενός τυπικού αστέρα νετρονίων βρίσκουμε ότι η ταχύτητα διαφυγής είναι $50\,\% - 70\,\%$ της ταχύτητας του φωτός! Αυτό έχει για συνέπεια ότι η ύλη από την επιφάνεια ενός αστέρα νετρονίων διαφεύγει πολύ δύσκολα στο διάστημα.
 
 \newpage
\subsection{Pulsars}
Ο όρος pulsar προέρχεται από τις λέξεις ``pulsating star'' καθώς αυτό το αντικείμενο χαρακτηρίζεται από την περιοδική επανάληψη ακτινοβολίας ραδιοφωνικών κυμάτων. Η περίοδος της επανάληψης έχει ένα μεγάλο εύρος από $\sim 1.5\,\text{msec}$ μέχρι $\sim 10\,\text{sec}$, ενώ σύντομα μετά την παρατήρηση του πρώτου pulsar αποδείχτηκε ότι πρόκεται για αστέρες νετρονίων.

Η καταπληκτική ακρίβεια της περιόδου επανάληψης των σημάτων των pulsar δεν αφήνει πολλά περιθώρια επιλογής για την ενδεχόμενη φύση της πηγής. Η περίπτωση να συνδέεται η εκπομπή των ραδιοφωνικών κυμάτων με την περιφορά των μελών ενός διπλού συστήματος απορρίφθηκε σχεδόν ασυζητητί για τους εξής δύο λόγους:
\begin{itemize}
    \item Η περίοδος περιφοράς ενός διπλού συστήματος αποτελούμενου από δύο λευκούς νάνους δεν είναι δυνατόν να είναι μικρότερη από $1.7\,\text{sec}$ (γιατί;). Όπως όμως έχουμε ήδη αναφέρει, έχουν παρατηρηθεί pulsars με πολύ μικρότερη περίοδο.
    
    \item Η περίοδος περιφοράς ενός διπλού συστήματος αποτελούμενου από δύο αστέρες νετρονίων ελαττώνεται λόγω εκπομπής βαρυτικών κυμάτων, η ενέργεια της οποίας αντλείται από την ελάττωση του μεγάλου ημιάξονα της σχετικής τροχιάς του συστήματος (τρίτος νόμος του Kepler). Αν λοιπόν η περιοδικότητα της ακτινοβολίας είχε σχέση με την περιφορά των μελών του συστήματος δύο αστέρων νετρονίων, τότε θα έπρεπε να παρατηρείται ελάττωση της περιόδου επανάληψης των παλμών, πράγμα που όμως δεν συμβαίνει. Στην πραγματικότητα μάλιστα συμβαίνει ακριβώς το αντίθετο. Η περίοδος των παλμών αυξάνει, και μάλιστα για μερικούς pulsars πολύ σημαντικά (π.χ. PSR 0531+21, που βρίσκεται στο νεφέλωμα του Καρκίνου)
\end{itemize}
Σύντομα αποκλείστηκε η περίπτωση \textit{αναπάλσεων} λευκών νάνων. Το πρότυπο αυτό βασιζόταν σε κύματα πυκνότητας (ακουστικά κύματα) που δημιουργούν ένα είδος αστρικού σεισμού (star quakes). Σε αυτή την περίπτωση η περίοδος των αναπάλσεων θα ήταν απλά:
$$P = \frac{2R}{c_s}$$
όπου $R$ η ακτίνα του αντικειμένου και $c_s$ η ταχύτητα με την οποία διαδίδονται τα ακουστικά κύματα (ταχύτητα του ήχου). Επειδή η ταχύτητα του ήχου εξαρτάται άμεσα από την πυκνότητα $(c_s \propto {\langle \rho \rangle}^{1/2})$  μπορούμε να δείξουμε ότι η μέση πυκνότητα αυτού του αντικειμένου είναι πολλές τάξεις μεγέθους μεγαλύτερη από αυτή ενός τυπικού λευκού νάνου. Επιπλέον, δεδομένου ότι οι περίοδοι των pulsars καλύπτουν πάνω από τρεις τάξεις μεγέθους, αυτό συνεπάγεται πολύ ευρεία κατανομή πυκνοτήτων ($\sim$ 6 τάξεις μεγέθους), που δεν είναι δυνατό να δικαιολογηθεί από μία, μόνο, κατηγορία αστέρων.

Τέλος  αποκλείστηκε και η περίπτωση περιστροφής λευκών νάνων, πάλι λόγω των μικρών τιμών που είχε η περίοδος ορισμένων pulsars. Πράγματι, η γραμμική ταχύτητα, $v$, της ύλης στον ισημερινό ενός αστέρα ακτίνας $R$ που περιστρέφεται με περίοδο $P$ δίνεται από την σχέση:
$$v = \frac{2\pi R}{P}$$
Αν οι pulsars ήταν όντως λευκοί νάνοι, τότε μπορούμε να αντικαταστήσουμε την ακτίνα με την εμπειρική σχέση ακτίνας-μάζας που γνωρίζουμε ότι ισχύει για τους λευκούς νάνους ώστε:
$$v = \frac{2\pi}{P} 0.01 R_\odot \left( \frac{M_{\text{WD}}}{0.7 M_\odot} \right)^{-1/3}$$
Για $P\sim 0.1\,\text{sec}$ που είναι μία ενδιάμεση τιμή περιόδου που παρατηρούμε για τους pulsars, η παραπάνω σχέση μας δίνει γραμμική ταχύτητα $v \simeq 1.5 c$. 

Έτσι απέμεινε η περίπτωση να οφείλονται τα περιοδικά σήματα των pulsars σε ταχύτητα \textit{περιστρεφόμενους αστέρες νετρονίων}, η ακτίνα των οποίων είναι σημαντικά μικρότερη από την ακτίνα των λευκών νάνων. Στους αστέρες νετρονίων η ακτίνα αυτή ορίζει έναν κύλινδρο, ο οποίος ονομάζεται \textit{κύλινδρος φωτός} (δες σχήμα \ref{fig:pulsar}).


\begin{figure}
	\centering
	\includegraphics[scale=0.25]{Figures/pulsar.jpeg}
    \caption{Εκπομπή ακτινοβολίας από pulsar σύμφωνα με το πρότυπο εκπομπής από τους μαγνητικούς πόλους ενός αστέρα νετρονίων. Ο άξονας του μαγνητικού πεδίου δεν συμπίπτει με τον άξονα περιστροφής. Ένας παρατηρήτης βλέπει την ακτινοβολία από τον έναν πόλο μόνο.}
    \label{fig:pulsar}
\end{figure}

Το πιο διαδεδομένο πρότυπο για pulsars δέχεται ότι η περιοδική ακτινοβολία που παρατηρούμε έχει τη μορφή κωνικής δέσμης και προέρχεται από την περιοχή των μαγνητικών πόλων (polar cap model) ενός ταχέως περιστρεφόμενου αστέρα νετρονίων με ισχυρό \textit{διπολικό μαγνητικό πεδίο}. Η εξαιρετικά μικρή περίοδος περιστροφής των αστέρων νετρονίων και το ισχυρό μαγνητικό πεδίο τους οφείλονται, αντίστοιχα, στην διατήρηση της στροφορμής και της μαγνητικής ροής στην επιφάνεια του αρχικού αστέρα. Το εύρος της κωνικής δέσμης της ακτινοβολίας καθορίζεται από τη γωνιώδη απόσταση του τελευταίου \textit{ανοικτού σωλήνα μαγνητικής ροής} από το μαγνητικό άξονα του αστέρα. Επειδή ο μαγνητικός άξονας των pulsars δεν συμπίπτει συνήθως με τον άξονα περιστροφής τους, η κωνική δέσμη ακτινοβολίας του κάθε πόλου του pulsar σαρώνει έναν κοίλο κώνο με κορυφή τον pulsar. Αν η Γη τυχαίνει να βρίσκεται στο εσωτερικό του κοίλου κώνου, τότε σε κάθε περίοδο περιστροφής του αστέρα παρατηρούμε έναν παλμό ακτινοβολίας η διάρκεια του οποίου είναι ανάλογη προς το εύρος της κωνικής δέσμης. Η γεωμετρία αυτή θυμίζει το μηχανισμό της περιοδικής ακτινοβολίας ενός φάρου. 

Ο βασικός μηχανισμός ακτινοβολίας του παραπάνω προτύπου είναι ο εξής: το περιστρεφόμενο μαγνητικό πεδίο του αστέρα παράγει (εξ επαγωγής) διαφορά δυναμικού, όπως ακριβώς και οι γεννήτριες ηλεκτρικού ρεύματος, η οποία εμφανίζεται μεταξύ των πόλων και του ισημερινού του αστέρα. Λόγω αυτής της διαφοράς δυναμικού, φορτισμένα σωματίδια αποσπώνται από την επιφάνεια του αστέρα και δημιουργούν έναν τεράστιο ``πυκνωτή'' στην περιοχή του κάθε πόλου, οι οπλισμοί του οποίου αποτελούνται από δύο ετερόσημα στρώματα φορτίων: ένα στην επιφάνεια του pulsar και ένα στην περιοχή πάνω από αυτήν. Όταν η τάση μεταξύ των οπλισμών του κάθε ``πυκνωτή'' γίνει $10^{12} - 10^{13}\,\text{V}$, τότε επέρχεται εκφόρτισή τους. Η ενέργεια που παράγεται είναι πολύ μεγάλη (πολλές τάξεις μεγέθους μεγαλύτερη από $511\,\text{keV}$, που είναι η ισοδύναμη ενέργεια της μάζας ηρεμίας ενός ηλεκτρονίου) με αποτέλεσμα να συμβαίνει \textit{δίδυμη γέννηση} σωματιδίων (e$^{-}$- e$^{+}$). Τα σωματίδια αυτά κινούμενα ελικοειδώς γύρω από τις μαγνητικές δυναμικές γραμμές του πεδίου παράγουν ακτινοβολία \textit{σύγχροτρον} στην περιοχή των ραδιοφωνικών κυμάτων.

Το παραπάνω πρότυπο εξηγεί ικανοποιητικά τρία βασικά παρατηρησιακά δεδομένα των pulsars:
\begin{enumerate}
    \item Η δέσμη ακτινοβολίας είναι πολύ στενή, όπως προκύπτει από τους ``στενούς'' περιοδικούς παλμούς που παρατηρούμε. Η διάρκεια των μεμονομένων παλμών είναι της τάξης του 1/100 της περιόδου επανάληψης του παλμού (που, σύμφωνα με το πρότυπο, ισούται με την περίοδο περιστροφής του αστέρα).
    \item Το φάσμα της ακτινοβολίας των παλμών δεν μοιάζει με φάσμα μελανού σώματος, αλλά είναι φάσμα ακτινοβολίας σύγχροτρον.
    \item Οι παλμοί είναι πολύ ισχυρά γραμμικά πολωμένοι, γεγονός που απαιτεί ισχυρό μαγνητικό πεδίο.
\end{enumerate}

Υπάρχουν μερικοί pulsars που εκπέμπουν κυρίως στην περιοχή των ακτίνων-X. Οι pulsars αυτοί είναι μέλη ημιαποχωρισμένων διπλών συστημάτων (Κεφάλαιο \ref{ch:Chapter7}). Η ακτινοβολία τους, η οποία έχει φάσμα μελανού σώματος, εκπέμπεται από ύλη η οποία ``βομβαρδίζει'' την επιφάνεια του αστέρα και θερμαίνεται, πέφτοντας επάνω της με μεγάλη ταχύτητα (λόγω του ισχυρού βαρυτικού πεδίου του pulsar). Η ύλη αυτή προέρχεται από το συνοδό αστέρα, ο οποίος είναι συνήθως ένας εξελιγμένος (γίγαντας) αστέρας που έχει γεμίσει το λοβό Roche (δες Κεφάλαιο \ref{ch:Chapter7}).

Σήμερα πιστεύουμε ότι στο Γαλαξία υπάρχουν περίπου 50000 pulsars, ότι η δημιουργία τους συμβαίνει συνήθως κατά την έκρηξη ενός υπερκαινοφανούς αστέρα με ρυθμό περίπου ίσο με 1 pulsar ανά 20 έτη και ότι ο μέσος όρος ζωής τους είναι $10^7$ έτη. Ένας αστέρας νετρονίων παύει να παρατηρείται ως pulsar, είτε όταν το μαγνητικό του πεδίο εξασθενήσει σημαντικά, είτε όταν ο άξονας του (διπολικού) πεδίου γίνει παράλληλος προς τον άξονα περιστροφής. Και στις δύο περιπτώσεις το μέχρι σήμερα δεκτό πρότυπο ακτινοβολίας των pulsars (πρότυπο μαγνητικών πόλων) δεν προβλέπει περιοδική εκπομπή ακτινοβολιας.




% {\color{red} \hrule}
% Αστέρας νετρονίων είναι ο εκφυλισμένος, γυμνός πυρήνας ενός αστέρα με σχετικά μεγάλη αρχική μάζα. Η εξέλιξη της θερμοκρασίας του ακολουθεί, όπως πιστεύουμε σήμερα, την πορεία της θερμοκρασίας των λευκών νάνων. Σ' αυτήν την περίπτωση η υδροστατική ισορροπία εξασφαλίζεται από την κβαντομηχανικής φύσεως πίεσης των νετρονίων. Δεν αποκλείεται ενάς τέτοιος αστέρας να παρουσιαστεί ενεργά στον ουρανό υπό την μορφή ενός pulsar, που γίνεται ορατός με παρατηρήσεις σε ραδιαφωνικά, κυρίως, μήκη κύματος.\\
% {\color{red} \hrule}





\section{Μελανές οπές}
Η ύπαρξη ενός ανώτατου ορίου της μάζας ενός ευσταθή αστέρα νετρονίων δημιουργεί ένα δίλημμα. Όπως έχουμε ήδη αναφέρει, είναι γνωστό ότι υπάρχουν και αστέρες πολύ μεγάλης μάζας, που φυσικά είναι και οι ταχύτερα εξελισσόμενοι αστέρες. Προκύπτει, συνεπώς, το ερώτημα, τι θα συμβεί, αν α) ο αστέρας χάσει μεν κατά κάποιον τρόπο το μεγαλύτερο μέρος της μάζας του, αλλά η μάζα του παραμένοντος πυρήνα είναι μεγαλύτερη απο $2 - 3\,M_\odot$,  ή β) επιπρόσθετη μάζα πέσει πάνω σε έναν αστέρα νετρονίων, ώστε η μάζα του να υπερβεί την οριακή μάζα των $\sim 3\,M_\odot$. Στην περίπτωση αυτή δεν υπάρχουν γνωστές φυσικές δυνάμεις, ικανές να αναχαιτίσουν την ολοκληρωτική βαρυτική κατάρρευση του αστέρα. Συνεπώς, οι διαστάσεις του αστέρα συνεχώς σμικρύνονται και η ένταση του βαρυτικού πεδίου του αστέρα αυξάνει σε υπερβολικό βαθμό, οπότε, φυσικά, η χρήση της Γενικής Θεωρίας της Σχετικότητας είναι αναγκαστική. Η καμπύλωση του τετραδιάστατου χωροχρόνου του αστέρα δεν επιτρέπει ούτε καν φως να διαφύγει από το ισχυρότατο βαρυτικό πεδίο του και ο αστέρας φαίνεται να εξαφανίζεται από το σύμπαν. Το αποτέλεσμα αυτό της ολοκληρωτικής βαρυτικής κατάρρευσης έχει ονομαστεί \textit{μελανή οπή}.

Η δημιουργία μιας μελανής οπής ως αποτέλεσμα της ολοκληρωτικής βαρυτικής κατάρρευσης ενός αστέρα αποτελεί μια από τις πιο εντυπωσιακές προβλέψεις της Γενικής Θεωρίας της Σχετικότητας και της σύγχρονης θεωρίας της αστρικής εξέλιξης. Η ακριβής σχετικιστική περιγραφή του φαινομένου της βαρυτικής κατάρρευσης και των μαθηματικών ιδιοτήτων των μελανών οπών, φυσικά, δεν ανήκει στους αντικειμενικούς σκοπούς αυτών των σημειώσεων. Σε γενικές γραμμές, όμως, μπορούμε να πούμε, στην περίπτωση ενός σφαιρικά συμμετρικά αστέρα, οτι, αν η λόγω της βαρυτικής κατάρρευσης συνεχώς ελλατούμενη ακτίνα $R(R > R_s)$ του αστέρα πάρει την οριακή τιμή
$$R = R_s$$
ο αστέρας θα γίνει μη-ορατός για κάθε παρατηρητή. Τότε η εξωτερική (πραγματική) επιφάνεια του αστέρα γίνεται μια \textit{παγιδευμένη επιφάνεια} για υλικά σωματίδια και ακτίνες φωτός. Η επιφάνεια
\begin{equation}
	r = R_s = \frac{2GM}{c^2}
    \label{eq:schwarzschild_radius}
\end{equation}
ονομάζεται \textit{ορίζοντας γεγονότων} και ορίζεται από την ακτινική απόσταση που δίνεται από τη σχέση \eqref{eq:schwarzschild_radius} και που έχει ονομαστεί \textit{ακτίνα Schwarzschild}. Η μορφή της σχέσης αυτής προέκυψε από την σχέση \eqref{eq:escape_velocity} για $v_{\text{esc}} = c$, η απόδειξη όμως αυτή δεν είναι σωστή, επειδή οι σχέσεις της κλασικής φυσικής δεν ισχύουν στο εν λόγω όριο ταχύτητας.

Η σωστή απόδειξη δόθηκε από το Γερμανό αστροφυσικό Karl Schwarzschild το 1917, με βάση τη Γενική Θεωρία της Σχετικότητας. Ο Schwarzschild έδειξε ότι το μήκος κύματος ενός φωτονίου αυξάνεται, όταν αυτό απομακρύνεται από μία πηγή βαρυτικού πεδίου. Αν $\lambda_0$ είναι το μήκος κύματος σε απόσταση $r_0 > R_s$ από ένα ομογενές, σφαιρικό, μη-περιστρεφόμενο και ηλεκτρικά ουδέτερο σώμα με μάζα Μ και $\lambda$ είναι το μήκος κύματος σε άπειρη απόσταση από το σώμα, τότε ισχύει η σχέση
\begin{equation}
	\frac{\lambda}{\lambda_0} = \left( 1 - \frac{2GM}{r_0 c^2} \right)^{-1/2}
    \label{eq:black_hole_wavelength}
\end{equation}
Από τη σχέση \eqref{eq:black_hole_wavelength} βλέπουμε ότι, όταν το $r_0$ γίνει ίσο με $R_s$, τότε το $\lambda$ τείνει προς το $\infty$, η ενέργεια των φωτονίων $(E = h\nu = hc/\lambda)$ τείνει προς το μηδέν και επομένως αυτά παύουν να είναι αντιληπτά. Για το λόγο αυτό, αν ένα φωτόνιο περιέχεται μέσα σε σφαίρα ακτίνας $r \leq R_s$, τότε η τροχιά των φωτονίων στη θέση αυτή είναι κλειστή, και το φωτόνιο θα παραμείνει δέσμιο του βαρυτικού πεδίου. Η περιοχή $r < R_s$, λοιπόν, που περικλύεται από τον ορίζοντα γεγονότων είναι απαγορευμένη για έναν εξωτερικό παρατηρητή, με την έννοια ότι η έξοδος από αυτήν είναι αδύνατη (αν και η είσοδος σε αυτήν είναι δυνατή).

Ο ορίζοντας γεγονότων δεν είναι μία υλική επιφάνεια της μελανής οπής. Εντούτοις ένας εξωτερικός παρατηρητής που βρίσκεται σε απόσταση $r > R_s$ δεν μπορεί να πάρει άλλες πληροφορίες για την εσωτερική περιοχή $(r < R_s)$ παρά μόνο για την συνολική μάζα $(M)$, το ηλεκτρικό φορτίο $(Q)$ και την ολική στροφορμή $(J)$ που η περιοχή αυτή περιέχει. Αυτό συμβαίνει, επειδή οι παραπάνω ποσότητες σχετίζονται με το βαρυτικό και το ηλεκτρικό πεδίο, τα οποία είναι τα μοναδικά, γνωστά στη σημερινή Φυσική, πεδία, που συνδέονται με δυνάμεις μεγάλης ακτίνας δράσης (long range forces). 

Είναι φανερό ότι αν $r_0 \gg 2GM/c^2$, μπορούμε να αναπτύξουμε κατά Taylor το δεξιό μέλος της σχέσης \eqref{eq:black_hole_wavelength}, οπότε έχουμε
\begin{equation}
	\frac{\lambda_0}{\lambda} = 1 - \frac{GM}{r_o c^2}
\end{equation}
Αν πολλαπλασιάσουμε και τα δύο μέλη της εξίσωσης επί $E = hc/\lambda_0$ και θέοσυμε $m=E/c^2$, τότε βρίσκουμε ότι
\begin{equation*}
	h\nu = h\nu_0 - GMm/r_0
\end{equation*}
Η παραπάνω σχέση θυμίζει το ολοκλήρωμα της ενέργειας στο πρόβλημα των δύο σωμάτων
\begin{equation*}
	E_{\infty} = E_0 - GMm/r_0
\end{equation*}
Είναι φανερό ότι η ποσότητα $GMm/r_0$ αντιστοιχεί στην ενέργεια που καταναλώνει ένα φωτόνιο που βρίσκεται σε απόσταση $r_0$ από μία μάζα $M$ για να διαφύγει, σε άπειρη απόσταση, από την επίδραση της μάζας αυτής.

% {\color{red} \hrule}
% Στην περίπτωση των μελανών οπών, η υδροστατική ισορροπία του αστέρα έχει καταστραφεί, επειδή η διαθέσιμη πίεση (θερμικής ή κβαντομηχανικής προέλευσης) δεν είναι ικανή να αντισταθμίσει την βαρυτική. Η μάζα του αστέρα έχει καταρρεύσει, δημιουργώντας ένα αντικείμενο εξαιρετικά μεγάλης πυκνότητας. Σε κάθε μελανή οπή μπορούμε να αντιστοιχήσουμε ένα χαρακτηριστικό μήκος, $R_S$, που ονομάζεται ακτίνα Schwarzschild, με τη σχέση $R_S = 2GM/c^2$. Το βαρυτικό πεδίο μιας μελανής οπής είναι τόσο ισχυρό, ώστε σε απόσταση μικρότερη από την ακτίνα Schwarzschild ακόμα και το φως δεν μπορεί να διαφύγει από την βαρυτική έλξη.\\
% {\color{red} \hrule}


%     \chapter{Αστρικά κατάλοιπα}
\label{ch:Chapter6}

Όταν σταματήσουν οι θερμοπυρηνικές αντιδράσεις στο εσωτερικό των άστρων, τότε ο πυρήνας αρχίζει να ψύχεται, επειδή δεν αναπληρώνονται τα ποσά της ενέργειας που ρέουν προς τα εξωτερικά στρώματα του αστέρα. Η ψύξη του πυρήνα, όμως, συνεπάγεται πτώση της θερμικής πίεσης στο εσωτερικό του, οπότε η πίεση των υπερκείμενων στρωμάτων αρχίζει να υπερισχύει της θερμικής πίεσης του αερίου, με αποτέλεσμα ο πυρήνας να αρχίσει να συστέλλεται. Αν η μάζα του είναι μικρή ($M < 1\,M_\odot$), η συστολή του δεν συνοδεύεται, συνήθως, από καταστροφικά φαινόμενα. Αντίθετα, η ύλη αστέρων μεγάλης μάζας υφίσταται καταστροφική "κατάρρευση", η οποία συνήθως ακολουθείται από έκρηξη, και η ισορροπία των δυνάμεων που διέπουν την ύπαρξη της τελικής κατάστασης, στην οποία θα περιπέσουν αυτοί οι αστέρες, είναι πολύ λεπτή.

Με τις σημερινές γνώσεις της Φυσικής πιστεύουμε ότι είναι δυνατόν να υπάρξουν τριών ειδών τελικές καταστάσεις, όταν σταματήσει οριστικά η παραγωγή ενέργειας από θερμοπυρηνικές αντιδράσεις, στις οποίες γενικά αναφερόμαστε ως \textbf{συμπαγείς αστέρας} (compact stars) επειδή έχουν μικρές τυπικές διαστάσεις και μεγάλες πυκνότητες. Μία τέταρτη περίπτωση κατά την οποία ο αστέρας διαλύεται, με την ύλη να διασκορπίζεται στο μεσοαστρικώ χώρο χωρίς να αφήνει πίσω κάποιο κατάλοιπο, θα ζηζητηθεί στο Κεφάλαιο \ref{ch:Chapter7}.


\section{Λευκοί νάνοι}
{\color{red} \hrule}
Λευκός νάνος είναι ο εκφυλισμένος, γυμνός πυρήνας ενός αστέρα με σχετικά μικρή αρχική μάζα ($M \leq 5\,M_\odot$). Αρχικά, ο λευκός νάνος έχει πολύ υψηλή θερμοκρασία. Με την πάροδο του χρόνου, όμως, η θερμοκρασία του συνεχώς μειώνεται, μέχρις ότου πάψει να ακτινοβολεί θερμικά. Στην περίπτωση αυτή η υδροστατική ισορροπία του αστέρα εξασφαλίζεται από την (κβαντομηχανικής και όχι θερμικής προέλευσης) πίεση των ηλεκτρονίων. \\
{\color{red} \hrule}




\section{Αστέρες νετρονίων}
{\color{red} \hrule}
Αστέρας νετρονίων είναι ο εκφυλισμένος, γυμνός πυρήνας ενός αστέρα με σχετικά μεγάλη αρχική μάζα. Η εξέλιξη της θερμοκρασίας του ακολουθεί, όπως πιστεύουμε σήμερα, την πορεία της θερμοκρασίας των λευκών νάνων. Σ' αυτήν την περίπτωση η υδροστατική ισορροπία εξασφαλίζεται από την κβαντομηχανικής φύσεως πίεσης των νετρονίων. Δεν αποκλείεται ενάς τέτοιος αστέρας να παρουσιαστεί ενεργά στον ουρανό υπό την μορφή ενός pulsar, που γίνεται ορατός με παρατηρήσεις σε ραδιαφωνικά, κυρίως, μήκη κύματος.\\
{\color{red} \hrule}





\section{Μαύρες τρύπες}




{\color{red} \hrule}
Στην περίπτωση των μελανών οπών, η υδροστατική ισορροπία του αστέρα έχει καταστραφεί, επειδή η διαθέσιμη πίεση (θερμικής ή κβαντομηχανικής προέλευσης) δεν είναι ικανή να αντισταθμίσει την βαρυτική. Η μάζα του αστέρα έχει καταρρεύσει, δημιουργώντας ένα αντικείμενο εξαιρετικά μεγάλης πυκνότητας. Σε κάθε μελανή οπή μπορούμε να αντιστοιχήσουμε ένα χαρακτηριστικό μήκος, $R_S$, που ονομάζεται ακτίνα Schwarzschild, με τη σχέση $R_S = 2GM/c^2$. Το βαρυτικό πεδίο μιας μελανής οπής είναι τόσο ισχυρό, ώστε σε απόσταση μικρότερη από την ακτίνα Schwarzschild ακόμα και το φως δεν μπορεί να διαφύγει από την βαρυτική έλξη.\\
{\color{red} \hrule}


%     \chapter{Διπλά συστήματα \& Μεταβλητοί αστέρες}
\label{ch:Chapter7}
{\hypersetup{linkcolor=black, pdfborder=0 0 1}
	\minitoc
	%\newpage
}

\section{Διπλά συστήματα αστέρων}


Ένα διπλό σύστημα αστέρων αποτελείται από δύο αστέρες που αλληλεπιδρούν βαρυτικά και το σύστημα είναι δέσμιο. Αν ένας αστέρας έχει μεγάλη ταχύτητα και περάσει από την γειτονιά ενός άλλου αστέρα, τότε τα 2 αστέρια θα αλληλεπιδράσουν βαρυτικά μεν, αλλά το σύστημα δεν θα είναι δέσμιο.
Το ότι το σύστημα είναι δέσμιο σημαίνει ότι η ενέργεια του συστήματος είναι αρνητική (θετική κινητική ενέργεια προφανώς, αλλά αρνητική δυναμική). Επίσης ισχύουν τα εξής:

\begin{itemize}
    \item Οι αστέρες εκτελούν ελλειπτικές τροχιές γύρων από το κέντρο μάζας (ΚΜ) του συστήματος.
    \item Το επίπεδο και η περίοδος της τροχιάς είναι κοινά και για τα δύο αστέρια.
    \item το ΚΜ βρίσκεται στην ευθεία που ενώνει τα δύο αστέρια, και η θέση του σ' αυτή καθορίζεται από την σχέση:
        \begin{equation}
            \label{eq:center_mass}
            \frac{a_B}{a_A} = \frac{M_A}{M_B}
        \end{equation}
        όπου $a_A, a_B$ είναι οι αποστάσεις των αστέρων από το ΚΜ.
    \item Για την περίοδο των τροχιών των αστέρων, P, ισχύει ο 3ος νόμος του Kepler
        \begin{equation}
            \label{eq:Kepler_third_law}
            P^2 = \frac{4\pi^2}{G}a^3 \frac{1}{(M_A + M_B)}
        \end{equation}
        όπου $a$ είναι ο μεγάλος ημιάξονας της τροχιάς της \textit{σχετικής} θέσης των δύο αστέρων.
\end{itemize}

Αν γνωρίζουμε λοιπόν τα χαρακτηριστικά του συστήματος (γωνία κλίσης επιπέδου τροχιάς, $P, a_B, a_A, a$) τότε έχουμε ένα σύστημα 2 εξισώσεων (τις σχέσεις \eqref{eq:center_mass} και \eqref{eq:Kepler_third_law}) για 2 αγνώστους ($M_A$, $M_B$).  Έτσι, μπορούμε να υπολογίσουμε τις μάζες των δύο αστέρων.






\subsection{Κατηγορίες διπλών συστημάτων}
Τα διπλά συστήματα αστέρων κατηγοριοποιούνται ανάλογα με την φαινόμενη απόσταση των δύο αστέρων και τη δυνατότητα διαχωρισμού τους από τα γήινα τηλεσκόπια. Έτσι προκύπτουν οι κάτωθι κατηγορίες.

\subsubsection{Οπτικά διπλοί αστέρες}
Τα συστήματα αυτά αποτελούνται από διπλούς αστέρες τα μέλη των οποίων είναι ορατά με γυμνό μάτι ή τηλεσκόπιο ως διακριτοί αστέρες. Είναι συνήθως κοντινά συστήματα με μεγάλες περιόδους περιστροφής και μεγάλες αποστάσεις μεταξύ των μελών του συστήματος.
Ένα τέτοιο σύστημα μπορεί να αποτελεί \textit{πραγματικά} διπλό σύστημα αστέρων όπως το έχουμε ορίσει, αλλά επίσης μπορεί δύο αστέρες να βρίσκονται σε τελείως διαφορετικές αποστάσεις από τη Γη και να φαίνεται σαν ένα διπλό σύστημα επειδή προβάλλεται πάνω στο επίπεδο της ουράνιας σφαίρας. Αυτοί ονομάζονται \textit{φαινομενικά} διπλοί αστέρες.

Η θεωρητική διακριτική ικανότητα, $\omega_{\text{min}}$, ενός οπτικού οργάνου εξαρτάται από το μήκος κύματος της παρατήρησης και τη διάμετρο, $D$, του αντικειμενικού φακού (ή κατόπτρου) σύμφωνα με τη σχέση:
\begin{equation}
    \omega_{\text{min}} = 1.22 \frac{\lambda}{D}
\end{equation}

\subsubsection{Μη-οπτικά διπλοί αστέρες}
Η κατηγορία αυτή διπλών αστέρων είναι αυτή των οποίων το ένα μέλος δεν διακρίνεται επειδή είναι εξαιρετικά αμυδρό. Η κατηγορία αυτή χωρίζεται σε 4 υποκατηγορίες ανάλογα με τη μέθοδο που χρησιμοποιούμε για να αντλήσουμε πληροφορίες από ένα τέτοιο σύστημα.

\begin{enumerate}[label=(\Roman*)]
    \item \textbf{Φασματοσκοπικά διπλοί αστέρες (spectroscopic binaries)} \\
    Από τον 3ο νόμο του Kepler (σχέση \eqref{eq:Kepler_third_law}) προκύπτει ότι όταν ο μεγάλος ημιάξονας της τροχιάς των μελών του συστήματος είναι μικρός, οι ταχύτητες περιφοράς των αστεριών γύρω από το κοινό ΚΜ είναι μεγάλες, με άμεσο αποτέλεσμα να μετατοπίζονται οι φασματικές γραμμές του συστήματος λόγω του φαινομένου Doppler.
        \begin{figure}
            \centering
            \includegraphics[scale=0.4]{Figures/spectroscopic_binary.png}
            \caption{Φασματοσκοπικά διπλό σύστημα αστέρων.}
            \label{fig:spectroscopic_binary}
        \end{figure}
    Από το φάσμα που παίρνουμε, βλέπουμε ότι περιέχει γραμμές που ανήκουν σε δύο αστέρες (αν οι λαμπρότητες των μελών είναι συγκρίσιμες) ή μόνο σε έναν αστέρα (αν η λαμπρότητα του ενός είναι πολύ μεγαλύτερη από του άλλου). Κάθε φασματική γραμμή ``ταλαντώνεται'' περιοδικά γύρω από ένα μέσο μήκος κύματος. Προφανώς οι φασματικές γραμμές μετατοπίζονται προς το ερυθρό όταν ο αντίστοιχος αστέρας βρίσκεται στο τμήμα της τροχιάς που απομακρύνεται από τη Γη, και προς το κυανό όταν βρίσκεται στο τμήμα της τροχιάς που προσεγγίζει τη Γη (σχήμα \ref{fig:spectroscopic_binary}). Επειδή οι 2 αστέρες βρίσκονται πάντα σε αντιδιαμετρική θέση ως προς το ΚΜ, όταν οι φασματικές γραμμές του ενός αστέρα είναι μετατοπισμένες προς το ερυθρό, οι φασματικές γραμμές του άλλου είναι μετατοπισμένες προς το κυανό και αντίστροφρα. Οι περισσότεροι γνωστοί διπλοί αστέρες είναι φασματοσκοπικά διπλοί χωρίς αυτό να σημαίνει ότι δεν μπορούν να ανήκουν ταυτόχρονα και σε κάποια άλλη κατηγορία.
        
    Από τον χρόνο που χρειάζεται για να έχουμε δύο διαδοχικές ταυτήσεις των φασματικών γραμμών, μπορούμε να βρούμε την \textit{περίοδο} του συστήματος. Επίσης, ξέρουμε ότι η μετατόπιση από τη θέση ισορροπίας (της φασματικής γραμμής) εξαρτάται από τη συνιστώσα της \textit{ταχύτητας κατά μήκος της γραμμής παρατήρησης} (line of sight). Δεν έχουμε και τις 3 συνιστώσες για το διάνυσμα της ταχύτητας. Μέσω αυτής της συνιστώσας της ταχύτητας μπορούμε να βρούμε και την απόσταση των δύο σωμάτων μεταξύ τους (ανάλογα με το πως αλλάζει το πλάτος της ταχύτητας).
        
    \underline{Σημείωση}: Τα μήκη κύματος στα οποία εμφανίζονται οι μετατοπισμένες γραμμές απορρόφησης δεν αντιστοιχούν σε κανένα χημικό στοιχείο που μπορεί να δώσει μετάπτωση από μία ενεργειακή στιβάδα σε κάποια άλλη και να παράξει αυτά τα μήκη κύματος.
        
    Μέσω της μελέτης των φασματοσκοπικά διπλών αστέρων \textit{δεν} είναι δυνατόν να υπολογισθεί η μάζα των αστέρων του συστήματος, παρά μόνο ο λόγος των μαζών τους και το γινόμενο της κάθε μάζας επί την άγνωστη ποσότητα $\sin^3 i$, όπου $i$ είναι η γωνία κλίσης του επιπέδου της τροχιάς του διπλού συστήματος (σχήμα \ref{fig:inclination_angle}).
        \begin{figure}[h]
            \centering
            \includegraphics[scale=0.6]{Figures/inclination.jpg}
            \caption{Η γωνία κλίσης $i$, του επιπέδου της τροχιάς ενός διπλού συστήματος αστέρων.}
            \label{fig:inclination_angle}
        \end{figure}
        
    \item \textbf{Εκλειπτικά διπλοί αστέρες (ecliptic binaries)}\\
    Όταν το επίπεδο της τροχιάς των δύο αστέρων είναι σχεδόν παράλληλο με τη διεύθυνση παρατήρησης, δηλαδή η γωνία κλίσης $i \simeq 90^{\circ}$, και η απόσταση μεταξύ των μελών του συστήματος είναι πολύ μικρή, τότε κατά την περιφορά τους γύρω από το ΚΜ, τα 2 μέλη διέρχονται διαδοχικά το ένα μπροστά από το άλλο έτσι ώστε το ένα να καλύπτει τμήμα (ή και το σύνολο) του φαινόμενου δίσκου του άλλου, προκαλώντας μερικές ή όλικές εκλείψεις.
    Η ύπαρξη του ζεύγους συνεπάγεται από τις περιοδικές μεταβολές (αυξομειώσεις) της φαινόμενης λαμπρότητας του --φαινομενικά απλού-- αστέρα, η οποία μειώνεται κατά τη διάρκεια της έκλειψης.
    
    Κατά τη διάρκεια μίας περιφοράς συμβαίνουν δύο εκλείψεις, ανάλογα με το ποιό αστέρι βρίσκεται μπροστά από το άλλο και ως εκ τούτου παρουσιάζονται δύο ελάχιστα λαμπρότητας (σχήμα \ref{fig:eclipsing_binary}). Τα δύο αυτά ελάχιστα διαφέρουν γενικά ως προς το πλάτος και το βάθος, ανάλογα με τη λαμπρότητα των μελών του συστήματος.
        \begin{figure}[h]
            \centering
            \includegraphics[scale=0.4]{Figures/eclipsing_binary.png}
            \caption{Καμπύλη φωτός για ένα εκλειπτικά διπλό σύστημα αστέρων. Όταν ο αμυδρότερος αστέρας καλύπτει τον λαμπρότερο έχουμε το \textit{πρωτεύον ελάχιστο}. Στην αντίθετη περίπτωση έχουμε το \textit{δευτερεύων ελάχιστο}.}
            \label{fig:eclipsing_binary}
        \end{figure}
        
    Και σε αυτή την περίπτωση δεν μπορούμε να υπολογίσουμε τη μάζα κανενός από τους αστέρες του συστήματος. Από την καμπύλη φωτός όμως (σχήμα \ref{fig:eclipsing_binary}) μπορούμε να υπολογίσουμε την κλίση της τροχιάς, τις ακτίνες των μελών του ζεύγους και τον λόγο των φωτεινοτήτων των δύο αστέρων. Αν επιπροσθέτως τα δύο αστέρια ανήκουν στην κύρια ακολουθία μπορούμε να υπολογίσουμε και τον λόγο των μαζών από τη σχέση μάζας-φωτεινότητας.
    
    \item \textbf{Αστρομετρικά διπλοί αστέρες (astrometric binaries)}\\
    Στη κατηγορία αυτή κατατάσσονται τα μη-οπτικώς διπλά συστήματα αστεριών, των οποίων ο αμυδρός συνοδός αστέρας εντοπίζεται μόνο μέσω των δυναμικών επιδράσεων του πάνω στην τροχιά του πρωτεύοντος αστέρα. Ουσιαστικά παρατηρούμε μόνο ένα άστρο, επειδή όμως η κίνηση στην τροχιά του παρουσιάζει παλινδρομήσεις, συμπεραίνουμε ότι υπάρχει αμυδρός συνοδός.
    
    \item \textbf{Φασματικά διπλοί αστέρες (spectral binaries)}\\
    Αν οι τυπικές ταχύτητες περιφοράς των μελών ενός διπλού συστήματος ή/και η γωνία κλίσης $i$ είναι πολύ μικρή, τότε δεν είναι δυνατόν να ανιχνευθεί η μετατόπιση Doppler των φασματικών γραμμών και επομένως το σύστημα δεν αναγνωρίζεται ως φασματοσκοπικά διπλός αστέρας. Παρόλα αυτά, αν τα δύο μέλη έχουν σημαντικά διαφορετικά φάσματα (ανήκουν δηλαδή σε διαφορετικούς φασματικούς τύπους) και συγκρίσιμες λαμπρότητες (έτσι ώστε και τα δύο φάσματα να είναι ορατά), τότε το σύστημα μπορεί να αναγνωριστεί ως φασματικά διπλός αστέρας.
    
    Είναι φανερό ότι οι φασματικά διπλοί αστέρες διαφέρουν από τους φασματοσκοπικά διπλούς στο ότι στους πρώτους δεν παρατηρείται μετατόπιση Doppler των γραμμών. Κανένα στοιχείο δεν μπορεί να βρεθεί για τους φασματικά διπλούς αστέρες αφού δεν παρατηρούμε σ' αυτούς κανένα φαινόμενο που να εμφανίζει χρονική μεταβολή.
\end{enumerate}

\subsection{Βασικοί υπολογισμοί}

\subsubsection{Υπολογισμός στοιχείων τροχιάς}
Μετατρέποντας τη \textit{σχετική φαινόμενη τροχιά} σε \textit{σχετική πραγματική τροχιά}, υπολογίζουμε τις εξής ποσότητες:
\begin{eqnarray*}
    \epsilon_{\pi} &=& \ \text{εκκεντρότητα} \\
    \alpha_{\pi} &=& \ \text{μεγάλος γωνιώδης ημιάξονας σε AU} \\
    i &=& \ \text{γωνία κλίσης}
\end{eqnarray*}

Αν γνωρίζουμε την παράλλαξη $p$, τότε υπολογίζουμε τον μεγάλο ημιάξονα $\alpha$ σε AU
\begin{equation}
    \alpha = \frac{\alpha_{\pi}}{p}
\end{equation}

\subsubsection{Υπολογισμός αθροίσματος μαζών}
Από τον 3ο νόμο του Kepler (σχέση \eqref{eq:Kepler_third_law}) έχουμε ότι 
$$M_1 + M_2 \simeq \frac{\alpha ^3}{P^2}$$ όπου $\alpha$ είναι ο μεγάλος ημιάξονας σε AU και $P$ η περίοδος του συστήματος.


\subsubsection{Υπολογισμός μαζών $M_1. M_2$}
Για φωτεινά διπλά συστήματα με σχετικά μεγάλη γωνιώδη απόσταση $\alpha_{\pi}$, μπορούμε να υπολογίσουμε και την \textit{απόλυτη φαινόμενη τροχιά} για καθένα από τα δύο μέλη. Άρα υπολογίζουμε δύο ημιάξονες $\alpha_1, \alpha_2$.

Από τον ορισμό του ΚΜ του συστήματος (σχέση \eqref{eq:center_mass}) και σε συνδυασμό με τον υπολογισμό του αθροίσματος $M_1 + M_2$ προκύπτουν οι μάζες $M_1, M_2$ για κάθε αστέρα.

\subsubsection{Δυναμικές παραλλάξεις}
 Από τις σχέσεις:
 \begin{eqnarray*}
    M_1 + M_2 &=& \frac{\alpha^3}{P^2} \\\\
    \alpha &=& \frac{\alpha_{\pi}}{p} 
 \end{eqnarray*}
 προκύπτει ότι:
 \begin{equation}
     \alpha_{\pi} = p \left[ (M_1 + M_2)P^2 \right]^{1/3}
 \end{equation}
 
 Αν οι δύο αστέρες ανήκουν στην κύρια ακολουθία, τότε για τον καθένα ισχύει ο νόμος μάζας-φωτεινότητας $L = f(M)$. Οπότε λύνουμε επαναληπτικά το σύστημα:
 
 \begin{eqnarray*}
    p &=& \alpha_{\pi} \left[ (M_1 + M_2)P^2 \right]^{-1/3} \\\\
    L_1 &=& f(M_1) \\\\
    L_2 &=& = f(M_2)
 \end{eqnarray*}
 για $\pi, M_1, M_2$ ξεκινώντας από κάποια εκτίμηση για το $M_1 + M_2$.  




\subsection{Στενά διπλά συστήματα \& απώλεια μάζας}
Κατά τη διάρκεια της ζωής τους, τα αστέρια χάνουν μάζα η οποία εμπλουτίζει τον μεσοαστρικό χώρο μέσω δύο διακασιών:
\begin{enumerate}[label=(\alph*)]
    \item \textbf{Συνεχώς}, μέσω των αστρικών ανέμων. Η μάζα που μπορεί να χαθεί με αυτόν τον τρόπο εξαρτάται από τη μάζα του αστέρα καθώς και το στάδιο της εξέλιξής του ($\dot{M}_{\text{Giant}} > \dot{M}_{MS}$).
    \item \textbf{Εκρηκτικώς}, μέσω καινοφανών και υπερκαινοφανών εκρήξεων.
\end{enumerate}

{\color{red} \hrule}
Στην περίπτωση ενός στενού διπλού συστήματος, οι αλληλεπιδράσεις μεταξύ των μελών μπορεί να οδηγήσει σε μεταφορά μάζας από το ένα αστέρι στο άλλο, επηρεάζοντας με αυτόν τον τρόπο την δομή, την μάζα, την γωνιακή στροφορμή καθώς και την τελική κατάσταση των αστέρων του συστήματος.\\
{\color{red} \hrule}

Όταν έχουμε ένα διπλό σύστημα, το πρόβλημα των πολλών σωμάτων (n-body problem) εκφυλίζεται στο ``περιορισμένο κυκλικό πρόβλημα των τριών σωμάτων'' (circular restricted three-body problem), με το ενεργό (effective) δυναμικό να δίνεται από τη σχέση:
\begin{equation}
    \label{eq:effective_potential}
    \Phi = - G \left( \frac{M_1}{r_1} + \frac{M_2}{r_2} \right) - \frac{1}{2} \Omega^2 r_3^2
\end{equation}
όπου  $r_1, r_2$ είναι οι αποστάσεις από το κέντρο των αστέρων $M_1, M_2$ αντίστοιχα, $\Omega$ είναι η τροχιακή γωνιακή ταχύτητα και $r_3$ είναι η απόσταση του άξονα περιστροφής τους συστήματος.

Η ποσότητα $\displaystyle V_g = - G \left( \frac{M_1}{r_1} + \frac{M_2}{r_2} \right)$ είναι ουσιαστικά το βαρυτικό δυναμικό ενώ το $\displaystyle V_F = - \frac{1}{2} \Omega^2 r_3^2$ περιγράφει το ``δυναμικό'' της φυγόκεντρου δύναμης. 

Αν απαιτήσουμε η συνολική δύναμη που ασκείται σε ένα δοκιμαστικό σωματίδιο, $m$, να είναι μηδέν: $$\boldsymbol{F_t} = - \nabla \Phi = 0$$
τότε η εξίσωση \eqref{eq:effective_potential} δίνει πέντε λύσεις όπου η βαρυτική δύναμη εξισορροπεί την φυγόκεντρο δύναμη που προκαλείται από τη σχετική κίνηση των δύο αστέρων του ενός γύρω από τον άλλον. Τα σημεία για τα οποία ισχύει αυτό, δηλαδή τα σημεία τα οποία έχουν μηδενική στιγμιαία ταχύτητα ονομάζονται \textit{σημεία Lagrange}, ($L_n, n = 1, \dots, 5$). Άρα αν το δοκιμαστικό σωματίδιο τοποθετούνταν σε κάποιο από αυτά τα σημεία, θα διατηρούσε τη θέση του σε σχέση ως προς τα δύο αστέρια.

Το σύνολο των σημείων μηδενικής ταχύτητας συγκροτεί μία ομάδα κλειστών επιφανειών, τις οποίες ονομάζουμε \textit{επιφάνειες μηδενικής ταχύτητας} (zero velocity surfaces) και περνούν από τα σημεία Lagrange. Η τομή μιας τέτοιας επιφάνειας μηδενικής ταχύτητας (ή ισοδυναμική επιφάνεια καθώς περιλαμβάνει όλα τα σημεία του συστήματος που μοιράζονται την ίδια τιμή του δυναμικού $\Phi$) με κάποιο συγκεκριμένο επίπεδο ονομάζεται \textit{καμπύλη μηδενικής ταχύτητας} (zero velocity curve). Συνήθως σαν επίπεδο τομής επιλέγεται το επίπεδο της σχετικής τροχιάς των δύο αστέρων του συστήματος (σχήμα \ref{fig:eq_sur}).
\\
{\color{red} \hrule}
Οι επιφάνειες μηδενικής ταχύτητας είναι σημαντικές γιατί ορίζουν τα όρια (boundaries) των περιοχών από τις οποίες το δοκιμαστικό σωματίδιο είναι δυναμικά αποκλεισμένο. Με άλλα λόγια, κάθε αστέρας ελέγχει βαρυτικά μόνο έναν περιορισμένο χώρο που καθορίζεται από μία ισοδυναμική επιφάνεια.\\
{\color{red} \hrule}

\begin{figure}[h]
   \centering
\begin{subfigure}[h]{0.45\textwidth}
	\centering
   	 \includegraphics[width = \linewidth]{Figures/equipotentials_mu_0_316.png} 
\end{subfigure}
\begin{subfigure}[h]{0.535\textwidth}
	\centering
	\includegraphics[width = \linewidth]{Figures/RochePotential_-_Colorized.png} 
    \end{subfigure}
    \caption{\textbf{Αριστερά}: Ισοδυναμικές γραμμές πάνω στο τροχιακό επίπεδο ενός περιστρεφόμενου διπλού συστήματος, με ανηγμένη μάζα $\mu = 0.316$ (reduced mass ratio). Σε αυτή τη διάταξη, $M_1$ είναι το αστέρι με τη μεγαλύτερη μάζα και βρίσκεται στη θέση $x = +a$, ενώ το $M_2$ είναι το αστέρι με τη μικρότερη μάζα και βρίσκεται στη θέση $x = -b$. Η εσωτερική ισοδυναμική επιφάνεια που περνάει από το σημείο Lagrange $L_1$, ορίζει τον λοβό Roche, που έχει σχήμα σταγόνας, για τον κάθε αστέρα. \textbf{Δεξιά}: 3D αναπαράσταση ισοδυναμικών επιφανειών.}
    \label{fig:eq_sur}
\end{figure}

Στο σχήμα \ref{fig:eq_sur} κάθε καμπύλη μηδενικής ταχύτητας έχει παρουσιαστεί με διαφορετικό χρώμα. Από τα πέντε σημεία Lagrange, το $L_1$ παίζει σημαντικό ρόλο στην εξέλιξη των διπλών συστημάτων καθώς η ισοδυναμική επιφάνεια που περνάει από αυτό ορίζει τους (ανεξάρτητους μεταξύ τους) \textit{λοβούς Roche} και κάθε ένας από αυτούς περιβάλλει έναν αστέρα του συστήματος.

Η επιφάνεια μηδενικής ταχύτητας που περνάει από το $L_1$ έχει ένα ιδιαίτερο χαρακτηριστικό: όλες οι επιφάνειες μηδενικής ταχύτητας που περιβάλλουν ένα μόνο σώμα, βρίσκονται μέσα στον αντίστοιχο λοβό Roche, ενώ όλες οι επιφάνειες μηδενικής ταχύτητας που βρίσκονται έξω από τους λοβούς Roche περιβάλλουν και τα δύο σώματα. Έτσι, ένα δοκιμαστικό σωματίδιο που βρίσκεται μέσα στο λοβό Roche ενός μέλους του διπλού συστήματος ``ανήκει'' βαρυτικά σε αυτόν τον αστέρα. Αντίθετα, ένα σωματίδιο που βρίσκεται έξω και από τους δύο λοβούς Roche ``ανήκει'' βαρυτικά και στους δύο αστέρες, αφού κινείται σε τροχιά που είναι δυνατόν να περιβάλλει και τους δύο. Οι τροχιές αυτές δεν είναι κωνικές τομές (κύκλοι, ελλείψεις, παραβολές, υπερβολές ή ευθείες).

Με βάση το λόγο της ακτίνας του κάθε αστέρα ως προς την ακτίνα του δικού του λοβού ($R_L$), τα διπλά συστήματα χωρίζονται στις εξής κατηγορίες:

\subsubsection{Αποχωρισμένα (detached binaries)}
Είναι ένα διπλό σύστημα στο οποίο ο κάθε ένας αστέρας βρίσκεται εξ' ολοκλήρου μέσα στον δικό του λοβό Roche και εξελίσσονται σχεδόν ανεξάρτητα ο ένας από τον άλλον. Ισχύει δηλαδή ότι $r_1 < R_{L,1}$ και $r_2 < R_{L,2}$ όπου $r_1, r_2$ οι ακτίνες των δύο αστέρων.

\subsubsection{Ημιαποχωρισμένα (semidetached binaries)}
Αυτά είναι τα συστήματα στα οποία μόνο ο ένας από τους δύο αστέρες ``γεμίζει'' τον λοβό Roche του (δηλαδή $r_1 < R_{L,1}$ και $r_2 \approx R_{L,2}$). Αυτό μπορεί να συμβεί σε μετεγενέστερα εξελικτικά στάδια στη διάρκεια ζωής ενός αστέρα, όταν μπει στη φάση των γιγάντων,
Έκεί, η ακτίνα του μεγαλώνει σε τόσο βαθμό όπου ο όγκος του αστέρα είναι συγκρίσιμος με  τον όγκο που ορίζει ο λοβός Roche του.

Σε αυτή την περίπτωση έχουμε εκροή μάζας από αυτόν τον αστέρα προς τον συνοδό του διαμέσου το εσωτερικού $L_1$ σημείο Lagrange. Η προσαύξηση του υλικού γίνεται με τον σχηματισμό ενός \textit{δίσκου επισυσσώρευσησ/προσαύξησης} και όχι με το να πέφτει απευθείας πάνω στον συνοδό αστέρα. Αυτό συμβαίνει διότο το υλικό που εκρύεται έχει την ίδια στροφορμή με τον αστέρα-δότη και εκτός αν υπάρχουν μη-συντηρητικές μηχανισμοί για να αφαιρέσουν ένα ποσό από τη στροφορμή του αερίου που εκρύεται, αυτό θα συνεχιστεί να περιφέρεται γύρω από τον συνοδό αστέρα. Για να καταφέρει να πέσει πάνω στον αστέρα, η ύλη πρέπει να χάσει στροφορμή από την εσωτερική τριβή (ιξώδες). Αυτή η εσωτερική τριβή του αερίου το αναγκάζει να θερμανθεί και να ακτινοβολεί (X-ray binary).

Η υπερχείλιση του αστέρα μέσα από τον λοβό του Roche οδηγεί σε σημαντική μείωση της μάζας του αστέρα και επηρεάζει τη μετέπειτα εξέλιξή του. Πέρα από αυτόν τον μηχανισμό όμως, μπορούμε να έχουμε μεταφορά μάζας με προσαύξηση από τον \textit{αστρικό άνεμο} του θερμότερου αστέρα προς τον ψυχρότερο. Παρόλα αυτά, η μεταφορά μάζας με αυτόν τον τρόπο δεν είναι το ίδιο αποδοτική όσο η υπερχείλιση του λοβού Roche.

\subsubsection{Εν επαφή (contact binary)}
Ένα διπλό σύστημα λέμε ότι είναι σε επαφή ότα και τα δύο αστέρια ``γεμίζουν'' τους λοβούς τους (δηλαδή $r_1 \approx R_{L,1}$ και $r_2 \approx R_{L,2}$). Σε αυτή την περίπτωση, το σύστημα μοιράζεται μία κοινή ατμόσφαιρα (common envelope, σχήμα \ref{fig:binary_cases}) η οποία μπορεί να εκδιωχθεί στερώντας έτσι από το σύστημα ένα σημαντικό μέρος της μάζας του.

\begin{figure}
    \centering
    \includegraphics[scale=0.3]{Figures/binary_cases.jpeg}
    \caption{Κατηγορίες διπλών συστημάτων βάσει του ποσοστού πλήρωσης των λοβών Roche.}
    \label{fig:binary_cases}
\end{figure}

Σε αυτό το σημείονα διευκρινήσουμε ότι δεν υπάρχει αναλυτική έκφραση για το μέγεθος του λοβού Roche σε ένα διπλό σύστημα. Παρόλα αυτά έχει προταθεί μια αριθμητική προσέγγιση της ακτίνας του λοβού Roche μέσω της σχέσης:
$$\frac{R_L}{a} = \frac{0.49q^{2/3}}{0.6q^{2/3} + \ln (1 + q^{1/3})}$$
όπου $a$ είναι ο τροχιακός διαχωρισμός (orbital separation) του συστήματος και $\displaystyle q \equiv \frac{M_{\text{donor}}}{M_{\text{accretor}}}$ είναι ο λόγος των μαζών των μελών του συστήματος.
Αυτές είναι και οι δύο πιο σημαντικές παράμετροι για να ακολουθήσουμε την εξέλιξη του συστήματος.













\section{Μεταβλητοί αστέρες}
Μεταβλητοί αστέρες ονομάζονται οι αστέρες, των οποίων η λαμπρότητα μεταβάλλεται μέσα σε ένα χρονικό διάστημα σημαντικά μικρότερο από την ηλικία τους. Με τον (γενικά ασαφή) αυτό ορισμό αποφεύγει κανείς να χαρακτηρίσει μεταβλητούς όλους τους αστέρες, αφού, η λαμπρότητα ενός αστέρα μεταβάλλεται σημαντικά, και μάλιστα κατά πολλές τάξεις μεγέθους, στη διάρκεια της ζώης τους (π.χ. εγκατάσταση στην κύρια ακολουθία, άνοδος στον κλάδο των γιγάντων κτλ.)

Οι μεταβλητοί αστέρες κατατάσσονται σε κατηγορίες, ανάλογα με το φαινόμενο που προκαλεί τη μεταβολή της λαμπρότητά τους, την περιοδικότητα (ή την έλλειψη αυτής) της μεταβολής και το μέγεθός της. Καταρχάς, οι μεταβλητοί αστέρες μπορούν να χωριστούν σε γνήσιους μεταβλητούς και σε μη-γνήσιους. Στην τελευταία περίπτωση, η μεταβολή της λαμπρότητας του αστέρα δεν οφείλεται σε κάποιον ενδογενή φυσικό μηχανισμό αλλά σε κάποιον εξωτερικό παράγοντα που επηρεάζει φαινομενικά την λαμπρότητα. 

Η απλούστερη περίπτωση ενός τέτοιου μη-γνήσιου μεταβλήτού αστέρα είναι εκείνη, κατά την οποία ο αστέρας είναι στην πραγματικότητα εκλειπτικά διπλός και η μεταβολή της λαμπρότητάς του οφείλεται στο γεγονός ότι, κατά την κίνησή τους, τα μέλη του διέρχονται διαδοχικά το ένα μπροστά από το άλλο, προκαλώντας εκλείψεις. Η διάρκεια των εκλείψεων, ο ρυθμός ελλάτωσης και αύξησης της λαμπρότητας του διπλού αστέρα κατά την αρχή και το τέλος των εκλείψεων, η αλλαγή του δείκτη χρώματος και η περίοδος του φαινομένου δίνουν πληροφορίες για την απόσταση μεταξύ των μελών, τη μάζα τους, τις διαστάσεις τους, το φασματικό τους τύπο κτλ.
Από την άλλη μεριά, οι γνήσιοι μεταβλητοί αστέρες χωρίζονται στις κατηγορίες που θα αναλύσουμε παρακάτω.

\subsection{Περιοδικοί}
\subsubsection{Κηφείδες}
Οι Κηφείδες αστέρες αποτελούν τη σημαντικότερη κατηγορία περιοδικών μεταβλητών αστέρων. Ονομάστηκαν έτσι καθώς ο πρώτος τέτοιος αστέρας που μελετήθηκε ήταν ο ``δ Κηφέως'' στον αστερισμό του Κηφέα. Οι Κηφείδες είναι γίγαντες αστέρες και παρουσιάζουν περιοδικές μεταβολές στη λαμπρότητά τους, με περίοδο από 1 εως 50 ημέρες. Κατά τη διάρκεια μιας περιόδου, εκτός από την λαμπρότητα του αστέρα, που μεταβάλλεται κατά $\Delta m \simeq 1$ μέγεθος, μεταβάλλονται επίσης και ο δείκτης χρώματός του, και επομένος ο φασματικός του τύπος (και άρα η επιφανειακή του θερμοκρασία). Συνέπεια του γεγονότος αυτού είναι ότι η θέση των αστέρων αυτών δεν μένει σταθερή στο διάγραμμα H-R αλλά μετατοπίζεται, εκτελώντας μία κλειστή τροχιά κατά τη διάρκεια μιας περιόδου.

Η μεταβολή της λαμπρότητας των Κηφείδων δεν οφείλεται μόνο στη μεταβολή της θερμοκρασίας των αστέρων αυτών, αλλά και σε μεταβολή της ακτίνας τους. Επομένως οι αστέρες αυτοί παρουσιάζουν \textbf{αναπάλσεις} (pulsations). Αυτές οι αναπάλσεις αφορούν τα εξωτερικά στρώματα του αστέρα και δεν οφείλονται σε μεταβολές του ρυθμού παραγωγής ενέργειας στον πυρήνα καθώς τέτοια φαινόμενα θα εξελίσσονταν στην θερμική χρονική κλίμακα που είναι της τάξης των δεκάδων χιλιάδων ετών. Γίνεται φανερό ότι οι Κηφείδες δεν είναι γεννημένοι μεταβλητοί, αλλά αποτελούν μία φάση στην εξέλιξη ορισμένων αστέρων, μετά την έξοδό τους από την κύρια ακολουθία.

Αποδεικνύεται ότι για τους Κηφείδες αστέρες ισχύει μία σχέση μεταξύ της περιόδου και της λαμπρότητας της μορφής
\begin{equation}
    \label{eq:cepheids_period_luminosity_relation}
    \log P = \alpha \log L + \beta 
\end{equation}
όπου $\alpha$ και $\beta$ είναι αριθμητικές σταθερές. Μερικές φορές αυτή η σχέση εκφράζεται και με όρους απόλυτου μεγέθους αντί λαμπρότητας. Σε συνδυασμό με τη παρατηρούμενη μεταβολή στο φαινόμενο μέγεθος των Κηφείδων, η σχέση \eqref{eq:cepheids_period_luminosity_relation} μπορεί να χρησιμοποιηθεί για την εκτίμηση της απόστασης μέσω του distance modulus. Μέχρι σήμερα, έχουν μετρηθεί οι αποστάσεις πολλών γαλαξιών που φιλοξενούν Κηφείδες αστέρες χρησιμοποιώντας αυτή την μέθοδο.

\subsubsection{RR Lyrae}
Οι αστέρες αυτοί είναι μεταβλητοί μικρής περιόδου, η μεταβολή της φωτεινότητας των οποίων οφείλεται επίσης σε σφαιρικα συμμετρικές (ακτινικές) αναπάλσεις. Είναι κιτρινόλευκοι γίγαντες αστέρες (οριζόντιος κλάδος στο διάγραμμα H-R) ενώ έχουν μάζα και περίοδο μικρότερες από αυτές των Κηφείδων ($M \simeq 0.5\,M_\odot$, $P \sim 0.2 - 1\,\text{ημέρα}$). Η λαμπρότητά τους είναι σχεδόν ανεξάρτητη της περιόδου αναπάλσεως και αντιστοιχεί σε απόλυτο μέγεθος $M_V = 0.6$. Οι αστέρες αυτοί, πάντως, είναι αμυδρότεροι από τους Κηφείδες, και έτσι δεν είναι ορατοί σε μεγάλες αποστάσεις. Επομένως είναι χρήσιμοι για μετρήσεις αποστάσεων μέσα στο Γαλαξία, αλλά όχι για μετρήσεις αποστάσεων άλλων γαλαξιών.

\subsubsection{Μακροπερίοδοι}
Κατά τα τελευταία στάδια της εξέλιξης αστέρων μικρών μάζας είναι δυνατόν να υπάρξει συντονισμός των εξωτερικών στρωμάτων ενός αστέρα με το ρυθμό παραγωγής ενέργειας κατά τις διαδοχικές εναλλαγές καύσης των φλοιών H και He. Ο αστέρας τότε παρατηρείται ως μακροπερίοδος \textbf{μεταβλητός τύπου Mira} (Mira variable) με περίοδο, $P$, μεταξύ 100 και 1000 ημερών και εύρος μεταβολής μεγέθους $\Delta m_V \simeq 2 - 6\,\text{μεγέθη}$.

Οι μεταβλητοί τύπου Mira είναι ψυχροί, ερυθροί γίγαντες αστέρες μικρής μάζας και φασματικού τύπου Μ. Στο φάσμα τους παρατηρούνται ιδιαίτερα ισχυρές μοριακές γραμμές απορρόφησης. Το μεγάλο εύρος της μεταβολής του μεγέθους αυτών των αστέρων μπορεί να εξηγηθεί από το γεγονός ότι ο συντελεστής απορρόφησης εξαρτάταται ισχυρά από τη θερμοκρασία.

\subsection{Μη-περιοδικοί}
\subsubsection{Ανώμαλοι - T Tauri}
Οι αστέρες αυτοί πήραν το όνομά τους από τον τυπικό εκπρόσωπό της κατηγορίας, το μεταβλητό αστέρα Τ στον αστερισμό του Ταύρου. Παρατηρησιακά οι αστέρες αυτοί χαρακτηρίζονται:
\begin{enumerate}
    \item από απότομες και βραχυχρόνιες αυξήσεις της φωτεινότητάς τους από λίγα δέκατα του μεγέθους μέχρι και 4 μεγέθη, τις οποίες διαδέχονται διαστήματα ηρεμίας που διαρκούν από λίγες ώρες μέχρι και 100 ημέρες,
    \item από φασματικές γραμμές λιθίου (οι οποίες δεν εμφανίζονται σε άλλους αστέρες καθώς το λίθιο εξαντλείται πολύ γρήγορα κατά την έναρξη των θερμοπυρηνικών αντιδράσεων στους αστέρες) και
    \item από ισχυρούς αστρικούς ανέμους.
\end{enumerate}

Από την ερυθρή χρώση του φωτός τους και από το φάσμα τους βρίσκουμε ότι περιβάλλονται από ένα νέφος αερίου και σκόνης και ότι η θέση τους στο διάγραμμα H-R είναι μεταξύ των ερυθρών γιγάντων και της κύριας ακολουθίας. Η παρουσία λιθίου και η θέση τους στο διάγραμμα H-R μας κάνουν να πιστεύουμε ότι οι αστέρες της κατηγορίας αυτής είναι αστέρες ``εν τη γενέσει τους'', δηλαδή αστέρες που βρίσκονται ακόμα στο τελευταίο στάδιο της βαρυτικής συστολής τους. Επομένως οι αστέρες αυτοί δεν έχουν φτάσει ακόμα στη κύρια ακολουθία.

\subsubsection{Ανώμαλοι - Αστέρες εκλάμψεων}
Οι αστέρες αυτοί είναι ψυχροί ερυθροί νάνοι της κύριας ακολουθίας, οι οποίοι παρουσιάζουν σε ακανόνιστα χρονικά διαστήματα απότομες και σύντομες αυξήσεις της φωτεινότητάς τους (κυρίως στα χρώματα B και U αλλά και στα ραδιοφωνικά μήκη κύματος) μέχρι και 6 μεγέθη, γνωστές ως \textbf{εκλάμψεις} (flares). Σήμερα πιστεύουμε ότι οι εκλάμψεις αυτών των αστέρων είναι της ίδιας φύσης με τις ηλιακές, οφείλονται δηλαδή σε απελευθέρωση μαγνητικής ενέργειας από επανασύνδεση μαγνητικών γραμμών (magnetic reconnection).

Η ενέργεια μιας τυπικής τέτοιας έκλαμψης είναι πολλές τάξεις μεγέθους μεγαλύτερη από την ενέργεια μιας τυπικής ηλιακής έκλαμψης. Η διαφορά αυτή αποδίδεται στο γεγονός ότι το μαγνητικό πεδίο των αστέρων εκλάμψεων είναι κατά πολύ ισχυρότερου του ηλιακού, επειδή η ζώνη μεταφοράς, στην ύπαρξη της οποίας οφείλεται το μαγνητικό πεδίο (σύμφωνα με τη θεωρία της μαγνητικής γεννήτριας του Parker), είναι πολύ μεγαλύτερη στους αστέρες εκλάμψεων. Πραγματικά, σήμερα πιστεύουμε ότι οι νάνοι της κύριας ακολουθίας φασματιών τύπων Μ3 - Μ9 δεν έχουν καθόλου ζώνη ακτινοβολίας, έτσι ώστε η ζώνη μεταφοράς καλύπτει όλο το εσωτερικό του αστέρα, εκτός από τον πυρήνα του.

\subsubsection{Ανώμαλοι - R Coronae Borealis}
Οι υπεργίγαντες αυτοί αστέρες φασματικού τύπου F χαρακτηρίζονται ως ``αντίστροφοι καινοφανείς'', διότι η φωτεινότητά τους ελλατώνεται ενίοτε κατά πολλά (έως και 9) μεγέθη μέσα σε πολύ μικρό χρονικό διάστημα (λίγες ημέρες), παραμένει χαμηλή για λίγο και στη συνέχεια επανέρχεται στην αρχική της τιμή. Γενικά το φάσμα τους είναι φτωχό σε γραμμές υδρογόνου και πλούσιο σε γραμμές άνθρακα. Η ασυνήθιστη αυτή χημική σύσταση οφείλεται είτε στη μεταφορά ύλης από τον πυρήνα προς τη φωτόσφαιρά τους (dredge up), είτε διότι κατά το στάδιο του γίγαντα έχασαν τους φλοιούς H και He που περιέβαλλαν τον πυρήνα.

Σύμφωνα με το πιο διαδεμένο πρότυπο, οι μεταβλητοί τύπου \textbf{R Coronae Borealis} (δηλαδή όμοιοι με τον μεταβλητό αστέρα R του αστερισμού του Βόρειου Στέφανου, R CBr) έχουν πολύ ισχυρούς αστρικούς ανέμους, με τους οποίους ο χώρος γύρω τους εμπλουτίζεται με ενώσεις άνθρακα. 'Οταν οι ενώσεις αυτές ψυχθούν, σχηματίζονται κόκκοι ανθρακούχου σκόνης οι οποίοι απορροφούν έντονα σε όλα τα μήκη κύματος. Σε αυτήν ακριβώς την απορρόφηση οφείλονται τα ελάχιστα της φωτεινότητας που παρατηρούμε. Η απορρόφηση όμως της ακτινοβολίας του αστέρα ανεβάζει τη θερμοκρασία των κόκκων της σκόνης, με αποτέλεσμα αυτοί να εξαχνωθούν και να επανέλθει ο αστέρας στην αρχική του φωτεινότητα.

\subsubsection{Καταστροφικοί - Γενικά στοιχεία}
Στους καταστροφικούς μεταβλητούς ανήκουν οι \textbf{καινοφανείς} (novae) και \textbf{υπερκαινοφανείς} (supernovae) αστέρες, κύριο χαρακτηριστικό των οποίων είναι η απότομη (σε διάστημα ωρών ως ημερών) και μεγάλη (πάνω από 4 και συχνά πάνω απο 10 μεγέθη) αύξηση της λαμπρότητάς τους. Η κατάταξη αυτών των αστέρων βασίζεται σε παρατηρησιακά κριτήρια και δεν αντανακλά απαραίτητα τους φυσικούς μηχανισμούς που προκαλούν τις τεράστιες αυτές αναλάμψεις. Αποτέλεσμα του γεγονότος αυτού είναι ότι αναλάμψεις που οφείλονται σε διαφορετικούς μηχανισμούς μπορεί να καταταγούν στην ίδια κατηγορία, ενώ αναλάμψεις που οφείλονται σε παρόμοιο μηχανισμό μπορεί να καταλήξουν σε διαφορετικές κατηγορίες.

Η απαίτηση που έχουμε από μια θεωρία για τη δημιουργία των καινοφανών και υπερκαινοφανών είναι να εξηγεί, τουλάχιστον, τη μορφή της καμπύλης φωτός (light curve) του αστέρα, δηλαδή να εξηγεί τον χρόνο ανόδου, τη μέγιστη λαμπρότητα, και το ρυθμό καθόδου, καθώς επίσης και το παρατηρούμενο φάσμα. Οι θεωρίες που έχουν προταθεί για την ερμηνεία των καινοφανών και υπερκαινοφανών μπορούν να καταταγούν σε δύο γενικές κατηγορίες:
\begin{enumerate}
    \item σε αυτές που αναφέρονται στην εξέλιξη \textbf{μεμονωμένων} αστέρων μεγάλης μάζας και
    \item σε αυτές που αναφέρονται στην εξέλιξη \textbf{διπλών συστημάτων} μέτριας μάζας.
\end{enumerate}
Σήμερα πιστεύουμε ότι οι υπερκαινοφανείς τύπου II ακολουθούν το πρώτο σενάριο, ενώ οι υπερκαινοφανείς τύπου I και οι καινοφανείς το δεύτερο σενάριο.

\begin{figure}
    \centering
    \includegraphics{Figures/sne_light_curves.jpeg}
    \caption{Καμπύλες φωτός υπερκαινοφανών αστέρων}
    \label{fig:sne_light_curves}
\end{figure}

\subsubsection{Καταστροφικοί - Καινοφανείς}
Σύμφωνα με το γενικά παραδεκτό σήμερα σενάριο, οι καινοφανείς αστέρες προέρχονται από την εξέλιξη διπλών συστημάτων. Σε ένα διπλό σύστημα με δύο αστέρες 1 και 2, που έχουν μάζες $M_1 > M_2$, ο αστέρας 1 εγκαταλείπει πρώτος την κύρια ακολουθία και αρχίζει να εξελίσσεται σε ερυθρό γίγαντα. Αν η απόσταση των δύο αστέρων ειναι αρκετά μικρή, τότε ο αστέρας 1 γεμίζει τον δικό του λοβό Roche, και μάζα από τα εξωτερικά του στρώματα μεταφέρεται στον αστέρα 2. 'Οταν τελειώσει αυτή η διαδικασία, τότε από τον αστέρα 1 δεν έχει απομείνει παρά μόνο ο πυρήνας του, ο οποίος τελικά καταλήγει (αν η μάζα του δεν υπερβαίνει το όριο Chandrasekhar) σε λευκό νάνο μάζας $M \sim 1\,M_\odot$. Παράλληλα ο αστέρας 2 έχει γίνει ήδη πολύ μεγαλύτερος (λόγω της προσαύξησης μάζας από τον αστέρα 1) και αρχίζει να εξελίσσεται με τη σειρά του σε ερυθρό γίγαντα. Κατά την εξέλιξη αυτή γεμίζει με τη σειρά του τον δικό του λοβό Roche, και αρχίζει πλέον η αντίστροφη μεταφορά μάζας από τον αστέρα 2 στον αστέρα 1 (λευκό νάνο).

'Οταν η μάζα της ύλης που συσσωρεύεται στη θερμή ατμόσφαιρα του λευκού νάνου ξεπεράσει κάποια κρίσιμη τιμή, τότε η πίεση και η θερμοκρασία στη βάση της ατμόσφαιράς του επιτρέπουν τη καύση του υδρογόνου σε ήλιο, αντίδραση που συμβαίνει σχεδόν ακαριαία. Η ενέργεια που εκλύεται από την έναρξη της θερμοπυρηνικής αυτής αντίδρασης απελευθερώνεται με τη μορφή μιας μεγάλης έκρηξης, που μπορεί να ανεβάσει τη λαμπρότητα του συστήματος κατά 10 μεγέθη, σε διάστημα μερικών ωρών.

Εφόσον η ροή ύλης από τον αστέρα 2 στον αστέρα 1 συνεχίζεται, μια τέτοια έκρηξη θα επαναλαμβάνεται κάθε φορά που η συσσωρευμένη μάζα ξεπερνά την κρίσιμη τιμή. Αν ο ρυθμός μεταφοράς είναι μεγάλος, τότε το πρότυπο αυτό προβλέπει ότι η κρίσιμη τιμή της μάζας είναι μικρή, επειδή η ύλη δεν προλαβαίνει να ``απλωθεί'' σε μεγάλη έκταση. Στην περίπτωση αυτή οι εκρήξεις αυτές επαναλαμβάνονται σε σύντομα χρονικά διαστήματα, αλλά είναι μικρές. Αντίθετα, αν ο ρυθμός μεταφοράς είναι μικρός, τότε οι εκρήξεις απέχουν χρονικά πολύ μεταξύ τους, αλλά είναι μεγάλες.

Σύμφωνα με το παραπάνω σενάριο, όλοι οι καινοφανείς πρέπει να είναι α) διπλοί και β) επαναληπτικοί. Τα μέχρι τώρα παρατηρησιακά δεδομένα υποστηρίζουν ικανοποιητικά την παραπάνω θεωρία σχηματισμού των υπερκαινοφανών. Ας σημειωθεί όμως ότι υπάρχουν και θεωρίες για την προέλευση των υπερκαινοφανών εκρήξεων που βασίζονται στην εξέλιξη απλών (μεμονωμένων) αστέρων, και δεν αποκλείεται μερικοί από τους καινοφανείς που παρατηρούμε να οφείλεται σε τέτοιες περιπτώσεις.

\subsubsection{Καταστροφικοί - Υπερκαινοφανείς I}
Για τους υπερκαινοφανείς τύπου I δεν υπάρχει σήμερα μια θεωρία τόσο γενικά παραδεκτή, όσο για τους υπερκαινοφανείς τύπου II ή τους καινοφανείς. Το πιο δημοφιλές σενάριο είναι παρόμοιο με αυτό των καινοφανών. Υπάρχει δηλαδή ένα διπλό σύστημα, το ένα μέλος του οποίου είναι ένας λευκός νάνος άνθρακα-οξυγόνου. Μάζα μεταφέρεται από τον συνοδό αστέρα στον λευκό νάνο και, έτσι, η μάζα του τελευταίου συνεχώς αυξάνει χωρίς την παρουσία εκρηκτικών αναλάμψεων αυτή το φορά. 'Οταν η μάζα του λευκού νάνου πλησιάζει το όριο Chandrasekhar, προκαλειται ανάφλεξη του άνθρακα στο εσωτερικό του λευκού νάνου και η παραγόμενη ενέργεια  θερμαίνει και εκτινάσσει τα ανώτερα στρώματα του αστέρα στο διάστημα. Η έκρηξη αυτή που ονομάζεται υπερκαινοφανής τύπου I είναι τόσο σφοδρή ώστε διαλύει το σύστημα και δεν αφήνει πίσω κάποιο συμπαγές αστρικό αντικείμενο, όπως π.χ. έναν αστέρα νετρονίων.

Το γεγονός ότι η έκρηξη συμβαίνει πάντα κοντά στην ίδια τιμή μάζας (όριο Chandrasekhar) σημαίνει ότι η ενέργεια από μία τέτοια έκρηξη (συγκεκριμένα από υπερκαινοφανείς τύπου Ia) είναι γνωστή (της τάξης των $10^{51}\,\text{erg}$). Αυτό μας επιτρέπει να χρησιμοποιήσουμε αυτές τις εκρήξεις ως ``σταθερά κεριά'' (standard candles) των οποίων γνωρίζουμε την λαμπρότητα, και άρα μπορούμε να τα χρησιμοποιήσουμε ως δείκτες για την μέτρηση αποστάσεων.


\subsubsection{Καταστροφικοί - Υπερκαινοφανείς II}
Σύμφωνα με την επικρατούσα σήμερα άποψη, η κατάληξη ενός αστέρα μεγάλης μάζας ($M > 5\,M_\odot$) είναι ένας υπερκαινοφανής τύπου II. Για αστέρες με μάζα στο διάστημα $5 < M/M_\odot < 10$ ο μηχανισμός του φαινομένου οφείλεται στην \textbf{έκρηξη άνθρακα} (carbon detonation) η οποία είναι παρόμοια με την λάμψη ηλίου που έχουμε περιγράψει αλλά πολύ πιο βίαιη. Για αστέρες με μάζα $M > 10\,M_\odot$ ο μηχανισμός του φαινομένου οφείλεται στην καταστροφική κατάρρευση ενός αδρανούς πυρήνα σιδήρου.

'Ενας αστέρας που έχει εξαντλήσει όλα τα πυρηνικά του καύσιμα (έχει δηλαδή δημιουργήσει έναν πυρήνα από σίδηρο), αρχίζει να καταρρέει, αφού δεν υπάρχει εσωτερική πηγή πίεσης που να αντισταθμίζει το βάρος των υπερκείμενων στρωμάτων. Επειδή η μάζα του πυρήνα είναι μεγαλύτερη από το όριο Chandrasekhar, η κατάρρευση δεν μπορεί να σταματήσει με τη δημιουργία ενός λευκού νάνου, και έτσι η κατάρρευση συνεχίζεται προς κατάστάσεις που χαρακτηρίζονται από ολοένα και μεγαλύτερη πυκνότητα. Η αδιαβατική συμπίεση του πυρήνα, που οφείλεται στη συνεχιζόμενη κατάρρευση, προκαλεί αύξηση της θερμοκρασίας του, η οποία ευνοεί τις ενδόθερμες πυρηνικές αντιδράσεις. 'Ετσι ο σίδηρος φωτοδιασπάται από τα φωτόνια υψηλών ενεργειών σύμφωνα με την αντίδραση
$$\rm Fe^{56} + h\nu \longrightarrow \rm 13 He^4 + 4n$$
ενώ στη συνέχεια διασπάται και το παραγόμενο ήλιο
$$\rm He^4 + h\nu \longrightarrow \rm 2p + 2n$$
Οι αντιδράσεις αυτές απορροφούν ενέργεια, ανακόπτουν την αύξηση της θερμοκρασίας και της πίεσης του πυρήνα και επιταχύνουν την κατάρρευση. Στα τελευταία στάδια η κατάρρευση επιταχύνεται ακόμη περισσότερο από τις πυρηνικές αντιδράσεις σύλληψης ηλεκτρονίου
$$\rm p + e^{-} \longrightarrow \rm n + \nu_e$$

Τα νετρίνα που παράγονται από τις παραπάνω αντιδράσεις διαφεύγουν έξω από τον πυρήνα, λόγω της μικρής ενεργού διατομής της αλληλεπίδρασης των νετρίνων με την ύλη (αδιαφάνεια ύλης στα νετρίνα $\kappa \simeq 0$). Με τη διαφυγή τους μεταφέρουν μεγάλα ποσά ενέργειας εκτός του πυρήνα και έτσι μετριάζουν ακόμα περισσότερο την πιεση και τη θερμοκρασία του. Κάτω από αυτές τις συνθήκες, η δύναμη της βαρύτητας σε κάθε στρώμα του αστέρα είναι πολύ μεγαλύτερη από τη δύναμη της πίεσης και η κατάρρευση των διάφορων στρωμάτων του αστέρα μοιάζει με ελεύθερη πτώση, η οποία για τον πυρήνα διαρκεί μερικά δευτερόλεπτα. Φυσικά η κατάρρευση αυτή, κατά την οποία η δυναμική βαρυτική ενέργεια της ύλης του αστέρα μετατρέπεται σε κινητική ενέργεια, δεν μπορεί να συνεχισθεί επ' άπειρο. 'Οταν η πυκνότητα του αστρικού πυρήνα φτάσει τα όρια της πυρηνικής πυκνότητας ($\sim 10^{14}\,\text{g cm$^{-3}$}$), η αδιαφάνεια των νετρίνων αυξάνει απότομα (η μέση ελεύθερη διαδρομή τους μειώνεται), έτσι ώστε να μην μπορούν πλέον να διαφύγουν ελεύθερα, μεταφέροντας ενέργεια έξω από τον πυρήνα. Επειδή ο πυρήνας δεν μπορεί να ψυχθεί μέσω της διαφυγής νετρίνων, η θερμοκρασία του αρχίζει να αυξάνει αδιαβατικά, προκαλώντας την αλματώδη αύξηση της πίεσης του αερίου στον πυρήνα, αλλά κυρίως την σχεδόν εκρηκτική αύξηση της πίεσης της ακτινοβολίας. Στο σημείο αυτό υπάρχουν δύο ενδεχόμενα:
\begin{enumerate}
    \item Η αρχική μάζα του αστέρα να ήταν εξαιρετικά μεγάλη. Στην περίπτωση αυτή η δύναμη της βαρύτητας παραμένει πάντα μεγαλύτερη της δύναμης της πίεσης, με αποτέλεσμα ο αστέρας να υποστεί ολοκληρωτική βαρυτική κατάρρευση και να καταλήξει σε μια μελανή οπή.
    \item Η αρχική μάζα του αστέρα δεν ήταν τόσο μεγάλη ώστε να μιλάμε για ολοκληρωτική νίκη της βαρύτητας. Στην περίπτωση αυτή η αύξηση της πίεσης δεν είναι δυνατόν να συνεχίζεται επ' άπειρον: φτάνει κάποια στιγμή, που η δύναμη της πίεσης προς τα έξω υπερισχύει κατά πολύ της δύναμης της βαρύτητας οπότε τα εξωτερικά στρώματα του πυρήνα ``αναπηδούν'', όπως μια ελαστική σφαίρα που προσκρούει σε μια σκληρή επιφάνεια, και αρχίζουν να διαστέλλονται με υπερηχητική ταχύτητα. Κατά την διαστολή τους δημιουργούν ένα κρουστικό κύμα το οποίο θερμαίνει και παρασύρει προς τα έξω τα υπόλοιπα στρώματα του αστέρα, που συνέχιζαν να καταρρέουν, και τα εκτινάσσει στο διάστημα. Ταυτόχρονα, τα άφθονα νετρόνια που είχαν δημιουργηθεί από τη φωτοδιάσπαση του σιδήρου και του ηλίου απορροφούνται από πυρήνες μεσαίου ατομικού αριθμού και σχηματίζουν όλα τα χημικά στοιχεία βαρύτερα του σιδήρου, τα οποία δεν είναι δυνατόν να σχηματιστούν με εξώθερμες θερμοπυρηνικές αντιδράσεις. 'Ενας υπερκαινοφανής τύπου II έχει μόλις δημιουργηθεί. Το μόνο που είναι δυνατόν αν απομείνει από την τεράστια αυτή έκρηξη είναι το κεντρικό τμήμα του αστρικού πυρήνα, που πιστεύουμε ότι σταθεροποιείται στην κατάσταση ενός αστέρα νετρονίων.
\end{enumerate}

  
    
    
%     \appendix
%     \chapter{Μαθηματικές Μέθοδοι}
\label{apx:math_tools}

Σε αυτό το παράρτημα παρουσιάζονται μερικά απαραίτητα μαθηματικά εργαλεία που θα χρησιμοποιηθούν για την απόδειξη βασικών νόμων.

\section{Διωνυμικοί και Πολυωνυμικοί συντελεστές}

\subsection{Διωνυμικοί συντελεστές}
Στην βασική άλγεβρα, το διωνυμικό θεώρημα περιγράφει την ανάπτυξη ενός πολυωνύμου της μορφής $(x+y)^n$, σε άθροισμα όρων της μορφής $a x^b y^c$, όπου οι εκθέτες b,c είναι μη-αρνητικοί αριθμοι και $b+c=n$. Ο συντελεστής $a$ ονομάζεται \textit{διωνυμικός συντελεστής} (binomial coefficient) και εξαρτάται από το n και το b (ή το c, καθώς όπως θα δούμε δεν αλλάζει το αποτέλεσμα).

Ο διωνυμικός συντελεστής $a = {n \choose b} = \binom{n}{c}$ προκύπτει από το πεδίο της Συνδυαστικής και εκφράζει τον αριθμό των διαφορετικών συνδυασμών $b$ στοιχείων που μπορεί να προκύψουν από ένα σύνολο $n$ στοιχείων. Ο διωνυμικός συντελεστής διαβάζεται ως "n ανά b" επειδή υπάρχουν ${n \choose b}$ δυνατοί τρόποι για να επιλεγούν $b$ στοιχεία από ένα σύνολο $n$ στοιχείων. Με τη χρήση παραγοντικών, ο διωνυμικός συντελεστής γράφεται 
\begin{equation}
    \label{eq:apx:binomial_coefficient_factorial}
    {n \choose k} = \frac{n!}{k! (n-k)!}
\end{equation}

Έτσι, μπορούμε να γράψουμε
\begin{eqnarray*}
    (x+y)^n &=& {n \choose 0}x^0 y^n + {n\choose 1} x^1 y^{n-1} + {n \choose 2} x^2 y^{n-2} + \dots + \\\\
    &+& {n \choose n-2}x^{n-2} y^2 + {n \choose n-1} x^{n-1} y^1 + {n \choose n} x^n y^0
\end{eqnarray*}
ή, σε πιο συμπαγή μορφή

\begin{equation}
    \label{eq:apx:general_form}
    (x+y)^n = \sum_{k=0}^{n} {n \choose k} x^k y^{n-k} = \sum_{k=0}^{n} {n \choose k} x^{n-k} y^k
\end{equation}

\textbf{Παράδειγμα}\\

Έστω ένα σύνολο τεσσάρων αριθμών $\{1,2,3,4\}$. Θέλουμε να βρούμε πόσους συνδυασμούς των δύο μπορούμε να επιλέξουμε. Αυτός ο αριθμός των 2-υποσυνόλων θα είναι
\begin{equation}
    {4 \choose 2} = \frac{4!}{2! (4-2)!} = 6
\end{equation}

Άρα δωθέντος τεσσάρων αριθμών, υπάρχουν έξι δυνατοί συνδυασμοί υποσυνόλων των δύο, τα οποία είναι
$\{1,2\}, \{1,3\}, \{1,4\}, \{2,3\}, \{2,4\}, \{3,4\}$.


\subsection{Πολυωνυμικοί συντελεστές}
Έχουμε δεί ότι αν θέλουμε να επιλέξουμε $k$ αντικείμενα από $n$, χωρίς επανάθεση, το πλήθος των τρόπων να γίνει αυτό είναι
$${n \choose k} = \frac{n!}{k!(n-k)!}$$
Τι γίνεται αν θέλουμε να επιλέξουμε, πάλι χωρίς επανάθεση, μια ομάδα στοιχείων του $ {\left\{{1,\ldots,n}\right\}}$ μεγέθους $ k_1$, μια ομάδα μεγέθους $ k_2$, κλπ, και τέλος μια ομάδα μεγέθους $ k_r$, όπου για $ j=1,\ldots,r$ έχουμε $ 0 \le k_j \le n$ και επιπλέον ισχύει $ k_1+\cdots +k_r = n$; Με πόσους τρόπους δηλ. μπορούμε να διαμερίσουμε το $ {\left\{{1,\ldots,n}\right\}}$ σε ένα σύνολο μεγέθους $ k_1$, σε ένα σύνολο μεγέθους $ k_2$ και τέλος σε ένα σύνολο μεγέθους $ k_r$;

\textit{\textbf{Θεώρημα}}:\\
Το πλήθος τρόπων να διαμερίσουμε ένα σύνολο με $ n$ στοιχεία σε $ r$ σύνολα με μεγέθη $ k_1,\ldots,k_r$, με $ k_1+\cdots +k_r = n$, όταν δε μας ενδιαφέρει η σειρά των στοιχείων μέσα στα σύνολα αυτά, είναι
\begin{equation}
    \label{eq:apx:polynomial_coefficient}
    {n \choose k_1, \dots, k_r} = \frac{n!}{k_1! k_2! \dots k_r!}
\end{equation}

\textit{\textbf{Απόδειξη}}:\\
Το πρώτο σύνολο μπορεί να επιλεγεί με
$${n \choose k_1} = \frac{n!}{k_1! (n-k_1)!}$$
τρόπους. Μετά από την επιλογή του πρώτου συνόλου απομένουν $ n-k_1$ στοιχεία αχρησιμοποίητα, άρα το δεύτερο σύνολο μπορεί να επιλεγεί με
$${n-k_1 \choose k_2} = \frac{(n-k_1)!}{n_2! (n-k_1-k_2)!}$$
τρόπους. Συνεχίζονας κατ' αυτόν τον τρόπο παίρνουμε ότι η επιλογή του προτελευταίου συνόλου (με $ k_{r-1}$ στοιχεία) μπορεί να γίνει με
\begin{align*}
    {n - k_1 - \dots - k_{r-2} \choose k_{r-1}} &= \frac{(n - k_1 - \dots - k_{r-2})!}{k_{r-1}! (n - k_1 - \dots - k_{r-2} - k_{r-1})!} = \\\\
    &= \frac{(n - k_1 - \dots - k_{r-2})!}{k_{r-1}! k_r!}
\end{align*}
τρόπους. Επίσης, αφού έχουν επιλεγεί τα $ r-1$ πρώτα σύνολα δεν υπάρχει πλέον καμιά επιλογή να γίνει αφου τα υπόλοιπα $ k_r$ στοιχεία που απομένουν ακόμη αχρησιμοποίητα αναγκαστικά πάνε στο τελευταίο σύνολο που πρέπει να επιλέξουμε.
Έτσι πολλαπλασιάζοντας τις δυνατότητες επιλογών μας για τα πρώτα $ r-1$ σύνολα, και κάνοντας τις απλοποιήσεις παίρνουμε τον τύπο \eqref{eq:apx:polynomial_coefficient}.

Το σύμβολο $ {n \choose k_1,\ldots, k_r}$ ονομάζεται πολυωνυμικός συντελεστής (κατ' αναλογία με τα $ {n \choose k}$ που ονομάζονται διωνυμικοί συντελεστές). Παρατηρήστε επίσης ότι

$$\displaystyle {n \choose k, n-k} = {n \choose k} = {n \choose n-k}.$$




\textbf{Παρατήρηση 1}: Ο πολυωνυμικός συντελεστής $ {n \choose k_1,\ldots, k_r}$ δεν αλλάζει αν τα $ k_1,\ldots,k_r$ αντικατασταθούν από μια μετάθεσή τους (αν αλλάξει δηλ. απλώς η σειρά τους).

\textbf{Παρατήρηση 2}: Πρέπει να τονίσουμε εδώ ότι, αν και δε μας ενδιαφέρει η εσωτερική σειρά των συνόλων των στοιχείων που επιλέγουμε, η σειρά των ίδιων των συνόλων είναι προκαθορισμένη. Αυτό είναι ίσως φανερό όταν όλα τα $ k_1, k_2, \ldots, k_m$ είναι μεταξύ τους διαφορετικά αλλά δημιουργεί κάποια σύγχυση όταν μερικά από αυτά είναι μεταξύ τους ίσα. Μια ακραία περίπτωση αυτού είναι όταν όλα είναι ίδια. Για παράδειγμα, ο πολυωνυμικός συντελεστής
$$\displaystyle {9 \choose 3, 3, 3}$$
μετράει με πόσους τρόπους μπορούμε να χωρίσουμε τους αριθμούς $ 1,2,\ldots,9$ σε τρείς ομάδες. Αν δύο τρόποι διαφέρουν μόνο ως προς τον εσωτερικό τρόπο γραφής της κάθε ομάδας τότε δε θεωρούνται διαφορετικοί. Έτσι οι τρόποι
$$ {\left\{{1,2,3}\right\}}, {\left\{{4,5,6}\right\}}, {\left\{{7,8,9}\right\}}
\hspace{0.25cm} \text{και} \hspace{0.25cm} {\left\{{3,2,1}\right\}}, {\left\{{4,5,6}\right\}}, {\left\{{7,8,9}\right\}}
$$
θεωρούνται ίδιοι και μετράνε ως ένα. Αν όμως δύο τρόποι διαφέρουν ως προς τον τρόπο γραφής των ομάδων τότε μετράνε ως διαφορετικοί. Οι τρόποι, π.χ.,
$$ {\left\{{1,2,3}\right\}}, {\left\{{4,5,6}\right\}}, {\left\{{7,8,9}\right\}}
\hspace{0.25cm} \text{και} \hspace{0.25cm} {\left\{{4,5,6}\right\}}, {\left\{{1,2,3}\right\}}, {\left\{{7,8,9}\right\}}
$$
μετράνε ως διαφορετικοί τρόποι.


\section{Πολλαπλασιαστές Lagrange}
Η μέθοδος των πολλαπλασιαστών Lagrange χρησιμοποιείται για την εύρεση ακρότατων μίας συνάρτησης $f(x,y,z)$  των οποίων οι μεταβλητές υπόκεινται σε περιορισμούς της μορφής $g_i(x,y,z) = 0, \ i=1,2, \dots r$.  Τα 
τοπικά ακρότατα προκύπτουν επιλύοντας το παρακάτω σύστημα εξισώσεων

\[
\begin{cases} 
\nabla f = \sum_{i=1}^{r} \lambda_i \nabla g_i \\ \\
g_i(x,y,z) = 0, \ \forall i=1,\dots,r 
\end{cases}
\]
ως προς $x,y,z,\lambda_1, \dots \lambda_r$.

\textbf{Παραδείγματα}\\

\begin{enumerate}
    \item\textbf{ Βρείτε τα μέγιστα και ελάχιστα της $f(x,y) = x^2 + y^2$ που βρίσκονται στην καμπύλη $g(x,y) = x^2 - 2x + y^2 - 4y = 0$.}
    
        Ισχύει ότι:
        \begin{align*}
            \nabla f & = \frac{\partial f}{\partial x} + \frac{\partial f}{\partial y} = 2x + 2y \longrightarrow \nabla f = (f_x, f_y) = (2x,2y) \\\\
            \nabla g & = (g_x, g_y) = (2x-2, 2y-4) \\\\
        \end{align*}
        Άρα, σύμφωνα με τη μέθοδο των πολλαπλασιστών Lagrange πρέπει να λύσω το σύστημα:
        \begin{align*}
            \begin{cases}
                \nabla f = \lambda \nabla g \\\\
                g(x,y) = 0
            \end{cases} &&\Rightarrow
            \begin{cases}
                2x = 2\lambda (x-1) \\\\
                2y = 2\lambda (y-2) \\\\
                x^2 -2x + y^2 - 4y = 0
            \end{cases} &\Rightarrow
            \begin{cases}
                x = \frac{\lambda}{\lambda - 1} \\\\
                y = \frac{2\lambda}{\lambda - 1} \\\\
                x^2 -2x + y^2 - 4y = 0
            \end{cases}
        \end{align*}
        Αντικαθιστώντας τις παραμετρικές τιμές των $x,y$ στην συνάρτηση $g(x,y) = 0$ καταλήγουμε ότι:
        
        \begin{align*}
            \frac{-5\lambda^2 + 10 \lambda}{(\lambda - 1)^2} = 0, \ \lambda \neq 1 \ \Rightarrow \\\\
            -5\lambda (\lambda - 2 ) = 0 \Rightarrow \begin{cases} \lambda = 0 \\\\ \lambda = 2 \end{cases}
        \end{align*}
    
      Άρα, για $\lambda = 0 \longrightarrow (x,y) = (0,0)$, η συνάρτηση $f$ παρουσιάζει πάνω στην $g(x,y) = 0$ ελάχιστο $f(0,0) = 0$. Για $\lambda = 2 \longrightarrow (x,y) = (2,4)$ και η συνάρτηση $f$ παρουσιάζει πάνω στην $g(x,y) = 0$ ελάχιστο $f(2,4) = 20$.
    
    \item \textbf{Βρείτε το μέγιστο της $f(x,y,z) = x^2 + 2y - z^2$ με τους περιορισμούς $2x-y=0$ και $y+z=0$.}
    
        Έστω $g_1(x,y) = 2x-y = 0$, $g_2(y,z) = y+z = 0$ οι δύο συναρτήσεις που δίνουν τους περιορισμούς. Έχουμε λοιπόν:
        \begin{align*}
            \nabla f & = (f_x, f_y, f_z) = (2x, 2, -2z) \\\\
            \nabla g_1 & = (g_{1x}, g_{1y}) = (2, -1) \longrightarrow \lambda_1 \nabla g_1 = (2\lambda_1, - \lambda_1) \\\\
            \nabla g_2 & = (g_{2y}. g_{2z}) = (1,1) \longrightarrow \lambda_2 \nabla g_2 = (\lambda_2, \lambda_2)
        \end{align*}
        Ισχύει ότι:
        \begin{align*}
            \nabla f = \sum_{i=1}^{2} \lambda_i \nabla g_i = (2\lambda_1, -\lambda_1 + \lambda_2, \lambda_2) = (2x, 2, -2z)
        \end{align*}
        Έτσι, προκύπτει το σύστημα:
        \begin{align*}
            \begin{cases}
                2x = 2\lambda_1 \\
                2 = - \lambda_1 + \lambda_2 \\
                -2z = \lambda_2 \\
                2x-y = 0 \\
                y+z = 0
            \end{cases} & \Rightarrow
            \begin{cases}
                \lambda_1  = 2/3 \\
                \lambda_2 = 8/3 \\
                x = 2/3 \\
                y = 4/3 \\
                z = -4/3
            \end{cases}
        \end{align*}
        
        Συνεπώς, η μέγιστη τιμή της $f$ υπό τους δοθέντες περιορισμούς είναι $f(2/3, 4/3, -4/3) = 4/3$
    \item \textbf{Σχεδιάστε ένα μεταλλικό κυλινδρικό δοχείο (με καπάκι) 1 λίτρου, χρησιμοποιώντας την ελάχιστη δυνατή ποσότητα μετάλλου.}
    
        Ο όγκος του κυλίνδρου δίνεται από τη συνάρτηση $$V(\rho, \upsilon) = \pi \rho^2 \upsilon$$ και η συνολική του επιφάνεια (παράπλευρη επιφάνεια και βάσεις) δίνεται από τη συνάρτηση $$S(\rho, \upsilon) = 2 \pi \rho \upsilon + 2 \pi \rho^2$$ όπου $\rho$ και $\upsilon$ είναι η ακτίνα της βάσης και το ύψος του κυλίνδρου, αντίστοιχα. Άρα, καλούμαι να βρω το ελάχιστο της $S(\rho, \upsilon)$ υπό τον περιορισμό $g(\rho, \upsilon) = V(\rho, \upsilon) - 1 = \pi \rho^2 \upsilon - 1 = 0$.
        
        Έτσι έχουμε:
        \begin{align*}
            \nabla S &= (S_{\rho}, S_{\upsilon}) = (2\pi \upsilon + 4 \pi \rho , 2\pi \rho) \\\\
            \nabla V &= (V_{\rho}, V_{\upsilon}) = (2\pi \rho \upsilon, \pi \rho^2) \longrightarrow \lambda \nabla V = (2 \lambda \pi \rho \upsilon,  \lambda \pi \rho^2)
        \end{align*}
        και το σύστημα που πρέπει να λύσω είναι:
        \begin{align*}
            \begin{cases}
                \nabla S = \lambda \nabla V \\
                \pi \rho^2 \upsilon - 1 = 0
            \end{cases} &\Rightarrow
            \begin{cases}
                \upsilon + 2 \rho = \lambda \rho \upsilon \\
                \lambda \rho = 2 \\
                \pi \rho^2 \upsilon - 1 = 0
            \end{cases} &\Rightarrow
            \begin{cases}
                \displaystyle \rho = \frac{1}{\sqrt[3]{2\pi}} \\\\
                \displaystyle \upsilon = \frac{2}{\sqrt[3]{2\pi}} \\\\
                \lambda = 2 \sqrt[3]{2\pi}
            \end{cases}
        \end{align*}
        δηλαδή το ζητούμενο κυλινδρικό δοχείο πρέπει να έχει ακτίνα βάσης $\displaystyle \rho = \frac{1}{\sqrt[3]{2\pi}}$ και ύψος $\displaystyle \upsilon = \frac{2}{\sqrt[3]{2\pi}}$.
\end{enumerate}
















%     \chapter{Συστήματα Συντεταγμένων}
\label{apx:coordinates}

\section{Καρτεσιανές συντεταγμένες}
Cartesian coordinates allow one to specify the location of a point in the plane, or in three-dimensional space. The Cartesian coordinates (also called rectangular coordinates) of a point are a pair of numbers (in two-dimensions) or a triplet of numbers (in three-dimensions) that specified signed distances from the coordinate axis.

\subsection{Καρτεσιανές συντεγαμένες στο επίπεδο}
The Cartesian coordinates in the plane are specified in terms of the x coordinates axis and the y-coordinate axis, as illustrated in the below figure (Σχήμα \ref{fig:apxA_cartesian2D}). The origin is the intersection of the x and y-axes. The Cartesian coordinates of a point in the plane are written as (x,y). The first number x is called the x-coordinate (or x-component), as it is the signed distance from the origin in the direction along the x-axis. The x-coordinate specifies the distance to the right (if x is positive) or to the left (if x is negative) of the y-axis. Similarly, the second number y is called the y-coordinate (or y-component), as it is the signed distance from the origin in the direction along the y-axis, The y-coordinate specifies the distance above (if y is positive) or below (if y is negative) the x-axis. The following figure, the point has coordinates (-3,2), as the point is three units to the left and two units up from the origin.

\begin{figure}[h]
    \centering
    \includegraphics[scale=0.5]{Figures/appendixA_cartesian2D.png}
    \caption{Καρτεσιανές συντεταγμένες στο επίπεδο. The Cartesian coordinates (x,y) of the blue point specify its location relative to the origin, which is the intersection of the x- and y-axis.}
    \label{fig:apxA_cartesian2D}
\end{figure}


\subsection{Καρτεσιανές συντεταγμένες στον χώρο}
In three-dimensional space, the Cartesian coordinate system is based on three mutually perpendicular coordinate axes: the x-axis, the y-axis, and the z-axis, illustrated below. The three axes intersect at the point called the origin. You can imagine the origin being the point where the walls in the corner of a room meet the floor. The x-axis is the horizontal line along which the wall to your left and the floor intersect. The y-axis is the horizontal line along which the wall to your right and the floor intersect. The z-axis is the vertical line along which the walls intersect.

\begin{figure}[h]
    \centering
    \includegraphics[scale=0.3]{Figures/appendixA_cartesian3D.jpg}
    \caption{Καρτεσιανές συντεταγμένες στον χώρο.}
    \label{fig:apxA_cartesian3D}
\end{figure}

With above definitions of the positive x, y, and z-axis, the resulting coordinate system is called right-handed; if you curl the fingers of your right hand from the positive x-axis to the positive y-axis, the thumb of your right hand points in the direction of the positive z-axis. Switching the locations of the positive x-axis and positive y-axis creates left-handed coordinate system. The right-handed and left-handed coordinate systems represent two equally valid mathematical universes. The problem is that switching universes will change the sign on some formulas. Since these pages are written in the right-handed universe, we suggest you live in our universe while studying from these pages.

In addition to the three coordinate axes, we often refer to three coordinate planes. The xy-plane is the horizontal plane spanned by the x and y-axes. It is identical to the two-dimensional coordinate plane and contains the floor in the room analogy. Similarly, the xz-plane is the vertical plane spanned by the x and z-axes and contains the left wall in the room analogy. Lastly, the yz-plane is the vertical plane spanned by the y and the z-axis and contains the right wall in the room analogy.
\\
{\color{red} \hrule}
Cartesian coordinates can be used not only to specify the location of points, but also to specify the coordinates of vectors. The Cartesian coordinates of two or three-dimensional vectors look just like those of points in the plane or three-dimensional space.

Από το σχήμα \ref{fig:apxA_cartesian3D} προκύπτει ότι το διάνυσμα θέσεως ,$\boldsymbol{a}$, θα δίνεται από τη σχέση: $$\boldsymbol{a} = a_x \boldsymbol{i} + a_y \boldsymbol{j} + a_z \boldsymbol{k}$$

But, there is no reason to stop at three-dimensions. We could define vectors in four, five, or higher dimensions by just specifying four, five, or more Cartesian coordinates. We can't visualize these higher dimensions like we did with the above applets, but we can easily write down the list of numbers for the coordinates.\\
{\color{red} \hrule}


\section{Πολικές συντεταγμένες}
In two dimensions, the Cartesian coordinates (x,y) specify the location of a point P in the plane. Another two-dimensional coordinate system is polar coordinates. Instead of using the signed distances along the two coordinate axes, polar coordinates specifies the location of a point P in the plane by its distance r from the origin and the angle $\theta$ made between the line segment from the origin to P and the positive x-axis. The polar coordinates $(r, \theta)$ of a point P are illustrated in the below figure (Σχήμα \ref{fig:apxA_polar_coordinates}).

\begin{figure}[h]
   \centering
\begin{subfigure}[h]{0.45\textwidth}
	\centering
   	 \includegraphics[width = \linewidth]{Figures/appendixA_polar_coordinates_whole.png} 
\end{subfigure}
\begin{subfigure}[h]{0.4\textwidth}
	\centering
	\includegraphics[width = \linewidth]{Figures/appendixA_polar_coordinates.png} 
    \end{subfigure}
    \caption{Πολικές συντεταγμένες στο επίπεδο.}
    \label{fig:apxA_polar_coordinates}
\end{figure}

As r ranges from 0 to infinity and $\theta$ ranges from 0 to $2\pi$, the point P specified by the polar coordinates $(r, \theta)$ covers every point in the plane. Adding $2\pi$ to $\theta$ brings us back to the same point, so if we allowed $\theta$ to range over an interval larger than $2\pi$, each point would have multiple polar coordinates. Hence, we typically restrict $\theta$ to be in the interval $0\leq \theta \leq 2\pi$. However, even with that restriction, there still is some non-uniqueness of polar coordinates: when $r=0$, the point P is at the origin independent of the value of $\theta$.

We can calculate the Cartesian coordinates of a point with polar coordinates $(r, \theta)$ by forming the right triangle illustrated in Figure \ref{fig:apxA_polar_coordinates}.  The hypotenuse is the line segment from the origin to the point, and its length is r. The projection of this line segment on the x-axis is the leg of the triangle adjacent to the angle $\theta$, so $x=r \cos \theta$. The y-component is determined by the other leg, so $y = r \sin \theta$. Our conversion formula is:
\begin{eqnarray}
    x &=& r \cos \theta \\
    y &=& r \sin \theta 
\end{eqnarray}

Για τους αντίστροφους μετασχηματισμούς προκύπτει από τις παραπάνω σχέσεις ότι:
\begin{eqnarray}
    x^2 + y^2 &=& r^2 (\cos^2 \theta + \sin^2 \theta) = r^2 \Rightarrow r = \sqrt{x^2 + y^2} \\ \nonumber \\
    \frac{y}{x} &=& \frac{r \sin \theta}{r \cos \theta} = \tan \theta \Rightarrow \theta = \arctan \left( \frac{y}{x} \right)
\end{eqnarray}

Παρατηρούμε ότι με βάση τη σχέση Α.4 για το σημείο $(x,y) = (0,0)$, η γωνία $\theta$ δεν ορίζεται. Σε αυτή την περίπτωση όμως παίρνουμε ότι $\theta = 0$.


\section{Κυλινδρικές συντεταγμένες}
Cylindrical coordinates are a simple extension of the two-dimensional polar coordinates to three dimensions. They simply combine the polar coordinates in the xy-plane with the usual z coordinate of Cartesian coordinates. To form the cylindrical coordinates of a point P, simply project it down to a point Q in the xy-plane (Σχήμα \ref{fig:apxA_cylindrical_coordinates}). Then, take the polar coordinates $(r, \theta)$ of the point Q, i.e., r is the distance from the origin to Q and $\theta$ is the angle between the positive x-axis and the line segment from the origin to Q. The third cylindrical coordinate is the same as the usual z-coordinate. It is the signed distance of the point P to the xy-plane (being negative is P is below the xy-plane). The below figure illustrates the cylindrical coordinates $(r, \theta, z)$ of the point P.

\begin{figure}[h]
    \centering
    \includegraphics[scale=0.5]{Figures/appendixA_cylindrical_coordinates.png}
    \caption{Κυλινδρικές συντεταγμένες στον χώρο.}
    \label{fig:apxA_cylindrical_coordinates}
\end{figure}

Οι μετασχηματισμοί δίνονται από τις σχέσεις:
\begin{eqnarray}
    x &=& r \cos \theta \\
    y &=& r \sin \theta \\
    z &=& z
\end{eqnarray}
ενώ για τους αντίστροφους μετασχηματισμούς, ισχύουν οι σχέσεις Α.3 και Α.4.

\section{Σφαιρικές συντεταγμένες}
Spherical coordinates can be a little challenging to understand at first. Spherical coordinates determine the position of a point in three-dimensional space based on the distance $\rho$ from the origin and two angles $\theta$ and $\phi$. If one is familiar with polar coordinates, then the angle $\theta$ isn't too difficult to understand as it is essentially the same as the angle $\theta$ from polar coordinates. But some people have trouble grasping what the angle $\phi$ is all about\footnote{Όλο το παράρτημα είναι μετάφραση της σελίδας \url{https://mathinsight.org/spherical_coordinates}, η οποία περιέχει και διάφορα applets για την καλύτερη κατανόηση των εννοιών.}. Στη συνέχεια θα εξάγουμε τις σχέσεις μεταξύ Καρτεσιανών και σφαιρικών συντενταγμένων.

Οι σφαιρικές συντεταγμένες ορίζονται βάσει του σχήματος \ref{fig:apxA_spherical_coordinates}, το οποίο δείχνει τις σφαιρικές συντεταγμένες στο σημείο P.

\begin{figure}[h]
    \centering
    \includegraphics[scale=0.5]{Figures/appendixA_spherical_coordinates_simple.png}
    \caption{Σφαιρικές συντεταγμένες στον χώρο.}
    \label{fig:apxA_spherical_coordinates}
\end{figure}

The coordinate $\rho$ is the distance from P to the origin. If the point Q is the projection of P to the xy-plane, then $\theta$ is the angle between the positive x-axis and the line segment from the origin to Q. Lastly, $\phi$ is the angle between the positive z-axis and the line segment from the origin to P.

We can calculate the relationship between the Cartesian coordinates $(x, y, z)$ of the point P and its spherical coordinates $(\rho, \theta, \phi)$ using trigonometry. Το ροζ τρίγωνο του Σχήματος \ref{fig:apxA_spherical_coordinates_derivation} is the right triangle whose vertices are the origin, the point P, and its projection onto the z-axis. As the length of the hypotenuse is $\rho$ and $\phi$ is the angle the hypotenuse makes with the z-axis leg of the right triangle, the z-coordinate of P (i.e., the height of the triangle) is $z = \rho \cos \phi$. The length of the other leg of the right triangle is the distance from P to the z-axis, which is $r = \rho \sin \phi$. The distance of the point Q from the origin is the same quantity.

\begin{figure}[h]
    \centering
    \includegraphics[scale=0.5]{Figures/appendixA_spherical_coordinates_analytical.png}
    \caption{Σχέση Καρτεσιανών και σφαιρικών συντενταγμένων.}
    \label{fig:apxA_spherical_coordinates_derivation}
\end{figure}

The cyan triangle, shown in both the original 3D coordinate system on the left and in the xy-plane on the right, is the right triangle whose vertices are the origin, the point Q, and its projection onto the x-axis. In the right plot, the distance from Q to the origin, which is the length of hypotenuse of the right triangle, is labeled just as r. As $\theta$ is the angle this hypotenuse makes with the x-axis, the x- and y-components of the point Q (which are the same as the x- and y-components of the point P) are given by $x = r \cos \theta$ and $y = r \sin \theta$. Since $r = \rho \sin \phi$, these components can be rewritten as $x = \rho \sin \phi \cos \theta$ and $y = \rho \sin \phi \sin \theta$. In summary, the formulas for Cartesian coordinates in terms of spherical coordinates are:

\begin{eqnarray}
    x &=& \rho \sin \phi \cos \theta \\
    y &=& \rho \sin \phi \sin \theta \\
    z &=& \rho \cos \phi
\end{eqnarray}
όπου $\rho \geq 0, 0 \leq \theta \leq 2\pi, 0 \leq \phi \leq \pi$.

Unfortunately, the convention for the notation of spherical coordinates is not standardized across disciplines. For example, in physics, the roles of $\theta$ and $\phi$ are typically reversed. In order to correctly understand someone's use of spherical coordinates, you must first determine what notational convention this are using. You cannot assume they follow the convention used here.

%     \chapter{Κινητική θεωρία και εξισώσεις κατάστασης}
\label{apx:kinetic_theory}

\section{Γενικές έννοιες}
\label{apx:sec:general}
Σε αυτό το Κεφάλαιο θα δώσουμε συνοπτικά κάποιες βασικές έννοιες Θερμοδυναμικής και Στατιστικής Φυσικής που θα χρησιμοποιηθούν τόσο για την κλασική όσο και για την κβαντική περιγραφή των αερίων.

\textbf{1ος θερμοδυναμικός νόμος}: Το πρώτο θερµοδυναµικό αξίωµα µπορεί να διατυπωθεί ως εξής: "\textit{Η θερµότητα
είναι µια µορφή ενέργειας, και η ενέργεια διατηρείται}". Αν σε ένα αέριο προσθέσουµε θερµότητα, $\dbar Q$, τότε στη γενικότερη περίπτωση, θα αυξηθεί ταυτόχρονα και η εσωτερική του ενέργεια κατά $dU$ και θα µεταβληθεί ο όγκος του κατά $dV$ καταναλώνοντας ενέργεια $\dbar W = P dV$. Με άλλα λόγια, το ποσό θερμότητας που απορροφά ή αποβάλλεται από ένα θερμοδυναμικό σύστημα είναι ίσο με το αλγεβρικό άθροισμα της μεταβολής της εσωτερικής ενέργειας και του έργου που δαπανά ή παράγει το σύστημα.
Το πρώτο θερµοδυναµικό αξίωµα µε βάση τα παραπάνω µπορεί να περιγραφεί από την εξίσωση\footnote{Προσέξτε ότι τα $\dbar Q$ και $\dbar W$ είναι μη-τέλεια διαφορικά (inexact differentials). Αυτό σημαίνει ότι οι ποσότητες $Q,W$ δεν είναι καταστατικές συναρτήσεις αλλά εξαρτώνται από την διαδρομή (path functions) και άρα και τα διαφορικά τους εξαρτώνται και αυτά από τη διαδρομή.}
\begin{equation}
    \dbar Q = dU + \dbar W = dU + P dV
\end{equation}

\textbf{2ος θερμοδυναμικός νόμος}: Για μία αντιστρεπτή μεταβολή, η αλλαγή στην εντροπία ισούται με την αλλαγή στο ποσό θερμότητας δια την θερμοκρασία
\begin{equation}
    dS = \frac{\dbar Q}{T}
\end{equation}


\textbf{Εσωτερική ενέργεια}: Ονομάζεται το άθροισμα της ενέργειας όλων των ατόμων, μορίων και ιόντων ενός συστήματος. Η εσωτερική ενέργεια περιλαμβάνει πάντα τους παρακάτω όρους:
\begin{itemize}
    \item Κινητική ενέργεια λόγω της άτακτης κίνησης των σωματιδίων (γνωστή και ως θερμική κίνηση) --- translational energy
    \item Ενέργεια λόγω της περιστροφικής κίνησης των μορίων --- rotational energy
    \item Ενέργεια δόνησης των ατόμων στο μόριο --- vibrational energy
    \item Δυναμική ενέργεια λόγω των ελκτικών ή απωστικών δυνάμεων ανάμεσα στα άτομα, μόρια ή ιόντα του συστήματος --- potential energy
\end{itemize}

\textbf{Ιδανικό αέριο}: Ένα υποθετικό αέριο που αποτελείται από μη-διαχωρίσιμα σημειακά σωματίδια τα οποία συγκρούονται ελαστικά και για τα οποία οι διασωματιδιακές δυνάμεις μπορούν να αγνοηθούν. Ένα ιδανικό αέριο υπακούει στο \textbf{νόμο των ιδανικών αερίων}
\begin{equation}
    PV = NkT \Rightarrow P = \frac{\rho}{\mu m_H} kT
\end{equation}

Για το ιδανικό αέριο, θεωρούμε ότι τα σωματίδια που το αποτελούν δεν έχουν εσωτερική δομή, άρα εσωτερικοί μηχανισμοί όπως η δόνηση και η περιστροφή δεν συνεισφέρουν στην εσωτερική ενέργεια. Επίσης, η ενέργεια του σωματιδίου δεν εξαρτάται από τη θέση του, άρα δεν υπάρχει και συνεισφορά από "συντεταγμένες". Αυτό μας αφήνει μόνο με την κινητική ενέργεια. Δηλαδή, για ένα ιδανικό αέριο, η εσωτερική του ενέργεια ισούται με την κινητική ενέργεια και χαρακτηρίζεται πλήρως από αυτήν.
\begin{equation}
    \langle U \rangle = \langle E_{\text{kin}} \rangle = \frac{3}{2} NkT
\end{equation}
όπου $N$ ο αριθμός των σωματιδίων.

Στις τρεις διαστάσεις, υπάρχουν 3 ανεξάρτητες διευθύνσεις για την ορμή: $p_x, p_y, p_z$. Σύμφωνα με το θεώρημα της ισοκατανομής (equipartition theorem) η μέση ενέργεια ανά σωματίδιο
\begin{equation}
    \frac{\langle U \rangle}{N} = \frac{3}{2}kT
\end{equation}
μοιράζεται ισόποσα σε κάθε βαθμό ελευθερίας, ώστε σε κάθε διεύθυνση να αντιστοιχεί $\frac{1}{2}kT$

\textbf{Αδιαβατικές διεργασίες}: Συχνά ερχόμαστε αντιμέτωποι με διεργασίες οι οποίες συμβαίνουν σε τόσο σύντομο χρονικό διάστημα (π.χ. σε δυναμικό χρόνο) ώστε δεν υπάρχει ανταλλαγή θερμότητας με το περιβάλλον ($\dbar Q = dS = 0$). Αυτές οι διεργασίες ονομάζοναι "αδιαβατικές" και υπακούουν σε μία σχέση της μορφής
\begin{equation}
    P \propto \rho^{\gamma} 
\end{equation}
όπου το $\gamma$ ονομάζεται "αδιαβατικός δείκτης" (adiabatic index) 
\begin{equation}
    \gamma = \frac{q+5}{q+3}
\end{equation}
όπου $q$ ο αριθμός των εσωτερικών βαθμών ελευθερίας (για ιδανικό αέριο που τα σωματίδια είναι σημειακά έχουμε $q = 0 \rightarrow \gamma = 5/3)$.

\textbf{Θερμοδυναμική ισορροπία}: Ένα σύστημα λέγεται ότι βρίσκεται σε κατάσταση θερμοδυναμικής ισορροπίας όταν βρίσκεται σε
\begin{itemize}
    \item \textbf{μηχανική ισορροπία}: το διανυσματικό άθροισμα όλων των εξωτερικών και εσωτερικών δυνάμεων είναι μηδέν.
    \item \textbf{θερμική ισορροπία}: ένα σύστημα είναι σε θερμική ισορροπία με τον εαυτό του όταν δεν υπάρχουν θερμοβαθμίδες. Δύο συστήματα βρίσκονται σε θερμική ισορροπία μεταξύ τους όταν δεν υπάρχει μεταφορά θερμότητας. Για ένα αέριο συγκεκριμένης θερμοκρασίας που βρίσκεται σε θερμική ισορροπία, ισχύει η κατανομή Maxwell-Boltzmann για την κατανομή των ταχυτήτων των σωματιδίων που το αποτελούν.
    \item \textbf{χημική ισορροπία}: σε μια χημική αντίδραση, η κατάσταση χημικής ισορροπίας είναι αυτή κατά την οποία τόσο τα αντιδρώντα όσο και τα προϊόντα βρίσκονται σε συγκεντρώσεις που δεν υπάρχει η τάση για αλλαγή με τον χρόνο. Συνήθως αυτή η κατάσταση επέρχεται όταν ο ρυθμός παραγωγής των προϊόντων ισούται με τον ρυθμό της αντίστροφης διαδικασίας κατά την οποία τα προϊόντα σχηματίζουν ξανά τα αντιδρώντα από τα οποία προήλθαν.
    \item \textbf{στατιστική ισορροπία}: ένα σύστημα βρίσκεται σε κατάσταση στατιστικής ισορροπίας όταν ο πληθυσμός των ατόμων και των ιόντων που το αποτελούν δεν αλλάζει με τον χρόνο. Σε ένα τέτοιο σύστημα, ο πληθυσμός των καταστάσεων δίνεται από τον νόμο του Boltzmann.
\end{itemize}
Σε μία κατάσταση θερμοδυναμικής ισορροπίας, δεν υπάρχει ροή ενέργειας ή ύλης, δεν υπάρχουν αλλαγές φάσης ή δυναμικά που θα οδηγήσουν σε αλλαγές μέσα στο σύστημα. Ένα τέτοιο σύστημα δεν υπόκειται σε καμία αλλαγή όταν βρίσκεται απομονωμένο από το περιβάλλον του. Αν το σύστημα είναι απομονωμένο τόσο για την ύλη όσο και για την ακτινοβολία, και βρίσκεται σε κατάσταση μηχανικής ισορροπίας, τότε σταδιακά θα επέλθει σε κατάσταση θερμοδυναμικής ισορροπίας. Τέλος, για ένα σύστημα που βρίσκεται σε κατάσταση θερμοδυναμικής ισορροπίας, η θερμοκρασία ακτινοβολίας $T_R$, η θερμοκρασία λόγω της κινητικής ενέργειας των σωματιδίων $T$, καθώς και η θερμοκρασία διέγερσης $T_{\text{ex}}$ είναι ίσες μεταξύ τους.

\textbf{Τοπική θερμοδυναμική ισορροπία}: Πραγματική θερμοδυναμική ισορροπία είναι δύσκολο να επιτευχθεί (σχεδόν σε όλες τις περιπτώσεις ενέργεια διαφεύγει από το σύστημα με τη μορφή ακτινοβολίας, δηλαδή το σύστημα ψύχεται), και συχνά υπάρχουν θερμοβαθμίδες. Παρόλα αυτά, σε πολλά συστήματα (π.χ. αστέρες, μεσοασαστρικό μέσο) μπορούμε να εφαρμόσουμε τοπική θερμοδυναμική ισορροπία, που σημαίνει ότι το σύστημα βρίσκεται σε θερμοδυναμική ισορροπία αλλά μόνο σε μια πολύ μικρή περιοχή ενδιαφέροντος. Σε ένα σύστημα που βρίσκεται σε κατάσταση τοπικής θερμοδυναμικής ισορροπίας, υπάρχουν βαθμίδες θερμοκρασίας, πυκνότητας, πίεσης κτλ, αλλά θα είναι αρκετά μικρές στο διάστημα που ορίζεται από τη μέση ελεύθερη διαδρομή ενός σωματιδίου του αερίου.
Το γεγονός ότι τα φωτόνια στο εσωτερικό του Ήλιου κάνουν πολλές χιλιάδες χρόνια να φτάσουν στην επιφάνεια και άρα είναι τοπικώς "παγιδευμένα" είναι αποτέλεσμα της κατάστασης τοπικής θερμοδυναμικής ισορροπίας που επικρατεί στο εσωτερικό του Ήλιου.

\subsection{Καταστατικές εξισώσεις αερίων}
Ως καταστατική εξίσωση εννοούμε μία θερμοδυναμική εξίσωση που περιγράφει την κατάσταση της ύλης υπό συγκεκριμένες φυσικές συνθήκες. Συνήθως είναι της μορφής $P = P(\rho, T)$. Πολλές φορές είναι χρήσιμο να γράψουμε την καταστατική εξίσωση σε διαφορική μορφή. Η διαφορική μορφή μπορεί να βρεθεί αν ξεκινήσουμε γράφοντας την καταστατική εξίσωση γενικά ως έναν εκθετικό νόμο (power law)
\begin{equation}
    P = \rho^{\chi_\rho} T^{\chi_T}
\end{equation}
και άρα το ολικό διαφορικό είναι
\begin{align}
    \nonumber dP &= \left( \frac{\partial P}{\partial \rho} \right)_T d\rho + \left( \frac{\partial P}{\partial T} \right)_\rho dT \Rightarrow \\\nonumber\\
    \nonumber &\Rightarrow dP = \chi_\rho T^{\chi_T} \rho^{\chi_\rho - 1} d\rho +  \chi_T \rho^{\chi_\rho} T^{\chi_T - 1} dT \Rightarrow \\\nonumber\\
    \nonumber &\Rightarrow dP = \underbrace{\rho^{\chi_\rho} T^{\chi_T}}_{P} \left(\chi_\rho \rho^{-1} d\rho + \chi_T T^{-1} dT \right) \Rightarrow \\\nonumber\\
    &\Rightarrow \boxed{\frac{dP}{P} = \chi_\rho \frac{d\rho}{\rho} + \chi_T \frac{dT}{T}}
\end{align}

Αντιπαραβάλοντας τις σχέσεις
\begin{align*}
    \label{apx:eq:differential_form_of_eos}
    dP &= \left( \frac{\partial P}{\partial \rho} \right)_T d\rho + \left( \frac{\partial P}{\partial T} \right)_\rho dT \\\\
    \frac{dP}{P} &= \chi_\rho \frac{d\rho}{\rho} + \chi_T \frac{dT}{T}
\end{align*}
βρίσκουμε ότι τα εκθετικά $\chi_\rho, \chi_T$ δίνονται από τις σχέσεις
\begin{align*}
    \chi_\rho &= \frac{\rho}{P} \left( \frac{\partial P}{\partial \rho} \right)_T \\\\
    \chi_T &= \frac{T}{P} \left( \frac{\partial P}{\partial T} \right)_\rho
\end{align*}

Μπορεί κανείς όμως να βρει μία ακόμα πιο κομψή έκφραση των παραπάνω εκθετών ως εξής:
\begin{align*}
     P &= \rho^{\chi_\rho} T^{\chi_T} \Rightarrow \log \,P = \log \,\left( \rho^{\chi_\rho} T^{\chi_T} \right) = \log \,\rho^{\chi_\rho} + \log \,T^{\chi_T} \Rightarrow \\\\
     &\Rightarrow \log \,P = \chi_\rho \log \,\rho + \chi_T \log \,T
\end{align*}
Το ολικό διαφορικό άρα είναι
\begin{equation*}
    d\log \,P = \left( \frac{\partial \log \,P}{\partial \log \,\rho} \right)_T d \log \,\rho + \left( \frac{\partial \log \,P}{\partial \log \,T} \right)_\rho d \log \,T
\end{equation*}
ώστε
\begin{equation*}
    d \log \,P = \chi_\rho d\log \,\rho + \chi_T d \log \,T
\end{equation*}
και με αντιπαραβολή των δύο τελευταίων σχέσεων προκύπτει τελικά ότι
\begin{align}
    \chi_\rho &= \frac{\rho}{P} \left( \frac{\partial P}{\partial \rho} \right)_T = \left( \frac{\partial \log \,P}{\partial \log \,\rho} \right)_T \label{apx:eq:chi_rho}\\\nonumber\\
    \chi_T &= \frac{T}{P} \left( \frac{\partial P}{\partial T} \right)_\rho = \left( \frac{\partial \log \,P}{\partial \log \,T} \right)_\rho \label{apx:eq:chi_T}
\end{align}

Στην πιο γενική περίπτωση, τα $\chi_\rho, \chi_T$ εξαρτώνται και τα ίδια από τα $\rho, T$ αλλά αν είναι (προσεγγιστικά) σταθερά, η καταστατική εξίσωση γράφεται
\begin{equation}
    \label{apx:eq:eos_chi_rho_chi_T}
    P = P_0 \,\rho^{\chi_\rho} \,T^{\chi_T}
\end{equation}

Στην συνέχεια θα εξάγουμε την καταστατική εξίσωση για ένα τέλειο αέριο από τις αρχές της στατιστικής μηχανικής. Έστω $n(p)$ η κατανομή των ορμών των σωματιδίων του αερίου, δηλαδή με άλλα λόγια, το $n(p)dp$ αναπαριστά τον αριθμό των σωματιδίων ανά μονάδα όγκου που έχουν ορμή $p \in [p, p+dp]$. Αν η $n(p)$ είναι γνωστή, τότε η αριθμητική πυκνότητα, η πυκνότητα (εσωτερικής) ενέργειας, και η πίεση θα δίνονται από τα ακόλουθα ολοκληρώματα:
\begin{align}
    n &= \int_{0}^{\infty} n(p) dp \label{apx:eq:number_density_integral} \\\nonumber\\
    u &= \int_{0}^{\infty} E_{\text{kin}} n(p) dp = n \langle E_{\text{kin}} \rangle \label{apx:eq:energy_density_integral} \\\nonumber\\
    P &= \frac{1}{3}\int_{0}^{\infty} p \,v_p \,n(p) dp = \frac{1}{3} n \langle p \,v_p \rangle \label{apx:eq:pressure_integral} 
\end{align}
όπου $E_{\text{kin}}$ η κινητική ενέργεια σωματιδίου με ορμή $p$ και ταχύτητα $v_p$. Η σχέση \eqref{apx:eq:number_density_integral} είναι τετριμμένη ενώ η σχέση \eqref{apx:eq:energy_density_integral} προκύπτει από τον ορισμό του ιδανικού αερίου. Όμως η σχέση \eqref{apx:eq:pressure_integral} χρειάζεται εξήγηση.

\begin{figure}[h]
    \centering
    \includegraphics[scale = 0.6]{Figures/elastic_collision_in_a_box.png}
    \caption{Σωματίδιο αερίου μέσα σε ένα κυβικό κουτί όγκου $1 \,\text{cm}^3$. Κάθε σύγκρουση με τα τοιχώματα του κουτιού οδηγεί σε μεταφορά ορμής. Η πίεση μέσα στο κουτί είναι το αποτέλεσμα της συνολικής μεταφοράς ορμής από όλα τα σωματίδια μέσα στο κουτί.}
    \label{apx:fig:elastic_collisions}
\end{figure}

Ας υποθέσουμε ένα αέριο που αποτελείται από $N$ σωματίδια και το οποίο περιέχεται σε κυβικό κουτί με μήκος πλευρών $L = 1 \,\text{cm}$. Κάθε σωματίδιο ανακλάται από τα τοιχώματα του κουτιού και η πίεση στην συγκεκριμένη επιφάνεια είναι το αποτέλεσμα της μεταφοράς της ορμής από όλα τα σωματίδια που συγκρούονται με αυτή. Έστω ένα σωματίδιο με ορμή $p$ και αντίστοιχη ταχύτητα $v_p$ το οποίο πλησιάζει την μία πλευρά του κουτιού υπο γωνία $\theta$, όπως φαίνεται στο σχήμα \ref{apx:fig:elastic_collisions}. Ο χρόνος μεταξύ δύο συγκρούσεων στην ίδια επιφάνεια του κουτιού είναι
\begin{equation*}
    \Delta t = \frac{2L}{v_p \cos \,\theta} = \frac{2}{v_p \cos \,\theta}
\end{equation*}
Επειδή οι συγκρούσεις είναι ελαστικές ($p_f = - p_i$) η μεταφορά ορμής είναι διπλάσια από τη συνιστώσα της ορμής που είναι κάθετη στην επιφάνεια
\begin{equation*}
    \Delta p = p_f - p_i = 2p \cos \,\theta
\end{equation*}
'Αρα, ο ρυθμός με τον οποίο μεταφέρεται ορμή ανά σωματίδιο είναι
\begin{equation*}
    \frac{\Delta p}{\Delta t} = p \,v_p \,\cos^2 \,\theta
\end{equation*}
Ο αριθμός των σωματιδίων με $p \in [p,p+dp]$ και $\theta \in [\theta, \theta+d\theta]$ συμβολίζεται με $n(p, \theta) dp \,d\theta$. Η συνεισφορά αυτών των σωματιδίων στην πίεση τότε είναι
\begin{equation*}
    dP =  p \,v_p \,\cos^2 \,\theta \,n(p, \theta) dp \,d\theta
\end{equation*}
Επειδή οι ορμές είναι κατανεμημένες ισοτροπικά σε όλες τις διευθύνσεις μέσα σε μια στερεά γωνία $2\pi$, και η στερεά γωνία $d\omega$ που αντιστοιχεί σε αυτά τα σωματίδια με $\theta \in [\theta, \theta+d\theta]$ ισούται με $2\pi \sin \,\theta \,d\theta$ έχουμε ότι $n(p, \theta) = n(p) \sin \,\theta \,d\theta$ και άρα
\begin{equation*}
    dP = p v_p \cos^2 \,\theta \,n(p) \sin \,\theta \,d\theta \,dp
\end{equation*}
Η συνολική πίεση βρίσκεται με ολοκλήρωση σε όλες τις γωνίες ($0 \leq \theta \leq \pi/2$) και ορμές ώστε
\begin{align*}
    P &= \int_{0}^{\frac{\pi}{2}} \int_{0}^{\infty} p v_p \cos^2 \,\theta \,n(p) \sin \,\theta \,d\theta \,dp = \\\\
    &=\int_{0}^{\frac{\pi}{2}} \cos^2 \,\theta \,\sin \,\theta \,d\theta \int_{0}^{\infty} p v_p n(p) dp = \\\\
    &= \int_{0}^{1} \cos^2 \,\theta \, d(\cos \,\theta) \int_{0}^{\infty} p v_p n(p) dp \Rightarrow \\\\
    &\Rightarrow P = \frac{1}{3} \int_{0}^{\infty} p v_p n(p) dp = \frac{1}{3} n \langle p v_p \rangle
\end{align*}

Έχοντας ξεκαθαρίσει τα παραπάνω, θέλουμε τώρα να βρούμε μία σχέση μεταξύ της πίεσης και της εσωτερικής ενέργειας. Σύμφωνα με την ειδική σχετικότητα, η ορμή και η ταχύτητα των σωματιδίων σχετίζονται με την ενέργειά τους σύμφωνα με τις σχέσεις:
\begin{equation}
    \epsilon^2 = p^2 c^2 + m^2 c^4
\end{equation}
\begin{align}
    \nonumber \frac{\partial}{\partial p} (\epsilon^2) &= \frac{\partial}{\partial p} (p^2 c^2 + m^2 c^4) \Rightarrow 2\epsilon \frac{\partial \epsilon}{\partial p} = 2p c^2 \Rightarrow \\\nonumber\\
    &\Rightarrow \frac{\partial \epsilon}{\partial p} \equiv v_p = \frac{pc^2}{\epsilon}
\end{align}
Χρησιμοποιώντας αυτές τις σχέσεις μπορούμε να βρούμε σχέσεις μεταξύ της πίεσης και της εσωτερικής ενέργειας ενός ιδανικού αερίου στο μη-σχετικιστικό (NR) όριο και στο εξαιρετικά σχετικιστικό (ER) όριο:

\begin{itemize}
    \item \textbf{NR όριο}: σε αυτή την περίπτωση οι ορμές είναι $p \ll mc$
    
    $$\epsilon^2 = p^2 c^2 + m^2 c^4 \Rightarrow \epsilon^2 = c^2 \left(p^2 + m^2c^2 \right)$$
    Επειδή ισχύει ότι $p \ll mc$, άρα $\epsilon \simeq mc^2$ και συνεπώς
    $$v_p = \frac{pc^2}{\epsilon} = \frac{p}{m}$$
    Η κινητική ενέργεια είναι όπως περιμένουμε $E_{\text{kin}} = \frac{p^2}{2m}$ οπότε $\langle p v_p \rangle = \langle \frac{p^2}{m} \rangle = 2 \langle E_{\text{kin}} \rangle $.
    
    Τελικά, από τις σχέσεις \eqref{apx:eq:energy_density_integral}, \eqref{apx:eq:pressure_integral}, προκύπτει
    \begin{equation}
        \label{apx:eq:pressure_internal_energy_relation_for_nr_case}
        \boxed{P = \frac{2}{3}u}
    \end{equation}
    
    \item \textbf{ER όριο}: σε αυτή την περίπτωση έχουμε $p \gg mc$
    
    Αυτό σημαίνει ότι $\epsilon = pc$ και άρα $v_p = c \longrightarrow \langle p v_p \rangle = \langle pc \rangle = \langle \epsilon \rangle$.
    
    Τελικά, από τις σχέσεις \eqref{apx:eq:energy_density_integral}, \eqref{apx:eq:pressure_integral}, προκύπτει
    \begin{equation}
        \label{apx:eq:pressure_internal_energy_relation_for_er_case}
        \boxed{P = \frac{1}{3}u}
    \end{equation}
\end{itemize}

Γενικά μπορούμε να γράψουμε ότι 
\begin{equation}
    P = \zeta n \langle E \rangle
\end{equation}
όπου $\zeta = 2/3$ ή $\zeta = 1/3$ για το μη-σχετικιστικό και το σχετικιστικό όριο αντίστοιχα. Αυτό έρχεται σε συμφωνία με την διαίσθησή μας καθώς θα περιμέναμε η πίεση που προέρχεται από την σύγκρουση (λόγω ταχυτήτων) σωματιδίων να είναι ανάλογη της πυκνότητας των σωματιδίων και της κινητικής τους ενέργειας
\begin{equation*}
    P \sim n mv^2 \sim n E_{\text{kin}}
\end{equation*}


\subsubsection{Το κλασικό ιδανικό αέριο}
Σε ένα ιδανικό αέριο η κατανομή των ορμών δίνεται από την κατανομή Maxwell-Boltzmann
\begin{equation}
    n(p) dp = \frac{n}{(2\pi m kT)^{3/2}} \exp\left(- \frac{p^2}{2mkT}\right) 4\pi p^2 dp
\end{equation}
με την οποία θα ασχοληθούμε εκτενώς παρακάτω. Θα αποδείξουμε τότε ότι $\langle E_{\text{kin}} \rangle = \frac{3}{2} kT $ και άρα σύμφωνα με τη σχέση \eqref{apx:eq:pressure_internal_energy_relation_for_nr_case} ισχύει ότι\textbf{ η καταστατική εξίσωση ενός ιδανικού αερίου είναι}
\begin{equation}
    \boxed{P_{\text{gas}} = \frac{2}{3}u = \frac{2}{3} n \langle E_{\text{kin}} \rangle = nkT = \frac{\rho}{\mu m_H}kT}
\end{equation}
που είναι ο γνωστός νόμος των ιδανικών αερίων. Αυτό το αποτέλεσμα προήλθε θεωρώντας μη-σχετικιστικά κλασικά σωματίδια αλλά μπορεί να δειχτεί ότι η ίδια σχέση ισχύει και για σχετικιστικά κλασικά σωματίδια.

Παρατηρούμε ότι ο νόμος των ιδανικών αερίων προκύπτει από τη σχέση \eqref{apx:eq:eos_chi_rho_chi_T} για $\chi_\rho = \chi_T = 1$


\subsubsection{Το αέριο φωτονίων}
Στο Κεφάλαιο \ref{ch:Chapter4}, αποδείξαμε ότι η πυκνότητα ενέργειας της ακτινοβολίας είναι
\begin{equation}
    u = \alpha T^4
\end{equation}
Επειδή το αέριο φωτονίων είναι εξ' ορισμού σχετικιστικό, χρησιμοποιούμε τη σχέση \eqref{apx:eq:pressure_internal_energy_relation_for_er_case} για να βρούμε την \textbf{καταστατική εξίσωση ενός αερίου φωτονίων}
\begin{equation}
    \boxed{P_{\text{rad}} = \frac{1}{3}u = \frac{1}{3} \alpha T^4}
\end{equation}

Παρατηρούμε ότι η καταστατική εξίωση ενός αερίου φωτονίων προκύπτει από τη σχέση \eqref{apx:eq:eos_chi_rho_chi_T} για $\chi_\rho = 0$ και $\chi_T = 4$.



\subsubsection{Το κβαντικό ιδανικό αέριο}
(εκφυλισμενο μη-σχετικιστικο αεριο φερμιονιων, εκφυλισμενο σχετικιστικο αεριο φερμιονιων)





\section{Νόμος του Boltzmann}
Εαν έχουμε έναν μεγάλο αριθμό ατόμων σε ένα ζεστό, πυκνό αέριο, τα άτομα αυτά συνεχώς θα συγκρούονται μεταξύ τους με αποτέλεσμα να διεγείρονται σε διάφορα πιθανά ενεργειακά επίπεδα. Η διέγερση λόγω συγκρούσεων θα ακολουθηθεί από την αποδιέγερση των ατόμων με εκπομπή ακτινοβολίας (σε χρονική κλίμακα της τάξης των νανοδευτερολέπτων). Αν η θερμοκρασία και η πίεση παραμείνουν σταθερά, θα υπάρχει κάποιου είδους δυναμικής ισορροπίας μεταξύ των διεγέρσεων από τις συγκρούσεις και τις αποδιεγέρσεις, οδηγώντας σε μια συγκεκριμένη κατανομή των ατόμων στα διάφορα ενεργειακά επίπεδα. Όσο χαμηλότερη είναι η θερμοκρασία, τόσο πιο γρήγορα θα μειώνεται ο πληθυσμός των ατόμων που καταλαμβάνουν υψηλές ενεργειακές στάθμες. Μόνο στις πολύ υψηλές θερμοκρασίες θα μπορούν οι υψηλές ενεργειακές στάθμες να καταλαμβάνονται από έναν σημαντικό αριθμό ατόμων. Η εξίσωση Boltzmann περιγράφει ποιά θα είναι η κατανομή των ατόμων ανάμεσα σε διάφορα ενεργειακά επίπεδα, ως συνάρτηση της ενέργειας και της θερμοκρασίας\footnote{Οι πληροφορίες αντλήθηκαν από \url{https://phys.libretexts.org/Bookshelves/Astronomy__Cosmology/Book\%3A_Stellar_Atmospheres_(Tatum)}}. 

Ας φανταστούμε ένα κουτί (σταθερός όγκος) το οποίο περιέχει N άτομα, το καθένα από τα οποία έχει m δυνατά ενεργειακά επίπεδα. Ας υποθέσουμε ότι υπάρχουν $ N_j$ άτομα στο ενεργειακό επίπεδο $ E_j$. Ο συνολικός αριμός N των ατόμων δίνεται από τη σχέση
\begin{equation}
    \label{eq:apx:total_number_of_atoms}
     N = \sum_{i=1}^{m} N_i
\end{equation}
όπου ο ακέραιος θετικός δείκτης i τρέχει από το 1 μέχρι το m, συμπεριλαμβάνοντας την τιμή j.

Η συνολική εσωτερική ενέργεια του συστήματος U θα δίνεται από τη σχέση
\begin{equation}
    \label{eq:apx:internal_energy}
     U = \sum_{i=1}^{m} N_i E_i
\end{equation}

Τώρα, πρέπει να βρούμε πόσοι τρόποι υπάρχουν να κατανείμουμε N άτομα έτσι ώστε να υπάρχουν $ N_1$ στο πρώτο ενεργειακό επίπεδο, $ N_2$ στο δεύτερο κ.ο.κ. Θα συμβολίσουμε αυτόν τον αριθμό με $ \Omega$. Στην στατιστική φυσική το μέγεθος αυτό ονομάζεται στατιστικό βάρος και είναι ο αριθμός των (τρόπων) μικροκαταστάσεων που αντιστοιχούν στην ίδια (ενεργεια) μακροκατάσταση.
\begin{equation}
    \label{eq:apx:statistical_weight}
     \Omega = \frac{N!}{N_1! N_2! N_3! \dots N_j! \dots N_m!} = \frac{N!}{\prod_{i=1}^{m}N_i!}
\end{equation}
Η σχέση αυτή δεν είναι προφανής, γι' αυτό θα επιχειρησούμε να τη δικαιολογήσουμε εν μέρει. Ο αριθμός των τρόπων που μπορούμε να διαλέξουμε $ N_1$ άτομα από ένα σύνολο N ατόμων ώστε να καταλάβουν το πρώτο ενεργειακό επίπεδο, θα δίνεται από τον διωνυμικό συντελεστή $ \Omega_1 = {N \choose N_1}$. Για κάθε έναν από αυτούς τους τρόπους, πρέπει να ξέρουμε με πόσους τρόπους μπορούν να διαμοιραστούν $ N_2$ άτομα από τα εναπομείναντα $ N - N_1$ τα οποία θα καταλαμβάνουν την δεύτερη ενεργειακή στοιβάδα, οι οποίοι είναι $ \Omega_2 = {N - N_1 \choose N_2}$. Άρα, ο αριθμός των τρόπων με τους οποίους μπορούμε να κατανείμουμε τα άτομα στις δύο πρώτες ενεργειακές στοιβάδες, είναι το γινόμενο 
\begin{align*}
     \Omega &= { N \choose N_1} { N - N_1 \choose N_2} =  \frac{N!}{N_1! (N-N_1)!} \frac{(N-N_1)!}{N_2! (N-N_1 -N_2)!} \\\\
    &=  \frac{N!}{N_1!} \frac{1}{N_2!(N_2! - N_2!)} = \frac{N!}{N_1!N_2!}
\end{align*}
Συνεχίζοντας αυτή τη λογική, καταλήγουμε στον πολυωνυμικό συντελεστή
$$ \Omega = \prod_{i=1}^{m} \Omega_i =  { N \choose N_1} { N - N_1 \choose N_2} \dots = \frac{N!}{N_1!N_2! \dots N_m!}$$
Ίσως ο παραπάνω συλλογισμός γίνεται πιο αντιληπτός αν έχουμε στο μυαλό μας πως ο διωνυμικός συντελεστής μας δίνει τον αριθμό των υπαρκτών τρόπων να χωρίσουμε τις ποσότητες $ N_1$ και $ N - N_1$ από ένα σύνολο $ N$, σε δύο διακριτές καταστάσεις (two distinct bins). Ο πολυωνυμικός συντελεστής αντίστοιχα μας δίνει τον αριθμό των υπαρκτών τρόπων να χωρίσουμε τις ποσότητες $ N_1, N_2, \dots , N_m$ από ένα σύνολο $ N$, σε $ N$ διακριτές καταστάσεις ($ N$ distinct bins).

Τώρα πρέπει να βρούμε τους πιο πιθανούς διαμερισμούς, δηλαδή τους πιο πιθανούς αριθμούς $ N_1, N_2, \dots , N_m$. Ο πιο πιθανός καταμερισμός θα είναι αυτός που μεγιστοποιεί το $ \Omega$ ως προς το κάθε $ N_j$, και που υπόκειται στους περιορισμούς των σχέσεων \eqref{eq:apx:total_number_of_atoms} και \eqref{eq:apx:internal_energy}. Η εύρεση ακροτάτων μιας συνάρτησης υπό περιορισμούς, μας οδηγεί στην χρήση των πολλαπλασιαστών Lagrange. Επειδή όμως ο χειρισμός του παραγοντικού ($ N!$) είναι δύσκολος στην ανάλυση, θα βρούμε το μέγιστο του λογαρίθμου $ \ln{\Omega}$, χωρίς καμία ποιοτική διαφορά στο αποτέλεσμα. Παίρνοντας τον λογάριθμο της σχέσης \eqref{eq:apx:statistical_weight}, προκύπτει ότι
\begin{equation}
    \label{eq:apx:log_statistical_weight}
     \ln{\Omega} = \ln{N!} - \ln{N_1!} - \ln{N_2!} - \dots
\end{equation}

Χρησιμοποιώντας τον τύπο του Stirling:
\begin{equation}
    \label{eq:apx:stirling_formula}
     \ln{X!} = X \ln{X} - X
\end{equation}
στη σχέση \eqref{eq:apx:log_statistical_weight}, προκύπτει ότι
\begin{align}
    \label{eq:apx:non_factorial_terms_multiplicity}
   \nonumber  \ln{\Omega} &=  N \ln{N} - \cancel{ N} - (N_1 \ln{N_1} - \cancel{ N_1}) - (N_2 \ln{N_2} - \cancel{ N_2}) - \dots \\\nonumber \\
   \nonumber &=  N \ln{ N} -  N_1 \ln{ N_1} -  N_2 \ln{  N_2} - \dots \\ \nonumber \\
   &=  N \ln{N} - \sum_{i=1}^{m} N_i \ln{N_i}
\end{align}

Τώρα μπορούμε να προχωρήσουμε με την εύρεση του μεγίστου της συνάρτησης \eqref{eq:apx:non_factorial_terms_multiplicity}, ως προς μία από τις μεταβλητές, για παράδειγμα την $ N_j$, με τρόπο τέτοιο ώστε να είναι συνεπής με τους περιορισμούς των σχέσεων \eqref{eq:apx:total_number_of_atoms} και \eqref{eq:apx:internal_energy}. Χρησιμοποιώντας την μέθοδο των πολλαπλασιαστών Lagrange, έχουμε ότι για τον πιθανό αριθμό κατάληψης (most probable occupation number) του j ενεργειακού επιπέδου, ισχύει η σχέση:

\begin{equation}
    \label{eq:apx:occupation_number_condition}
     \frac{\partial \ln{\omega}}{\partial N_j} = \lambda \frac{\partial N}{\partial N_j} + \mu \frac{\partial U}{\partial N_j}
\end{equation}
Αναπτύσοντας τους όρους έχουμε:
\begin{align*}
    \frac{\partial \ln{\Omega}}{\partial N_j} &= \frac{\partial}{\partial N_j} \left( -N_j \ln{N_j}\right) = -\cancelto{1}{\frac{\partial N_j}{\partial N_j}} \ln{N_j} - \cancelto{1}{N_j \frac{\partial \ln{N_j}}{\partial N_j}} = - \ln{N_j} - 1 \\\\
    \lambda \frac{\partial N}{\partial N_j} &= \lambda \frac{\partial}{\partial N_j} \left( \sum_{i=1}^{m} N_i \right) = \lambda \sum_{i=1}^{m} \frac{\partial N_i}{\partial N_j} = \lambda \\\\
    \mu \frac{\partial U}{\partial N_j} &= \mu \frac{\partial}{\partial N_j} \left( \sum_{i=1}^{m} N_i E_i \right) = \mu \sum_{i=1}^{m} \frac{\partial \left( N_i E_i \right)}{\partial N_j} = \mu \sum_{i=1}^{m} \left( \frac{\partial N_i}{\partial N_j} E_i + N_i \cancelto{0}{\frac{\partial E_i}{\partial N_j}} \right) \\  
    &= \mu \sum_{i=1}^{m} \frac{\partial N_i}{\partial N_j}E_i = \mu E_j
\end{align*}

Αντικαθιστώντας τα παραπάνω στην σχέση \eqref{eq:apx:occupation_number_condition}, προκύπτει
\begin{equation}
    \label{eq:apx:occupation_number}
    - \ln{N_j} - 1 = \lambda + \mu E_j \Rightarrow N_j = C e^{-\mu E_j}
\end{equation}

Το μόνο που μένει να κάνουμε, είναι να προσδιορίσουμε τους πολλαπλασιαστές Lagrange $\lambda$ (ή ισοδύναμα $C = e^{-\lambda - 1}$) και $\mu$. Πολλαπλασιάζοντας την σχέση \eqref{eq:apx:occupation_number} με $N_j$, ενώ ταυτόχρονα αλλάζοντας τον δείκτη από $j$ σε $i$ και παίρνοντας το άθροισμα, έχουμε
\begin{align}
    \label{eq:apx:final_form}
   \nonumber N_j \ln{N_j} &+ \lambda N_j + N_j \mu E_j + N_j = 0 \Rightarrow \\ \nonumber \\
    \nonumber \Rightarrow \sum_{i=1}^{m} N_i \ln{N_i} &+ \lambda \sum_{i=1}^{m} N_i + \mu \sum_{i=1}^{m} N_i E_i + \sum_{i=1}^{m} N_i = 0 \\\nonumber \\
     \nonumber = \sum_{i=1}^{m} N_i \ln{N_i} &+ \lambda N + \mu U + N = 0 \Rightarrow \\\nonumber \\
   \Rightarrow N \ln{N} &- \ln{\Omega} + \lambda N + \mu U + N = 0
\end{align}

όπου στο τελευταίο βήμα κάναμε χρήση της σχέσης \eqref{eq:apx:non_factorial_terms_multiplicity}. Από τη Θερμοδυναμική και τη Στατιστική Φυσική γνωρίζουμε οτι ισχύουν για την εντροπία τα εξής:
\begin{equation}
    \label{eq:apx:thermodynamics_2nd_law}
    dU = TdS - PdV \longrightarrow \left( \frac{\partial U}{\partial S} \right)_V = T \Rightarrow \left( \frac{\partial S}{\partial U} \right)_V = \frac{1}{T}
\end{equation}
\begin{equation}
    \label{eq:apx:multiplicity}
    S = k \ln{\Omega}
\end{equation}

Συνδυάζοντας τις σχέσεις \eqref{eq:apx:thermodynamics_2nd_law} και \eqref{eq:apx:multiplicity} με την \eqref{eq:apx:final_form}, προκύπτει ότι:
\begin{equation}
    \label{eq:apx:mu_value}
    \left( \frac{\partial S}{\partial U} \right)_V = k \mu \Rightarrow \mu = \frac{1}{kT}
\end{equation}

Τέλος, αθροίζοντας τους όρους της σχέσης \eqref{eq:apx:occupation_number}, βρίσκουμε τον δεύτερο πολλαπλασιαστή Lagrange:
\begin{equation}
    \label{eq:apx:lambda_value}
    \sum_{i=1}^{m} N_i = C \sum_{i=1}^{m} e^{-E_i/(kT)} \Rightarrow C = \frac{N}{\sum_{i=1}^{m} e^{-E_i/(kT)}}
\end{equation}
που μας οδηγεί στην έκφραση:
\begin{equation}
    \label{eq:apx:unweighted_boltzmann}
    \frac{N_j}{N} = \frac{e^{-E_j/(kT)}}{\sum_{i=1}^{m}e^{-E_i/(kT)}}
\end{equation}

Παρόλα αυτά, μας μένει να λάβουμε υπόψιν ένα ακόμα πράγμα. Τα περισσότερα ενεργειακά επίπεδα σε ένα άτομο είναι εκφυλισμένα, δηλαδή υπάρχουν πολλές καταστάσεις με την ίδια ενέργεια. Γι' αυτό το λογο, για να βρούμε τον πληθυσμό ενός επιπέδου, πρέπει να προσθέσουμε τους πληθυσμούς των επιμέρους καταστάσεων. Έτσι, η εξίσωση \eqref{eq:apx:unweighted_boltzmann} πρέπει να πολλαπλασιαστεί με το στατιστικό βάρος g του επιπέδου
\begin{equation}
    \label{eq:apx:boltmann_equation}
   \boxed{ \frac{N_j}{N} = \frac{g_j e^{-E_j/(kT)}}{\sum_{i=1}^{m} g_i e^{-E_i/(kT)}} = \frac{g_j}{Z}e^{-E_j/(kT)}}
\end{equation}

Η εξίσωση \eqref{eq:apx:boltmann_equation} είναι γνωστή ως \textit{εξίσωση Boltzmann} ή κατανομή Boltzmann και μας δίνει την πιθανότητα να βρούμε $N_j$ άτομα στο ενεργειακό επίπεδο j, με ενέργεια $E_j$. Ο όρος $Z = \sum_{i=1}^{m} g_i e^{-E_i/(kT)}$ ονομάζεται συνάρτηση επιμερισμού (partition function). Ο υπολογισμός της, αν και πολύπλοκος, είναι καταρχήν δυνατός, αφού από τη θεωρία και το πείραμα είναι γνωστά τόσο το στατιστικό βάρος $g_i$, όσο και η ενέργεια διέγερσης $E_i$, κάθε στάθμης.

Το στατιστικό βάρος ενός ενεργειακού επιπέδου ενός ατόμου με μηδενικό πυρηνικο σπιν είναι $2J+1$, όπου το $J$ συμβολίζει την ολική στροφορμή του ατόμου. Αν το πυρηνικό σπιν είναι $I$, το στατιστικό βάρος του επιπέδου θα είναι $(2I+1)(2J+1)$. Παρόλα αυτά, ο παράγοντας $2J+1$ εμφανίζεται τόσο στον αριθμητή όσο και σε κάθε όρο του παρονομαστή της σχέσης \eqref{eq:apx:boltmann_equation}, άρα μπορεί να εξαλειφθεί. Άρα, όταν δουλεύουμε με την κατανομή Boltzmann, στις περισσότερες περιπτώσεις δεν είναι απαραίτητο να ανησυχούμε για το αν το άτομο έχει κάποιο συγκεκριμένο πυρηνικό σπιν, και το στατιστικό βάρος του κάθε επιπέδου στην εξίσωση \eqref{eq:apx:boltmann_equation} μπορεί να θεωρηθεί ότι είναι $g_j = 2J+1$.

Στην εξίσωση \eqref{eq:apx:boltmann_equation} συγκρίναμε τον αριθμό των ατόμων στο ενεργειακό επίπεδο $j$ με τον αριθμό των ατόμων σε όλα τα υπόλοιπα επίπεδα. Μπορούμε το ίδιο εύκολα να συγκρίνουμε τον αριθμό των ατόμων στο επίπεδο $j$ με τον αριθμό των ατόμων στη βασική στάθμη $0$:

\begin{equation}
    \label{eq:apx:ground_state}
    \frac{N_j}{N_0} = \frac{g_j}{g_0} e^{-E_j/(kT)}
\end{equation}
ή ακόμα μπορούμε να συγκρίνουμε τους πληθυσμούς δύο οποιοδήποτε ενεργειακών επιπέδων $i,j \ \text{με} \ j>i$:
\begin{equation}
    \label{eq:apx:boltzmann_two_levels}
    \frac{N_j}{N_i} = \frac{g_j}{g_i} e^{- (E_j - E_i)/ (kT)} = \frac{g_j}{g_i} e^{-h\nu/(kT)}
\end{equation}

\section{Συνάρτηση επιμερισμού}
Σε κάποια συστήματα, η ολική ενέργεια μπορεί να αποτελείται από διάφορες πηγές ενέργειας. Ας αναλογιστούμε, ως παράδειγμα, το απλό διατομικό μόριο. Η ενέργειά του μπορεί να είναι το άθροισμα της ενέργειας διέγερσης (λόγω ηλεκτρονίων), ενέργεια δόνησης, κινητική ενέργεια λόγω περιστροφής, αλλά και ενέργειας λόγω μεταφοράς. Αν αυτές οι πηγές ενέργειας είναι ανεξάρτητες η μία από τις άλλες, η ολική ενέργεια θα είναι το άθροισμα αυτών των συνεισφορών:
\begin{equation}
    \label{eq:apx:total_energy_diatomic_molecule}
    E_{\rm tot} = E_{\rm tran} + E_{\rm el} + E_{\rm vib} + E_{\rm rot}
\end{equation}
Στην πράξη, για πραγματικά μόρια, αυτό είναι απλά μία πρώτη προσέγγιση --- στην πραγματικότητα, υπάρχει κάποια αλληλεπίδραση μεταξύ των διάφορων συνεισφορών στην ολική ενέργεια, αλλά η πρόθεσή μας εδώ δεν είναι να επικεντρωθούμε στις λεπτομέρειες της μοριακής δομής αλλά να επισημάνουμε ένα μικρό σημείο που αφορά τις συναρτήσεις επιμερισμού. Δεδομένου λοιπόν ότι οι διάφορες ενεργειακές συνεισφορές είναι ανεξάρτητες μεταξύ τους και δεν αλληλεπιδρούν σημαντικά, η ολική ενέργεια όπως είπαμε είναι το άθροισμα των επιμέρους ενεργειακών συνεισφορών. Η ολική κυματοσυνάρτηση σε αυτή την περίπτωση είναι το γινόμενο των τριών επιμέρους κυματοσυναρτήσεων
\begin{equation}
    \label{eq:apx:wavefunctions}
    \psi = \psi_{\rm el} \psi_{\rm vib} \psi_{rot}
\end{equation}
Οι ποσότητες $\psi_{\rm vib}$ και $ E_{\rm vib}$ ελιναι οι κυματοσυναρτήσεις και τα ενεργειακά επίπεδα για έναν απλό αρμονικό ταλαντωτή, ενώ $\psi_{rot}$ και $E_{\rm rot}$ είναι οι κυματοσυναρτήσεις και τα ενεργειακά επίπεδα ενός στερεού ρότορα. Είναι αντιληπτό ότι ένα μόριο δεν μπορεί να είναι ταυτόχρονα στερεός ρότορας και αρμονικός ταλατωτής, γι' αυτό οι εξισώσεις \eqref{eq:apx:total_energy_diatomic_molecule} και \eqref{eq:apx:wavefunctions} είναι απλά προσεγγιστικές. Παρόλα αυτά, σε γενικές γραμμές, οι ενεργειακές διαφορές μεταξύ ηλεκτρονιακών ενεργειακών επιπέδων είναι οι μεγαλύτερες, οι διαφορές μεταξύ επιπέδων δονήσεων είναι ενδιάμεσες, και οι διαφορές μεταξύ ενεργειακών επιπέδων λόγω περιστροφής είναι οι μικρότερες, έτσι ώστε αυτή η πρώτη προσέγγιση είναι μια καλή αρχή. Ενεργειακά επίπεδα λόγω μεταφορικής κίνησης είναι πρακτικά συνεχή και μπορούν να υπολογισθούν ως κινητική ενέργεια. Επικεντρώνοντας την προσοχή μας γύρω από το σημείο ενδιαφέροντος για αυτό το κεφάλαιο, κάθε όρος της συνάρτησης επιμερισμού είναι της μορφής $e^{-E/(kT)}$, και άρα αν η ολική ενέργεια είναι το άθροισμα των επιμέρους συνεισφορών, όπως δείχνει η εξίσωση \eqref{eq:apx:total_energy_diatomic_molecule}, τότε η ολική (εσωτερική) συνάρτηση επιμερισμού είναι το γινόμενο των συναρτήσεων επιμερισμού των διάφορων συνειφορών:
\begin{equation}
    \label{eq:apx:total_partition_function}
    u = u_{\rm el} u_{\rm vib} u_{\rm rot}
\end{equation}

Σε εισαγωγικά μαθήματα κβαντομηχανικής, διδασκόμαστε --χωρίς ίσως να ξέρουμε γιατί-- το παράδειγμα "σωματίδιο σε κουτί". Αυτό ήταν χρήσιμο και θα το χρειαστούμε σε ό,τι ακολουθεί. Ίσως ήταν καλύτερο να ονομαζόταν "κύματα σε κουτί" καθώς αύτο που αναλύουμε στο συγκεκριμένο παράδειγμα είναι οι κυματοσυναρτήσεις που περιγράφουν την κυματική φύση ενός σωματιδίου. Αν το σωματίδιο (και το αντίστοιχο κύμα του) είναι περιορισμένο σε ένα κουτί, οι κυματοσυναρτήσεις περιορίζονται σε συναρτήσεις που έχουν έναν ακέραιο αριθμό αντι-κόμβων (antinodes) ανάμεσα σε δύο τοίχους, και κατά συνέπεια τα ενεργειακά επίπεδα μπορούν να έχουν μόνο διακριτές τιμές ---ένα ακόμα παράδειγμα της μη-αποφυγής των διακριτών ενεργειακών επιπέδων που προκύπτουν από συνοριακούς περιορισμούς. Αν το κουτί είναι ένας κύβος με πλευρά $a$, τότε τα ενεργειακά επίπεδα δίνονται από:
\begin{equation}
    \label{eq:apx:discrete_energy_levels_box}
    E_{n_x n_y n_z} = \frac{h^2}{8ma^2} \left(n^2_x + n^2_y + n^2_z \right)
\end{equation}

Αν έχουμε πολλά σωματίδια μέσα στο κουτί, μπορεί να ενδιαφερόμαστε για το πως τα διάφορα σωματίδια είναι κατενεμημένα (διαμοιρασμένα) μεταξύ των ενεργειακών επιπέδων. Με άλλα λόγια, πρέπει να ξέρουμε τη συνάρτηση επιμερισμού, η οποία είναι απλά:
\begin{equation}
    \label{eq:apx:box_partition_function_discrete}
    \sum_{n_x=0}^\infty \sum_{n_y=0}^\infty \sum_{n_z=0}^\infty \exp \left[ - \frac{h^2 \left(n_x^2 + n_y^2 + n_z^2 \right)}{8ma^2kT} \right]
\end{equation}

Αν το κουτί είναι πολύ μεγάλο, τα ενεγειακά επίπεδα είναι πολύ κοντά μεταξύ τους και σχεδόν συνεχή (αν το κουτί είναι απείρων διαστάσεων, δεν υπάρχουν συνοριακοί περιορισμοί και γι' αυτό υπάρχει ένα συνεχές φάσμα δυνατών ενεργειών). Αν τα επίπεδα είναι σχεδόν συνεχή, μπορούμε να αντικαταστήσουμε τα αθροίσματα με ολοκληρώματα και να πάρουμε το μεταφορικό κομμάτι της συνάρτησης επιμερισμού (που οφείλεται σε κίνηση του κέντρου μάζας):
\begin{align}
    \label{eq:apx:box_partition_function_integrals}
    \nonumber Q &= \int_0^\infty \int_0^\infty \int_0^\infty \exp \left[ - \frac{h^2 \left( n_x^2 + n_y^2 + n_z^2 \right)}{8ma^2kT } \right] dn_x dn_y dn_z \nonumber \\ \nonumber \\
    &=\left[ \int_0^\infty \exp \left( - \frac{h^2 n^2}{8ma^2kT } \right) dn \right]^3 \nonumber \\ \nonumber \\
    &= \left(\frac{2 \pi mkT}{h^2} \right)^{\frac{3}{2}} V
\end{align}
όπου $V = a^3$ είναι ο όγκος του κουτιού και $Z=Qu$ είναι η συνολική συνάρτηση επιμερισμού που περιλαμβάνει και το μεταφορικό αλλά και το "εσωτερικό" κομμάτι.




\section{Κατανομή Maxwell-Boltzmann}
Όπως είδαμε, η κατανομή Boltzmann μας δείχνει ότι η πιθανότητα ένα σωματίδιο να βρεθεί σε μία ενεργειακή κατάσταση $E$, μειώνεται εκθετικά όσο η ενέργεια αυξάνει. Μαθηματικά, η κατανομή Boltzmann γράφεται
\begin{equation}
    f(E) = A e^{-E/(kT)}
\end{equation}

Εάν αυτή η κατανομή εφαρμοστεί σε μία διάσταση για την ταχύτητα ενός σωματιδίου σε ένα αέριο, τότε επειδή $E = mu_x^2/2$, προκύπτει ότι
\begin{equation}
    \label{eq:apx:non_normalized_maxwell}
    f(u_x) = A e^{-mu_x^2/(2kT)}
\end{equation}
Για να βρούμε τον συντελεστή κανονικοποίησης $A$, πρέπει να λάβουμε υπόψιν ότι η πιθανότητα να βρούμε ο σωματίδιο σε κάποια τιμή της ταχύτητας είναι ίση με τη μονάδα. Έτσι:
\begin{equation}
    \label{eq:apx:normalize_integral}
    \int_{- \infty}^{+ \infty} f(u_x) du_x = 1 \Rightarrow A \int_{- \infty}^{+ \infty} e^{-mu_x^2/(2kT)} du_x = 1
\end{equation}
Κάνοντας την αντικατάσταση 
$$x^2 = \frac{m u_x^2}{2kT} \Rightarrow x = \sqrt{\frac{m}{2kT}} u_x \longrightarrow dx = \sqrt{\frac{m}{2kT}} du_x$$
από τη σχέση \eqref{eq:apx:normalize_integral} προκύπτει ότι:
\begin{align}
    \label{eq:apx:normalization_coefficient}
    \nonumber & A \sqrt{\frac{2kT}{m}} \int_{- \infty}^{+ \infty} e^{-mu_x^2/(2kT)} \sqrt{\frac{m}{2kT}} du_x = 1 \Rightarrow  \\ \nonumber \\ \nonumber
    & A \sqrt{\frac{2kT}{m}} \underbrace{\int_{- \infty}^{+ \infty} \sqrt{\pi}{e^{-x^2} dx}}_{=\sqrt{\pi}} = 1 \\ \nonumber \\
    & A = \sqrt{\frac{m}{2\pi kT}}
\end{align}
ώστε η σχέση \eqref{eq:apx:non_normalized_maxwell} να γράφεται τελικά:
\begin{equation}
    \label{eq:apx:normalized_maxwell_x_direction}
    f(u_x) = \sqrt{\frac{m}{2\pi kT}} e^{-mu_x^2/(2kT)}
\end{equation}

Η σχέση \eqref{eq:apx:normalized_maxwell_x_direction} ονομάζεται γραμμική πυκνότητα πιθανότητας (εφόσον μιλάμε για μία διάσταση). 'Ετσι, η πιθανότητα $dp(u_x)$ του να έχει ένα σωματίδιο ταχύτητα κατά τον άξονα των $x$ μεταξύ των τιμών $u_x$ και $u_x+du_x$ είναι:
\begin{equation}
    \label{eq:apx:gauss_probability_x}
    dp(u_x) = \sqrt{\frac{m}{2\pi kT}} e^{-mu_x^2/(2kT)} du_x = f(u_x) du_x
\end{equation}

Παρατηρούμε ότι η συνάρτηση που δίνει τη γραμμική πυκνότητα πιθανότητας $f(u_x)$ είναι συνάρτηση κατανομής Gauss, η οποία έχει τη μορφή
\begin{equation}
    \label{eq:apx:gauss_formula}
    G = C e^{-b^2u^2}
\end{equation}
όπου ο συντελεστής κανονικοποίησης $C = \sqrt{\frac{m}{2\pi kT}}$ καθορίζει το ύψος της καμπύλης, ενώ ο παράγοντας $b^2 = m/(2kT)$ καθορίζει το εύρος της κατανομής. Η κατανομή είναι κεντραρισμένη γύρω από το σημείο $u_x = 0$ καθώς λόγω των τυχαίων κινήσεων, η μέση διανυσματική ταχύτητα είναι μηδέν.

Επειδή ο ολικός αριθμός των σωματιδίων $N$, παραμένει σταθερός το ποσοστό $dN_x/N$ (ή ισοδύναμα η αριθμητική πυκνότητα $n=N/V$) των σωματιδίων που έχουν ταχύτητες κατά τον άξονα των $x$ στο διάστημα μεταξύ $u_x$ και $u_x+du_x$, θα ισούται με την πιθανότητα που υπολογίσαμε στη σχέση \eqref{eq:apx:gauss_probability_x}. Επομένως, ο αριθμός των μορίων που έχουν ταχύτητες στο διάστημα $\left[u_x, u_x + du_x \right]$ θα είναι:
\begin{align}
    \label{eq:apx:number_density_x_axis}
    \nonumber dN_x &= N dp(u_x) \longrightarrow dn_x = n dp(u_x) \Rightarrow\\ \nonumber \\ 
     dn_x &= n \sqrt{\frac{m}{2\pi kT}} e^{-mu_x^2/(2kT)} du_x = n f(u_x) du_x
\end{align}

Είναι προφανές ότι οι αντίστοιχες σχέσεις που δίνουν την πιθανότητα και την αριθμητική πυκνότητα των σωματιδίων με ταχύτητες κατά τη διεύθυνση $y$ και $z$ στα διαστήματα $\left[u_y, u_y + du_y \right]$ και $\left[u_z, u_z + du_z \right]$ αντίστοιχα, βρίσκονται αν αντικαταστήσουμε το σύμβολο $x$ με τα σύμβολα $y$ και $z$.

Τις περισσότερες φορές όμως δεν ενδιαφερόμαστε για τον αριθμό των σωματιδίων με ορισμένες ταχύτητες ως προς κάποια συγκεκριμένη διεύθυνση, κυρίως μάλιστα όταν δεν υπάρχει προτιμητέα κατεύθυνση στον χώρο από τη φυσική ή τη γεωμετρία του προβλήματος που μελετούμε. Συνήθως μας ενδιαφέρει η πιθανότητα $dp(u)$ ή το ποσοστό των σωματιδίων $dn(u)/n$ με συνολική ταχύτητα μεταξύ των τιμών $u$ και $u+du$, η οποία ταχύτητα προφανώς συνδέεται με τις ταχύτητες $u_x, u_y$ και $u_z$ μέσω της σχέσης:
\begin{equation}
    \label{eq:apx:total_velocity}
    u^2 = u_x^2 + u_y^2 + u_z^2
\end{equation}
Η πιθανότητα αυτή μπορεί να υπολογιστεί από τις σχέσεις \eqref{eq:apx:gauss_probability_x}, \eqref{eq:apx:number_density_x_axis} και \eqref{eq:apx:total_velocity}, αν χρησιμοποιήσουμε την αρχή του μοριακού χάους, σύμφωνα με την οποία οι ταχύτητες (άρα και οι κατανομές) κατά τις τρεις διευθύνσεις $x,y,z$ είναι ανεξάρτητες μεταξύ τους. Τότε, σύμφωνα με την θεωρία των πιθανοτήτων, η κοινή συνάρτηση κατανομής $f(u_x, u_y, u_z) $ που δίνει την πιθανότητα $dp(u)$ του να έχει ένα σωματίδιο ταχύτητες στα διαστήματα $[u_x, u_x + du_x], [u_y, u_y + du_y], [u_z, u_z + du_z]$, ισούται με το γινόμενο των $f(u_x), f(u_y)$ και $f(u_z)$. Έτσι, στις 3 διαστάσεις η πυκνότητα πιθανότητας είναι:
\begin{equation}
    \label{eq:apx:common_gauss_distribution}
    f(\boldsymbol{u}) = f(u_x, u_y, u_z) = f(u_x)f(u_y)f(u_z)
\end{equation}
και επομένως η πιθανότητα $dp(u)$ δίνεται από τη σχέση:
$$dp(u) = dp(u_x,u_y,u_z) =$$
\begin{align}
    \label{eq:apx:total_gauss_probability}
  \nonumber &= \sqrt{\frac{m^3}{(2 \pi kT)^3}} \exp \left(- \frac{m(u_x^2 + u_y^2 + u_z^2)}{2kT} \right) du_x du_y du_z \Rightarrow \\ \nonumber \\
    &\Rightarrow dp(u) = \sqrt{\frac{m^3}{(2 \pi kT)^3}} e^{- mu^2/(2kT)} du_x du_y du_z
\end{align}
όπου το γινόμενο $dV(u) = du_x du_y du_z$ εκφράζει τον στοιχειώδη όγκο στον χώρο των ταχυτήτων. Η σχέση \eqref{eq:apx:total_gauss_probability} απλοποιείται σημαντικά αν, αντί για καρτεσιανές, χρησιμοποιήσουμε σφαιρικές συντεταγμένες. Στην περίπτωση αυτή, ορίζεται έναν σφαιρικό κέλυφος με "ακτίνα" $u$ και συνιστώσες που δίνονται από τους μετασχηματισμούς:
\begin{align*}
    u_x & = u \sin{\theta} \cos{\phi} \\
    u_y & = u \sin{\theta} \sin{\phi} \\
    u_z & = u \cos{\theta}
\end{align*}
ενώ ο στοιχειώδης όγκος στο χώρο των ταχυτήτων μετασχηματίζεται βάσει της σχέσης $$dV = du_x du_y du_z = \left[ \frac{\partial (u_x, u_y, u_z)}{\partial (u, \theta, \phi)} \right] du d\theta d\phi$$

Η Ιακωβιανή υπολογίζεται ως:
\begin{equation*}
\displaystyle 
    \begin{bmatrix}
      \frac{\partial u_x}{\partial u} & 
        \frac{\partial u_x}{\partial \theta} & 
        \frac{\partial u_x}{\partial \phi} \\[2ex] % <-- 1ex more space between rows of matrix
      \frac{\partial u_y}{\partial u} & 
        \frac{\partial u_y}{\partial \theta} & 
        \frac{\partial u_y}{\partial \phi} \\[2ex]
      \frac{\partial u_z}{\partial u} & 
        \frac{\partial u_z}{\partial \theta} & 
        \frac{\partial u_z}{\partial \phi}
\end{bmatrix}
= \dots = u^2 \sin{\theta}
\end{equation*}
και άρα ο στοιχειώδης όγκος στο χώρο των ταχυτήτων εκφρασμένος σε σφαιρικές συντενταγμένες είναι $dV = u^2 \sin{\theta} du d\theta d\phi$ και, επειδή δεν μας ενδιαφέρει η διεύθυνση της ταχύτητας των σωματιδίων, μπορούμε να ολοκληρώσουμε την πιθανότητα ως προς τις γωνίες κατεύθυνσης
$$\int_{0}^{2\pi} \int_{0}^{\pi} d\phi \sin{\theta} d\theta = 4 \pi$$

Έτσι τελικά το ποσοστό των σωματιδίων με ταχύτητες μεταξύ $u$ και $u+du$ \textit{ανεξαρτήτως διεύθυνσης} είναι:
\begin{equation}
    \label{eq:apx:maxwell_boltzmann_speed_distribution}
    \frac{dn(u)}{n} = f(u) du = 4\pi \left(\frac{m}{2\pi kT}\right)^{\frac{3}{2}} e^{-\frac{mu^2}{2kT}} u^2 du
\end{equation}
Η συνάρτηση $f(u)$ είναι η συνάρτηση κατανομής της πυκνότητας πιθανότητας (πιθανότητα ανά μονάδα ταχύτητας) των ταχυτήτων και ονομάζεται κατανομή Maxwell-Boltzmann. Στο σημείο αυτό πρέπει να αναφέρουμε ότι η συνάρτηση κατανομής 
$$ f(u) = \left(\frac{m}{2\pi kT}\right)^{\frac{3}{2}} e^{-\frac{mu^2}{2kT}} u^2 $$
που εμφανίζεται στη σχέση \eqref{eq:apx:maxwell_boltzmann_speed_distribution}, αναφέρεται στην κατανομή των \textbf{μέτρων} των ταχυτήτων (Maxwell-Boltzmann speed distribution) και όχι στην κατανομή των διανυσματικών ταχυτήτων (Maxwell-Boltzmann velocity distribution) που δίνεται από τη συνάρτηση κατανομής της σχέσης \eqref{eq:apx:common_gauss_distribution}:
$$f(\boldsymbol{u}) = \left(\frac{m}{2\pi kT}\right)^{\frac{3}{2}} e^{- mu^2/(2kT)}$$

Παρατηρώντας τις δύο εκφράσεις, βλέπουμε ότι η συνάρτηση κατανομής των διανυσματικών ταχυτήτων είναι Γκαουσιανής φύσης (ως το γινόμενο των τριών επιμέρους Γκαουσιανών για την κάθε συνιστώσα), ενώ η συνάρτηση κατανομής των μέτρων των ταχυτήτων είναι κατανομή $\chi^2$ (chi square distribution) ως γινόμενο του Γκαουσιανού όρου και του πολυωνυμικού όρου $u^2$. Αυτό έχει ως αποτέλεσμα η κατανομή των μέτρων των ταχυτήτων να παρουσιάζει μια ασυμμετρία προς μεγάλες ταχύτητες (right-skewed) όπως φαίνεται και στο σχήμα \ref{fig:MBD_distribution}. Επίσης, ενώ η κατανομή των διανυσματικών ταχυτήτων είναι κεντραρισμένη γύρω από το μηδέν, η κατανομή των μέτρων των ταχυτήτων ξεκινάει από το μηδέν λόγω του όρου $u^2$. Αυτό είναι λογικό καθώς δεν νοείται αρνητικό μέτρο της ταχύτητας.


\begin{figure}[h]
   \centering
\begin{subfigure}[h]{0.45\textwidth}
	\centering
   	 \includegraphics[width = \linewidth]{Figures/MBD_temperatures.png} 
\end{subfigure}
\begin{subfigure}[h]{0.54\textwidth}
	\centering
	\includegraphics[scale=0.6]{Figures/MBD_velocities.png} 
    \end{subfigure}
    \caption{Κατανομή Maxwell-Boltzmann των μέτρων ταχυτήτων. Ο κάθετος άξονας εκφράζει τον αριθμό των σωματιδίων, αν και πολλές φορές επιλέγεται να δείχνει την πυκνότητα πιθανότητας αντ' αυτού η οποία είναι η πιθανότητα ανά μονάδα ταχύτητας να βρούμε ένα σωματίδιο με ταχύτητα κοντά στην $u$. \textbf{Left panel}: Εξάρτηση της κατανομής από τη θερμοκρασία. \textbf{Right Panel}: Ορισμός της πιο πιθανής ταχύτητας, της μέσης ταχύτητας, και της ταχύτητας $u_{\text{rms}}$.}
    \label{fig:MBD_distribution}
\end{figure}

Το ότι δεν υπάρχει προτιμητέα διεύθυνση στον χώρο και άρα η κατανομή των διανυσματικών ταχυτήτων ακολουθεί τη συμμετρική Γκαουσιανή κατανομή είναι εύκολα αντιληπτό. Γιατί όμως η συνάρτηση κατανομής των μέτρων των ταχυτήτων να είναι ασύμμετρη προς τα δεξιά, δηλαδή προς τις μεγάλες ταχύτητες; Η απάντηση είναι γιατί υπάρχουν πολλοί τρόποι να πάρουμε το μέτρο μιας μεγάλης ταχύτητας όταν συνυπολογίζουμε όλες τις κατευθύνσεις. Θα προσπαθήσουμε να το εξηγήσουμε δίνοντας ένα απλό παράδειγμα στον 2D χώρο:

Έστω δύο διανύσματα ταχύτητας $\boldsymbol{u_1}, \boldsymbol{u_2}$ με συνιστώσες $u_{1x}=1, u_{1y}=2$ και $u_{2x}=2, u_{2y}=1$ αντίστοιχα, ώστε:
\begin{align*}
    \boldsymbol{u_1} &= u_{1x} \hat{i} + u_{1y} \hat{j} = \hat{i} + 2 \hat{j} \\\\
    \boldsymbol{u_2} &= u_{2x} \hat{i} + u_{2y} \hat{j} = 2 \hat{i} +  \hat{j}
\end{align*}
Και τα δύο διανύσματα της ταχύτητας έχουν το ίδιο μέτρο: $$u_x = u_y = \sqrt{1^2 + 2^2} = \sqrt{2^2 + 1^2} = \sqrt{5}$$
Γενικεύοντας σε όλο το χώρο, γίνεται αντιληπτό ότι όσο μεγαλύτερο είναι το μέτρο της ταχύτητας, τόσο περισσότεροι συνδυασμοί των συνιστωσών $u_x, u_y, u_z$ (δηλαδή τόσο περισσότερα διανύσματα) υπάρχουν που θα δίνουν το ίδιο μέτρο. Άρα, ναι μεν όλες οι διευθύνσεις των ταχυτήτων είναι ισοπίθανες και άρα η συνάρτηση κατανομής τους θα είναι Γκαουσιανή, αλλά υπάρχουν περισσότερες καταστάσεις υψηλής ταχύτητας γεγονός που δίνει στην κατανομή την χαρακτηριστική ασυμμετρία προς τα δεξιά. 
Ο παράγοντας $4\pi u^2$ που υπάρχει στη σχέση \eqref{eq:apx:maxwell_boltzmann_speed_distribution}, λαμβάνει υπόψιν ακριβώς τη πυκνότητα καταστάσεων της ταχύτητας που είναι διαθέσιμη για τα σωματίδια.


\subsection{Χαρακτηριστικά και ιδιότητες}
Το ύψος (amplitude) της κατανομής Maxwell-Boltzmann δίνεται από τον συντελεστή κανονικοποίησης για τον οποίο γενικά ισχύει:
$$A \propto T^{-3/2}$$ 
Αντίστοιχα, για το εύρος της κατανομής ισχύει:
$$b^2 \propto T^{-1}$$
Άρα, υπό σταθερή μάζα, αν αυξάνεται η θερμοκρασία τότε ελλατώνεται το ύψος της καμπύλης ενώ παράλληλα φαρδαίνει όπως φαίνεται στο σχήμα \ref{fig:MBD_distribution}. Αυτό είναι λογικό καθώς το εμβαδόν κάτω από την καμπύλη της κατανομής μας δίνει τον συνολικό αριθμό των σωματιδίων. Έτσι, αν μειώσουμε τη θερμοκρασία του αερίου, η καμπύλη θα μετακινηθεί προς τα αριστερά και το ύψος θα αυξηθεί ώστε το εμβαδόν να παραμείνει σταθερό.

\subsubsection{Εύρεση πιθανότερης ταχύτητας}
Αν $u_p$ είναι η πιθανότερη ταχύτητα, τότε η καμπύλη παρουσιάζει μέγιστο στο $u_p$. Δηλαδή αρκεί
\begin{align*}
    u_p &= \frac{df(u)}{du} = 0 \Rightarrow \frac{d}{du} \left( A e^{-\frac{mu^2}{2kT}} u^2 \right) = 0 \\\\
    &\Rightarrow A \left[ \left( - \frac{mu}{kT} \right) e^{-\frac{mu^2}{2kT}} u^2 + 2u e^{-\frac{mu^2}{2kT}}  \right] = 0 \\\\
    &\Rightarrow 2u A e^{-\frac{mu^2}{2kT}} \left(1 - \frac{mu^2}{2kT} \right) = 0
\end{align*}
Η εξίσωση αυτή έχει τρεις λύσεις:
\begin{itemize}
    \item Αν $u=0$, τότε η κατανομή παρουσιάζει ελάχιστο.
    \item Αν $e^{-\frac{mu^2}{2kT}} = 0$, τότε $u \rightarrow \infty$ οπότε η κατανομή πάλι παρουσιάζει ελάχιστο.
    \item Αν $\frac{mu^2}{2kT} = 1$, τότε η κατανομή παρουσιάζει μέγιστο.
\end{itemize}
Έτσι, και σύμφωνα με το σχήμα \ref{fig:MBD_distribution}, η πιο πιθανή ταχύτητα είναι η 
\begin{equation}
    \label{eq:apx:most_probable_speed}
    \boxed{u_p = \sqrt{\frac{2kT}{m}}}
\end{equation}
και εκφράζει την πιο πιθανή τιμή της ταχύτητας που μπορεί να έχει ένα σωματίδιο στο αέριο.



\subsubsection{Στατιστικές ροπές}
Η συνάρτηση κατανομής Maxwell-Boltzmann περιέχει όλες τις πληροφορίες για τις στατιστικές ιδιότητες ενός τέλειου αερίου, δεν είναι όμως εύχρηστη επειδή ο υπολογισμός των πιθανοτήτων απαιτεί την ολοκλήρωση της σχέσης \eqref{eq:apx:maxwell_boltzmann_speed_distribution}, η οποία δεν είναι δυνατόν να γίνει αναλυτικά. Στην πράξη χρησιμοποιούμε συνήθως τις \textbf{στατιστικές ροπές} (moments) της σχέσης \eqref{eq:apx:maxwell_boltzmann_speed_distribution}, οι οποίες προκύπτουν πολλαπλασιάζοντας την συνάρτηση κατανομής επί της διάφορες δυνάμεις της ταχύτητας $u$ και ολοκληρώνοντας ως προς $u$. 
Η ροπή \textbf{μηδενικής τάξης} (μονοπολική/monopole) της σχέσης \eqref{eq:apx:maxwell_boltzmann_speed_distribution}

\begin{equation}
    \label{eq:apx:zeroth_order_moment}
    \int_{0}^{\infty} f(u) u^0 du = \int_{0}^{\infty} f(u) du = 1
\end{equation}
είναι εξ' ορισμού μονάδα, αφού παριστάνει την πιθανότητα του να έχει ένα σωματίδιο ταχύτητα $0 < u < \infty$, γεγονός που είναι προφανώς βέβαιο.
Η ροπή \textbf{πρώτης τάξης} (διπολική/dipole moment) μας δίνει, σύμφωνα με την θεωρία των πιθανοτήτων, τη \textbf{μέση ταχύτητα} $\langle u \rangle$ των σωματιδίων του αερίου
\begin{equation}
    \label{eq:apx:first_order_moment}
    \langle u \rangle = \int_{0}^{\infty} f(u) u du = A \int_{0}^{\infty} e^{-\frac{mu^2}{2kT}} u^3 du
\end{equation}
όπου $A = 4\pi \frac{m}{2\pi kT}^{3/2}$. Από πίνακες ολοκληρωμάτων γνωρίζουμε ότι $$\int_{0}^{\infty} x^3 e^{-ax^2} dx = \frac{1}{2a^2}$$
Χωρίς τη χρήση αυτού του τύπου, το ολοκλήρωμα της σχέσης \eqref{eq:apx:first_order_moment} λύνεται με την διαδοχική ολοκλήρωση κατά μέρη ως εξής:
\begin{align*}
    \langle u \rangle &= A \int_{0}^{\infty} e^{-\frac{mu^2}{2kT}} u^3 du = A \left( -\frac{kT}{m} \right) \int_{0}^{\infty} u^2 \underbrace{e^{-\frac{mu^2}{2kT}} d \left( -\frac{mu^2}{2kT} \right)}_{d\left(e^{- \frac{mu^2}{2kT}} \right)} = \\\\
   &= A \left( -\frac{kT}{m} \right) \int_{0}^{\infty} u^2 d\left(e^{- \frac{mu^2}{2kT}} \right) =  A \left( -\frac{kT}{m} \right) \left[ \cancelto{0}{\left. u^2 e^{-\frac{mu^2}{2kT}} \right|_{0}^{\infty}} - \right.\\\\
   &- \left. \int_{0}^{\infty} e^{-\frac{mu^2}{2kT}} 2u du \right] = 2A \left(\frac{kT}{m} \right) \int_{0}^{\infty} e^{-\frac{mu^2}{2kT}} u du = \\\\
   &= -2A \left(\frac{kT}{m} \right)^2 \int_{0}^{\infty} e^{-\frac{mu^2}{2kT}} d \left( -\frac{mu^2}{2kT} \right) = -2A \left(\frac{kT}{m} \right)^2 \int_{0}^{\infty} d\left(e^{- \frac{mu^2}{2kT}} \right) = \\\\
   &= -2A \left(\frac{kT}{m} \right)^2 \cancelto{-1}{\left[ e^{- \frac{mu^2}{2kT}} \right]_{0}^{\infty}} = 2A \left(\frac{kT}{m} \right)^2 
\end{align*}
Αντικαθιστώντας την τιμή της σταθεράς $A$ προκύπτει ότι η μέση ταχύτητα είναι
\begin{equation}
    \label{eq:apx:mean_speed}
     \boxed{\langle u \rangle = \sqrt{\frac{8kT}{\pi m}}}
\end{equation}
Από το σχήμα \ref{fig:MBD_distribution} φαίνεται ότι η μέση ταχύτητα βρίσκεται πιο δεξιά από την πιο πιθανή τιμή της ταχύτητας. Αυτό συμβαίνει γιατί η κατανομή παρουσιάζει μια "ουρά" προς τα δεξιά, η οποία "τραβάει" την μέση τιμή προς τα δεξιά της κορυφής. Με άλλα λόγια, υπάρχουν περισσότερες υψηλές από χαμηλές ταχύτητες.

Τέλος, η ροπή \textbf{δεύτερης τάξης} (τετραπολική ροπή/quadruple moment) μας δίνει τη \textbf{μέση τετραγωνική ταχύτητα} των σωματιδίων του αερίου
\begin{equation}
    \label{eq:apx:quadruple_moment}
    \langle u^2 \rangle = u_{\text{rms}} = \int_{0}^{\infty} f(u) u^2 du = A \int_{0}^{\infty} e^{-\frac{mu^2}{2kT}} u^4 du
\end{equation}
Η λύση του ολοκληρώματος έχει ως εξης:
\begin{align*}
    u_{\text{rms}} & = A \int_{0}^{\infty} e^{-\frac{mu^2}{2kT}} u^4 du = -A \left( \frac{kT}{m} \right) \int_{0}^{\infty} e^{-\frac{mu^2}{2kT}} u^3 d \left( - \frac{mu^2}{2kT} \right) = \\\\
    &= -A \left( \frac{kT}{m} \right) \int_{0}^{\infty} u^3 d \left( e^{- \frac{mu^2}{2kT}} \right) = -A \left( \frac{kT}{m} \right) \left [ \cancelto{0}{\left. u^3 e^{- \frac{mu^2}{2kT}} \right|_{0}^{\infty}} - \right. \\\\
    &- \left. 3 \int_{0}^{\infty} u^2 e^{-\frac{mu^2}{2kT}} du \right] = \frac{3kT}{m} \int_{0}^{\infty} A e^{-\frac{mu^2}{2kT}} u^2 du = \\\\
    &= \frac{3kT}{m} \cancelto{1}{\int_{0}^{\infty} f(u) du} = \frac{3kT}{m}
\end{align*}
Άρα, η μέση τετραγωνική ταχύτητα των σωματιδίων είναι
\begin{equation}
    \label{eq:apx:rms_speed}
    \boxed{u_{\text{rms}} = \frac{3kT}{m}}
\end{equation}

Από τη σχέση \eqref{eq:apx:rms_speed} προκύπτει και ότι η μέση κινητική ενέργεια των σωματιδίων ενός τέλειου αερίου δίνεται από τη σχέση
\begin{equation}
    \label{eq:apx:mean_kinetic_energy}
    \langle E \rangle = \frac{1}{2} m \langle u^2 \rangle = \frac{3}{2} kT
\end{equation}
που συνδέει την μέση κινητική ενέργεια με τη θερμοκρασία του αερίου.

Στο σημείο αυτό αξίζει να αναφέρουμε ότι η τιμή της συνάρτησης κατανομής Maxwell-Boltzmann πέφτει πολύ γρήγορα με την ταχύτητα. Έτσι συνήθως δεχόμαστε πως υπάρχει "αισθητή" αριθμητική πυκνότητα σωματιδίων μέχρι ταχύτητα ίση με $3 \sqrt{u_{\text{rms}}}$.
%     \chapter{Πυρηνική Αστροφυσική}
\label{apx:nucleosynthesis}

\section{Πυρηνοσύνθεση βαρέων στοιχείων}
Στην πυρηνική αστροφυσική, ο όρος "βαρέα στοιχεία" αναφέρεται στους πυρήνες που εντοπίζονται μετά την ομάδα του σιδήρου καθώς γι' αυτούς τους πυρήνες τα πράγματα αλλάζουν δραματικά. Γνωρίζουμε από παρατηρήσεις ότι ο σίδηρος\footnote{Στην πραγματικότητα, το $^{62}$Ni είναι αυτό με τη μεγαλύτερη ενέργεια σύνδεσης (M. P. Fewell. "\textit{The atomic nuclide with the highest mean binding energy}". American Journal of Physics, 63:653–658, July 1995). Ο Fe$^{56}$ όμως έχει μικρότερη μάζα ανα νουκλεόνιο, σε σχέση με το νικέλιο, λόγω του ότι έχει λιγότερα νετρόνια και αυτό του δίνει το "προβάδισμα" έναντι του Ni$^{62}$ όσον αφορά την ευστάθεια του πυρήνα.} έχει την μεγαλύτερη ενέργεια σύνδεσης ανα νουκλεόνιο και από εκεί και πέρα μειώνεται συνεχώς (σχήμα \ref{fig:apx:binding_energy_per_nucleon}) καθιστώντας έτσι την διαδικασία της φωτοδιάσπασης --χρησιμοποιώντας ενεργειακά κριτήρια-- μη-ευνοϊκή. Επιπρόσθετα, το συνεχώς αυξανόμενο φράγμα Coulomb μειώνει την πιθανότητα εμφάνισης του φαινομένου σήραγγος και άρα η πυρηνοσύνθεση που οφείλεται σε επαγωγή φορτισμένων σωματιδίων σταματά με την καύση του πυριτίου. Σε αντίθετη περίπτωση, οι αναμενόμενες πυρηνικές αναλογίες θα ήταν πολύ μικρότερες από αυτές που παρατηρούμε (Χ. Ελευθεριάδης, \textit{Πυρηνοσύνθεση: Δημιουργία των Στοιχείων στο Σύμπαν}, pp. 46). Με λίγα λόγια, όσον αφορά την αστρική πυρηνοσύνθέση, πυρήνες με μαζικό αριθμό μεγαλύτερο του 56 δεν μπορούν να σχηματιστούν μέσω θερμοπυρηνικών αντιδράσεων (π.χ. πυρηνική σύντηξη) και άλλοι μηχανισμοί πρέπει να ληφθούν υπόψιν.

\begin{figure}
  \centering
    \includegraphics[scale=0.7]{Figures/benergy.jpg}
    \caption{Ενέργεια σύνδεσης ανα νουκλεόνιο συναρτήσει του μαζικού αριθμού Α.}
    \label{fig:apx:binding_energy_per_nucleon}
\end{figure}

Για την πυρηνοσύνθεση των βαρέων στοιχείων (βαρύτερων του σιδήρου) και την ερμηνεία των παρατηρούμενων ισοτοπικών αναλογιών οι Burbidge, Burbidge, Fowler και Hoyle πρότειναν το 1957 (το περίφημο B$^2$FH paper) ότι τα στοιχεία αυτά πρέπει να παράγονται κυρίως με αντιδράσεις αρπαγής νετρονίων και συνεπακόλουθες β-διασπάσεις (E. Margaret Burbidge, G. R. Burbidge, William A. Fowler, and F. Hoyle. \textit{Synthesis of the elements in stars}. Rev. Mod. Phys., 29:547–650, Oct 1957).
%% --------------------------------------------------------------------------------------------------------%%
%% --------------------------------------------------------------------------------------------------------%%
%% --------------------------------------------------------------------------------------------------------%%
\subsection{Αρπαγή νετρονίων}
Με τον όρο "αρπαγή νετρονίων" εννοούμε την πυρηνική αντίδραση κατά την οποία ένα ή περισσότερα νετρόνια συγκρούονται και "απορροφούνται" από τον πυρήνα προς σχηματισμό ενός βαρύτερου πυρήνα. Η αρπαγή νετρονίων αποτελεί την πιο απλή περίπτωση περιγραφής πυρηνικών αντιδράσεων λόγω της απουσίας δυνάμεων Coulomb. Η αντίδραση αυτή παίζει σημαντικότατο ρόλο, όπως θα δούμε παρακάτω, στην πυρηνοσύνθεση βαρέων στοιχείων στα άστρα μέσω δύο βασικών διεργασιών, την αργή διαδικασία (slow-process) και την γρήγορη διαδικασία (rapid-process). Οι όροι προέρχονται από την συσχέτιση του μέσου χρόνου που απαιτείται σε δεδομένο αστρικό περιβάλλον για αρπαγή νετρονίου, με το μέσο χρόνο ζωής του υπάρχοντος πυρήνος με β-διάσπαση (Χ. Ελευθεριάδης, \textit{Πυρηνοσύνθεση: Δημιουργία των Στοιχείων στο Σύμπαν}, pp. 47). 

Υποθέτοντας ότι η ενεργός διατομή για αρπαγή νετρονίου είναι ανεξάρτητη της ενέργειας, τότε ο μέσος χρόνος που απαιτείται για να γίνει η αρπαγή είναι:

\begin{eqnarray}
\label{eq:apx:mean_time_neutron_capture}
\tau_n = \frac{1}{N_n \langle \sigma u \rangle}\approx \frac{1}{N_n \langle \sigma \rangle u_T} = \frac{1}{N_n \langle \sigma \rangle} \left ( \frac{\mu_n}{2kT} \right )^{1/2} 
\end{eqnarray}     
όπου  $N_n$ το πλήθος των νετρονίων, $\langle \sigma \rangle$ η μέση ενεργός διατομή για σύλληψη νετρονίων, $u_T$ η ταχύτητα των θερμικών νετρονίων και $\mu_n$ η ανηγμένη μάζα των νετρονίων. Για μια τυπική ενεργό διατομή νετρονίων της τάξης των $\langle \sigma \rangle \ \sim 10^{-25}$ cm$^2$ και θερμοκρασίας των $5 \times 10^8 \,\text{K}$, προκύπτει ότι $\tau_n \approx 10^9/N_n$ χρόνια.

Ακόμα, ο μέσος χρόνος για μία β-διάσπαση όπως φαίνεται στην εξίσωση \eqref{eq:apx:beta_decay_reaction}, είναι της τάξης των μερικών ωρών.

\begin{eqnarray}
\label{eq:apx:beta_decay_reaction}
(Z, A+1) \longrightarrow (Z+1, A+1) + e^{-} + \overline{\nu}_e
\end{eqnarray}
%% --------------------------------------------------------------------------------------------------------%%
%% --------------------------------------------------------------------------------------------------------%%
%% --------------------------------------------------------------------------------------------------------%%
\subsubsection{Η s-διεργασία}
Ας υποθέσουμε ένα αστρικό περιβάλλον στο οποίο η πυκνότητα των νετρονίων είναι χαμηλή, $N_n \sim 10^5 $ cm$^{-3}$, και άρα ο χρόνος που απαιτείται για να γίνει σύλληψη νετρονίου από έναν πυρήνα είναι μεγάλος. Μέσα σε αυτό το χρονικό διάστημα, ο πυρήνας θα κάνει αρπαγή ενός νετρονίου και ταυτόχρονα θα εκπέμψει ένα φωτόνιο σύμφωνα με την αντίδραση

\begin{eqnarray}
\label{eq:apx:neutron_capture_reaction}
(Z, A) + n \longrightarrow (Z, A+1) + \gamma
\end{eqnarray}
Το ισότοπο που θα προκύψει μπορεί να είναι είτε σταθερό, είτε ασταθές. Εάν είναι σταθερό τότε, μέσω της ίδιας διαδικασίας, θα δημιουργηθεί ένα νέο ισότοπο $(Z, A+2)$ κοκ. Στην περίπτωση όμως που το ισότοπο που δημιουργείται είναι ασταθές όσον αφορά τη β-διάσπαση (λόγω περισσείας νετρονίων), τότε θα διασπαστεί πριν προλάβει να λάβει χώρα η επόμενη αρπαγή νετρονίου και το ισότοπο που θα δημιουργηθεί θα είναι το $(Z+1, A+1)$ οδεύοντας με αυτό τον τρόπο προς την κοιλάδα σταθερότητας. Αυτή είναι η λεγόμενη \textbf{s-διεργασία} και συμβαίνει όταν ο χρόνος διάσπασης ενός ασταθούς πυρήνα είναι μικρότερος από τον χρόνο που απαιτείται για να κάνει αρπαγή νετρονίου. Τα ισότοπα που εξελίσσονται βάσει αυτής της διεργασίας βρίσκονται επάνω ή πολύ κοντά στην κοιλάδα σταθερότητας, οπότε η s-διεργασία είναι υπεύθυνη για την παραγωγή της πλειονότητας των στοιχείων με μαζικό αριθμό από 63 εως 209 καθώς ο $^{209}$Bi είναι ο πιο βαρύς σταθερός πυρήνας. Το ισότοπο με $A=210$ παράγεται με νετρονική αρπαγή από το $^{209}$Bi και διασπάται με α-εκπομπή στο $^{206}$Pb, έτσι ένας μικρός κύκλος που αποτελείται από μια αρπαγή νετρονίου και μια εκπομπή σωματιδίου άλφα τερματίζει την s-διεργασία\footnote{Πλήρης ανάλυση για τον τερματισμό της s-διεργασίας μπορεί να βρεθεί στο paper των Clayton και Rassbach, \textit{Termination of the s-Process}, \textit{The Astrophysical Journal, Vol. 148, April 1967.}} (D. Clayton, \textit{Principles of Stellar Evolution and Nucleosynthesis}, McGraw-Hill Book Company, New York, 1968, pp. 560).

\begin{figure}
    \centering
    \includegraphics[scale=0.7]{Figures/sprocess.png}
    \caption{Η διαδρομή της s-διεργασίας.}
    \label{fig:apx:sprocess_path}
\end{figure}
%% --------------------------------------------------------------------------------------------------------%%
%% --------------------------------------------------------------------------------------------------------%%
%% --------------------------------------------------------------------------------------------------------%%
\subsubsection{Η r-διεργασία}
Στη συνέχεια ας αναλογιστούμε ένα διαφορετικό περιβάλλον όπου η ροή των νετρονίων είναι πολύ υψηλή, $N_n \sim$ $10^{23} \,\text{cm}^{-3}$. Τόσο υψηλές ροές νετρονίων παρατηρούνται κατά την έκρηξη υπερκαινοφανών αστέρων (supernovae). Σε αυτό το σενάριο, έχουμε διαδοχικές συλλήψεις νετρονίων σε πολύ μικρή χρονική κλίμακα, της τάξης των millisecond, και έτσι τα ενδιάμεσα ισότοπα δεν προλαβαίνουν να διασπασθούν οπότε δημιουργούνται πυρήνες εξαιρετικά πλούσιοι σε νετρόνια. Οι πυρήνες αυτοί εντοπίζονται μακριά από την κοιλάδα σταθερότητας και συνεπώς αυξάνεται η αστάθεια τους όσο αυξάνεται ο αριθμός των νετρονίων που συλλαμβάνουν. Φυσικά υπάρχει ένα ανώτατο όριο στο πόσο μπορεί να αυξηθεί ο αριθμός των νετρονίων στα διάφορα στοιχεία. Το όριο αυτό ονομάζεται \textbf{neutron drip line} και είναι το όριο στο οποίο η ενέργεια σύνδεσης του τελευταίου νετρονίου μηδενίζεται. 
Όσο πλησιάζουμε αυτό το όριο, οι πυρήνες γίνονται όλο και πιο ασταθείς και ο μέσος χρόνος που απαιτείται για να πραγματοποιηθεί μία β-διάσπαση γίνεται συγκρίσιμος με τον μέσο χρόνο που απαιτείται για να γίνει μία αρπαγή νετρονίου. Έτσι, καθίσταται δυνατόν να προηγηθεί μία β-διάσπαση από μία σύλληψη νετρονίου οπότε ο πυρήνας κάνει ένα βήμα προς την κοιλάδα σταθερότητας. Το στοιχείο όμως που δημιουργήθηκε από την β-διάσπαση δεν προχωράει περαιτέρω καθώς είναι πιο σταθερό από το μητρικό του και άρα ο μέσος χρόνος διάσπασης του είναι μεγαλύτερος από τον αντίστοιχο της αρπαγής νετρονίων. Πρέπει να περιμένουμε λοιπόν μέχρι να δημιουργηθεί ξανά ένα εξόχως βραχύβιο ισότοπο το οποίο με την σειρά του θα κάνει το δικό του βήμα προς τη κοιλάδα σταθερότητας κοκ. 
Στην περίπτωση που δημιουργηθεί ένα ισότοπο με μαγικό αριθμό νετρονίων (κλειστοί νετρονικοί φλοιοί), εμφανίζει μεγάλη αντίσταση στην αρπαγή νετρονίων καθώς τείνει να διασπαστεί αμέσως. Όταν όμως υπάρχει μεγάλη ροή νετρονίων δεν είναι απίθανο να γίνουν διαδοχικές συλλήψεις πριν προλάβει να διασπαστεί και έτσι να συνεχιστεί η r-διεργασία. Οι κορυφές όπου δημιουργούνται στοιχεία με κλειστούς νετρονικούς φλοιούς καθώς και η επίδραση που έχουν στην εξέλιξη της r-διεργασίας φάινεται στο σχήμα \ref{fig:apx:rprocess_path}.
Η διαδικασία αυτή ονομάζεται \textbf{r-διεργασία} και συμβαίνει όταν ο μέσος χρόνος διάσπασης ενός πυρήνα είναι μεγαλύτερος από τον μέσο χρόνο στον οποίο κάνει αρπαγή νετρονίων. Ο τερματισμός της διεργασίας αυτής πραγματοποιείται μέσω φαινομένων σχάσης.

\begin{figure}
    \centering
    \includegraphics[scale=0.7]{Figures/processes.jpg}
    \caption{Η διαδρομή της r-διεργασίας. Η διαφορά της με την s-διεργασία είναι εμφανής καθώς φαίνεται ο τρόπος που απομακρύνεται από την κοιλάδα σταθερότητας. Τα κάθετα "σκαλοπατάκια" που εμφανίζονται στο διάγραμμα αντιστοιχούν σε μαγικούς αριθμούς όπου η r-διεργασία εμφανίζει ισχυρή αντίσταση.}
    \label{fig:apx:rprocess_path}
\end{figure}

Είναι φανερό ότι, μόλις σταματήσει η ακτινοβόληση με νετρόνια, τα r-ισότοπα που έχουν δημιουργηθεί θα αρχίσουν να υπόκεινται σε β-διασπάσεις --όντας ασταθή-- λόγω του δυσανάλογα μεγάλου αριθμού νετρονίων που έχουν. Γενικά, μπορούμε να έχουμε την παραγωγή ισοτόπων και με τις δύο διεργασίες που έχουμε αναφέρει μέχρι στιγμής, όπως επίσης και ισότοπα που προέρχονται αποκλειστικά μόνο από μία από αυτές. Αν για παράδειγμα, ένας r-πυρήνας καταλήξει μετά από τις διαδοχικές β-διασπάσεις σε σταθερό ισότοπο $(Z, A)$, τότε αν το επόμενο ισότοπο $(Z+1, A)$ είναι επίσης σταθερό, πρέπει αναγκαστικά να έχει δημιουργηθεί με την s-διεργασία και όχι με την r. 
Το τελικό αποτέλεσμα στις αναλογίες των στοιχείων είναι να έχουμε ισότοπα που προέρχονται στην πλειοψηφία τους από την s-διεργασία αλλά και ισότοπα που προέρχονται αποκλειστικά από την r-διεργασία. Οι κοσμικές αναλογίες των βαρέων στοιχείων φαίνονται στο σχήμα \ref{fig:apx:cosmic_abundances}.


Έχοντας εξασφαλίσει μία βασική γνώση για τον τρόπο δημιουργίας των βαρέων στοιχείων σε διάφορες αστρικές συνθήκες, μπορούμε να κάνουμε κάποιες βασικές παρατηρήσεις. Όπως αναφέρει ο Clayton στο βιβλίο του \textit{Principles of Stellar Evolution and Nucleosynthesis}, η σύνθεση βαρέων στοιχείων μέσω της s-διεργασίας μπορεί να ξεκινήσει από οποιονδήποτε πυρήνα με $A > 8$. Όμως η ενεργός διατομή για αρπαγή νετρονίου στους ελαφρείς πυρήνες είναι κατά μέσο όρο πολύ μικρότερη από την αντίστοιχη ενεργό διατομή για πυρήνες μετά την ομάδα του σιδήρου. Έτσι, απαιτείται πολύ μεγαλύτερη ροή νετρονίων για την δημιουργία ενός βαρέως πυρήνα ξεκινώντας από το πυρίτιο, για παράδειγμα, απ' ότι ξεκινώντας από το σίδηρο. Γίνεται αντιληπτό λοιπόν ότι λόγω του περιορισμένου αριθμού ελεύθερων νετρονίων που είναι διαθέσιμα, είναι πολύ πιο ωφέλιμο να έχουμε σύνθεση βαρέων στοιχείων χρησιμοποιώντας την ομάδα του σιδήρου σαν "δότες" (seed nuclei) αντί τα ελαφρύτερα στοιχεία. 

Στο σημείο αυτό, γίνεται ξεκάθαρο ότι οι αντιδράσεις και οι μηχανισμοί παραγωγής ελεύθερων νετρονίων σε ένα δεδομένο αστρικό περιβάλλον είναι μείζωνος σημασίας για την εξέλιξη της s-διεργασίας. Γενικά, τα ελεύθερα νετρόνια δεν βρίσκονται σε αφθονία κατά την διάρκεια των φάσεων καύσης των πυρηνικών καυσίμων ενός αστέρα όπως περιγράφηκαν στο κεφάλαιο 3. Γι' αυτό το λόγο είναι σημαντική η ύπαρξη αναλυτικών μοντέλων αστρικής εξέλιξης ώστε να γίνει κατανοητή η πηγή αυτών των νετρονίων. Σήμερα, θεωρείται παραδεκτό ότι η κύρια πηγή ελεύθερων νετρονίων προέρχεται από αρπαγές σωματιδίων-α στα διάφορα στρώματα φλοιών ενός αστέρα σύμφωνα με τις αντιδράσεις:

\begin{eqnarray}
\rm C^{13}  &(\alpha, n)&  \rm O^{16} \label{eq:apx:c13+alpha_o16+n}\\
\rm O^{17}  &(\alpha, n)&  \rm Ne^{20} \label{eq:apx:o17+alpha_ne20+n}\\
\rm Ne^{21}  &(\alpha, n)& \rm Mg^{24} \label{eq:apx:ne21+alpha_mg24+n}
\end{eqnarray}
όπου η \eqref{eq:apx:o17+alpha_ne20+n} είναι η πιο πολλά υποσχόμενη από αυτές, απλά γιατί το O$^{17}$ φαίνεται να παράγεται και να υπάρχει σε μεγαλύτερη αφθονία σε σχέση με τα άλλα δύο.    

Το τελικό συμπέρασμα που πρέπει να κρατήσουμε για την s-διεργασία είναι ότι: 

\begin{itemize}
\item γίνεται στο εσωτερικό των άστρων.
\item απαιτεί μικρή ροή ελεύθερων νετρονίων ώστε τα ασταθή ισότοπα που προκύπτουν να προλαβαίνουν να υποστούν β-διάσπαση.
\item μπορούμε να υποθέσουμε ρεαλιστικά ότι τροφοδοτείται από την ομάδα του σιδήρου.
\item λαμβάνει χώρα κοντά στη κοιλάδα σταθερότητας και δημιουργεί ισότοπα με κλειστούς νετρονικούς φλοιούς.
\end{itemize}


Αντίστοιχα τα βασικά σημεία που πρέπει να μας μείνουν σχετικά με την r-διεργασία είναι:

\begin{itemize}
\item γίνεται κατά τα τελευταία στάδια πριν την καταστρεπτική έκρηξη ενός υπερκαινοφανούς αστέρα (supernova).
\item απαιτεί πολύ μεγάλη ροή ελεύθερων νετρονίων ώστε τα ασταθή ισότοπα που προκύπτουν να μην προλαβαίνουν να υποστούν β-διάσπαση. Τέτοιες συνθήκες κυριαρχούν στα supernovae.
\item λαμβάνει χώρα μακριά από τη κοιλάδα σταθερότητας και είναι υπεύθυνη για ισότοπα ιδαιτέρως πλούσια σε νετρόνια.
\item  Τα τελικά σταθερά ισότοπα που δημιουργούνται έχουν νετρονικό αριθμό μικρότερο του αντιστοιχούντος σε κλειστό νετρονικό φλοιό.
\end{itemize}


\begin{figure}
    \centering
    \includegraphics[scale=0.7]{Figures/abundance.jpg}
    \caption{Κοσμικές αναλογίες των βαρέων στοιχείων συναρτήσει του μαζικού αριθμού. Σημαντική λεπτομέρεια ότι οι κορυφές αριστερά των αντιστοιχουσών σε κλειστούς νετρονικούς φλοιούς (Ν=50, 82, 126) οφείλονται σε ισότοπα προερχόμενα από την r-διεργασία.}
    \label{fig:apx:cosmic_abundances}
\end{figure}
%% --------------------------------------------------------------------------------------------------------%%
%% --------------------------------------------------------------------------------------------------------%%
%% --------------------------------------------------------------------------------------------------------%%
\subsection{Αρπαγή πρωτονίων}
Μέχρι στιγμής είδαμε το πως μπορούν να δημιουργηθούν πυρήνες πλούσιοι σε νετρόνια, μέσω της αρπαγής νετρονίων. Υπάρχουν όμως και κάποιοι πυρήνες, πλούσιοι σε πρωτόνια, η παρουσία των οποίων δεν μπορεί να ερμηνευθεί με καμία διαδικασία σύλληψης νετρονίων. Ένας τρόπος δημιουργίας τέτοιων πυρήνων είναι η αρπαγή πρωτονίου προς σχηματισμό βαρύτερων ισοτόπων για την οποία θα μιλήσουμε σε αυτό το υποκεφάλαιο. Οι πυρήνες οι οποίοι είναι πλούσιοι σε πρωτόνια και δεν μπορούν να παραχθούν μέσω της s ή της r-διεργασίας, ονομάζονται \textbf{p-πυρήνες}.
%% --------------------------------------------------------------------------------------------------------%%
%% --------------------------------------------------------------------------------------------------------%%
%% --------------------------------------------------------------------------------------------------------%%
\subsubsection{Η p-διεργασία}
Ο όρος p-διεργασία παρουσιάζεται για πρώτη φορά στο διάσημο B$^2$FH paper στο οποίο προτείνεται σαν ο μοναδικός μηχανισμός παραγωγής p-πυρήνων σε υπερκαινοφανείς αστέρες τύπου ΙΙ (type-II supernovae). Αργότερα, αποδείχτηκε ότι δεν πληρούνται οι συνθήκες σε τέτοιου είδους υπερκαινοφανών.
Η διεργασία έχει ως αφετηρία τα βαρέα ισότοπα τα οποία προήλθαν από την s ή την r διεργασία. Στη συνέχεια, ένα πρωτόνιο συλλαμβάνεται μέσω της γενικής αντίδρασης $(p, \gamma)$, αλλάζοντας έτσι το ισότοπο σ' ένα καινουργιο χημικό στοιχείο ενώ ταυτόχρονα, αλλάζει και ο λόγος νετρονίων-πρωτονίων οδηγώντας μας έτσι σε πυρήνες πλούσιους σε πρωτόνια.

Είναι φανερό ότι η λογική αυτή δεν θα οδηγήσει σε πολύ βαρείς πυρήνες λόγω του ότι η θερμοκρασία του πλάσματος δεν αυξάνει αυθαίρετα με αποτέλεσμα τα πρωτόνια να μην έχουν την κατάλληλη ενέργεια για να ξεπεράσουν το συνεχώς αυξανόμενο φράγμα Coulomb. Αλλά ακόμα κι αν η θερμοκρασία αυξάνονταν με αυθαίρετο τρόπο, οι επικείμενες φωτοδιασπάσεις δεν θα επέτρεπαν τον σχηματισμό βαρέων πυρήνων επειδή θα αφαιρούσαν πρωτόνια με πιο γρήγορο ρυθμό από τις αρπαγές πρωτονίων. Ένας εναλλακτικός τρόπος από την p-διεργασία θα μπορούσε να γίνει σ' ένα περιβάλλον όπου υπάρχει υπερβολικά μεγάλη ροή πρωτονίων με αποτέλεσμα να αυξάνεται ο ρυθμός των αρπαγών χωρίς να χρειαστεί να αυξηθεί σημαντικά η θερμοκρασία του πλάσματος. Τέτοιους μηχανισμούς θα εξετάσουμε στη συνέχεια.
%% --------------------------------------------------------------------------------------------------------%%
%% --------------------------------------------------------------------------------------------------------%%
%% --------------------------------------------------------------------------------------------------------%%
\subsubsection{Η rp-διεργασία}
Η συγκεκριμένη διαδικασία παίρνει το όνομά της από τον όρο "\textbf{ταχεία σύλληψη πρωτονίου}" (rapid proton capture process) και είναι προφανές ότι απαιτεί συγκεκριμένες συνθήκες ροής πρωτονίων και θερμοκρασίας προκειμένου να ξεπεραστεί το φράγμα Coulomb. Τέτοιες συνθήκες πιστεύεται ότι υπάρχουν σε διπλά συστήματα αστέρων, που αποτελούνται συνήθως από έναν αστέρα νετρονίων και έναν συνοδό αστέρα\footnote{Τα συστήματα αυτά ονομάζονται αλλιώς και x-ray bursters.}. Σ' ένα τέτοιο σύστημα, αστρική ύλη από το συνοδό αστέρα έλκεται από το έντονο βαρυτικό πεδίο του αστέρα νετρονίων και δημιουργείται έτσι ένας περιστρεφόμενος δίσκος συσσώρευσης (accretion disk). Με αυτό τον τρόπο ενεργοποιείται μία εκρηκτική καύση υδρογόνου στα πλαίσια της οποίας εξελίσσεται και η rp-διεργασία. Δεν είναι απίθανο να έχουμε παρόμοια φαινόμενα και στον δίσκο συσσώρευσης γύρω από μία μαύρη τρύπα (R. Surman ,et.al., \textit{Heavy element synthesis in neutrino-processed black hole accretion disk ejecta}, Proceedings of Science, XIII Nuclei in the Cosmos, 7-11 July 2014, Debrecen, Hungary).
Λόγω της συνεχούς αύξησης των πρωτονίων, η διαδικασία αυτή οδηγεί σε παραγωγή ισοτόπων που απομακρύνονται από την κοιλάδα σταθερότητας και πλησιάζουν προς την λεγόμενη \textbf{proton drip line} η οποία ορίζεται σε αντιστοιχία με την neutron drip line. Όσο πλησιάζουμε σε αυτό το όριο, οι άλφα διασπάσεις, η β$^{+}$ διάσπαση καθώς και η εκπομπή πρωτονίου γίνονται όλο και πιο συχνές και οδηγούν τα ασταθή ισότοπα πίσω προς την κοιλάδα σταθερότητας. Το τελευταίο ισότοπο το οποίο μπορεί να παράγει αυτή η διεργασία είναι το Te $^{107}$ το οποίο τείνει να διασπαστεί με α-διάσπαση, οπότε κλείνει με αυτόν τον τρόπο ο κύκλος της rp-διεργασίας.
%% --------------------------------------------------------------------------------------------------------%%
%% --------------------------------------------------------------------------------------------------------%%
%% --------------------------------------------------------------------------------------------------------%%
\subsubsection{Η pn-διεργασία}
Μία ακόμα διεργασία η οποία μπορεί να συμβαίνει παράλληλα με την rp-process είναι η λεγόμενη \textbf{pn-διεργασία} (neutron-rich rapid proton capture) και πρόκειται ουσιαστικά για αντιδράσεις τύπου $(n, p)$. Μέσω των αντιδράσων αυτών μπορεί εύκολα να ξεπεραστούν τα σημεία αναμονής (waiting points)\footnote{Όταν ένας πυρήνας δεν μπορεί να προχωρήσει παρακάτω προς σχηματισμό βαρύτερου ισοτόπου είτε λόγω φωτοδιασπάσεων, είτε λόγω προσέγγισης στη drip line, τότε πρέπει να περιμένει μέχρι να β-διασπαστεί ή μέχρι να βρεθεί τρόπος να "ξεφύγει" από αυτό το σημείο μέσω κάποιας διαδικασίας. Τα σημεία αυτά στα οποία κάποια ισότοπα φαίνεται να "περιμένουν" για την εξέλιξη της πυρηνοσύνθεσης ονομάζονται σημεία αναμονής.} της rp-διεργασίας καθώς ο χρόνος για μια τέτοια αντίδραση είναι μικρότερος τόσο από τον μέσο χρόνο αρπαγής πρωτονίου/φωτοδιάσπασης όσο και από τον μέσο χρόνο β-διάσπασης στα σημεία αυτά.
Φυσικά, αυτός ο μηχανισμός απαιτεί μία πηγή ελεύθερων νετρονίων τα οποία δεν είναι συνήθως παρόντα σε περιβάλλον με τόσο πυκνό πλάσμα πρωτονίων.
%% --------------------------------------------------------------------------------------------------------%%
%% --------------------------------------------------------------------------------------------------------%%
%% --------------------------------------------------------------------------------------------------------%%
\subsubsection{Η $\nu$p-διεργασία}
Το 2006 προτάθηκε από την Fr\"ohlich και άλλους (C. Fr\"ohlich, et.al., \textit{Neutrino-induced nucleosynthesis of A $\geq$ 64 nuclei: The $\nu$p process}, Phys. Rev. Lett., 96:142502, Apr 2006) ένας καινούργιος μηχανισμός παραγωγής νετρονίων σε περιβάλλον πλούσιο σε πρωτόνια μέσω της αντίδρασης:

\begin{equation}
\overline{\nu}_e + p \longrightarrow e^{+} + n
\end{equation}

Αυτός ο μηχανισμός ομάστηκε \textit{$\nu$p-διεργασία} και θα μπορούσε να παρέχει τα απαραίτητα νετρόνια για την εξέλιξη της pn-διεργασίας σε ένα περιβάλλον όπου εκτός από την παρουσία πλάσματος με μεγάλη πυκνότητα πρωτονίων, θα υπάρχει επίσης και μεγάλη ροή (αντι)-νετρίνων καθώς τα τελευταία αλληλεπιδρούν εξαιρετικά ασθενώς με την ύλη. Τέτοιες συνθήκες πιστεύεται ότι επικρατούν σε υπερκαινοφανείς αστέρες και σε εκλάμψεις ακτίνων γαμμα (gamma-ray bursts).
%% --------------------------------------------------------------------------------------------------------%%
%% --------------------------------------------------------------------------------------------------------%%
%% --------------------------------------------------------------------------------------------------------%%
\subsection{Φωτοδιάσπαση}
Είδαμε πως μπορεί να δημιουργηθεί ένας πυρήνας πλούσιος σε πρωτόνια μέσω της συλλήψης πρωτονίου. Ένας δεύτερος τρόπος δημιουργίας τέτοιων πυρήνων είναι η \textbf{φωτοδιάσπαση} ή αλλιώς \textbf{γ-διεργασία}, κατά την οποία έχουμε μείωση του αριθμού των νετρονίων από έναν πυρήνα μέσω αντιδράσεων τύπου (γ,n). Γενικά, στις συνθήκες υψηλής ενέργειας που μελετάμε, το φαινόμενο της φωτοδιάσπασης εμφανίζεται πολύ συχνά και εμποδίζει κατά κάποιον τρόπο τους πυρήνες να φτάσουν στις drip lines. 
Φαίνεται ότι η φωτοδιάσπαση και η αρπαγή πρωτονίου είναι ανταγωνιστικοί μεταξύ τους μηχανισμοί δημιουργίας βαρέων πυρήνων και σήμερα είναι γενικά αποδεκτό ότι η φωτοδιάσπαση είναι η βασική διαδικασία σύνθεσης p-πυρήνων.
%% --------------------------------------------------------------------------------------------------------%%
%% --------------------------------------------------------------------------------------------------------%%
%% --------------------------------------------------------------------------------------------------------%%
\subsection{Η l-διεργασία}
Όπως έχει αναφερθεί, η ύπαρξη των ελαφρών στοιχείων όπως το λίθιο, το δευτέριο, το βόριο και το βηρύλλιο, δεν δικαιολογείται λόγω των συνθηκών που επικρατούν στο εσωτερικό των άστρων. Σε αυτές τις συνθήκες, τα στοιχεία αυτά θα έπρεπε να καταστρέφονται πολύ γρήγορα και άρα να μην παρατηρούνται σε αυτές τις κοσμικές αναλογίες, έστω κι αν αυτές είναι πολύ μικρές. Άρα, είτε πρέπει να δημιουργήθηκαν σε ψυχρό περιβάλλον, είτε να απομακρύνθηκαν από το αστρικό περιβάλλον αμέσως μετά την δημιουργία τους. 
Ένας μηχανισμός που μπορεί να εξηγήσει τη σύνθεση αυτών των στοιχείων στο μεσοαστρικό χώρο (άρα σε ψυχρή περιοχή) είναι η \textbf{ l-διεργασία}\footnote{Στη διεθνή βιβλιογραφία μπορεί να συναντήσει κάποιος και τον όρο x-διεργασία αντί του όρου l-διεργασία.} η οποία πρόκειται ουσιαστικά για μία αντίδραση θρυμματισμού (spallation). Κατά την διαδικασία αυτή, κοσμικές ακτίνες προσπίπτουν σε στοιχεία με σημαντική κοσμική αναλογία (C, N, O) και τα διασπούν με κυρίως παράγωγα τα Li, Be και Β.

Στο περιβάλλον στο οποίο λαμβάνουν χώρα οι διαδικασίες πυρηνοσύνθεσης, δεν μπορούμε να τοποθετήσουμε κάποιον ανιχνευτή και να πάρουμε πρωτογενή δεδομένα. Γι' αυτό το λόγο δημιουργείται η ανάγκη να μπορέσουμε με κάποιο τρόπο να τις εκφράσουμε μέσω των μαθηματικών και στη συνέχεια να μπορέσουμε με την βοήθεια αριθμητικών μεθόδων να τις προσομοιώσουμε, να τις αξιολογήσουμε και να τις συγκρίνουμε με τα παρατηρησιακά μας δεδομένα. Αν θέλουμε να προσομοιώσουμε όμως την δημιουργία όλων των στοιχείων, θα πρέπει να γνωρίζουμε αρχικά όλα εκείνα τα φυσικά χαρακτηριστικά που μπορούν να την περιγράψουν. Για παράδειγμα, θα πρέπει να γνωρίζουμε τα χαρακτηριστικά των πυρήνων που μπορεί να λαμβάνουν μέρος στην πυρηνοσύνθεση, τον χρόνο ημιζωής τους, το σπιν τους, τα ενεργειακά τους επίπεδα κ.α. Πρέπει να ξέρουμε επίσης και τα χαρακτηριστικά των αντιδράσεων μεταξύ πυρήνων, νουκλεονίων, λεπτονίων, φωτονίων ή νετρίνων. 

 \begin{figure}
     \centering
     \includegraphics[scale=0.6]{Figures/Screenshot_2.jpg} 
    \caption{Διάγραμμα Segre και συσχέτιση των χημικών στοιχείων με τους μηχανισμούς παραγωγής τους.}
    \label{fig:apx:segre_diagram}
\end{figure} 

Παρακάτω θα περιγράψουμε πως μπορούμε να κατασκευάσουμε τη  διαφορική εξίσωση, που στην συνέχεια κάποιος πρέπει να επιλύσει αριθμητικά με σκοπό να προσομοιώσει τις διαδικασίες της πυρηνοσύνθεσης.  
%------------------------------------------------------------------------------------------------------------
%------------------------------------------------------------------------------------------------------------
%------------------------------------------------------------------------------------------------------------
\section{Ρυθμός Αντιδράσεων}
Aς υποθέσουμε λοιπόν ότι έχουμε στο εργαστήριό μας ένα αέριο που αποτελείται από δυο σωματίδια i και j τα οποία έχουν αριθμητικές πυκνότητες $n_{i}$ (cm$^{-3}$) και $n_{j}$ (cm$^{-3}$) με \textit{n} να είναι ο αριθμός των σωματιδίων στον όγκο και την ενεργό διατομή μεταξύ αυτων των δυο $\sigma$ (cm$^{2}$). Αρχικά και για λόγους απλότητας, θεωρούμε το ένα σωματίδιο ακίνητο, ως στόχο (έστω το σωμάτιο j), και αυτόν τον στόχο θα τον "βομβαρδίσουμε" με το άλλο σωμάτιο, το οποίο θα κινείται με γνωστή ταχύτητα. 

\begin{figure}
    \centering
    \includegraphics[scale=0.6]{Figures/target.jpg}  
    \caption{Διάταξη βλήματος-στόχου.}
    \label{fig:apx:target_projectile}
\end{figure} 

Aυτό που θα μετρήσουμε πειραματικά είναι η ενεργός διατομή μεταξύ αυτών των σωματιδίων. Οι ποσότητες που μας ενδιαφέρουν ωστόσο από αυτή την διαδικασία είναι το πόσες αντιδράσεις γίνονται στη μονάδα του χρόνου και στη μονάδα του όγκου. Η πληροφορία αυτή όμως, σχετίχεται με την σχετική ροή των σωματιδίων i, τον αριθμό των σωματιδίων j και την μετρούμενη ενεργό διατομή μεταξύ τους. Συνδυάζοντας τα παραπάνω μπορούμε να γράψουμε τις αντιδράσεις που θα πραγματοποιηθούν από αυτή την σύγκρουση των δυο σωματιδίων με τη μορφή:

\begin{equation}
\label{eq51}
r = n_{i}\cdot u \cdot n_{j}\cdot \sigma (u) \ \text{cm}^{-3}\text{/s}
\end{equation}
Στη Φύση όμως είναι σχεδόν απίθανο να βρεθεί ένα σωματίδιο ακίνητο. Το πιθανότερο σενάριο είναι και τα δυο σωμάτια να βρίσκονται σε κίνηση τη στιγμή της μεταξύ τους σύγκρουσης. Έτσι, για να εκφράσουμε  και αυτή την κίνηση του δεύτερου σωματίου θα πρέπει να υπολογίσουμε την σχετική ταχύτητα μεταξύ τους. Πλέον λοιπόν, οι αντιδράσεις ανά μονάδα χρόνου και όγκου θα είναι:

\begin{equation}
\label{eq52}
r_{i,j}= \int \sigma\cdot \vert \vec{u_{i}}- \vec{u_{j}} \vert dn_{i}dn_{j}
\end{equation}
Γενικά όμως, τα σωματίδια δεν έχουν σταθερή ταχύτητα κατά τη διάρκεια της κίνησής τους αλλά παρουσιάζουν μια κατανομή ταχυτήτων, έτσι είναι αναγκαίο να συμπεριλάβουμε και αυτή την παράμετρο γράφοντας:

\begin{equation}
\label{eq53}
r_{i,j}= \int \sigma( \vert \vec{u_{i}}- \vec{u_{j}} \vert) \cdot \vert \vec{u_{i}}- \vec{u_{j}} \vert \cdot \phi(\vec{u_{i}}) \cdot \phi(\vec{u_{j}})d^{3}u_{i}d^{3}u_{j}
\end{equation}
όπου το $r_{i,j}$ εκφράζει τον ρυθμό των αντιδράσεων, το $\sigma( \vert \vec{u_{i}}- \vec{u_{j}} \vert)$ την ενεργό διατομή συναρτήσει της σχετικής ταχύτητας μεταξύ των δυο σωματιδίων, το $\vert \vec{u_{i}}- \vec{u_{j}} \vert $ την σχετική ταχύτητα μεταξύ των δυο σωματιδίων και τέλος τα $\phi(\vec{u_{i}}), \phi(\vec{u_{j}})$ την κατανομή των ταχυτήτων με τις οποίες κινούνται τα σωματίδια.
Το πρόβλημά μας όμως ακόμα δεν έχει παραμετροποιηθεί πλήρως. Οι πυρήνες που βρίσκονται μέσα στο αστροφυσικό πλάσμα ενός αστέρα για παράδειγμα, δεν έχουν όλοι την ίδια ενέργεια. Ακολουθούν μια κατανομή ενεργειών και πιο συγκεκριμένα μπορούμε να εκφράσουμε αυτή την πληθώρα ενεργειών μεσω της γνωστής μας κατανομής Maxwell-Boltzmann.

\begin{figure}[h]
    \centering
    \includegraphics[scale=0.7]{Figures/mb.jpg} 
    \caption{Κατανομή Maxwell-Boltzmann.}
    \label{fig:apx:mb}
\end{figure} 
Η κατανομή Maxwell-Boltzmann όπως βλέπουμε και στο σχήμα \ref{fig:apx:mb}, στον οριζόντιο άξονα εκφράζει την ενέργεια και στον κάθετο άξονα εκφράζει την πιθανότητα να συναντήσουμε αυτή την ενέργεια στο σύστημα που μελετάμε. Παρατηρούμε ότι το μέγιστο της κατανομής εμφανίζεται σε περιοχή χαμηλών ενεργειών ενώ μια μακριά και φθίνουσα ουρά αντιστοιχεί σε μεγαλύτερες τιμές ενέργειας. Για να μπορέσουμε να κάνουμε ένα βήμα παραπέρα πάνω στην προσομοίωση της πυρηνοσύνθεσης, θα πρέπει να κάνουμε κάποιες απαραίτητες απλοποιήσεις. Ορίζουμε λοιπόν ως κέντρο αυτού του συστήματος το κέντρο μάζας του σωματιδίου που μελετάμε λαμβάνοντας υπόψιν ότι η κατανομή των ταχυτήτων κανονικοποιείται στη μονάδα

\begin{equation}
\label{eq54}
\int \phi \left( \vec{u}  \right) d^{3}u = 1
\end{equation} 
και κανονικοποιούμε την κατανομή πιθανότητας ενεργειών συναρτήσει του όγκου. Έπειτα από τα παραπάνω ο ρυθμός αντίδρασης έχει πλέον την μορφή

\begin{equation}
\label{eq55}
r_{i,j}=n_{i}n_{j}\langle \sigma u \rangle_{i,j}
\end{equation}
όπου ο όρος $\langle \sigma u \rangle_{i,j}$ εκφράζει την μέση θερμοπυρηνική ενεργό διατομή. Αν θέλουμε τώρα να υπολογίσουμε την θερμοπυρηνική ενεργό διατομή για σωματίδια που ανήκουν σε όλο το ενεργειακό φάσμα που φαίνεται στην κατανομή Maxwell-Boltzmann, λαμβάνοντας υπόψιν τα προηγούμενα βήματα, καταλήγουμε στη σχέση:
\begin{equation}
\label{eq56}
\langle \sigma u \rangle (T) = \left(  \frac{8}{\mu \pi} \right)^{1/2} \frac{1}{kT^{3/2}} \int_{0}^{\infty} E \sigma(E) \exp (-E/kT)dE
\end{equation}
όπου $\mu$ το χημικό δυναμικό του συστήματος που μελετάμε, Τ η θερμοκρασία, $\sigma(E)$ η ενεργός διατομή των ταχυτήτων συναρτήσει της ενέργειας και k η σταθερά Maxwell-Boltzmann.
Η σχέση μας τώρα εξαρτάται μόνο από τη θερμοκρασία. Αυτό σημαίνει για εμάς ότι αν ξέρουμε την ενεργό διατομή $\sigma(E)$ για όλες τις ενέργειες, αν δηλαδή μπορέσουμε και την μετρήσουμε πειραματικά, τότε μπορούμε να βρούμε εύκολα και τον ρυθμό των θερμοπυρηνικών αντιδράσεων. Όμως, στο εργαστήριο μπορούμε να μετρήσουμε ένα μικρό εύρος ενεργειών και μόνο, ενώ στο εσωτερικό ενός αστέρα υπάρχει μια ολόκληρη κατανομή ενεργειών. Ένα χαρακτηριστικό παράδειγμα που εκφράζει αυτή την δυσκολία είναι το παράδειγμα της p-process. Αν θέλουμε να μελετήσουμε αυτή την διαδικασία πειραματικά, θα πρέπει να ξεπεράσουμε το δυναμικό Coulomb του πυρήνα-στόχου. Για τον σκοπό αυτό τον "χτυπάμε" με πρωτόνια συγκεκριμένης ενέργειας, ώστε να γίνει η αρπαγή του πρωτονίου, ενώ ο πυρήνας-στόχος είναι ακίνητος. Για να είναι τα αποτελέσματά μας ρεαλιστικά όμως, θα πρέπει να λάβουμε υπόψιν ότι σε ένα φυσικό αστρικό περιβάλλον, όπως αναφέραμε και παραπάνω, οι διαθέσιμοι πυρήνες καλύπτουν ένα μεγάλο φάσμα ταχυτήτων, όπως και τα ελεύθερα πρωτόνια. Συνεπώς και στο εργαστήριο θα πρέπει να δημιουργήσουμε μια παρόμοια κατανομή πρωτονίων και στη συνέχεια να μετρήσουμε την ενεργό διατομή μεταξύ αυτών και των πυρήνων, κάτι που είναι πρακτικά αδύναντο.

 Για να μπορέσουμε να προσομοιώσουμε όλη εκείνη την μπάντα των ενεργειών και στη συνέχεια τον ρυθμό των αντιδράσεων, θα πρέπει να ξαναγράψουμε τον όρο $\sigma(E)$ με κάποιον άλλο τρόπο, χρησιμοποιώντας όλες τις πληροφορίες που έχουμε γι' αυτόν. Ένα χαρακτηριστικό που δεν έχουμε χρησιμοποιήσει ακόμα είναι το φορτίο των σωματιδίων που μελετάμε. Μπορεί για τα φορτισμένα σωματίδια η ενεργός διατομή να έχει εξάρτηση από αρκετούς παράγοντες, ωστόσο όμως ξέρουμε με σιγουριά ότι το φορτίο Ζ εξαρτάται από:
 
\begin{itemize}
\item Το δυναμικό Coulomb με μια εξάρτηση της μορφής: $\sim \exp (-E^{1/2}) $
\item Το μέγεθος του πυρήνα με μια εξάρτηση της μορφής: $\sim 1/E$
\end{itemize}
Για να διευκολύνουμε τις πράξεις, θεωρούμε ότι η ενεργός διατομή εξαρτάται μόνο από αυτούς τους δυο παράγοντες. Όλες τις άλλες εξαρτήσεις τις αθροίζουμε και τις εκφράζουμε με έναν παράγοντα που ονομάζουμε \textit{αστροφυσικό παράγοντα, S} (astrophysical factor S).
%------------------------------------------------------------------------------------------------------------
%------------------------------------------------------------------------------------------------------------
%------------------------------------------------------------------------------------------------------------
\section{Aστροφυσικός Παράγοντας}
Για non-resonant αντιδράσεις, δηλαδή για αντιδράσεις που δεν εμφανίζουν μέγιστο σε κάποιο ενεργειακό επίπεδο, όλες οι υπόλοιπες επιδράσεις εκτός από τις δύο που αναφέραμε μεταβάλλονται με πολύ αργό ρυθμό συναρτήσει των ενεργειακών μεταβολών. Έτσι, σε τέτοιες περιπτώσεις ο αστροφυσικός παράγοντας S μπορεί να θεωρηθεί σχεδόν ως σταθερός. Εύκολα λοιπόν μπορούμε να εκφράσουμε την συνεισφορά αυτών των παραγόντων. Με αυτόν τον τρόπο, η ενεργός διατομή παίρνει τη μορφή:

\begin{equation}
\label{eq57}
\sigma = E^{-1}\cdot \exp (-E^{1/2})\cdot S(E)
\end{equation} 
και ο ρυθμός αντιδράσεων γίνεται:
\begin{align}
\label{eq58_59}
\nonumber\langle \sigma u \rangle &= \left(  \frac{8}{\mu \pi} \right)^{1/2} \frac{1}{kT^{3/2}} \int_{0}^{\infty} E \sigma(E) \exp (-E/kT)dE \Longleftrightarrow \\ \nonumber \\
\langle \sigma u \rangle &=\left(  \frac{8}{\mu \pi} \right)^{1/2} \frac{1}{kT^{3/2}} \int_{0}^{\infty} S(E) \exp (-bE^{-1/2})\exp(-E/kT)dE
\end{align}
Στην τελευταία σχέση, όπως αναφέραμε, ο αστροφυσικός παράγοντας S μεταβάλλεται πολύ αργά με την ενέργεια, έτσι οι κυρίαρχοι όροι μέσα στο ολοκλήρωμα είναι οι δύο εκθετικοί οροι. Αν κάνουμε γραφική παράσταση της παραπάνω σχέσης θα πάρουμε το διάγραμμα που φαίνεται στο σχήμα \ref{fig:apx:gamow}.
\begin{figure}[h!]
    \centering
    \includegraphics[scale=0.7]{Figures/gamow.jpg} 
     \caption{Κορυφή Gamow.}
     \label{fig:apx:gamow}
\end{figure}
Aν συνδυάσουμε αυτές τις δύο καμπύλες, η ενεργός διατομή γίνεται μέγιστη για τιμές ενέργειας που βρίσκονται μέσα στην μπλε περιοχή. Η καμπύλη που οριοθετεί την μπλε περιοχή ονομάζεται \textit{κορυφή Gamow} και το μέγιστο εύρος αυτής της περιοχής ονομάζεται \textit{παράθυρο ενεργειών Gamow}. Ο ρυθμός των αντιδράσεων φαίνεται, μέσα από αυτό το διάγραμμα, να είναι πιο δραστικός (effective) όταν τα σωματίδια παίρνουν τιμές ενέργειας που βρίσκονται μέσα στην μπλε περιοχή. Με αυτό τον τρόπο, μπορούμε εύκολα να διακρίνουμε ποιές τιμές ενεργειών είναι οι πιο δραστικές. Παρατηρούμε ότι τυπικά αυτές οι τιμές είναι σχετικά χαμηλές.
%------------------------------------------------------------------------------------------------------------
%------------------------------------------------------------------------------------------------------------
%------------------------------------------------------------------------------------------------------------
\section{Δίκτυο πυρηνικών αντιδράσεων}
 Ο αρχικός μας σκοπός ήταν να κατασκευάσουμε μια διαφορική εξίσωση, η αριθμητική επίλυση της οποίας θα μας προσομοιώνει την διαδικασία της πυρηνοσύνθεσης. Έτσι το επόμενο βήμα είναι να εκφράσουμε τα παραπάνω ως μια διαφορική εξίσωση της μεταβολής των αριθμητικών πυκνοτήτων συναρτήσει της χρονικής τους μεταβολής. Η ποσότητα που θέλουμε να δούμε πως μεταβάλλεται και είναι και η πιο εύκολα μετρήσιμη παρατηρησιακά, είναι η μεταβολή της αφθονίας (abundance) ενός στοιχείου, πυρήνα, σωματίου ή ισοτόπου με το χρόνο. Για να φτάσουμε όμως σε αυτό το σημείο πρέπει να κάνουμε κάποιες παραδοχές. Εκφράζουμε λοιπόν τον αριθμό των αντιδράσεων ανά μονάδα χρόνου και όγκου μέσω μιας διαφορικής εξίσωσης και βλέπουμε το πως μεταβάλλεται με τον χρόνο η αριθμητική πυκνότητα του i-σωματίου για μια αντίδραση $i(j,o)m$, δηλαδή για μια αντίδραση των $i+j$ όπου θα μας δώσει $ o+m $ σωμάτια. Θεωρώντας ότι τα $i,j$ μεταβάλλονται με τον ίδιο ρυθμό, έχουμε:
 
\begin{equation}
\label{eq60}
r_{i,j}=\frac{1}{1+\delta_{ij}}n_{i}n_{j}\langle\sigma u \rangle \longrightarrow
\end{equation} 
\begin{align}
\label{eq61_62}
\left( \frac{\partial n_{i}}{\partial t} \right)_{\rho} &= \left( \frac{\partial n_{j}}{\partial t} \right)_{\rho}= -r_{i,j} \\ \nonumber \\
\left( \frac{\partial n_{o}}{\partial t} \right)_{\rho} &= \left( \frac{\partial n_{m}}{\partial t} \right)_{\rho}= +r_{i,j}
\end{align}
Δηλάδή για κάθε αντίδραση μεταξύ των i και j, έχουμε καταστροφή των i, j και δημιουργία των ο, m. Ο όρος $\displaystyle \frac{1}{1+\delta_{ij}}$ είναι απαραίτητος στην περίπτωση που μελετάμε αντιδράσεις μεταξύ δύο ίδιων σωματιδίων και μας βοηθά να μην μετρήσουμε αυτή την αντίδραση δύο φορές. Η ποσότητα $\delta_{ij}$ δεν είναι άλλη από το δέλτα του Κronecker. Ένας πολύ σημαντικός παράγοντας όμως σε αυτόν τον συλλογισμό που πρέπει να ληφθεί σοβαρά υπόψιν είναι ότι η αριθμητική πυκνότητα ενός σωματίου εξαρτάται άμεσα από την πυκνότητα της ύλης. Έτσι, η πλήρης έκφραση της αριθμητικής πυκνότητας μιας ποσότητας είναι αυτή που δίνεται από την σχέση
\begin{equation}
\label{eq63}
\dot{n_{i}}= \left( \frac{\partial n_{i}}{\partial t} \right)_{\rho} + n_{i}\frac{\dot{\rho}}{\rho}
\end{equation}

Βλέπουμε ότι η $\dot{n_{i}}$ εκφράζεται από δύο όρους. Έναν όρο στον οποίο η παράμετρος της πυκνότητας ύλης είναι σταθερή και έναν στον οποίο εκφράζεται η μεταβολή της. Στη μελέτη μας όμως των διαδικασιών της πυρηνοσύνθεσης, η μεταβολή της αριθμητικής πυκνότητας ως προς την πυκνότητα ύλης δεν έχει μεγάλη συμβολή. Η μεταβολή που επηρρεάζει σημαντικά τη μελέτη μας είναι εκείνη η οποία μας δίνει την αλλαγής της αριθμητικής πυκνότητας, του αριθμού δηλαδή των σωματιδίων i, j κατά τη διάρκεια των αντιδράσεων. Συνεπώς, προς το παρόν θα αγνοήσουμε την ποσότητα $n_{i}\frac{\dot{\rho}}{\rho}$ και να επικεντρώσουμε την μελέτη μας στην μεταβολή των αντιδράσεων, δηλαδή του όρου $\left( \frac{\partial n_{i}}{\partial t} \right)_{\rho} $.
Για τον λόγο αυτό χρησιμοποιούμε την ποσότητα της αφθονίας ($Y=n/\rho N_{A}$) ή του λόγου μάζας ($X_{i}=A_{i}Y_{i}$). Σχετικά με τον λόγο μάζας, γνωρίζουμε ότι το άθροισμα όλων των μαζών κανονικοποιείται στην μονάδα. Έτσι, ελέγχοντας τον λόγο μάζας κατά τη διάρκεια προσομοίωσης των αντιδράσεων, βλέπουμε τις μεταβολές που έχουμε στην μάζα. Έτσι λοιπόν, όταν το άθροισμα  αυτό γίνεται μικρότερο από μονάδα ξέρουμε ότι έχουμε απώλεια μάζας και μετατροπή της κατά πάσα πιθανότητα σε ενέργεια.
Τώρα πλέον, μπορούμε να συμπεριλάβουμε στους υπολογισμούς μας και την μικρή συνεισφορά που προέρχεται από την μεταβολή της αφθονίας του σωματίου που μας ενδιαφέρει συναρτήσει της πυκνότητας ύλης. Χρησιμοποιώντας τον ορισμό της αφθονίας έχουμε
\begin{equation}
\label{eq64}
\dfrac{{dY}_{i}}{dt}=\frac{d }{dt} \frac{n_{i}}{\rho N_{A}}\Rightarrow \dot{Y}_{i}= \frac{\dot{n_{i}}}{\rho N_{A}}-\frac{n_{i}}{\rho N_{A}} \frac{\dot{\rho}}{\rho}
\end{equation}

Μπορούμε τώρα να αντικαταστήσουμε το $\dot{n_{i}}$ από το πρώτο μέλος της σχέσης \eqref{eq63} και αν θέλουμε να το εκφράσουμε με όρους σχετικούς με την θερμοπυρηνική ενεργό διατομή, θα αντικαταστήσουμε τις αριθμητικές πυκνότητες των i, j με τις αφθονίες τους. Έτσι, σε συνδυασμό με τη σχέση \eqref{eq61_62} προκύπτει ότι
\begin{equation}
\label{eq65}
\dot{Y}_{i}= \frac{1}{\rho N_{A}} \left( \frac{\partial n_{i}}{\partial t} \right)_{\rho} = - \frac{r_{i,j}}{\rho N_{A}}=-\frac{1}{1+ \delta_{ij}}\rho N_{A} \langle\sigma u \rangle _{i,j} Y_{i}Y_{j}
\end{equation}

Μέχρι τώρα στη μελέτη μας θεωρήσαμε αντιδράσεις που γίνονται μεταξύ νουκλεονίων. Στη Φύση όμως, έχουμε και αντιδράσεις που γίνονται από αλληλεπίδρασεις νουκλεονίων ή πυρήνων με φωτόνια, λεπτόνια ή νετρίνα. Επιπλέον, μέχρι τώρα δεν λάβαμε υπόψιν μας τις διασπάσεις των πυρήνων. Για να λάβουμε και αυτές τις αλληλεπιδράσεις και διεργασίες υπόψιν, ορίζουμε μια ειδική παράμετρο, την παράμετρο λ. Η παράμετρος αυτή θα πάρει τη θέση του όρου $r_{i,j}$ και η χρονική παράγωγος της αφθονίας θα έχει τη μορφή
\begin{equation}
\label{eq66}
\dot{Y_{i}}=-\lambda _{i} Y_{i}
\end{equation}
Έτσι λοιπόν, μέσω αυτής της παραμέτρου μπορούμε να εκφράσουμε τον ρυθμό των διασπάσεων (decay rates) ή αν έχουμε αντιδράσεις με λεπτόνια, φωτόνια ή νετρίνα με μια αντίστοιχη παράμετρο λ. Για παράδειγμα, αν έχουμε μια φωτοδιάσπαση ατόμων οξυγόνου $^{16}$O από φωτόνια, ο ρυθμός με τον οποίο θα γίνονται αυτές οι διασπάσεις θα εκφραστούν μέσω της λ. Αξίζει να αναφέρουμε ότι υπολογιστικά, για έναν κώδικα οι αντιδράσεις που γίνονται από την αλληλεπίδραση νουκλεονίων με φωτόνια, λεπτόνια ή νετρίνα αντιμετωπίζονται με τον ίδιο τρόπο και επηρρεάζουν την αφθονία όπως φαίνεται στη σχέση \eqref{eq66}. Σύμφωνα με τα παραπάνω, γίνεται σαφές ότι οι αλληλεπιδράσεις που μπορούν να λάβουν χώρα χωρίζονται σε  δύο κατηγορίες

\begin{itemize}
\item Mεταξύ δύο πυρήνων ή νουκλεονίων.
\item Μεταξύ πυρήνων/νουκλεονίων και φωτονίων, λεπτονίων, νετρίνων - διασπάσεων.
\end{itemize}

Φτάνοντας πλέον πολύ κοντά στην τελική γραφή της ζητούμενης διαφορικής εξίσωσης που θέλουμε να λύσουμε έχουμε να κάνουμε λίγες σκέψεις ακόμα.
%------------------------------------------------------------------------------------------------------------
%------------------------------------------------------------------------------------------------------------
%------------------------------------------------------------------------------------------------------------
\section{Αντίστροφες Αντιδράσεις}
Αντίστροφες (inverse) είναι οι χημικές αντιδράσεις κατά τις οποίες τα αντιδρώντα μετασχηματίζονται  σε προϊόντα σε έναν χρόνο t αλλά αν αντιστρέψουμε αυτόν το χρόνο και πάμε να δούμε τι αντίδραση γίνεται σε -t, η αλληλεπίδραση των προϊόντων θα μας δώσει σαν αποτέλεσμα τα αρχικώς αντιδρώντα. Για παράδειγμα, αν μέσα στον αστέρα αντιδράσει ένας  $^{12}$C με ένα σωμάτιο α ($^{4}$He) θα δώσει ένα οξυγόνο ($^{16}$O) και ένα φωτόνιο-γ αλλά και το αντίστροφο. Για να μελετήσουμε αυτές τις αντιδράσεις, θα πρέπει να ξαναγράψουμε την ενεργό διατομή για μια αντίστροφη αντίδραση η οποία θα έχει τη μορφή:

\begin{equation}
\label{eq67}
\langle \sigma u \rangle _{i,j,o} = \frac{1+\delta_{ij}}{1+\delta_{om}} \frac{G_{m}g_{o}}{G_{i}g_{j}} \left(  \frac{\mu_{om}}{\mu_{ij}} \right) ^{3/2} \exp \left( -Q_{o,j}/kT \right) \langle \sigma u \rangle_{m,o,j}
\end{equation}

Στην τελευταία σχέση βλέπουμε τον τύπο της θερμοπυρηνικής ενεργού διατομής για την κατεύθυνση $\langle \sigma u \rangle _{i,j,o}$ και στο τέλος, την θερμοπυρηνική ενεργό διατομή για την αντίστροφη κατευθυνση $\sigma u \rangle_{m,o,j}$. Όπως παρατηρούμε, αυτές οι δύο διατομές σχετίζονται. Ο εκθετικός όρος $Q$ εκφράζει την ενεργειακή διαφορά μεταξύ αντιδρώντων και προϊόντων (άρα και το είδος της αντίδρασης, αν είναι δηλαδή ενδόθερμη ή εξώθερμη) και ο πρώτος όρος εκφράζει την πυκνότητα καταστάσεων και τη συνάρτηση επιμερισμού του συστήματος που μελετάμε. Αυτό σημαίνει για τους υπολογισμούς μας ότι αν μετρήσουμε την ενεργό διατομή για μια κατεύθυνση --πειραματικά ή και θεωρητικά μέσω κάποιου μοντέλου-- και αν γνωρίζουμε επίσης την συνάρτηση επιμερισμού του συστήματος και την τιμή της $Q$, μπορούμε εύκολα να υπολογίσουμε την ενεργό διατομή της αντίστροφης κατεύθυνσης μέσω της τελευταίας σχέσης.

\section{Θεμελιώδης Διαφορική Εξίσωση}
Συνδυάζοντας όλα τα παραπάνω, είμαστε πλέον σε θέση να γράψουμε με σαφήνεια και λαμβάνοντας υπόψιν όλες τις σημαντικές παραμέτρους, την θεμελιώδη διαφορική εξίσωση που μπορεί στη συνέχεια να μας βοηθήσει να προσομοιώσουμε τις διαδικασίες της πυρηνοσύνθεσης, η μορφή της οποίας είναι

\begin{align}
\label{eq68}
\nonumber \dot{Y} &= \sum_{j} N^{i}_{j} \lambda_{j} Y_{j}+ \sum_{j,k} N^{i}_{jk}\rho N_{A} \langle\sigma u \rangle _{jk}Y_{j}Y_{k} + \\ \nonumber \\
&+ \sum_{j,k,l} N^{i}_{jkl}\rho ^{2} N_{A} ^{2} \langle\sigma u \rangle _{jkl}Y_{j}Y_{k}Y_{l}
\end{align}
Έτσι, για κάθε πυρηνικό είδος που θέλουμε να μελετήσουμε, έχουμε μια διαφορική εξίσωση που μπορεί να μας βοηθήσει η οποία μοιάζει με την σχέση \eqref{eq68}. Καθώς λοπόν μεταβάλλεται με το χρόνο η αφθονία ενός είδους, βλέπουμε ότι αυτή η μεταβολή εξαρτάται απ' όλες τις αλληλεπιδράσεις που εμπλέκονται και συμβάλλουν στην δημιουργία ή την καταστροφή του i-στοιχείου. Συνήθως, ομαδοποιούμε τους όρους ανά είδος αντιδράσεων. Έτσι, στην εξίσωση \eqref{eq68} ο πρώτος όρος εκφράζει τις αλληλεπιδράσεις του i-στοιχείου με φωτόνια, λεπτόνια, νετρίνα ή πιθανές διασπάσεις, ο δεύτερος εκφράζει τις αντιδράσεις μεταξύ των i-στοιχείων. Οι όροι $N$, είναι ουσιαστικά μετρητές σωματιδίων και μας πληροφορούν για το πόσα σωματίδια δημιουργούνται ή καταστρέφονται από την κάθε αλληλεπίδραση. Ο τρίτος όρος τώρα, εκφράζει την περίπτωση εκείνη όπου στην αντίδραση συμμετέχουν περισσότερα από δύο σωμάτια, όπως είναι η περίπτωση της αντίδρασης τρία-α. Σε μια τέτοια αντίδραση πρέπει να μετρήσουμε την θερμοπυρηνική ενεργό διατομή και του τρίτου σωματίου καθώς και την αφθονία του. Εξαιτίας τώρα των 3 αφθονιών έχουμε τους όρους της πυκνότητας και του αριθμού Avogadro υψωμένους στο τετράγωνο. 
Απ' αυτό το σημείο και μετά, είμαστε σε θέση, λαμβάνοντας υπόψιν μερικές παραμέτρους ακόμα σχετικά με τα αστροφυσικά φαινόμενα στα οποία έχουμε πυρηνοσύνθεση καθώς και τα επιμέρους χαρακτηριστικά της κάθε διαδικασίας, να μελετήσουμε υπολογιστικά την πιθανή εξέλιξη και προέλευση των διαφόρων βαρέων --και όχι μόνο-- χημικών στοιχείων.


    
%     %\printbibliography[heading=bibintoc]

% \end{document}

\documentclass[a4paper, 10pt, twoside, openright]{book}



%\usepackage[grid]{eso-pic} % Uncomment this to display a grid over the whole page
\usepackage[pscoord]{eso-pic}
\usepackage{titlesec} % Format title, chapter, sections etc
                      % check out: https://ctan.org/pkg/titlesec?lang=en
\usepackage{minitoc}
\usepackage{lscape}
\usepackage{afterpage}
\usepackage{mathtools}
\usepackage{csquotes}
\usepackage{dirtytalk}
\usepackage{amsmath, esint}
\usepackage{amssymb}
\usepackage{cancel}
\usepackage{subcaption}
\usepackage{xcolor}
\usepackage[unicode]{hyperref}
\hypersetup{colorlinks,linkcolor={black},citecolor={black},urlcolor={blue}}
\usepackage[LGR, T1]{fontenc}
\usepackage[utf8]{inputenc}



% Set the depth and numbering
% for table of contents (toc)
% \setcounter{tocdepth}{5}
% \setcounter{secnumdepth}{2}


%%% FORMAT THE STYLE OF CHAPTERS, SECTIONS ETC %%%
% Add a horizontal line in the header
\newpagestyle{main}{%
  \sethead[\thepage][][\chaptertitle]{\thesection\ \sectiontitle}{}{\thepage}
  \headrule
}
\pagestyle{main}

% Display format of the chapter
\titleformat{\chapter}[display]
{\large}
{\filleft\MakeUppercase{\chaptertitlename} \Huge\thechapter}
{2ex}
{\LARGE\bfseries\filleft}
[\vspace{2ex}%
\titlerule]

% Display format of the section
% \titleformat{\section}[block]
% {\filcenter\large
% \addtolength{\titlewidth}{2pc}%
% \titleline*[c]{\titlerule*[.6pc]{\tiny\textbullet}}%
% \addvspace{6pt}%
% \normalfont\bfseries\sffamily}
% {\thesection}{1em}{}
% \titlespacing{\section}
% {5pc}{*2}{*2}[5pc]

% Display format of the subsection
\titleformat{\subsection}[block]
{\normalfont\sffamily}
{\thesubsection}{.5em}{\titlerule\\[.8ex]\bfseries}

% Display format of the subsubsection
\titleformat{\subsubsection}[wrap]
{\normalfont\fontseries{b}\selectfont\filright}
{\thesubsubsection.}{.5em}{}
\titlespacing{\subsubsection}
{12pc}{1.5ex plus .1ex minus .2ex}{1pc}

%%% END OF FORMATING %%%


\usepackage[              % Left and right margins
    DIV=14,               % You can play with these numbers for a better
    BCOR=1cm,             % reference look into KOMA-Script CTAN manual
    headinclude=true,
    footinclude=false,
    paper=A4
    ]{typearea}
    

% commands below work only for twoside option of \documentclass
\makeatletter
\if@twoside
    \newlength{\textblockoffset}
    \setlength{\textblockoffset}{10mm}
    \addtolength{\hoffset}{\textblockoffset}
    \addtolength{\evensidemargin}{-2.0\textblockoffset}
\fi
\makeatother

% PACKAGES
\usepackage{subfiles}
\usepackage{natbib}                    
\bibpunct{(}{)}{;}{a}{}{,}
\usepackage{fancyhdr, graphicx}
\usepackage{upgreek}                   % More greek letters
\usepackage{enumerate}
\usepackage{commath}
% \usepackage[dvipsnames]{xcolor}
%\usepackage{blindtext}

% to know the page width and adjust the image size!
\usepackage{printlen} % \uselengthunit{in}\printlength{\textwidth}

% TABLE PACKAGES
\usepackage{etoolbox}
\BeforeBeginEnvironment{tabular}{\small}
\usepackage{booktabs}
\usepackage{multirow}

% IMAGES PACAKGES
\usepackage{graphicx}
\usepackage[font=small]{caption}
\newtheorem{theorem}{Theorem}
\usepackage{subcaption}

% TIKZ & PGFPLOTS
\usepackage{tikz}
\usepackage{tikz-3dplot}
\usepackage{pgfplots}
%\pgfplotsset{compat=1.14}
\usetikzlibrary{
    calc,
    patterns,
    circuits.logic.US,
    circuits.ee.IEC,
    intersections,
    angles,
    quotes,
    arrows,
    shapes,
    }


   
% WRITE GREEK TEXT
\usepackage[greek,english]{babel}
\usepackage{alphabeta}
%\selectlanguage{greek}


% Inline enumerated lists
\usepackage[inline]{enumitem}
\usepackage{pdfpages}
%\includepdfset{offset=0.7cm -0cm}
%\usepackage[fulladjust]{marginnote}

% PDF CONFIGURATION
% \definecolor{xlinkcolor}{cmyk}{1,1,0,0}
% \usepackage{hyperref}
% \hypersetup{
% 	breaklinks,
%     pdffitwindow=true,
%     pdfstartview={FitH},
%     pdftitle={PhD Thesis},                    % TITLE
%     pdfauthor={Savvas Chanlaridis},          % AUTHOR
%     pdfsubject={\title},                      % DOCUMENT SUBJECT
%     pdfcreator={Savvas Chanlaridis},            % CREATOR
%     pdfproducer={\thesisauthor},              % PRODUCER
%     pdfkeywords={Astronomy, Astrophysics},
%     pdfnewwindow=true,
%     colorlinks=true,                          % COMMENT FOR BLACK LINKS
%     linkcolor=xlinkcolor,
%     citecolor=xlinkcolor,
%     urlcolor=blue
%     }

% Paragraph Packages
\usepackage{indentfirst}            % Indent at the start of every paragraph
\setlength{\parindent}{0em}         % Size of indent
\setlength{\parskip}{1em}           % Space between paragraphs

\newcommand{\dbar}{{d\mkern-7mu\mathchar'26\mkern-2mu}} % Define inexact differential


%%%%%%%%%%%%%%%%%%%%%%%%%%%%%%%%%%%%%%%%%%%%%%%%%%%%%%%%%%%%%%%%%%%%%%%%%%%%%%%%%%%%%%%%%%%%%%%%%%%%%%%%%%%%

% % Document Details
\title{Εισαγωγή στην Αστροφυσική}
\author{Χανλαρίδης Σάββας}

\begin{document}
%     % Define names of report parts
    \renewcommand{\contentsname}{Περιεχόμενα}
    \renewcommand{\listfigurename}{Λίστα Σχημάτων}
    \renewcommand{\listtablename}{Λίστα Πινάκων}
    \renewcommand{\chaptername}{Κεφάλαιο}
    \renewcommand{\appendixname}{Παράρτημα}
    \renewcommand{\bibname}{Βιβλιογραφία}
    
    \frontmatter  % front and main matter to make roman numbers in between.
    


    % FIRST PAGE TITLE
    \pdfbookmark[0]{Cover}{title}
    
    \begin{titlepage}
        \begingroup
        \thispagestyle{empty}
        \AddToShipoutPicture*{\put(0,0){\includegraphics[scale=1.25]{Figures/esahubble}}} % Image background
        \centering
        \vspace*{5cm}
        \par\normalfont\fontsize{35}{35}\sffamily\selectfont
        \textbf{Εισαγωγή στην Αστροφυσική}\\
        {\LARGE Σάββας Χανλαρίδης}\par % Book title
        \vspace*{1cm}
        {\Huge Σημειώσεις Μαθήματος}\par % Author name
        \endgroup
    \end{titlepage}

    \newpage  % Inserting empty page
    \mbox{}
    \thispagestyle{empty}
    
    \newpage
    \thispagestyle{empty}
    \begin{center}
        \textbf{\LARGE Σημείωση}
    \end{center}
    \noindent Το εγχειρίδιο αυτό αποτελεί μία συλλογή από εδάφια παρμένα (άλλες φορές αυτούσια και άλλες φορές σε ελεύθερη μετάφραση) από την ελληνική και διεθνή βιβλιογραφία, διαδικτυακές πηγές καθώς και προσωπικές σημειώσεις από πανεπιστημιακές διαλέξεις.
    
    Το εγχείρημα επιχειρήθηκε, αρχικά, έχοντας ως στόχο τη δημιουργία ενός αρχείου που θα λειτουργούσε ως σημειώσεις για ιδιωτική χρήση και μελέτη. Δεν υπήρχε καμία πρόθεση αυτές οι σημειώσεις να δουν το φως της δημοσιότητας --- υπάρχουν πολύ καλύτερα εγχειρίδια και βιβλία που εξυπηρετούν αυτόν τον σκοπό --- και γι' αυτό το λόγο δεν δώθηκε καμία έμφαση στην παράθεση πηγών.
    
    Σε μια προσπάθεια εξευμενισμού, παρατίθεται στο τέλος μία λίστα με βιβλιογραφικές αναφορές που χρησιμοποιήθηκαν κατά τη διάρκεια συγγραφής αυτών των σημειώσεων.

    \begin{flushright}
        Σ. Χ.
    \end{flushright}
    \newpage
    \thispagestyle{empty}
    
    
    


    % TABLE OF CONTENT
{   \cleardoublepage
    \phantomsection
    
    %\addcontentsline{toc}{chapter}{Contents}
    
    % CHANGE COLOR LINK FOR TOC ONLY

  \hypersetup{linkcolor=black, pdfborder=0 0 1}
  \dominitoc
  \tableofcontents
%  \adjustmtc % necessary for displaying the minitoc after using \addcontentsline
}


	


    \mainmatter  % front and main matter to make roman numbers in between

    % BODY
    \subfile{Chapters/Chapter1}
    \subfile{Chapters/Chapter2}
    \subfile{Chapters/Chapter3}
    \subfile{Chapters/Chapter4}
    \subfile{Chapters/Chapter5}
    \subfile{Chapters/Chapter6}


    % APPENDICES
    \appendix
    \subfile{Appendixes/AppendixA}
    \subfile{Appendixes/AppendixB}
    \subfile{Appendixes/AppendixC}
    \subfile{Appendixes/AppendixD}

    % REFERENCES
    % If you are using a windows machine and the bibliography doesn't work
    % remove the extension for .bst and .bib files here!

    % \bibliographystyle{bibliography/bibstyle_apj.bst}
    % \bibliography{bibliography/bibfile_thesis}
    % \addcontentsline{toc}{chapter}{Bibliography}



\newpage
\thispagestyle{empty}
\begin{flushright}
    \textbf{\LARGE Βιβλιογραφία} 
\end{flushright}

\hrule
\textbf{Διεθνής βιβλιογραφία}
\begin{itemize}
    
    \item Arcones A., Thielemann F. K., Neutrino-Driven Wind Simulations and Nucleosynthesis of Heavy Elements, arXiv:1207.2527v1 [astro-ph.SR], 11 July 2012.
    
    \item Arnett D., Supernovae and Nucleosynthesis, Princeton University Press, New Jersey, 1996.
    
    \item Bethe H. and Critchfield C., The Formation of Deuteron by Proton Combination, Phys. Rev., 54:248–254, August 1938.
    
    \item BlattM.J., and Weisskopf F.V.,Theoretical Nuclear Physics, Springer-Verlag, NewYork, 1979.
    
    \item Branch, D., \& Wheeler, J. C. 2017, Supernova Explosions (Springer).
    
    \item Burbidge B. M., et.al., Synthesis of the Elements in Stars, Rev. Mod. Phys., 29:547-650, October 1957.
    
    \item Cameron A., Stellar Evolution, Nuclear Astrophysics and Nucleogenesis, Dover Publications, 2013.
    
    \item Cameron A., Some Nucleosynthesis Effects Associated with r-Process Jets, The Astrophysical Journal, 587:327–340, 10 April 2003.
    
    \item Clayton D., Principles of Stellar Evolution and Nucleosynthesis, McGraw-Hill Book Company, New York, 1968.
    
    \item Clayton D., and RassbachM.E., Termination of the s-Process, The Astrophysical Journal, Vol. 148, April 1967.
    
    \item Cox, J. P., \& Giuli, R. T. 1968, Principles of stellar structure.
    
    \item Cruz M.A., Nucleosynthesis in Extremely Metal-Poor and Zero Metallicity Stars, Academic Dissertation, Ludwig-Maximilian University of Munich, 2012, Munich.
    
    \item Eggleton, P. 2006, Evolutionary Processes in Binary and Multiple Stars (Cambridge University Press).
    
    \item Fewell M. P., The Atomic Nuclide with the Highest Mean Binding Energy, American Journal of Physics, 63:653-658, July 1995.

    \item Freiburghaus C., Rosswog S., and Thielemann F.,r-Process in Neutron Star Mergers, ApJ, Vol. 525, pp. L121-L124.
    
    \item Fr\"ohlich C., et.al., Neutrino-Induced Nucleosynthesis of $A \geq 64$ Nuclei: The $\nu$p-Process, Phys. Rev. Lett., 96:142502, Apr 2006.
    
    \item Fujimoto S., et.al., Heavy-Element Nucleosynthesis in a Collapsar, The Astrophysical Journal, 656:382-392, 10 February 2007.
    
    \item Gribbin J., Stardust, Yale University Press, 11 August 2001.
    
    \item Jaikumar P., et.al., Nucleosynthesis in neutron-rich ejecta from quark-novae, http://arxiv.org/abs/nucl-th/0610013, 04 October 2006.
    
    \item K\"appeler F., The Origin of the Heavy Elements: The s-Process, Progress in Particle and Nuclear Physics, Volume 43, 1999, pp. 419–483.
    
    \newpage
    \thispagestyle{empty}
    
    \item Kippenhahn R., Weigert A., and Weiss, A., \textit{Stellar Structure and Evolution}, 2nd. edition, Springer-Verlag Berlin Heidelberg, 2012.
    
    \item Korobkin O., et.al., On the Astrophysical Robustness of Neutron Star Merger r-Process, arXiv:1206.2379v2 [astro-ph.SR], 4 August 2012.
    
    \item Lamers, H., \& Cassinelli, J. 1999, Introduction to Stellar Winds (Cambridge University Press).
    
    \item Langer, N. 2012, Annual Review of Astronomy and Astrophysics, 50, 107.
    
    \item Lodders K., Solar System Abundances of the Elements, Springer-Verlag Berlin Heidelberg, pp. 379-417 (ISBN 978-3-642-10351-3), 2010.
    
    \item Nishimura S., et.al., r-Process Nucleosynthesis in Magnetohydrodynamic Jet Explosions of Core - Collapse Supernovae, The Astrophysical Journal, 642:410-419, 01 May 2006.
    
    \item Podsiadlowski, P. 2014, The evolution of binary systems, ed. I. Gonzlez Martnez-Pas, T. Shahbaz, \& J. Casares Velzquez, Canary Islands Winter School of Astrophysics (Cambridge University Press), 4588.
    
    \item Prialnik, D. 2000, An Introduction to the Theory of Stellar Structure and Evolution (Cambridge Uni- versity Press)
    
    \item Surman R., et.al. Heavy Element Synthesis in Neutrino-Processed Black Hole Accretion Disk Ejecta, Proceedings of Science, XIII Nuclei in the Cosmos, 7-11 July 2014, Debrecen, Hungary.
    
    \item Tauris, T. M., \& van den Heuvel, E. P. J. 2006, in Compact stellar X-ray sources, ed. W. H. G. Lewin \& M. van der Klis, 623.
    
    \item Wanajo S., et.al., The r-Process in Neutrino-Driven Winds from Nascent, ”Compact” Neutron Stars of Core-Collapse Supernovae, The Astrophysical Journal, 554 : 578È586, 10 June 2001.
    
    \item Woosley S., Macfadyen A., and Heger A., Collapsars, Gamma-Ray Bursts,and Supernovae, arXiv:astro-ph/9909034, 199.
    
    \item Witti J., et.al., Nucleosynthesis in Neutrino-Driven Winds from Protoneutron Stars, I.The α-process, Astronomy and Astrophysics 286, 841-856, 1994.

\end{itemize}

\textbf{Ελληνική βιβλιογραφία}

\begin{itemize}

    \item Βάρβογλης Χ., Σειραδάκης Γ., Εισαγωγή στη Σύγχρονη Αστρονομία, Εκδόσεις Γαρταγάνη, 1994

    \item Δανέζης Μ., Θεοδοσίου Σ., Το Σύμπαν που Αγάπησα - Εισαγωγή στην Αστροφυσική, Τόμος Α, Εκδόσεις Δίαυλος, 1999.
    
    \item Δανέζης Μ., Θεοδοσίου Σ., Το Σύμπαν που Αγάπησα - Εισαγωγή στην Αστροφυσική, Τόμος Β, Εκδόσεις Δίαυλος, 1999.

    \item Ελευθεριάδης Χ., Πυρηνική Φυσική: Βασικές Αρχές και Πυρηνοσύνθεση, Publish City, 2014.
    
    \item Shu F., Αστροφυσική - Δομή και Εξέλιξη του Σύμπαντος, Τόμος I: Αστέρες, Πανεπιστημιακές Εκδόσεις Κρήτης, 2009.
    
    \item Shu F., Αστροφυσική - Δομή και Εξέλιξη του Σύμπαντος, Τόμος II: Γαλαξίες - Ηλιακό Σύστημα, Πανεπιστημιακές Εκδόσεις Κρήτης, 2009.
    
    \item Σπύρου Ν., Αρχές Αστρικής Εξέλιξης, 3η έκδοση, Εκδόσεις Παρατηρητής, 2003.
    
    \item Χριστοπούλου - Μαυρομιχαλάκη Ε., Κοσμική ακτινοβολία, Εκδόσεις Συμμετρία, 2009.
    
\end{itemize}

\end{document}