\chapter{Μαθηματικές Μέθοδοι}
\label{apx:math_tools}

Σε αυτό το παράρτημα παρουσιάζονται μερικά απαραίτητα μαθηματικά εργαλεία που θα χρησιμοποιηθούν για την απόδειξη βασικών νόμων.

\section{Διωνυμικοί και Πολυωνυμικοί συντελεστές}

\subsection{Διωνυμικοί συντελεστές}
Στην βασική άλγεβρα, το διωνυμικό θεώρημα περιγράφει την ανάπτυξη ενός πολυωνύμου της μορφής $(x+y)^n$, σε άθροισμα όρων της μορφής $a x^b y^c$, όπου οι εκθέτες b,c είναι μη-αρνητικοί αριθμοι και $b+c=n$. Ο συντελεστής $a$ ονομάζεται \textit{διωνυμικός συντελεστής} (binomial coefficient) και εξαρτάται από το n και το b (ή το c, καθώς όπως θα δούμε δεν αλλάζει το αποτέλεσμα).

Ο διωνυμικός συντελεστής $a = {n \choose b} = \binom{n}{c}$ προκύπτει από το πεδίο της Συνδυαστικής και εκφράζει τον αριθμό των διαφορετικών συνδυασμών $b$ στοιχείων που μπορεί να προκύψουν από ένα σύνολο $n$ στοιχείων. Ο διωνυμικός συντελεστής διαβάζεται ως "n ανά b" επειδή υπάρχουν ${n \choose b}$ δυνατοί τρόποι για να επιλεγούν $b$ στοιχεία από ένα σύνολο $n$ στοιχείων. Με τη χρήση παραγοντικών, ο διωνυμικός συντελεστής γράφεται 
\begin{equation}
    \label{eq:apx:binomial_coefficient_factorial}
    {n \choose k} = \frac{n!}{k! (n-k)!}
\end{equation}

Έτσι, μπορούμε να γράψουμε
\begin{eqnarray*}
    (x+y)^n &=& {n \choose 0}x^0 y^n + {n\choose 1} x^1 y^{n-1} + {n \choose 2} x^2 y^{n-2} + \dots + \\\\
    &+& {n \choose n-2}x^{n-2} y^2 + {n \choose n-1} x^{n-1} y^1 + {n \choose n} x^n y^0
\end{eqnarray*}
ή, σε πιο συμπαγή μορφή

\begin{equation}
    \label{eq:apx:general_form}
    (x+y)^n = \sum_{k=0}^{n} {n \choose k} x^k y^{n-k} = \sum_{k=0}^{n} {n \choose k} x^{n-k} y^k
\end{equation}

\textbf{Παράδειγμα}\\

Έστω ένα σύνολο τεσσάρων αριθμών $\{1,2,3,4\}$. Θέλουμε να βρούμε πόσους συνδυασμούς των δύο μπορούμε να επιλέξουμε. Αυτός ο αριθμός των 2-υποσυνόλων θα είναι
\begin{equation}
    {4 \choose 2} = \frac{4!}{2! (4-2)!} = 6
\end{equation}

Άρα δωθέντος τεσσάρων αριθμών, υπάρχουν έξι δυνατοί συνδυασμοί υποσυνόλων των δύο, τα οποία είναι
$\{1,2\}, \{1,3\}, \{1,4\}, \{2,3\}, \{2,4\}, \{3,4\}$.


\subsection{Πολυωνυμικοί συντελεστές}
Έχουμε δεί ότι αν θέλουμε να επιλέξουμε $k$ αντικείμενα από $n$, χωρίς επανάθεση, το πλήθος των τρόπων να γίνει αυτό είναι
$${n \choose k} = \frac{n!}{k!(n-k)!}$$
Τι γίνεται αν θέλουμε να επιλέξουμε, πάλι χωρίς επανάθεση, μια ομάδα στοιχείων του $ {\left\{{1,\ldots,n}\right\}}$ μεγέθους $ k_1$, μια ομάδα μεγέθους $ k_2$, κλπ, και τέλος μια ομάδα μεγέθους $ k_r$, όπου για $ j=1,\ldots,r$ έχουμε $ 0 \le k_j \le n$ και επιπλέον ισχύει $ k_1+\cdots +k_r = n$; Με πόσους τρόπους δηλ. μπορούμε να διαμερίσουμε το $ {\left\{{1,\ldots,n}\right\}}$ σε ένα σύνολο μεγέθους $ k_1$, σε ένα σύνολο μεγέθους $ k_2$ και τέλος σε ένα σύνολο μεγέθους $ k_r$;

\textit{\textbf{Θεώρημα}}:\\
Το πλήθος τρόπων να διαμερίσουμε ένα σύνολο με $ n$ στοιχεία σε $ r$ σύνολα με μεγέθη $ k_1,\ldots,k_r$, με $ k_1+\cdots +k_r = n$, όταν δε μας ενδιαφέρει η σειρά των στοιχείων μέσα στα σύνολα αυτά, είναι
\begin{equation}
    \label{eq:apx:polynomial_coefficient}
    {n \choose k_1, \dots, k_r} = \frac{n!}{k_1! k_2! \dots k_r!}
\end{equation}

\textit{\textbf{Απόδειξη}}:\\
Το πρώτο σύνολο μπορεί να επιλεγεί με
$${n \choose k_1} = \frac{n!}{k_1! (n-k_1)!}$$
τρόπους. Μετά από την επιλογή του πρώτου συνόλου απομένουν $ n-k_1$ στοιχεία αχρησιμοποίητα, άρα το δεύτερο σύνολο μπορεί να επιλεγεί με
$${n-k_1 \choose k_2} = \frac{(n-k_1)!}{n_2! (n-k_1-k_2)!}$$
τρόπους. Συνεχίζονας κατ' αυτόν τον τρόπο παίρνουμε ότι η επιλογή του προτελευταίου συνόλου (με $ k_{r-1}$ στοιχεία) μπορεί να γίνει με
\begin{align*}
    {n - k_1 - \dots - k_{r-2} \choose k_{r-1}} &= \frac{(n - k_1 - \dots - k_{r-2})!}{k_{r-1}! (n - k_1 - \dots - k_{r-2} - k_{r-1})!} = \\\\
    &= \frac{(n - k_1 - \dots - k_{r-2})!}{k_{r-1}! k_r!}
\end{align*}
τρόπους. Επίσης, αφού έχουν επιλεγεί τα $ r-1$ πρώτα σύνολα δεν υπάρχει πλέον καμιά επιλογή να γίνει αφου τα υπόλοιπα $ k_r$ στοιχεία που απομένουν ακόμη αχρησιμοποίητα αναγκαστικά πάνε στο τελευταίο σύνολο που πρέπει να επιλέξουμε.
Έτσι πολλαπλασιάζοντας τις δυνατότητες επιλογών μας για τα πρώτα $ r-1$ σύνολα, και κάνοντας τις απλοποιήσεις παίρνουμε τον τύπο \eqref{eq:apx:polynomial_coefficient}.

Το σύμβολο $ {n \choose k_1,\ldots, k_r}$ ονομάζεται πολυωνυμικός συντελεστής (κατ' αναλογία με τα $ {n \choose k}$ που ονομάζονται διωνυμικοί συντελεστές). Παρατηρήστε επίσης ότι

$$\displaystyle {n \choose k, n-k} = {n \choose k} = {n \choose n-k}.$$




\textbf{Παρατήρηση 1}: Ο πολυωνυμικός συντελεστής $ {n \choose k_1,\ldots, k_r}$ δεν αλλάζει αν τα $ k_1,\ldots,k_r$ αντικατασταθούν από μια μετάθεσή τους (αν αλλάξει δηλ. απλώς η σειρά τους).

\textbf{Παρατήρηση 2}: Πρέπει να τονίσουμε εδώ ότι, αν και δε μας ενδιαφέρει η εσωτερική σειρά των συνόλων των στοιχείων που επιλέγουμε, η σειρά των ίδιων των συνόλων είναι προκαθορισμένη. Αυτό είναι ίσως φανερό όταν όλα τα $ k_1, k_2, \ldots, k_m$ είναι μεταξύ τους διαφορετικά αλλά δημιουργεί κάποια σύγχυση όταν μερικά από αυτά είναι μεταξύ τους ίσα. Μια ακραία περίπτωση αυτού είναι όταν όλα είναι ίδια. Για παράδειγμα, ο πολυωνυμικός συντελεστής
$$\displaystyle {9 \choose 3, 3, 3}$$
μετράει με πόσους τρόπους μπορούμε να χωρίσουμε τους αριθμούς $ 1,2,\ldots,9$ σε τρείς ομάδες. Αν δύο τρόποι διαφέρουν μόνο ως προς τον εσωτερικό τρόπο γραφής της κάθε ομάδας τότε δε θεωρούνται διαφορετικοί. Έτσι οι τρόποι
$$ {\left\{{1,2,3}\right\}}, {\left\{{4,5,6}\right\}}, {\left\{{7,8,9}\right\}}
\hspace{0.25cm} \text{και} \hspace{0.25cm} {\left\{{3,2,1}\right\}}, {\left\{{4,5,6}\right\}}, {\left\{{7,8,9}\right\}}
$$
θεωρούνται ίδιοι και μετράνε ως ένα. Αν όμως δύο τρόποι διαφέρουν ως προς τον τρόπο γραφής των ομάδων τότε μετράνε ως διαφορετικοί. Οι τρόποι, π.χ.,
$$ {\left\{{1,2,3}\right\}}, {\left\{{4,5,6}\right\}}, {\left\{{7,8,9}\right\}}
\hspace{0.25cm} \text{και} \hspace{0.25cm} {\left\{{4,5,6}\right\}}, {\left\{{1,2,3}\right\}}, {\left\{{7,8,9}\right\}}
$$
μετράνε ως διαφορετικοί τρόποι.


\section{Πολλαπλασιαστές Lagrange}
Η μέθοδος των πολλαπλασιαστών Lagrange χρησιμοποιείται για την εύρεση ακρότατων μίας συνάρτησης $f(x,y,z)$  των οποίων οι μεταβλητές υπόκεινται σε περιορισμούς της μορφής $g_i(x,y,z) = 0, \ i=1,2, \dots r$.  Τα 
τοπικά ακρότατα προκύπτουν επιλύοντας το παρακάτω σύστημα εξισώσεων

\[
\begin{cases} 
\nabla f = \sum_{i=1}^{r} \lambda_i \nabla g_i \\ \\
g_i(x,y,z) = 0, \ \forall i=1,\dots,r 
\end{cases}
\]
ως προς $x,y,z,\lambda_1, \dots \lambda_r$.

\textbf{Παραδείγματα}\\

\begin{enumerate}
    \item\textbf{ Βρείτε τα μέγιστα και ελάχιστα της $f(x,y) = x^2 + y^2$ που βρίσκονται στην καμπύλη $g(x,y) = x^2 - 2x + y^2 - 4y = 0$.}
    
        Ισχύει ότι:
        \begin{align*}
            \nabla f & = \frac{\partial f}{\partial x} + \frac{\partial f}{\partial y} = 2x + 2y \longrightarrow \nabla f = (f_x, f_y) = (2x,2y) \\\\
            \nabla g & = (g_x, g_y) = (2x-2, 2y-4) \\\\
        \end{align*}
        Άρα, σύμφωνα με τη μέθοδο των πολλαπλασιστών Lagrange πρέπει να λύσω το σύστημα:
        \begin{align*}
            \begin{cases}
                \nabla f = \lambda \nabla g \\\\
                g(x,y) = 0
            \end{cases} &&\Rightarrow
            \begin{cases}
                2x = 2\lambda (x-1) \\\\
                2y = 2\lambda (y-2) \\\\
                x^2 -2x + y^2 - 4y = 0
            \end{cases} &\Rightarrow
            \begin{cases}
                x = \frac{\lambda}{\lambda - 1} \\\\
                y = \frac{2\lambda}{\lambda - 1} \\\\
                x^2 -2x + y^2 - 4y = 0
            \end{cases}
        \end{align*}
        Αντικαθιστώντας τις παραμετρικές τιμές των $x,y$ στην συνάρτηση $g(x,y) = 0$ καταλήγουμε ότι:
        
        \begin{align*}
            \frac{-5\lambda^2 + 10 \lambda}{(\lambda - 1)^2} = 0, \ \lambda \neq 1 \ \Rightarrow \\\\
            -5\lambda (\lambda - 2 ) = 0 \Rightarrow \begin{cases} \lambda = 0 \\\\ \lambda = 2 \end{cases}
        \end{align*}
    
      Άρα, για $\lambda = 0 \longrightarrow (x,y) = (0,0)$, η συνάρτηση $f$ παρουσιάζει πάνω στην $g(x,y) = 0$ ελάχιστο $f(0,0) = 0$. Για $\lambda = 2 \longrightarrow (x,y) = (2,4)$ και η συνάρτηση $f$ παρουσιάζει πάνω στην $g(x,y) = 0$ ελάχιστο $f(2,4) = 20$.
    
    \item \textbf{Βρείτε το μέγιστο της $f(x,y,z) = x^2 + 2y - z^2$ με τους περιορισμούς $2x-y=0$ και $y+z=0$.}
    
        Έστω $g_1(x,y) = 2x-y = 0$, $g_2(y,z) = y+z = 0$ οι δύο συναρτήσεις που δίνουν τους περιορισμούς. Έχουμε λοιπόν:
        \begin{align*}
            \nabla f & = (f_x, f_y, f_z) = (2x, 2, -2z) \\\\
            \nabla g_1 & = (g_{1x}, g_{1y}) = (2, -1) \longrightarrow \lambda_1 \nabla g_1 = (2\lambda_1, - \lambda_1) \\\\
            \nabla g_2 & = (g_{2y}. g_{2z}) = (1,1) \longrightarrow \lambda_2 \nabla g_2 = (\lambda_2, \lambda_2)
        \end{align*}
        Ισχύει ότι:
        \begin{align*}
            \nabla f = \sum_{i=1}^{2} \lambda_i \nabla g_i = (2\lambda_1, -\lambda_1 + \lambda_2, \lambda_2) = (2x, 2, -2z)
        \end{align*}
        Έτσι, προκύπτει το σύστημα:
        \begin{align*}
            \begin{cases}
                2x = 2\lambda_1 \\
                2 = - \lambda_1 + \lambda_2 \\
                -2z = \lambda_2 \\
                2x-y = 0 \\
                y+z = 0
            \end{cases} & \Rightarrow
            \begin{cases}
                \lambda_1  = 2/3 \\
                \lambda_2 = 8/3 \\
                x = 2/3 \\
                y = 4/3 \\
                z = -4/3
            \end{cases}
        \end{align*}
        
        Συνεπώς, η μέγιστη τιμή της $f$ υπό τους δοθέντες περιορισμούς είναι $f(2/3, 4/3, -4/3) = 4/3$
    \item \textbf{Σχεδιάστε ένα μεταλλικό κυλινδρικό δοχείο (με καπάκι) 1 λίτρου, χρησιμοποιώντας την ελάχιστη δυνατή ποσότητα μετάλλου.}
    
        Ο όγκος του κυλίνδρου δίνεται από τη συνάρτηση $$V(\rho, \upsilon) = \pi \rho^2 \upsilon$$ και η συνολική του επιφάνεια (παράπλευρη επιφάνεια και βάσεις) δίνεται από τη συνάρτηση $$S(\rho, \upsilon) = 2 \pi \rho \upsilon + 2 \pi \rho^2$$ όπου $\rho$ και $\upsilon$ είναι η ακτίνα της βάσης και το ύψος του κυλίνδρου, αντίστοιχα. Άρα, καλούμαι να βρω το ελάχιστο της $S(\rho, \upsilon)$ υπό τον περιορισμό $g(\rho, \upsilon) = V(\rho, \upsilon) - 1 = \pi \rho^2 \upsilon - 1 = 0$.
        
        Έτσι έχουμε:
        \begin{align*}
            \nabla S &= (S_{\rho}, S_{\upsilon}) = (2\pi \upsilon + 4 \pi \rho , 2\pi \rho) \\\\
            \nabla V &= (V_{\rho}, V_{\upsilon}) = (2\pi \rho \upsilon, \pi \rho^2) \longrightarrow \lambda \nabla V = (2 \lambda \pi \rho \upsilon,  \lambda \pi \rho^2)
        \end{align*}
        και το σύστημα που πρέπει να λύσω είναι:
        \begin{align*}
            \begin{cases}
                \nabla S = \lambda \nabla V \\
                \pi \rho^2 \upsilon - 1 = 0
            \end{cases} &\Rightarrow
            \begin{cases}
                \upsilon + 2 \rho = \lambda \rho \upsilon \\
                \lambda \rho = 2 \\
                \pi \rho^2 \upsilon - 1 = 0
            \end{cases} &\Rightarrow
            \begin{cases}
                \displaystyle \rho = \frac{1}{\sqrt[3]{2\pi}} \\\\
                \displaystyle \upsilon = \frac{2}{\sqrt[3]{2\pi}} \\\\
                \lambda = 2 \sqrt[3]{2\pi}
            \end{cases}
        \end{align*}
        δηλαδή το ζητούμενο κυλινδρικό δοχείο πρέπει να έχει ακτίνα βάσης $\displaystyle \rho = \frac{1}{\sqrt[3]{2\pi}}$ και ύψος $\displaystyle \upsilon = \frac{2}{\sqrt[3]{2\pi}}$.
\end{enumerate}















