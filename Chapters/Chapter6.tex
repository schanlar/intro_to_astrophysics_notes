\chapter{Αστρικά κατάλοιπα}
\label{ch:Chapter6}

Όταν σταματήσουν οι θερμοπυρηνικές αντιδράσεις στο εσωτερικό των άστρων, τότε ο πυρήνας αρχίζει να ψύχεται, επειδή δεν αναπληρώνονται τα ποσά της ενέργειας που ρέουν προς τα εξωτερικά στρώματα του αστέρα. Η ψύξη του πυρήνα, όμως, συνεπάγεται πτώση της θερμικής πίεσης στο εσωτερικό του, οπότε η πίεση των υπερκείμενων στρωμάτων αρχίζει να υπερισχύει της θερμικής πίεσης του αερίου, με αποτέλεσμα ο πυρήνας να αρχίσει να συστέλλεται. Αν η μάζα του είναι μικρή ($M < 1\,M_\odot$), η συστολή του δεν συνοδεύεται, συνήθως, από καταστροφικά φαινόμενα. Αντίθετα, η ύλη αστέρων μεγάλης μάζας υφίσταται καταστροφική "κατάρρευση", η οποία συνήθως ακολουθείται από έκρηξη, και η ισορροπία των δυνάμεων που διέπουν την ύπαρξη της τελικής κατάστασης, στην οποία θα περιπέσουν αυτοί οι αστέρες, είναι πολύ λεπτή.

Με τις σημερινές γνώσεις της Φυσικής πιστεύουμε ότι είναι δυνατόν να υπάρξουν τριών ειδών τελικές καταστάσεις, όταν σταματήσει οριστικά η παραγωγή ενέργειας από θερμοπυρηνικές αντιδράσεις, στις οποίες γενικά αναφερόμαστε ως \textbf{συμπαγείς αστέρας} (compact stars) επειδή έχουν μικρές τυπικές διαστάσεις και μεγάλες πυκνότητες. Μία τέταρτη περίπτωση κατά την οποία ο αστέρας διαλύεται, με την ύλη να διασκορπίζεται στο μεσοαστρικώ χώρο χωρίς να αφήνει πίσω κάποιο κατάλοιπο, θα ζηζητηθεί στο Κεφάλαιο \ref{ch:Chapter7}.


\section{Λευκοί νάνοι}
{\color{red} \hrule}
Λευκός νάνος είναι ο εκφυλισμένος, γυμνός πυρήνας ενός αστέρα με σχετικά μικρή αρχική μάζα ($M \leq 5\,M_\odot$). Αρχικά, ο λευκός νάνος έχει πολύ υψηλή θερμοκρασία. Με την πάροδο του χρόνου, όμως, η θερμοκρασία του συνεχώς μειώνεται, μέχρις ότου πάψει να ακτινοβολεί θερμικά. Στην περίπτωση αυτή η υδροστατική ισορροπία του αστέρα εξασφαλίζεται από την (κβαντομηχανικής και όχι θερμικής προέλευσης) πίεση των ηλεκτρονίων. \\
{\color{red} \hrule}




\section{Αστέρες νετρονίων}
{\color{red} \hrule}
Αστέρας νετρονίων είναι ο εκφυλισμένος, γυμνός πυρήνας ενός αστέρα με σχετικά μεγάλη αρχική μάζα. Η εξέλιξη της θερμοκρασίας του ακολουθεί, όπως πιστεύουμε σήμερα, την πορεία της θερμοκρασίας των λευκών νάνων. Σ' αυτήν την περίπτωση η υδροστατική ισορροπία εξασφαλίζεται από την κβαντομηχανικής φύσεως πίεσης των νετρονίων. Δεν αποκλείεται ενάς τέτοιος αστέρας να παρουσιαστεί ενεργά στον ουρανό υπό την μορφή ενός pulsar, που γίνεται ορατός με παρατηρήσεις σε ραδιαφωνικά, κυρίως, μήκη κύματος.\\
{\color{red} \hrule}





\section{Μαύρες τρύπες}




{\color{red} \hrule}
Στην περίπτωση των μελανών οπών, η υδροστατική ισορροπία του αστέρα έχει καταστραφεί, επειδή η διαθέσιμη πίεση (θερμικής ή κβαντομηχανικής προέλευσης) δεν είναι ικανή να αντισταθμίσει την βαρυτική. Η μάζα του αστέρα έχει καταρρεύσει, δημιουργώντας ένα αντικείμενο εξαιρετικά μεγάλης πυκνότητας. Σε κάθε μελανή οπή μπορούμε να αντιστοιχήσουμε ένα χαρακτηριστικό μήκος, $R_S$, που ονομάζεται ακτίνα Schwarzschild, με τη σχέση $R_S = 2GM/c^2$. Το βαρυτικό πεδίο μιας μελανής οπής είναι τόσο ισχυρό, ώστε σε απόσταση μικρότερη από την ακτίνα Schwarzschild ακόμα και το φως δεν μπορεί να διαφύγει από την βαρυτική έλξη.\\
{\color{red} \hrule}

